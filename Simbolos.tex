\chapter{LISTA DE SÍMBOLOS}

\begin{longtable}{@{}l@{\extracolsep{\fill}} p{4.75in} @{}}
  \textbf{Símbolo} & \textbf{Significado}\\[12pt]
  \endhead
%LETRA C

    {\boldmath $a \mid b$} & $a$ divide a $b$\\[3pt]
    {\boldmath $a \equiv b$} & $a$ es congruente con $b$\\[3pt]
    {\boldmath $A \simeq B$} & $A$ es isomorfo a $B$\\[3pt]
    {\boldmath $A \to B$} & $A$ se mapea en $B$\\[3pt]
    {\boldmath $a \nmid b$} & $a$ no divide a $b$\\[3pt]
    {\boldmath $A \triangleleft B$} & $A$ es subgrupo normal de $B$\\[3pt]
    {\boldmath $Ann_d(X)$} & Aniquilador del conjunto $X$ por la derecha\\[3pt]
	{\boldmath $Ann_i(X)$} & Aniquilador del conjunto $X$ por la izquierda\\[3pt]
	{\boldmath $\car(K)$} & Característica del campo $K$\\[3pt]
	{\boldmath $\mathds{Z}$} & Conjunto de números enteros\\[3pt]
	{\boldmath $\mathds{Q}$} & Conjunto de números racionales\\[3pt]
	{\boldmath $\mathds{R}$} & Conjunto de números reales\\[3pt]
	{\boldmath $\emptyset$} & Conjunto vacío\\[3pt]
	{\boldmath $[a,b]$} & Conmutador de a y b\\[3pt]
	{\boldmath $\cos$} & Función trigonométrica coseno\\[3pt]
	{\boldmath $\sin$} & Función trigonométrica seno\\[3pt]
	{\boldmath $\grado$} & Grado de una representación de un polinomio\\[3pt]
	{\boldmath $\hom_K(A,A)$} & Conjunto de homomorfismos de $A$ en $A$ como $K\mbox{-módulos}$\\[3pt]
	{\boldmath $\implies$} & Implicación\\[3pt]
	{\boldmath $\infty$} & Infinito\\[3pt]
	{\boldmath $\cap$} & Intersección de conjuntos\\[3pt]
	{\boldmath $\ker(f)$} & Kernel de la función $f$	\\[3pt]
	{\boldmath $\neq$} & No igual a \\[3pt]
	{\boldmath $\notin$} & No pertenece a\\[3pt]
	{\boldmath $\|a\|$} & Norma del vector $a$\\[3pt]
	{\boldmath $\circ(p)$} & Orden del elemento $p$\\[3pt]
	{\boldmath $\in$} & Pertenencia \\[3pt]
	{\boldmath $\subseteq$} & Subconjunto\\[3pt]
	{\boldmath $\subset$} & Subconjunto propio\\[3pt]
	{\boldmath $\oplus$} & Suma directa\\[3pt]
	{\boldmath $\sum$} & Sumatoria\\[3pt]
	{\boldmath $\colon$} & Tal que\\[3pt]
	{\boldmath $\tr(A)$} & Traza de la matriz $A$\\[3pt]
	{\boldmath $\cup$} & Unión de conjuntos\\[3pt]
	{\boldmath $x \mapsto y$} & $x$ se mapea en $y$\\[3pt]
	
\end{longtable}
