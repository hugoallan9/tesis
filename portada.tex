% para generar en el pdf la barra naranja que identifica a la escuela
% utilizar el siguiente c�digo, de lo contrario no hacer cambios
%\vspace*{-1.63in}\hspace{-1.97in}{\color[rgb]{1,.5,0}{\rule{8.5in}{5mm}}}
\vspace*{-1.63in}\hspace{-1.97in}%{\color[rgb]{1,.5,0}{\rule{8.5in}{5mm}}}
%\vspace*{-1.63in}\hspace{-1.9in}{\color[rgb]{1,1,1}{\rule{1in}{5mm}}}
\vspace{12pt} \pdfbookmark[0]{PORTADA}{prt} \thispagestyle{empty}

\begin{figure}[h]
\setbox0 \vbox{\begin{flushleft}
\hspace{-10mm}\includegraphics[width = 2.5cm]{Pictures/escudo}\\
\end{flushleft}}
\wd0 = 0pt \ht0 = 0pt \box0
\end{figure} \hangindent = 18mm \hangafter = -3\vspace{-4mm}
\noindent Universidad de San Carlos de Guatemala\\
Facultad de Ingeniería \\ Escuela de Ciencias

\vspace{5.5cm}

\begin{center}
{\large \textbf{TEORÍA DE LOS GRUPOS-ANILLOS Y SUS APLICACIONES}}

\vspace{3.5cm}

\textbf{Hugo Allan García Monterrosa}\\[6pt]
Asesorado por el Lic. William Roberto Gutiérrez Herrera \\[5cm]

Guatemala, FECHA
\end{center}
