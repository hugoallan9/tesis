\chapter{GLOSARIO}
%\addcontentsline{toc}{chapter}{GLOSARIO}
\vspace{.5cm}
\begin{flushleft}
\renewcommand{\arraystretch}{1} % Espacio entre elementos 
\begin{longtable}{@{}l@{\extracolsep{1.6cm}}  p{4.0in}@{} l@{}}
\textbf{Aplicación} & Se refiere a una regla que asigna a cada elemento de un primer conjunto, un único elemento de un segundo conjunto. \\
&\\
\multirow{2}{3cm}{\textbf{Campo ciclotómico}} & Es un cuerpo numérico que se obtiene al añadir una raíz primitiva de la unidad compleja a el campo de los números racionales.\\
&\\
\textbf{Cubierta} & Es una colección de subconjuntos $A$ de un conjunto $X$, tal que la unión de los elementos de la colección $A$ es igual a $X$. \\
&\\
\multirow{2}{2cm}{\textbf{Grupo soluble}} & Es un grupo para el cual existe una cadena finita de subgrupos $\{G_i\}_{i=1}^{n}\subset G$ tal que
 $\{1_G\}=G_0\subseteq G_1 \subseteq \dots \subseteq G_n = G$ donde para cada $i=0,1,\dots,n-1$ se cumple que $G_i$  es subgrupo normal en $G_{i+1}$ y
 el grupo cociente  $G_{i+1}/G_i$  es abeliano.\\
&\\
\multirow{2}{3cm}{\textbf{Inducción matemática}} & Es un razonamiento que permite demostrar una infinidad de proposiciones, o una proposición que depende de un parámetro $n$, que toma una infinidad de valores enteros.\\
&\\
\textbf{Norma} & Es la función que determina el tamaño de un elemento de un espacio vectorial.\\
&\\
\multirow{2}{3.5cm}{\textbf{Registro de desplazamiento}} &  Es un circuito digital secuencial consistente en una serie de biestables, generalmente de tipo D, conectados en cascada que basculan de forma sincrónica con la misma señal de reloj.\\
&\\
\textbf{Tupla} & Es una secuencia ordenada de objetos, esto es, una lista con un número limitado de objetos.\\
&\\
\end{longtable}
\end{flushleft}



