\chapter{\hspace{1cm} ELEMENTOS ALGEBRAICOS}

\section{\quad Generalidades y definiciones}
En este capítulo será de especial interés estudiar algunos elementos algebraicos en grupo-álgebras usando la representación regular que puede ser definida para un álgebra finito dimensional con unidad sobre un campo $K$ de la siguiente manera.

\begin{definicion}
Sea $T \colon A \to Hom_k(A,A)$ tal que $a \mapsto T_a \in Hom_k(A,A)$, definida mediante multiplicación por la izquierda por $a$. Esto es, $T_a$ es una aplicación tal que $T_a(x) = ax$, para cualquier $x \in A$.  
\end{definicion}

Se puede observar a partir de la definición que
\begin{eqnarray*}
T_{a+b} &=& T_a + T_b \\
T_{ab} &=& T_aT_b \\
T_{ka} &=& kT_a
\end{eqnarray*}
para todo $a,b \in A,$  $k\in K$. Mas aún, la aplicación $a \mapsto T_a$ es inyectiva debido a que $T_a(1) = a$. Eligiendo una base $\{ a_1, \dots, a_n \}$ de $A$ sobre $K$ se puede representar a $T_a$ con una matriz $\rho(a)$, con lo que  se obtiene la representación matricial:
\begin{equation*}
a \mapsto \rho(a) \in M_n(K).
\end{equation*}
Si $a$ es un elemento algebraico de $A$, esto es, si existe un polinomio no nulo $f(X) \in K[X]$ tal que $f(a) = 0$, entonces los valores propios de la matriz $\rho(a)$ también anulan a $f(X)$, debido al teorema de Cayley-Hamilton (véase \cite[p 241]{bib:lang}). De esta manera, por ejemplo, si $a$ es un elemento nilpotente entonces los valores propios de $\rho(a)$ son todos cero. Si $a$ es de orden multiplicativo finito, es decir, si $a^m = 1$ para algún $m$ entero positivo, entonces los valores propios de $\rho(a)$ son raíces de la unidad de orden $m$. 
\begin{lema}\label{lem:traza}
Sea $G$ un grupo finito y $K$ un campo. Sea $p$ la representación regular de $KG$ y $\gamma = \sum_{g \in G}\gamma(g)g \in KG$. Entonces la traza de $\rho(\gamma)$ viene dada por
\begin{equation*}
tr\rho(\gamma) = |G|\gamma(1).
\end{equation*}
\end{lema} 
\begin{proof}
Se sabe que $tr\rho(\gamma)$ es independiente de la base elegida, así que se elige $G=\{g_1, \dots, g_n \}$ como $K$-base para $KG$ y se asume que $g_1=1$.
Entonces
\begin{equation*}
\rho(\gamma) = \rho\left( \sum_{g \in G}\gamma(g)g \right) = \sum_{g \in G}\gamma(g)\rho(g) .
\end{equation*}
Para un elemento $g\neq 1 \in G$, se tiene $gg_i \neq g_i$, para $1\leq i\leq n$, de donde se sigue que los elementos de la diagonal de la matriz $\rho(g)$ son todos nulos si $g \neq 1$. Así $\tr\rho(g) = 0$ si $g \neq 1$. Más aún, como $\rho(1)$ es la matriz identidad, se tiene que $\tr\rho(1) = n$. Entonces
\begin{equation*}
\tr\rho(\gamma) = \sum_{g \in G}\gamma(g)\tr(g) = \gamma(1)\tr\rho(1) = \gamma(1)|G|.
\end{equation*} 
\end{proof}

\begin{lema}\label{lem:BH}
Sea $\gamma = \sum_{g \in G}\gamma(g)g$ una unidad de orden finito en el grupo-anillo integral $\mathds{Z}G$ con $G$ un grupo finito y asúmase que $\gamma(1)\neq 0$. Entonces $\gamma = \gamma(1) = \pm 1.$
\end{lema}
\begin{proof}
Sea $|G| = n$ y supóngase que $\gamma^{m} = 1$ para algún entero positivo $m$. Si se considera la representación regular $\rho$ del grupo-álgebra $\mathds{C}G$ y a $\mathds{Z}G$ como un subanillo de la misma, se tiene que $\tr\rho(\gamma) = n\gamma(1)$.
Como $\gamma^{m} = 1$, entonces $\left( \rho(\gamma) \right)^m = \rho(\gamma^{m}) = I$, de esto se sigue que $\rho(\gamma)$ es raíz del polinomio $X^m-1$, cuyas raíces son todas distintas. Esto implica, por el teorema espectral(véase \cite[p 214]{bib:lang}) que existe una base de $\mathds{C}G$ donde la matriz de $\rho(\gamma)$ es diagonal de la forma
\begin{equation*}
\mathds{A} = \begin{pmatrix}
\xi_1 & & & &\\
 & \xi_2 & & \\
 & & \ddots & \\
  & & & \xi_n
\end{pmatrix}, \quad \xi_i^m = 1.
\end{equation*}
 Entonces $\tr\rho(\gamma) = \sum\limits_{i = 1}^{n}\xi_i$ y así
 \begin{equation*}
 n\gamma(1) = \sum\limits_{i =1}^{n}\xi_i.
 \end{equation*}
 Por lo tanto, aplicando valor absoluto,
 \begin{equation*}
 \mid n\gamma(1) \mid = \left| \sum_{i = 1}^{n}\xi_i \right| \leq \sum_{i = 1}^{n}\mid \xi \mid = n.
 \end{equation*}
 Como $\mid n\gamma(1)\mid = n\mid\gamma(1)\mid \leq n$, entonces $\mid \gamma(1) \mid = 1$ y $\left| \sum_{i = 1}^{n}\xi_i \right| = \sum\limits_{i = 1}^{n}\mid \xi_i \mid$, lo cual sucede si y sólo si $\xi_1=\xi_2=\cdots = \xi_n$.
 
 Así $n\gamma(1) = n\xi_1$ y $\gamma(1) = \xi_1=\pm 1$. Se concluye que $\rho(\gamma) = \pm I$, de donde, $\gamma = \pm 1$.
\end{proof}
\begin{corolario}
Supóngase que $\gamma = \sum_{g \in G}\gamma(g)g$ es una unidad central en el grupo-anillo integral $\mathds{Z}G$ con $G$ un grupo finito de orden finito. Entonces $\gamma$ es de la forma $\gamma = \pm g$ con $g \in  \mathcal{Z}(G)$.
\end{corolario}
\begin{proof}
Sea $\gamma = \sum_{g \in G}\gamma(g)g$ una unidad central de orden $m$. Supóngase que $\gamma(g_0) \neq 0$, para algún $g_0 \in G$. Entonces $\gamma g_0^{-1}$ es también una unidad de orden finito en $\mathds{Z}G$. Mas aún, el coeficiente de $1$ en la expresión de $\gamma g_0^{-1}$ es $\gamma(g_0) \neq 0$, de donde $\gamma g_0^{-1} = \pm 1$ y por lo tanto $\gamma = \pm g_0$.
\end{proof}
Una consecuencia inmediata del corolario anterior es el siguiente
\begin{teorema}\label{teo:Graham-Higman}
Sea $A$ un grupo abeliano finito. Entonces, el grupo de torsión de las unidades del grupo-anillo integral $\mathds{Z}A$ es igual $\pm A$.
\end{teorema}

Ahora se desea hacer un estudio de los elementos idempotentes. Es evidente que en cualquier anillo con unidad el $0$ y el $1$ son elementos idempotentes. Estos elementos son llamados \textbf{idempotentes triviales} de un anillo. Se verá a continuación que los elementos idempotentes $e$ en un grupo-álgebra están fuertemente influenciados por su primer coeficiente $e(1)$.

\begin{teorema}\label{teo:idme}
Sea $G$ un grupo finito y $K$ un campo de característica cero. Supóngase que $e \in KG$ y $e$ es idempotente. Entonces:
\begin{enumerate}
\item $e(1) \in \mathds{Q}$
\item $0 \leq e(1) \leq 1$
\item $e(1) = 0 \Leftrightarrow e = 0$ y $e(1) = 1 \Leftrightarrow e=1$.   
\end{enumerate} 
\end{teorema}
\begin{proof}
Considérese la representación regular de $KG$ escrita con respecto a la base $G$ de $KG$. Entonces, por el lema \ref{lem:traza}, se tiene que $\tr\rho = \mid G \mid e(1)$. Más aún, como $e^2 = e$, $\rho(e)$ satisface el polinomio $X^2 - X = X(X-1)$ y por lo tanto $\rho(e)$ puede ser diagonalizada. Los valores propios de $\rho(e)$ son $0$ o $1$ ya que $\rho(e)$ es idempotente. Debido a que la traza es la suma de los valores propios, se tiene que $\tr\rho(e) = r$, donde $r$ es el número de valores propios iguales a $1$ y por lo tanto también es el rango de $\rho(e)$. Se concluye entonces que $0\leq e(1) \leq 1$.

Nótese que $e(1) = 0 $ si y sólo si el rango de $\rho(e)$ es $0$ y eso pasa sólo si $e = 0$. Similarmente $e(1) = 1$ si y sólo si el rango de $\rho(e)$ es $\mid G \mid $, lo cual pasa sólo si $\rho(e)$ es la matriz identidad, es decir, si $e = 1$. 
\end{proof} 

\section{\quad Elementos Idempotentes}
Se ha demostrado en el teorema \ref{teo:idme} que si $K$ es un campo de característica cero y $G$ es un grupo finito, entonces cualquier elemento idempotente $e \in KG$ cumple que $e(1) \in \mathds{Q}$. Se dará, en esta sección, un resultado análogo a este resultado, donde $K$ tiene característica $p > 0$.

\begin{teorema}\label{teo:genidem}
Sea $K$ un campo de característica $p > 0$ y sea $G$ cualquier grupo. Supóngase que $e \in KG$ es un idempotente. Entonces $e(1) \in F_p$, donde $F_p$ es el subcampo primo de $K$. 
\end{teorema}

La demostración de este resultado está fuera del alcance de este trabajo, pero se recomienda al lector consultar \cite{bib:passman}. 

En el teorema \ref{teo:idme} se demostró el teorema anterior cuando la característica del campo es cero con la condición de que el grupo sea finito, pero dicho resultado es válido aún cuando el grupo es infinito, pero su demostración requiere el uso de resultados previos de teoría de números. Para la demostración del siguiente resultado, se sugiere al lector consultar \cite{bib:passman}

\begin{teorema}\label{teo:genidem23}
Sea $G$ un grupo cualquiera y $K$ un campo de característica cero. Supóngase que $e = e^2 = \sum e(g)g \in KG$. Entonces:
\begin{enumerate}
\item $e(1) \in \mathds{Q}$
\item $0\leq e(1) \leq 1$
\item $e(1) = 0 \Leftrightarrow e= 0 $ y $e(1) = 1 \Leftrightarrow e = 1$. 
\end{enumerate}
\end{teorema} 

Supóngase que $e = e^2 \in \mathds{Z}G$, como $e(1)$ es un entero, se sigue del teorema anterior que $e = 0$ o $e = 1$. Así, se obtiene el siguiente
\begin{corolario}
El grupo-anillo integral $\mathds{Z}G$ sólo contiene idempotentes triviales, para cualquier grupo $G$.
\end{corolario}

\section{\quad Unidades de Torsión}
Se demostró en el lema \ref{lem:BH} que si $G$ es un grupo finito, $\gamma \in \mathds{Z}G$ es una unidad de orden finito y $\gamma(1) \neq 0$ entonces $\gamma = \pm 1$.Dicho resultado es válido también cuando $G$ es un grupo infinito. 

\begin{teorema}\label{teo:Passman-Bass}
Sea $\gamma = \sum\gamma(g)g \in \mathds{Z}G$ que satisface $\gamma^n = 1$, para algún entero positivo $n$. Si $\gamma(1) \neq 0$ entonces $\gamma = \pm 1$.
\end{teorema} 

\begin{proof}
Sea $\mathds{C}[X]$ el anillo de polinomios con coeficientes en $\mathds{C}$. Considérese el homomorfismo $\phi \colon \mathds{C}[X] \to \mathds{C}[\gamma]$ dada por $X \mapsto \gamma$. El kernel de este homomorfismo es el ideal $\textless f(X) \textgreater$ generado por el polinomio minimal $f(X)$ de $\gamma$. Entonces $f(X)$ divide a $X^n - 1$ y por lo tanto tiene sus raíces distintas. 
Así, se tiene 
\begin{equation*}
\mathds{C}[\gamma] \simeq \frac{\mathds{C}[X]}{\textless f(X) \textgreater} \simeq \mathds{C}\oplus\mathds{C}\oplus\cdots\oplus\mathds{C} \simeq \oplus_i\mathds{C}e_i,
\end{equation*}
donde los $e_i$ son idempotentes ortogonales primitivos de $\mathds{C}[\gamma]$. De lo anterior, se puede escribir $\gamma = \sum_{i}\xi_ie_i$ donde $\xi_i\in \mathds{C}$, $\xi_i^n = 1$ y $e_ie_j = \delta_{ij}e_j$, con $\delta_{ij}$ la función delta de Kronecker. 

Calculando el primer coeficiente en ambos lados de la ecuación y usando el teorema \ref{teo:genidem23} se obtiene
\begin{equation*}
\gamma(1) = \sum\xi_ie_i(1) = \sum\xi_i\frac{r_i}{s}, \quad \mbox{con } r_i, s \in \mathds{Z}, \quad r_i,s \geq 0.
\end{equation*}
Entonces, $s\gamma(1) = \sum\xi_ir_i$. De igual manera; como $\sum e_i = 1$ se tiene que $1 = \sum \frac{r_i}{s}$, así $\sum r_i = s$. De donde
\begin{equation*}
\mid s\gamma(1) \mid = \left| \sum\xi_ir_i  \right| \leq \sum \mid \xi \mid \mid r_i \mid = \sum \mid r_i \mid = s.
\end{equation*}
Como $\mid s\gamma(1) \mid \leq s$, se tiene $\mid \gamma(1) \mid = 1$ y también $\left| \sum \xi_ir_i \right| = \sum \mid \xi_i \mid \mid r_i \mid$. Se sigue que todos los $\xi_i$ son iguales y $\gamma = \sum \xi_1e_1 = \xi_1 = \gamma(1) \in \mathds{Z}$. 
\end{proof}

Este último resultado tiene bastantes consecuencias útiles. El lector deberá recordar que, como se mostró en la preposición \ref{prep:conjugacion}, hay una involución estándar en $\mathds{Z}G$ dada por 
\begin{equation*}
\gamma = \sum\gamma(g)g \mapsto \gamma* = \sum\gamma(g)g^{-1},
\end{equation*}
tal que
\begin{eqnarray*}
(\gamma^*)^* &=& \gamma \\
(\gamma + \mu)^* &=& \gamma^* + \mu^*\\
(\gamma\mu)^* &=& \mu^*\gamma^*\\
(c\gamma)^* &=& c\gamma^*,
\end{eqnarray*}
para todo $\gamma, \mu \in \mathds{Z}G$ y $c \in \mathds{Z}$. Más aún, 
\begin{equation*}
(\gamma\gamma^*)(1) = \sum(\gamma(g))^2
\end{equation*}
lo cual implica que $\gamma\gamma^* = 0 $ si y sólo si $\gamma = 0$.

\begin{corolario}
Supóngase que $\gamma \in \mathds{Z}G$ tiene la propiedad de conmutar con $\gamma^*$. Si $\gamma$ es una unidad central de orden finito, entonces $\gamma = \pm g_0$ para algún $g_0 \in G$.
\end{corolario}
\begin{proof}
Por hipótesis $\gamma^n = 1$ para algún entero positivo $n$ y $\gamma\gamma^* = \gamma^*\gamma$, por lo tanto $(\gamma\gamma^*) = 1$. Más aún, $(\gamma\gamma^*)(1) = \sum \gamma(g)^2 \neq 0$. Entonces, por el teorema anterior, $\gamma\gamma^* = 1$. De esta manera, existe un único coeficiente $\gamma(g_0)$ que es distinto de cero. Se concluye entonces que $\gamma = \pm g_0$. 
\end{proof}
Como consecuencia inmediata se tiene 
\begin{corolario}
Todas las unidades centrales de orden finito en $\mathds{Z}G$ son triviales.
\end{corolario}
\begin{corolario}
Si $A$ es un grupo abeliano cualquiera, entonces todas las unidades de torsión de $\mathds{Z}A$ son triviales.
\end{corolario}

\section{\quad Elementos nilpotentes}
Ahora se desea clasificar los grupo-álgebras $KG$ de un grupo finito $G$ sobre un anillo $K$ tal que $KG$ no tiene elementos nilpotentes no triviales. Es posible observar que si $\car(K) = p > 0$ y $G$ contiene un elemento $g$ tal que $g^{p^{n}} = 1$ para algún entero positivo $n$, entonces $(g - 1 )^{p^{n}} = g^{p^{n}} - 1 = 0$. De esto se sigue el resultado
\begin{proposicion}\label{prop:nilpotentes}
Si $K$ es un campo de característica $p > 0 $ y $G$ contiene $p$-elementos, entonces $KG$ contiene elementos nilpotentes.  
\end{proposicion}
A partir de este punto se asumirá que $G$ es finito, $p 
\leq 0$ y que si $p >0 $ entonces $G$ no tiene $p$-elementos. Supóngase que $KG$ no contiene elementos nilpotentes a excepción de los triviales y sea $e \in KG$ un idempotente. Entonces, para cada $x \in KG$, el idempotente $\eta = ex(1-e)$ satisface $\eta^2 = 0$ y por lo tanto $ ex = exe $. Similarmente si $\eta = (1 - e)xe$ entonces $\eta^2 = 0$ y $xe = exe$, con lo que se comprueba que $e$ es central. Ahora para cualquier $g \in G$, el elemento $e = \frac{\hat{g}}{\circ (g)} = \frac{\sum_{i = 1}^{\circ(g)}g^i}{\circ(g)}$ es idempotente y por lo tanto es central. Esto significa que el subgrupo $\textless g \textgreater$ es normal para cualquier $g \in G$. Se sigue del teorema \footnote{poner el teorema en el capítulo 1 que sustenta esta parte y luego hacer referencia} que $G$ es abeliano o hamiltoniano. 

En el caso que $G$ sea hamiltoniano, $G = K_8 \times E \times A$, donde $K_8$ es el grupo de cuaterniones de orden ocho, $E^2 = 1$ y $A$ es un grupo abeliano de orden impar. Para obtener más información de este caso es necesario hacer un estudio mas profundo del grupo-álgebra $FK_8$, donde $F$ es un campo.

\begin{proposicion}\label{prop:AH}
Sea $K$ un campo de característica $p \geq 0$ y sea $G$ un grupo finito. Si $KG$ no tiene elementos nilpotentes entonces todos los idempotentes de $KG$ son centrales y $G$ es abeliano o Hamiltoniano.
\end{proposicion}

En lo que sigue el lector deberá recordar el concepto de números cuaterniones. Los números cuaterniones, con coeficientes racionales, se escriben como sumas directas de espacios vectoriales:
\begin{equation*}
H_{\mathds{Q}} = \mathds{Q}\oplus\mathds{Q}i\oplus\mathds{Q}j\oplus\mathds{Q}k,
\end{equation*}
con su ya conocida multiplicación (véase \cite[p. 31]{bib:herstein}). Se puede definir formalmente una estructura similar sobre cualquier campo $F$. Considérese el espacio vectorial
\begin{equation*}
H(F) = F\oplus Fi\oplus Fj\oplus Fk 
\end{equation*} 
y defínase la multiplicación distributivamente con $i^2 = j^2 = k^2 = -1$, $ij = k = -ji$, $jk = i = -kj$ y $ki = j = -ik$. De esta forma $H(F)$ es un anillo no conmutativo.

Para un elemento $\alpha = a + bi + cj + dk \in H(F)$ se define:
\begin{equation*}
\overline{\alpha} = a - bi - cj -dk
\end{equation*}
y 
\begin{equation*}
\alpha'= a - bi +cj + dk.
\end{equation*}
Luego de hacer algunos cálculos sencillos se obtiene
\begin{lema}\label{lem:propq} Sea $\alpha \in H(F)$. Entonces: 
\begin{enumerate}
\item $\alpha\overline{\alpha} = a^2 + b^2 + c^2 + d^2$
\item $\alpha'\alpha = (a^2 + b^2 - c^2 -d^2) + (2ac + 2bd)j + (2ad -2bc)k$
\end{enumerate}
Al escalar $N(\alpha) := \alpha\overline{\alpha}$ se le llama la norma de $\alpha$.
\end{lema}
\begin{proposicion}
El álgebra de los cuaterniones $H(F)$ tiene divisores de cero si y sólo si la ecuación $X^2 + Y^2 = -1$ tiene solución en $F$.
\end{proposicion}
\begin{proof}
Supóngase que existen elementos $a,b \in F$ tal que $a^2 + b^2 = -1$. Entonces, para $\alpha = a + bi + j$ se tiene $N(\alpha) = \alpha\overline{\gamma} = 0$; $\alpha$ es divisor de cero en $H(F)$.

Falta demostrar que si $H(F)$ tiene divisores de cero, entonces la ecuación $X^2 + Y^2 = -1$ tiene soluciones en $F$. Cuando $F$ es de característica dos se tiene $1 + 0 = -1$ entonces se asumirá que $\car (F) \neq 2$. Supóngase que existen elementos $\alpha = a + bi + cj + dk \neq 0$ y $\beta \neq 0 \in H(F)$ tal que $\alpha\beta = 0 $. Entonces, $\overline{\alpha}\alpha\beta = N(\alpha)\beta = 0 $ de donde $N(\alpha) = a^2 +b^2 + c^2 + d^2 = 0$. Si alguno de los coeficientes de $\alpha$ es cero, se tiene que la ecuación $X^2 + Y^2 =-1$ solución en $F$. Si, por el contrario, todos los coeficientes de $\alpha$ son distintos de cero se puede considerar $\alpha'\alpha\beta = 0$ y del lema \ref{lem:propq} se sabe que $\gamma = \alpha'\alpha$ a lo sumo tiene tres coeficientes no nulos y por lo tanto $N(\gamma)\beta = 0$ implica que $X^2 + Y^2 = -1$ tiene solución en $F$. Por último falta considerar el caso en que $\gamma = 0$, en cuyo caso se tiene, por el lema \ref{lem:propq}, que $a^2 + b^2 -c^2 -d^2 = 0 $ y de $N(\alpha) = 0 $ se sigue  que $a^2 + b^2 + c^2 + d^2 = 0 $ entonces $a^2 + b^2 = 0$, de donde $\left( \frac{a}{b} \right)^2 + 0 = -1$.
\end{proof}
Este resultado se puede ampliar de la siguiente manera.
\begin{proposicion}
Las siguientes proposiciones son equivalentes:
\begin{enumerate}
\item El álgebra de los cuaterniones $H(F)$ no tiene divisores de cero
\item La ecuación $X^2 + Y^2 = -1$ no tiene solución en $F$
\item $H(F)$ es un anillo de división
\end{enumerate}
\end{proposicion}

\begin{proof}
Para la demostración de esta proposición solo falta probar que si $H(F)$ no tiene divisores de cero entonces es un anillo de división. Sea $0 \neq \alpha \in H(F)$ entonces $\overline{\alpha}\alpha = N(\alpha) \neq 0$ y por lo tanto $\alpha\left( \frac{\overline{\alpha}}{N(\alpha)} = 1\right)$.
\end{proof}

\begin{teorema}\label{teo:caracterizacion}
Supóngase que $\car(F) \neq 2$. Entonces el álgebra de los cuaterniones $H(F)$ es un anillo de división o isomorfo a $M_2(F)$, el anillo de matrices de $2\times 2$ sobre el campo $F$. La última opción sucede únicamente si $X^2 + Y^2 = -1$ tiene solución en $F$.
\end{teorema}

\begin{proof}
Se debe demostrar que si $H(F)$ no es un anillo de división entonces es isomorfo a $M_2(F)$. En efecto, supóngase que existen $x,y \in F$ tal que $x^2 + y^2 = -1$. Considérese la aplicación $\theta \colon H(F) \to M_2(F)$ dada por:
\begin{eqnarray*}
\theta(i) &=& \begin{pmatrix}
x & y\\
y & -x
\end{pmatrix}\\
\theta(j) &=& \begin{pmatrix}
0 & 1\\
-1 & 0
\end{pmatrix}\\
\theta(k) &=& \begin{pmatrix}
-y & x \\
x & y 
\end{pmatrix} \\
\theta(1) &=& \begin{pmatrix}
1 & 0 \\
0 & 1
\end{pmatrix}
\end{eqnarray*}
y por extensión lineal en $F$.
Para demostrar que $\theta$ es biyectiva, basta demostrar que las cuatro matrices dadas anteriormente son linealmente independientes en $F$, es decir, si existen $a, b, c, d \in F$ tal que 
\begin{equation*}
a\begin{pmatrix}
1 & 0\\
0 & -1
\end{pmatrix} + b\begin{pmatrix}
x & y \\
y & -x
\end{pmatrix} + c\begin{pmatrix}
0 & 1 \\
-1 & 0
\end{pmatrix} + d\begin{pmatrix}
-y & x\\
x & y
\end{pmatrix} = 0
\end{equation*}
entonces $a = b = c = d = 0$. En efecto, de la ecuación anterior se obtiene
\begin{eqnarray*}
a + bx - dy &=& 0\\
by + c + dx &=& 0 \\
by -c + dx &=& 0 \\
a -bx + dy &=& 0
\end{eqnarray*}
un sistema de ecuaciones en $a, b, c,d$ don determinante $4(x^2 + y^2) \neq 0$, entonces la única solución de dicho sistema es la trivial, de donde $H(F) \simeq M_2(F)$.
\end{proof}
Por lo expuesto en esta sección, el lector podrá intuir que es esencial poder trabajar con álgebras de cuaterniones sobre campos ciclótomicos de la forma $F = \mathds{Q}(\xi)$, donde $\xi$ es una raíz primitiva de la unidad. 

\begin{teorema}
Sea $F = \mathds{Q}(\xi_m)$ un campo ciclotómico, con $\xi_m$ una raíz primitiva de la unidad de orden $m$, donde $m$ es un entero impar mayor que $1$. Entonces, la ecuación $X^2 + Y^2 = -1$ tiene solución en $F$ si y sólo si el orden multiplicativo de $2$ módulo $m$ es par.
\end{teorema}

Para la demostración de dicho teorema y una lectura más profunda de este tema el lector puede consultar \cite{bib:moser}. Como consecuencia de este teorema se tiene
\begin{lema}
Sea $F = \mathds{Q}(\xi_m)$ como en el teorema anterior. Entonces $H(F)$ tiene divisores de cero si y sólo si el orden multiplicativo de $2$ módulo $m$ es par.
\end{lema}

Ahora se describe la estructura algebraica del grupo-álgebra $F(K_8)$.

\begin{lema}\label{lem:K8}
Sea $F$ un campo de característica distinta de $2 $. Entonces,
\begin{equation*}
FK_8 \simeq 4F\oplus H(F).
\end{equation*}
\end{lema}
\begin{proof}
Se escribe $K_8$ como
\begin{equation*}
K_8 = \textless a, b \colon a^4 = 1, a^2 = b^2, bab^{-1} = a^{-1} \textgreater
\end{equation*}
y del hecho que $\overline{K_8} = \frac{K_8}{K_8'}$ \footnote{dar referencia al capítulo 1} es el grupo de Klein de cuatro elementos se tiene
\begin{equation*}
F\overline{K_8} \simeq F \oplus F \oplus F \oplus F.
\end{equation*}
Por otro lado, como existe $\phi \colon FK_8 \to H(F)$ endomorfismo dada por $a \mapsto i$, $b \mapsto j$, se sigue que $H(F)$ es isomorfo a algún sumando simple de $FK_8$ y contando las dimensiones se obtiene el resultado. 
\end{proof}

Ahora se tiene la capacidad de clasificar las grupo-álgebras $FG$ con la propiedad que $FG$ no contenga elementos nilpotentes.
\begin{teorema}
Sea $F$ un campo de característica $p > 0$ y sea $G$ un grupo finito. Entonces $FG$ no tiene elementos nilpotentes si y sólo si $G$ es un $p'$-grupo abeliano. 
\end{teorema}
\begin{proof}
Supóngase que $FG$ no tiene elementos nilpotentes. Entonces de las proposiciones \ref{prop:nilpotentes} y \ref{prop:AH} se tiene que $G$ es un $p'$-grupo y que $G$ es abeliano o Hamiltoniano. Supóngase que $G$ es Hamiltoniano. Entonces $p \neq 2$.Más aún, siempre se puede resolver $X^2 + Y^2 = -1$ en un campo con $p$ elementos (y por lo tanto también en $F$). Entonces $FK_8$ tiene elementos nilpotentes por el teorema \ref{teo:caracterizacion} y el lema \ref{lem:K8}. Así, $G$ debe ser abeliano.
Para el converso basta notar que $FG$, siendo semisimple y conmutativo, es suma directa de campos. 
\end{proof}

Existen caracterización cuando el campo tiene característica cero, a continuación se presentan los resultados sin su demostración, pero se recomienda al lector consultar \cite{bib:Sehgal}.
\begin{teorema}
Sea $G$ un grupo finito de orden $2^km$ con $(2,m) = 1$. Entonces $\mathds{Q}G$ no tiene elementos nilpotentes si y sólo si $G$ es abeliano o Hamiltoniano con la propiedad de que el orden de $2$ módulo $m$ sea impar. 
\end{teorema}

\begin{teorema}
Sea $G$ un grupo nilpotente finitamente generado. El grupo-anillo $\mathds{Z}G$ no tiene elementos nilpotentes si y sólo si cada subgrupo finito de $G$ es normal y sucede alguna de las siguientes:
\begin{enumerate}
\item $T(G)$, el conjunto de los elementos de torsión de $G$, es un subgrupo abeliano
\item $T(G) = K_8 \times E \times A$, donde $E$ es un $2$-grupo elemental abeliano y $A$ es un grupo abeliano de orden impar $m$ tal que el orden multiplicativo de $2$ módulo $m$ es impar.  
\end{enumerate}
\end{teorema}