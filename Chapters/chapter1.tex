\chapter{CONCEPTOS PRELIMINARES}
En este capítulo se presentará la teoría básica del álgebra abstracta necesaria para la comprensión del contenido a desarrollar más adelante. Dicha exposición no pretende ser una guía de estudios del álgebra, más bien refresca resultados básicos de teoría de grupos, anillos y álgebras. En la medida de lo posible se evitará dar demostraciones de los resultados de estos conceptos, a menos que no sean materia de estudio de la licenciatura en Matemática.
\section{\hskip 1.1em Antecedentes}
La teoría de grupos como la conocemos actualmente tiene sus orígenes en los trabajos de Ruffini, Abel, Lagrange y Galois a inicios siglo \lsc{xix}, quienes trabajaron con el concepto de permutación (en su tiempo Cauchy las llamaba \textsf{sustituciones}, ver \cite[104]{bib:libroGuti}). Con Cayley \cite[ 104]{bib:libroLosGrandes} se formalizó el concepto de grupo y además se dieron muchos avances significativos que impulsaron la investigación de este tema. Entre los avances hechos por Cayley figuran:
\begin{bulletList}
\newItem Definición formal de grupo usando la notación de multiplicación.
\newItem Utilizar una tabla para mostrar como actúa una operación.
\newItem Demostración de existencia de dos grupos no isomorfos de orden cuatro, dando ejemplos explícitos. 
\newItem Demostración de existencia de dos grupos no isomorfos de orden seis, uno de los cuales es conmutativo y el otro es isomorfo a $\mathcal{S}_3$, el grupo de permutaciones de tres elementos.
\newItem Demostración de que el orden de todo elemento es divisor del orden del grupo, cuando éste es finito. 
\end{bulletList}
%-----------------------> teoría de grupos
\section{\hskip 1.1em Teoría de grupos}
\begin{definicion}
Un grupo es un conjunto no vacío $G$ junto con una operación binaria, denotada como $\cdot$, tal que para cada $a,b \in G$ se cumplen las siguientes condiciones:
\begin{bulletList}
\newItem $(a\cdot b)\cdot c = a \cdot (b\cdot c)$,
\newItem Existe un elemento único $1\in G$, tal que $a\cdot 1 = 1 \cdot a = a$.
\newItem Para cada $a\in G$ existe un elemento único $a^{-1} \in G$, tal que \[a\cdot a^{-1} =a^{-1}\cdot a =1.\]
\end{bulletList}
Si además de las tres propiedades anteriores, se cumple  \[ a\cdot b = b \cdot a , \mbox{ para cada } a,b \in G \] entonces se dice que el grupo es abeliano  o conmutativo. 
Si el conjunto $G$ es finito, entonces el número de elementos de $G$ es llamado el orden de $G$ y se denota como $\circ(G)$. 
\end{definicion}
\begin{ejemplo}\label{ejemplo:simetrias}
Sea $M$ un conjunto finito. El lector deberá recordar que una aplicación biyectiva de $M$ a $M$ es llamada permutación de $M$. Es claro entonces que la aplicación identidad de $M$ a $M$ es una permutación, que la composición de dos permutaciones es una permutación y la inversa de una permutación también es permutación. 

A partir de estos hechos, es evidente que dado un conjunto $M$ se puede construir conjunto de permutaciones y que este constituye un grupo respecto a la composición de funciones. Este grupo usualmente es denotado como $\mathcal{S}_M$ y es llamado el grupo de permutaciones de $M$. 
Si $M = \{  1,2,\dots, n \}$ entonces $\mathcal{S}_M$ es llamado el grupo de simetrías de grado $n$ y se denotada como $\mathcal{S}_n$. Dado un elemento $\psi \in \mathcal{S}_n$, si se elige que $i_k = \psi (k), \ 1 \leq k \leq n$, entonces se puede representar $\psi$ en la forma:
\[ \psi = \begin{pmatrix}
1 & 2 & 3 & \cdots & n \\
i_1 & i_2 & i_3 & \cdots & i_n
\end{pmatrix}, \]
la cual es una notación introducida por Cauchy en 1845 \cite[64-90]{bib:Cauchy}. Usando esta notación, la inversa de $\psi$ se representa como: \[ \psi^{-1} = \begin{pmatrix}
i_1 & i_2 & i_3 & \cdots & i_n \\
1 & 2 & 3 & \cdots & n
\end{pmatrix}.
 \]
 Dadas, por ejemplo, \[ \phi = \begin{pmatrix}
 1 & 2 & 3 & 4 & 5 \\
 3 & 5 & 2 & 4 & 1
 \end{pmatrix} \mbox{\qquad y \qquad } \psi = \begin{pmatrix}
 1 & 2 & 3 & 4 & 5 \\
  2 & 1 & 4 & 5 & 3
 \end{pmatrix}, \]
 se tiene que $(\phi \circ \psi)(1) = \phi(2) = 5$. Haciendo el cálculo para el resto de los números, se obtiene \[ \phi \circ \psi = \begin{pmatrix}
 1 & 2 & 3 & 4 & 5 \\
  5 & 3 & 4 & 1 & 2
 \end{pmatrix}. \]
 De la misma manera \[ \psi \circ \phi = \begin{pmatrix}
 1 & 2 & 3 & 4 & 5 \\
  4 & 3 & 1 & 5 & 2
 \end{pmatrix}. \]
\newpage Este simple cálculo demuestra que, en general, $\mathcal{S}_n$ no es conmutativo. De hecho, es fácil demostrar que $\mathcal{S}_n$ es conmutativo si y sólo si $n \leq 2$. 
\end{ejemplo}
\begin{definicion}
Un subconjunto no vacío $H$ de un grupo $G$ es llamado subgrupo de $G$ si es cerrado bajo la operación de $G$ y $H$, con la restricción de la operación de $G$, es un grupo por sí mismo.
\end{definicion}
\begin{ejemplo}[Subgrupos cíclicos]
Sea $a$ un elemento del grupo $G$. Para un exponente entero $n$ se definen las potencias de $a$ como
\[ a^n = \left\{ \begin{array}{lr}
\underset{n\mbox{ veces}}{\underbrace{a \cdot a \cdots a}} & \mbox{ si } n >0\\
\underset{\mid n \mid \mbox{ veces }}{\underbrace{a^{-1}\cdot a^{-1}\cdots a^{-1}}} & \mbox{ si } n<0 \\
1 & \mbox{ si } n = 0.
\end{array} \right.  \]
Como $a^m \cdot a^n = a^{m+n}$, se sigue que el conjunto
\[\langle a \rangle = \{ a^n : n \in \mathds{Z} \} \]
es un subgrupo de $G$, llamado subgrupo cíclico de $G$ generado por $a$.
Si este grupo es finito, entonces existen enteros positivos $n,m$ distintos tales que $a^n = a^m$, de esta cuenta, se tiene $a^{n-m} = a^{m-n} = 1$. El entero positivo más pequeño $n$  tal que $a^n = 1$ se le llama orden de $a$ y se denota como $\circ(a)$. Si $\langle a \rangle$ es infinito se dice que $a$ es de orden infinito.
Si existe un elemento $a$ en $G$ tal que $G = \langle a \rangle$, entonces se dice que $G$ es un grupo cíclico y que $a$ es un generador de $G$. Nótese que $\circ(a) = | \langle a \rangle | $.
\end{ejemplo}
\begin{ejemplo}
Sea $X$ un subconjunto no vacío de un grupo $G$. Se define el subgrupo generado por $X$ como la intersección de todos los subgrupos de $G$ que contienen a $X$. Esta familia de subgrupos es no vacía, ya que por lo menos $G$ pertenece a ella. 

Es fácil demostrar que esta intersección definida previamente es un subgrupo de $G$. Este subgrupo es denotado como $\langle X \rangle$. Se propone al lector demostrar que: 
\[ \langle X \rangle = \{ x_1^{\epsilon_1} \cdots x_k^{\epsilon_k} \colon x_i \in X, \ \epsilon_i = \pm 1, \ k \geq 1 \} \cup \{1\} .\]
Si $\langle X \rangle = G$ se dice que $X$ es un conjunto de generadores de $G$. Si $X$ es finito, entonces se dice que $G$ es un grupo finitamente generado.
\end{ejemplo}
 \begin{lema}
 Un subconjunto no vacío $H$ de un grupo $G$ es un subgrupo de $G$ si y sólo si para cualesquiera $x,y \in H$ se tiene que $x^{-1}y \in H$.
 \end{lema}
 \begin{definicion}
 El centro de un grupo $G$ es el subgrupo \[ \mathcal{Z}(G) = \{ a \in G \colon ax=xa, \mbox{ para cada }  x \in G \}. \]
\end{definicion}
Dado un subgrupo $H$ de un grupo $G$, se puede definir una partición de $G$, es decir una cubierta de $G$ hecha de subconjuntos disjuntos. 
\begin{definicion}
Sea $H$ un subgrupo de un grupo $G$. Dado un elemento $a \in G$, los subconjuntos de la forma 
\begin{eqnarray*}
aH &=& \{ ah \colon h \in H \}, \\
Ha &=& \{ ha \colon h \in H \} 
\end{eqnarray*}
son llamados clases lateral izquierda y derecha del subgrupo $H$ determinadas por $a$, respectivamente.
\end{definicion}
\newpage
\begin{proposicion}
Sea $H$ un subgrupo de un grupo $G$ y $a,b$ elementos arbitrarios de $G$. Entonces se cumple:
\begin{bulletList}
\newItem Si $b \in aH$ entonces $bH = aH$.
\newItem Si $b \notin aH$ entonces $aH \cap bH = \emptyset$.
\end{bulletList}
\end{proposicion}
\begin{corolario}
Sea $H$ un subgrupo de un grupo $G$. Dados $a,b \in G$ se cumple que $b \in aH$ si y sólo si $aH = bH$.
\end{corolario}
Todo elemento en una clase lateral es un representante de la misma. Un conjunto completo de representantes de una clase lateral izquierda (derecha), es llamado transversal izquierdo (derecho) de $H$ en $G$.
\begin{definicion}
Sea $H$ un subgrupo de un grupo $G$. Si el número de clases izquierdas (derechas) de $H$ en $G$ es finito, entonces este número es llamado  índice de $H$ en $G$ y se denota como $(G:H)$.
\end{definicion}
\begin{teorema}[Lagrange]
Sea $H$ un subgrupo de un grupo finito $G$. Entonces, el orden de $H$ divide a el orden de $G$. Más aún, de manera más formal, se tiene
\[ \mid G \mid = (G:H) \mid H \mid. \]
\end{teorema}
\begin{corolario}
Sea $a$ un elemento de un grupo finito $G$. Entonces $\circ(a)$ es un divisor de $\mid G \mid$.
\end{corolario}
\begin{ejemplo}
Considérese nuevamente a $\mathcal{S}_3$, el grupo de simetrías de grado tres. Se sabe que $|  \mathcal{S}_3 | = 6$. Explícitamente, este grupo se expresa como \[ \mathcal{S}_3 = \{ I, (12), (13), (23), (123), (132) \}, \mbox{ donde } I = (1) .\]
Sea $H = \{ I, (12) \} $ y $\alpha = (123)$. Entonces; \[ \alpha H = \{ (123), (13) \}   \mbox{ y } H\alpha = \{ (123), (23) \} ,\] con esto se demuestra que, en general las clases laterales derechas e izquierdas determinadas por el mismo elemento, no son iguales.
\end{ejemplo}
Los subgrupos cuyas clases laterales derechas e izquierdas generadas por el mismo elemento son iguales son de especial importancia. Nótese que para un elemento $a$ y un subgrupo $H$ de un grupo $G$, se tiene que $aH = Ha$ si y sólo si $a^{-1}Ha=H$. Esto sugiere la siguiente:
\begin{definicion}
Sea $H$ un subgrupo de un grupo $G$. Se dice que $H$ es normal en $G$, y se escribe $H \triangleleft G$ si $a^{-1}Ha = H$ para cualquier $a \in G$.
\end{definicion}

%-------------subsección --------------------------
\subsection{\hskip 1.1em Homomorfismos y grupos cocientes}
\begin{definicion}
Sean $G_1$ y $G_2 $ grupos. Una aplicación $f \colon G_1 \to G_2$ es llamada un homomorfismo de grupos si para cada $g, h \in G$ se cumple que \[ f(g \cdot h) = f(g) \cdot f(h) . \]
\end{definicion}
\begin{definicion}
Sea $f \colon G_1 \to G_2$ un homomorfismo de grupos. Entonces, la imagen de $f$ es el conjunto \[ \Ima(f) = \{ y \in G_2 \colon \mbox{ existe }x \in G_1, f(x) =y \}. \]
El kernel de $f$ es el conjunto \[ \ker(f) =  \{ x \in G_1 \colon f(x) = 1 \}.\] 
\end{definicion}
\begin{definicion}
Un homomorfismo de grupos $f \colon G_1 \to G_2$ es llamado un epimorfismo si es sobreyectivo. Se llama a $f$ un monomorfismo si es inyectivo. Por último, se dice que $f$ es un isomorfismo si es sobreyectivo e inyectivo. Dados dos grupos $G_1$ y $G_2$, se dice que son isomorfos, y se denota como $G_1 \simeq G_2$ si existe un isomorfismo $f \colon G_1 \to G_2$. 
\end{definicion} 
Un homomorfismo de un grupo $G$ en sí mismo es llamado un endomorfismo y si a su vez es un isomorfismo se llama  automorfismo de $G$. 
El siguiente resultado se debe al famoso matemático británico Arthur Cayley, el cual demuestra la relevancia de los grupos de permutación en la teoría de grupos.
\begin{teorema}[Cayley]
Todo grupo $G$ es isomorfo a un grupo de permutaciones.
\end{teorema}
\begin{definicion}
	Sea $H$ un subgrupo normal de un grupo $G$ y $a, b \in G$. Se dice que $a\equiv b\pmod{H}$ si $b^{-1}a \in H$. Es fácil demostrar que esta relación es de equivalencia. Para un elemento $a \in G$ se denota su clase de equivalencia como \[ \bar{a} = \{ x \in G \colon x \equiv a \pmod{H}  \} = \{ x \in G \colon a^{-1}x \in H \} = aH.  \]
\indent Se denota como $G/H$ al conjunto de clases de equivalencia de los elementos de $G$. Se define el producto de elementos en $G/H$ como \[ \bar{a}\cdot \bar{b} = \overline{ab}. \] 
\indent Esta operación es bien definida y $G/H$ es un grupo, llamado \textbf{grupo cociente}. 
\end{definicion}
Considérese la aplicación $\omega \colon G \to G/H$ dada por: \[G \ni a \mapsto \omega(a) = 	\bar{a} = aH. \]
\indent Es evidente que $\omega$ es un epimorfismo de grupos llamado homomorfismo canónico de $G$ hacia el grupo cociente $G/H$. Este homomorfismo satisface que $\omega(1) = 1H = H$ y $\ker(\omega) = H.$ 
\begin{teorema}[Primer teorema de isomorfía de grupos]
Sea $f \colon G_1  \to G_2$ un homomorfismo de grupos, $\omega$ el homomorfismo canónico de $G_1$ hacia el grupo cociente $X = G_1/\ker(f)$ e $i$ la inclusión de $\Ima(f)$ en $G_2$. Entonces existe un homomorfismo único $\bar{f} \colon G_1/\ker(f) \to \Ima(f)$ tal que $f = i \circ \bar{f}\circ \omega$, es decir que diagrama  de la figura \ref{fig:primerTeoremaIsomofia} conmuta.
\begin{figure}
\caption{\hskip 2em Primer teorema de isomorfía de grupos}
\vskip -0.59em
\[\xymatrix { G_1 \ar[r]^f 
\ar[d]_{\omega}
 & G_2\\
X \ar[r]_{\bar{f}} & \Ima(f) \ar[u]_i }\]
\vskip 0.20em
\caption*{Fuente: elaboración propia con paquete xymatrix para \LaTeX.}
\label{fig:primerTeoremaIsomofia}
\end{figure}
Además $\bar{f}$ es un isomorfismo. 
\end{teorema}
\begin{corolario}
Sea $f \colon G_1 \to G_2$ un epimorfismo. Entonces \[ G_1/\ker(f) \simeq G_2. \]
\end{corolario}
\newpage
\begin{lema}
Sea $H$ un subgrupo normal de $G$. Entonces
\begin{bulletList}
\newItem Para cada subgrupo $K$ de $G$ que contiene a $H$, el conjunto $K/H = \{ xH \colon x \in K \}$ es un subgrupo de $G/H$ que es normal si y sólo si $K$ es normal.
\newItem Si $\mathcal{K}$ es un subgrupo de $G/H$, entonces la preimagen $K = \{ x \in G \colon xH \in \mathcal{K} \}$ es un subgrupo de $G$ que contiene a $H$, tal que $\mathcal{K} = K/H$.
\end{bulletList}
\end{lema}
\begin{teorema}
Sea $f \colon G_1 \to G_2$ un epimorfismo de grupos, entonces existe un biyección entre el conjunto de subgrupos de $G_2$ y el conjunto de subgrupos de $G_1$ que contienen a $\ker(f)$.
\end{teorema}
\begin{teorema}[Segundo teorema de isomorfía de grupos]
Sea $H$ y $K$ subgrupos de un grupo $G$ y supóngase que $K$ es normal. Entonces: \[ \frac{H}{H \cap K} \simeq \frac{HK}{K}, \] donde $HK = \{ hk \colon h \in H, \ k \in K \}$.
\end{teorema}
\begin{teorema}[Tercer teorema de isomorfía de grupos]
Sean $H \subset K$ subgrupos normales de un grupo $G$. Entonces:
\[\frac{G/H}{K/H} \simeq \frac{G}{K}. \]
\end{teorema}
\subsection{\hskip 1.1em Productos directos}
\begin{definicion}
Sean $H,\ K$ subgrupos de un grupo $G$. Se dice que $G$ es el producto directo interno de $H$ y $K$ y se escribe $G = H \times K$ si se cumplen las siguientes condiciones:
\begin{bulletList}
\newItem $G=HK$.
\newItem $H \cap K = \{1\}$.
\newItem $H \triangleleft G$ y $K \triangleleft G$.
\end{bulletList}
\end{definicion}
La definición anterior se puede extender a familias arbitrarias de subgrupos normales, como se muestra en la siguiente definición.
\begin{definicion}
Sea $\{H_i\}_{i \in I}$ un familia de subgrupos normales de un grupo $G$. Entonces $G$ es llamado el producto directo interno de los subgrupos $\{H_i\}_{i \in I}$ si se cumplen las siguientes condiciones:
\begin{bulletList}
\newItem $G =\langle H_i \colon i \in I \rangle$, es decir que cada elemento $g$ de $G$ se puede escribir como producto de un número finito de elementos de los subgrupos $\{H_i\}_{i \in I}$.
\newItem $H_i \cap \langle H_j \colon j \in I, \ j \neq i \rangle = \{1\}$ para cada índice $i \in I$.
\end{bulletList}
\end{definicion}
\begin{definicion}
Dada una familia de grupos $G_1, \dots, G_n$ considérese el producto cartesiano $G = G_1 \times \cdots \times G_n$. Considérese la operación entre elementos de $G$ componente a componente, usando la operación de cada $G_i, 1 \leq i \leq n$. Con la operación definida previamente es fácil demostrar que $G$ es un grupo. A este se le llama producto directo externo de $G_1, \dots, G_n$.
\end{definicion} 
Si $G_1, \dots, G_n$ es una familia de grupos y $G = G_1\dot{\times}\cdots\dot{\times}G_n$ su producto directo externo, entonces los conjuntos
\[ H_1=\{(x,1,\dots,1)\colon x \in G_1\},\dots,H_n=\{(1,1,\dots,x)\colon x \in G_n \} \] son subgrupos normales de $G$ tales que $G_i \simeq H_i,1\leq i \leq n$ y $G$ es también el producto directo interno de los subgrupos $H_1, \dots, H_n$.
De manera similar, si $G$ es el producto directo interno de una familia de subgrupos normales $H_1,\dots,H_n$ y se construye el producto directo externo $\bar{G} = H_1\dot{\times}\cdots\dot{\times}H_n$, entonces se tiene que $\bar{G} \simeq G$. Debido a este hecho, no se hace distinción entre producto interno y externo.
%-------------subsección de grupos abelianos
\subsection{\hskip 1.1em Grupos abelianos}
Resultan de mucho interés para el desarrollo del capítulo \ref{chap:unidades}, así que se remarca sus importancia.
Sea $G$ un grupo abeliano. Un elemento de $G$ es llamado un elemento de torsión si este es de orden finito. Si dos elementos $g,h \in G$ son de torsión, de órdenes $m$ y $n$ respectivamente, entonces es inmediato que $(g^{-1}h)^{mn} = 1$, lo cual demuestra que el conjunto de elementos de torsión de $G$ es un subgrupo de $G$. Nótese que este hecho también demuestra que dado un número primo $p$, el conjunto de elementos de $G$ cuyos órdenes son potencias de $p$ también constituyen un subgrupo de $G$.
\begin{definicion}
Sea $G$ un grupo abeliano, entonces el subgrupo 
\[T(G) = \{ g \in G \colon \circ(g) < \infty \}  \]
es llamado el subgrupo de torsión de $G$ y el subgrupo \[ G(p) = \{g \in G \colon \circ(g) \mbox{ es una potencia de } p  \} \] es llamado la componente $p$-primaria de $G$.
\end{definicion}
Se dice que $G$ es un grupo libre de elementos de torsión si $T(G) = (1)$.
Un grupo abeliano que es producto directo de grupos cíclicos infinitos se llama abeliano libre. Al número de factores directos se le llama rango del grupo abeliano libre. Si dicho número no es finito, se dice que tiene rango infinito.
\begin{teorema}
Un grupo $G$ que es abeliano, finitamente generado y libre de elementos de torsión debe ser libre.
\end{teorema}
\begin{teorema}
Sea $G$ un grupo abeliano finitamente generado. Entonces $T(G)$ es finito, $G/T(G)$ es libre de rango finito y \[ G \simeq T(G)\times \frac{G}{T(G)}. \]
\end{teorema}
\begin{lema}
Sea $g$ un elemento de orden $\circ(g) = p_1^{n_1}\cdots p_t^{n_t}$ de un grupo $G$. Entonces se puede escribir $g = g_1\cdots g_t$ con $\circ(g_i) = p_i^{n_i}, i\leq i \leq t$. Más aún, los elementos $g_1, \dots, g_t$ que están determinados de manera única son potencias de $g$ y por lo tanto conmutan entre ellos. 
\end{lema}
Un elemento cuyo orden es potencia de un número primo $p$ es llamado $p$-elemento. Por otro lado, si $p$ no divide el orden del elemento, se dice que es un $p'$-elemento.
\begin{lema}
Sea $G$ un grupo abeliano finito de orden $|G| = p_1^{n_1}\cdots p_t^{n_t}$. Entonces \[ G = G(p_1)\times \cdots \times G(p_t). \]
\end{lema}
\begin{definicion}
Sea $p$ un primo. Un grupo finito $G$ es llamado $p$-grupo si su orden es una potencia de $p$.
Se dice que un grupo abeliano $G$ es un abeliano elemental si existe un primo $p$ tal que todos los elementos distintos al elemento identidad, son de orden $p$.
\end{definicion}
Es interesante notar que los $p$-grupos abelianos elementales también pueden ser vistos como espacios vectoriales.
\begin{lema}
Sea $G$ un $p$-grupo abeliano elemental. Entonces $G$ es un espacio vectorial sobre $\mathds{Z}_p$. Más aún, si $G$ es finito entonces puede ser escrito como producto directo de un número finito de grupos cíclicos de orden $p$.
\end{lema}
Para un grupo $G$ se define su exponente, y se denota como $\exp(G)$, como el entero positivo más pequeño $m$, tal que $g^m=1$ para cualquier $g \in G$. Nótese que $G$ es un $p$-grupo abeliano elemental si y sólo $\exp(G)=p$ y que si $G$ es un grupo abeliano con $\exp(G) = p^m$, entonces $\exp(G^p) = p^{m-1}$.
\begin{teorema}\label{teo:estructuraAbelianos}
Sea $G$ un $p$-grupo abeliano finito. Entonces $G$ se puede escribir como producto directo de $p$-subgrupos cíclicos. Esta descomposición es única, en el sentido que si
\[ G = C_1 \times \dots \times C_t = D_1 \times \cdots \times D_s \]
donde $C_i,D_j, 1\leq i \leq t, 1\leq j \leq s,$ son $p$-grupos cíclicos de órdenes $p^{n_1} \geq \cdots \geq p^{n_t}>1$ y $p^{m_1}\geq \cdots \geq p^{m_s}>1$ respectivamente, entonces $t=s$ y $n_i = m_i, 1\leq i \leq t$.
\end{teorema} 
\begin{proposicion}
Sea $G$ un grupo abeliano finito de orden $n$. Entonces para cada divisor $d$ de $n$, el número de subgrupos cíclicos de $G$ de orden $d$ es igual al número de factores cíclicos de $G$ del mismo orden.
\end{proposicion}

%--------------Seccion hamiltonianos
\subsection{\hskip 1.1em Grupos hamiltonianos}
Para hacer un estudio de los grupos hamiltonianos se procede a dar algunas definiciones previas.
Se define el grupo de cuaterniones de orden 8 como:\[ K_8 =  \langle a,b \colon a^4 =1, \ a^2 = b ^2, \ bab^{-1} = a^{-1} \rangle.  \]
Los elementos de este grupo se enumeran de la siguiente manera:
\[ K_8 = \{ 1,a,a^2, a^3, b, ab, a^2b, a^3b \}. \]
\begin{definicion}
Dados dos elementos $x,y$ en un grupo $G$, el conmutador de $x$ con $y$ es el elemento $[x,y] = x^{-1}y^{-1}xy \in G$. Dados dos subconjuntos $H$ y $K$ de un grupo $G$, se denota como $[H,K]$ el subgrupo de $G$ generado por el conjunto:
\[\{[h,k] \colon h \in H, k \in K\}. \] 
En particular, el grupo $G'=[G,G]$ es llamado subgrupo conmutador o subgrupo derivado de $G$.
\end{definicion}
Un cálculo directo demuestra que $K'_8 = \mathcal{Z}(K_8) = \{1, a^2\}$. Además, $a^2$ es el único elemento de $K_8$ de orden 2, entonces si $H$ es un subgrupo cualquiera de $K_8$, entonces se tiene que $K'_8 = \{1,a^2\} \subset H$. Además cualquier subgrupo de $K_8$ es normal.
\begin{definicion}
El 4-grupo de Klein está definido por:
\[ G = \langle a, b \colon a^2 = b^2 = (ab)^2 = 1 \rangle. \] 
\end{definicion}
\begin{ejercicio}\label{ejer:klein}
Demostrar que $K_8/K'_8$ es isomorfo al 4-grupo de Klein.
\end{ejercicio} 
\begin{solucion}
Se sabe que $K'_8 = \{1,a^2\}$, entonces calculando las clases de \nopagebreak[0] equivalencia se tiene:
\begin{eqnarray*}
\bar{1} &=& K'_8\\
\bar{a} &=& \{a, a^3\} \\
\bar{b} &=&  \{b, a^2b \} \\
\overline{ab} &=& \{ ab, a^3b \}
\end{eqnarray*}
lo cual genera una partición, por lo tanto $K_8/K'_8 = \{\bar{1},\bar{a},\bar{b}, \bar{ab}  \}$. Ahora bien, $\bar{a}\cdot\bar{a} =\bar{a^2}=\bar{1}, \ \bar{b}\bar{b} = \bar{b^2} = \bar{a^2} = \bar{1}$ y finalmente $\overline{ab}\cdot\overline{ab} = \overline{(ab)^2} = \bar{1}$ con lo que concluye la demostración. \qedsymbol
\end{solucion}
\begin{definicion}
Un grupo G es hamiltoniano si no es conmutativo y todos sus subgrupos son normales. 
\end{definicion}
\begin{lema}
Todo grupo hamiltoniano contiene un subgrupo isomorfo a $K_8$.
\end{lema}
\begin{teorema}\label{teo:abelianohamiltoniano}
Un grupo $G$ es hamiltoniano si y sólo si $G$ es producto directo de un grupo cuaterniones de orden 8, un 2-grupo abeliano elemental $E$ y un grupo abeliano $A$ en el cual todos sus elementos son de orden impar.
\end{teorema}
La demostración de este importante teorema, se puede consultar en \cite[130]{bib:groupBook}.
%----------------> teoria de anillos, módulos  y algebras
\section{\hskip 1.1em Anillos, módulos y álgebras}
En lo subsiguiente será de vital importancia conocer propiedades importantes de los anillos, módulos y álgebras, es por eso que se ha dedicado esta sección para su estudio, la mayoría de veces, no se darán demostraciones.
\subsection{\hskip 1.1em Anillos}
\begin{definicion}
Un anillo es un conjunto no vacío $R$ dotado con dos operaciones binarias, llamadas suma y multiplicación y denotadas como $+$ y $\cdot$ respectivamente, tal que para cada $a,b\in R$ se cumplen las siguientes propiedades:
\begin{bulletList}
\newItem $a+(b+c) = (a+b)+c$.
\newItem Existe un único elemento $0\in R$ tal que $a+0=0+a=a$.
\newItem $a+b=b+a$.
\newItem Para cada $a \in R$ existe un elemento $-a\in R$ tal que $a+(-a) = (-a)+a=0$.
\newItem $a\cdot(b\cdot c) = (a\cdot b)\cdot c$.
\newItem $a\cdot(b+c) = a\cdot b+ a\cdot c$.
\newItem $(a+b)\cdot c= a\cdot c+b\cdot c $.
\end{bulletList}
Además, si se satisface:
\begin{bulletList}
\newItem $a \cdot b = b\cdot a$.
\end{bulletList}
se dice que el anillo es conmutativo.
Un anillo es llamado dominio si satisface:
\begin{bulletList}
\newItem $a\cdot b = 0$ implica que $a = 0$  o $b = 0$.
\end{bulletList}
\end{definicion}
Dos elementos $a,b$ distintos de cero de un anillo $R$ tales que $ab = 0$ son llamados divisores de cero. Así que un dominio es un anillo sin divisores de cero.
Un anillo $R$ que contiene un elemento $1\neq 0$ tal que: \[ 1\cdot a = a \cdot 1 = a, \mbox{ para todo } a \in R \]\newpage \noindent es llamado anillo con unidad. A un anillo con unidad que es un dominio conmutativo se le llama dominio entero.
\begin{definicion}
Un elemento $a$ de un anillo $R$ es invertible si existe un elemento $a^{-1} \in R$, conocido como su inverso, tal que $a\cdot a^{-1} = a^{-1} \cdot a = 1$. El conjunto \[ \mathcal{U}(R) = \{a \in R \colon a \mbox{ es invertible} \} \] se llama el grupo de unidades de $R$.
\end{definicion}
A un anillo se le llama de división si todos los elementos distintos de cero son invertibles, dicho de otra manera, si $R\backslash \{0\} = \mathcal{U}(R)$. Un anillo de división conmutativo es un campo.
\begin{ejemplo}
El conjunto $\mathds{Z}_m = \{ \bar{0}, \bar{1}, \dots, \overline{m-1} \}$ de enteros módulo $m$ es un anillo conmutativo. Más aún $\mathds{Z}_m$ es un campo si y sólo si $m$ es un número primo.
\end{ejemplo}
\begin{ejemplo}
Sea $R$ un anillo. Entonces el conjunto $R[X]$ de todos los polinomios con coeficientes en $R$ e indeterminada  $X$ con la operación usual de suma y multiplicación de polinomios es un anillo. El anillo de polinomios $R[X]$ es un dominio si y sólo si $R$ lo es. 
\end{ejemplo}
\begin{ejemplo}
Sea $R$ un anillo, entonces el conjunto $M_n(R)$ de todas la matrices de $n\times n$ con entradas en $R$, con suma y producto usual de matrices es un anillo. Este anillo es llamado anillo completo de matrices de $n\times n$ sobre $R$.
\end{ejemplo}
\begin{ejemplo}
Sean $R_1, R_2, \dots, R_n$ anillos. El anillo \[ R_1\dot{\oplus}R_2\dot{\oplus}\cdots\dot{\oplus} R_n = \{ (a_1,a_2,\dots,a_n),\   a_i \in R_i, 1\leq i \leq n \},  \] con la suma y el producto definido componente a componente es llamado  suma directa de los anillos $R_1, R_2,\dots,R_n$.
\end{ejemplo}
El siguiente ejemplo es el primero de un anillo no conmutativo en la historia de las matemáticas. 
\begin{ejemplo}
Sean $i,j,k$ símbolos dados y considérese el conjunto $\mathcal{H}$ de todas las expresiones de la forma $x_0 + x_1i+x_2j+x_3k$ donde los coeficientes $x_0,x_1,x_2,x_3$ son números reales.
Se define la suma de dos elementos de este conjunto como \[ (x_0 + x_1i+x_2j+x_3k)+(y_0 + y_1i+y_2j+y_3k) = (x_0 + y_0) + (x_1 + y_1)i + (x_2+y_2)j + (x_3+y_3)k.  \] La multiplicación está definida de manera distributiva, con las siguientes reglas:
\begin{equation*}
i^2 = j^2 = k^2 =-1
\end{equation*}
\vspace*{-5.1em}
\begin{alignat*}{3}
ij &= k &= -ji\\
jk &= i &= -kj\\
ki &= j &= -ik.
\end{alignat*}
 \indent Por medio de un cálculo sencillo se demuestra que $\mathcal{H}$ es un anillo, llamado \textbf{anillo de cuaterniones reales}. Dado un cuaternión $\alpha = x_0 + x_1i + x_2j + x_3k$, se define el \textbf{conjugado} $\bar{\alpha}$ de $\alpha$ como \[ \bar{\alpha} = x_0-x_1i - x_2j-x_3k. \] 
 
 La norma de $\alpha$ se define como: \[ \lVert \alpha \rVert = \alpha\bar{\alpha} = x_0 ^2 + x_1^2 + x_ 2^2 + x_ 3^2. \]
Al igual que en el caso de los números complejos, se cumple que:
\begin{bulletList}
\newItem $\lVert \alpha\beta \rVert = \lVert \alpha \rVert \lVert \beta \rVert = \lVert \alpha\beta \rVert$.
\newItem $\lVert \alpha \rVert \geq 0$ y $\lVert \alpha \rVert  = 0 $ si y sólo si $\alpha = 0 $.
\end{bulletList}

Si $\alpha \in \mathcal{H}$ es distinto de cero, se tiene que $\lVert \alpha \rVert \neq 0$, entonces se puede definir $\alpha' = \bar{\alpha}/\lVert \alpha \rVert$. Entonces \[ \alpha\alpha'= \alpha \frac{\bar{\alpha}}{\lVert \alpha \rVert} = 1 \] y de manera similar se demuestra que $\alpha'\alpha = 1$, de donde se sigue que $\alpha^{-1} = \alpha'$. Este argumento demuestra que $\mathcal{H}$ es un anillo de división.
Se define $\mathcal{H}_{\mathds{Q}}$ restringiendo $\mathcal{H}$ al campo de los números racionales. A este conjunto se le conoce como cuaterniones racionales.
Finalmente se definen los cuaterniones enteros como: \[ \mathcal{H}_{\mathds{Z}} = \{ x_0 + x_1i + x_2j + x_3k \colon x_0, x_1, x_2, x_3 \in \mathds{Z} \}. \]
\indent Es evidente que $\mathcal{H}_{\mathds{Z}}$ es un anillo de división.
Sea $\alpha \in \mathcal{U}(\mathcal{H}_{\mathds{Z}})$ entonces es inmediato, de la definición de norma, que $\lVert \alpha \rVert$ es un entero positivo, más aún como $\alpha\alpha'= 1$ entonces $\lVert \alpha \rVert \lVert \alpha'\rVert = 1$ y consecuentemente $\lVert \alpha \rVert = 1$, es decir, $x_i = 1$ para algún índice $0\leq i \leq 3 $ y $x_j = 0 $ para $j \neq i$. De lo anterior se tiene \[ \mathcal{U}(\mathcal{H}_{\mathds{Z}}) = \{ \pm1, \pm i, \pm j, \pm k \}. \] 
\end{ejemplo} 
\newpage
\begin{definicion}
Un subconjunto no vacío $S$ de un anillo $R$ es un subanillo de $R$ si es cerrado bajo las operaciones de $R$ y es un anillo respecto a estas operaciones.
\end{definicion}
\begin{definicion}
El centro de un anillo $R$ es el subanillo \[ \mathcal{Z}(R) = \{a \in R \colon ax = xa, \mbox{ para todo } x \in R \}. \]
\end{definicion}
\begin{definicion}
Un subconjunto no vacío $L$ de un anillo $R$ es un ideal izquierdo de $R$ si cumple las siguientes propiedades:
\begin{bulletList}
\newItem Si $x,y \in L$ entonces $x-y \in L$.
\newItem Si $x \in L$ y $a \in R$ entonces $ax \in L$.
\end{bulletList}
\end{definicion}
De manera similar se define un ideal derecho de un anillo $R$. Un subconjunto no vacío $L$ de un anillo $R$ se llama ideal de $R$, si es un ideal derecho e izquierdo de $R$. 
Los subconjuntos $\{0\}$ y $R$ de un anillo $R$ siempre son ideales de $R$. Un ideal $L$ de $R$ distinto de estos se llama ideal propio. Nótese que si un ideal $L$ de un anillo $R$ contiene un elemento invertible $a$ entonces $L=R$ no es ideal propio. 
\begin{definicion}
Sea $R$ un anillo y $a \in R$. El conjunto \[ RA = \{ xa \colon x \in R \} \] es llamado  el ideal izquierdo generado por $a$. Al elemento $a$ se le conoce como generador de este ideal. Los ideales generados por un elemento $a \in R$ se llaman ideales principales y se denotan como $(a)$.
\end{definicion}
\begin{proposicion}
Sea $D$ un anillo de división y $n$ un entero positivo. Entonces el anillo completo de matrices $M_n(D)$ no contiene ideales propios.
\end{proposicion}
\begin{definicion}
Sean $R, S$ anillos. Una aplicación $f \colon R \to S$ se llama homomorfismo de anillos si para cualesquiera $a,b \in R$ se cumple:
\begin{bulletList}
\newItem $f(a+b) = f(a) +f(b)$.
\newItem $f(ab) = f(a)f(b)$.
\end{bulletList}
\end{definicion}
\begin{definicion}
Sea $f \colon R \to S$ un homomorfismo de anillos, entonces la imágen de $f$ es el subanillo \[\Ima (f) = \{ y \in S \colon \mbox{existe } x \in R, f(x)=y \}. \]
El kernel de $f$ es el ideal\[ \ker(f) = \{ x \in R \colon f(x) = 0 \}. \]
\end{definicion}
\begin{definicion}
Un homomorfismo de anillos $f \colon R \to S$ es un epimorfismo si es sobreyectivo. Se dice que $f$ es un monomorfismo si es inyectivo. Finalmente, si $f$ es inyectivo y sobreyectivo, entonces se dice que $f$ es un isomorfismo.
\end{definicion}
Un homomorfismo de un anillo $R$ en sí mismo es un endomorfismo y si también es un isomorfismo entonces se llama automorfismo de $R$.
\begin{definicion}
Sea $I$ un ideal de $R$. El grupo cociente aditivo $R/I$ con la multiplicación definida por $\bar{r}\bar{s} = \overline{rs}$ es un anillo llamado cociente de $R$ por $I$. 
\end{definicion}

%----------------Inicio de algebras y módulos
\subsection{\hskip 1.1em Módulos y álgebras}
\begin{definicion}
Sea $R$ un anillo. Un grupo aditivo abeliano $M$ es llamado un $R\mbox{-módulo}$ izquierdo si para todo $a \in R$ y $m \in M$ se cumple que $am \in M$ y 
\begin{bulletList}
\newItem $(a+b)m = am  + bm$
\newItem $a(m_1 + m_2) = am_1 + am_2$
\newItem $a(b)m = (ab)m$
\newItem $1m = m$
\end{bulletList}
para cualesquiera $a,b \in R$ y $m, m_1, m_2 \in M$.
\end{definicion}
De manera similar, dado un anillo $R$, se define un $R\mbox{-módulo}$ derecho considerando la multiplicación de elementos de $M$ por elementos de $R$ por la derecha. De acá en adelante, un $R\mbox{-módulo}$ izquierdo será abreviado como $R\mbox{-módulo}$. 
\begin{ejemplo}
Un ideal izquierdo $L$ de un anillo $R$ es un $R\mbox{-módulo}$, ya que el producto de elementos de $R$ por elementos de $L$ pertenece a $L$. Así mismo, los ideales derechos también son $R\mbox{-módulos}$ derechos.
En particular, un anillo siempre es un módulo sobre sí mismo. Cuando se considere un anillo como un $R\mbox{-módulo}$ izquierdo o derecho sobre sí mismo, se denotará como $_{R}R$ y $R_R$ respectivamente.
\end{ejemplo}
\begin{ejemplo}
Sea $L$ un ideal derecho de un anillo $R$ y $R/L$ el grupo cociente aditivo. Entonces $R/L$ es un $R\mbox{-módulo}$ con \[  r(a+L) = ra +L, \mbox{ para todo } r, a \in R  \]
\end{ejemplo} 
\begin{definicion}
Sea $R$ un anillo conmutativo. Un $R\mbox{-módulo}$ $A$ es un $R\mbox{-álgebra}$ si existe una operación de multiplicación definida en $A$ tal que con su adición y esta multiplicación $A$, es un anillo que cumple: \[ r(ab) = (ra)b = a(rb), \] para todo $r \in R$ y $a,b \in A$.
\end{definicion}
\begin{definicion}
Sea $M$ un módulo sobre un anillo $R$. Un conjunto no vacío $N \subset M$ es llamado un $R\mbox{-submódulo}$ de $M$ si se cumplen las siguientes condiciones:
\begin{bulletList}
\newItem Para todos $x,y \in N$ se tiene $x + y \in N$
\newItem Para cualquier $r \in R$ y todo $n \in N$, $rn \in N$
\end{bulletList}
Si $R$ es conmutativo y $M$ es un $R\mbox{-álgebra}$, entonces  se dice que $N$ es un $R\mbox{-subálgebra}$ de $M$ si es un submódulo y un subanillo de $M$ al mismo tiempo. 
\end{definicion}
Todo módulo $M \neq \{0\}$ contiene al menos dos submódulos, a saber, $M$ y $\{0\}$, que son llamados triviales. Un submódulo no trivial es llamado submódulo propio. Un módulo, distinto al módulo que sólo contiene al $0$, que no contiene submódulos propios es un  módulo simple. 
Sea $N$ un submódulo de un $R\mbox{-módulo}$ $M$. Al igual que en el caso de los anillos, el grupo cociente aditivo $M/N$ es un $R\mbox{-módulo}$ con $r\bar{m} = \overline{rm}, r\in R, m \in M$. Este es el módulo cociente entre $M$ y $N$. 

%-----------------subsection modulos libres
\subsection{\hskip 1.1em Módulos libres}
\begin{definicion}
Un conjunto $S = \{ s_i\}_{i \in I}$, $I$  un conjunto de índices, de elementos de un $R\mbox{-módulo}$ $M$ es un conjunto de generadores de $M$ si $M = RS$, es decir; si todo elemento de $M$ se puede escribir como un combinación lineal finita de elementos de $S$ con coeficientes en $R$.
\end{definicion}
\begin{definicion}
Un conjunto $S = \{ s_i\}_{i \in I}$ de elementos de un $R\mbox{-módulo}$ $M$ es  linealmente independiente o $R\mbox{-libre}$ si toda combinación lineal de elementos de $S$ con coeficientes en $R$ de la forma \[ r_{i_1}s_{i_1} + \cdots + r_{i_t}s_{i_t} = 0   \] implica que $r_{i_1} = \cdots = r_{i_t} = 0$. 
\end{definicion}
\begin{definicion}
Un conjunto $S = \{ s_i\}_{i \in I}$, $I$ un conjunto de índices, de elementos de un $R\mbox{-módulo}$ $M$ es una base de $M$ sobre $R$ o una $R\mbox{-base}$ si es linealmente independiente y un conjunto de generadores. 
\end{definicion}
\begin{definicion}
Un $R\mbox{-módulo}$ $M$ es libre si tiene una base.
\end{definicion}
\begin{definicion}
Sea $\{M_i\}_{i \in I}$, $I$ un conjunto de índices,  un familia de submódulos de un $R\mbox{-módulo}$ $M$. Se dice que $M$ es la suma directa de submódulos de esta familia y se escribe $M = \oplus_{i \in I}M_i$ si se cumplen las siguientes condiciones:
\begin{bulletList}
\newItem Para todo $i \in I$ se satisface que $M_i \cap \left( \sum_{j \neq i} M_j \right) = \emptyset$.
\newItem $M = \sum_{i \in I}m_i$.
\end{bulletList}
\indent particular, si $\{ m_i \}_{i \in I}$ es un $R\mbox{-base}$ de $M$, entonces $M$ es la suma directa de $M = \oplus_{i \in I}Rm_i$.
\end{definicion}
\begin{definicion}
Un submódulo $N$ de un $R\mbox{-módulo}$ $M$ es un sumando directo si existe otro módulo $N'$ tal que $M = N \oplus N'$. Un módulo que no contiene sumandos directos, a excepción de los triviales, se llama indescomponible. 
\end{definicion}
Caso contrario a los espacio vectoriales, no todo submódulo de un módulo dado es un sumando directo.
\begin{lema}
Sea $N$ un submódulo de un $R\mbox{-módulo}$ $M$. Entonces $N$ es un sumando directo de $M$ si y sólo si existe un endomorfismo $f \colon M \to M$ tal que $f \circ f = f$ e $\Ima (f) = N$.
\end{lema}
El homomorfismo $f \colon M \to M$ del lema anterior es la  proyección de $M$ en $N$. 
\begin{proposicion}\label{prop:bimodulos}
Sea $R$ un anillo. Todo $\modulo{R}$ $M$ es una imagen epimórfica de un $\modulo{R}$ libre. 
\end{proposicion}

% ------------Inicio subseccion semisimplicidad
\subsection{\hskip 1.1em Semisimplicidad}
En álgebra lineal se demuestra que todo subespacio de espacio vectorial es un sumando directo. Esta aseveración no es válida en el caso de módulos sobre anillos arbitrarios, por ejemplo, $\mathds{Z}$ no es un sumando directo de $\mathds{Q}$ como $\mathds{Z}\mbox{-módulo}$. Será de interés conocer los módulos que cumplen la propiedad de tener submódulos que sean sumandos directos. 
\begin{definicion}
Un $R\mbox{-módulo}$ $M$ es semisimple si todo submódulo de $M$ es un sumando directo. 
\end{definicion}  
\begin{proposicion}
Sea $N \neq (0)$ un submódulo de un módulo semisimple $M$. Entonces $N$ es semisimple y contiene a un módulo simple.
\end{proposicion}
\begin{proof*}
Sea $S$ un submódulo arbitario de $N$. Entonces $N$ es también submódulo de $M$, así que existe $S'$ tal que $M = S \oplus S'$. Se asegura que $N = S \oplus (S'\cup N)$. En efecto, por definición $S \cap (S'\cup N) \subset S\cap S' = (0)$. Por otro lado, dado un elemento $n \in N$ se puede escribir $n = x + y$ con $x \in S, y \in S'$, pero $y = n - x \in N$, entonces $y \in N \cup S'$, con lo que se demuestra que $N$ es semisimple. 

Para demostrar que $N$ contiene a un submódulo semisimple elíjase un elemento $x \in N, x \neq 0$. Considérese la familia de submódulos de $N$ que tienen a $x$ como elemento y nótese que dicha familia es no vacía, está parcialmente ordenada por inclusión y además toda cadena está acotada por $N$, por lo que, usando el lema de Zorn, existe un elemento maximal $N_1$. Como $N$ es semisimple, existe $N_2$ submódulo de $N$ tal que $N = N_1 \oplus N_2$. Se requiere demostrar que $N_2$ es simple.
Si $N_2$ no fuera simple, entonces existe $W$ submódulo propio de $N_2$ tal que $N_2 = W \oplus W'$, con $W'$ submódulo de $N_2$. De esta manera, $N = N_1 \oplus W \oplus W'$ y $N_1 = (N_1 + W)\cup(N_1 + W')$. Como $x \notin N_1$ entonces $x \notin N_1 + W$ ni $x \in N_1 + W'$, lo cual contradice el hecho que $N_1$ es maximal.   
\end{proof*}
\begin{teorema}\label{teo:caracSemi}
Sea $M$ un $\modulo{R}$. Entonces, las siguientes condiciones son equivalentes:
\begin{centeredList}
\item\label{item:ssimple1} $M$ es semisimple.
\item\label{item:ssimple2} $M$ es suma directa de submódulos simples.
\item\label{item:ssimple3} $M$ es suma (no necesariamente directa) de submódulos simples.
\end{centeredList}
\end{teorema}
\begin{proof*}
(\ref{item:ssimple1}) $\implies$ (\ref{item:ssimple2}). Sea $\mathcal{F}$ la colección de todos los submódulos de $M$ que se pueden escribir como suma directa de submódulos simples. La proposición anterior asegura la existencia de dichos submódulos. Se define un orden en $\mathcal{F}$ de la siguiente manera: dados dos elementos $\oplus_{i \in I}M_i$ y $\oplus_{i \in J}M_i$ de $\mathcal{F}$, se tiene que $\oplus_{i \in I}M_i  \prec \oplus_{i \in J}M_i$ si y sólo si $I \subset J$. 
Ahora como $(\mathcal{F, \prec})$ satisface las condiciones del lema de Zorn, existe un elemento maximal $M_0 \in \mathcal{F}$ que se puede escribir como $M_0 = \oplus_{i \in I}M_i$ con $M_i, i \in I$ simple.
Ahora sólo falta demostrar que $M_0 = M$. En efecto, si $M_0 \neq M$, entonces existe un submódulo $N$ de $M$ tal que $M = M_0 \oplus N$, pero, por la proposición anterior, $N$ contiene un submódulo simple $S$ y por lo tanto $M_0 \oplus S = \oplus_{i \in I}M_i \oplus S \supset M_0$, lo cual contradice la maximalidad de $M_0$.

(\ref{item:ssimple2}) $\implies$ ($\ref{item:ssimple3}$) es trivial.

(\ref{item:ssimple3}) $\implies$ (\ref{item:ssimple1}). Supóngase que $M = \sum_{i \in I}M_i$, donde cada componente $M_i, i \in I$ es simple. Sea $N$ un submódulo propio cualquiera de $M$. Se demostrará que $N$ es sumando directo. 
Considérese la familia \[ \mathcal{J} = \left\{ \sum_{i \in J} M_i \colon J \subset I, \left(\sum_{i \in J}M_i \right) \cap N = (0)  \right\} \] y nótese que si $N \cap M_i \neq (0)$ entonces $M_i \subset N$. Como $N \neq M$, se deduce que existe al menos un submódulo $M_i$ tal que $N \cap M_i = (0)$ y $\mathcal{J} \neq \emptyset$. Por el lema de Zorn, se puede encontrar un submódulo maximal en $\mathcal{J}$, a saber $M_0 = \sum_{i \in \mathcal{J}_0}M_i$. Como $\left( \sum_{i \in \mathcal{J}_0}M_i  \right) \cap N = (0)$, sólo queda por demostrar que $M = M_0 + N$. Para ello se demostrará que $M_i \subset M_0 + N$, para todo $i \in I$. Supóngase por el absurdo que esto no es cierto, entonces existe un índice $i_{0}$, tal que $M_{i_0} \nsubseteq M_0 + N $. Ahora bien, $M_{i_0}$ es simple, se tiene que $M_{i_0} \cap (M_0 + N) = (0)$. Entonces $(M_{i_0} + M_0)\cap N =(0)$, lo que implica que $M_{i_0} + M_0 \in \mathcal{J}$, lo cual contradice la maximalidad de $M_0$.
\end{proof*}
Si se conoce la descomposición de un módulo semisimple como suma directa de módulos simples, entonces se puede determinar la estructura de todos sus submódulos. 
\begin{corolario}
Sea $M = \oplus_{i \in I}M_i$ una descomposición de un módulo semisimple $M$ como suma directa de submódulos y sea $N$ un submódulo de $M$. Entonces, existe un subconjunto de índices $J \subset I $ tal que $N \simeq \oplus_{i \in J}M_i$.
\end{corolario}
\begin{proof*}
Como se vió en la demostración de la última implicación del teorema anterior, dado un submódulo $N$ de $M$ se puede encontrar un subconjunto de índices $J_0 \subset I$  tal que $M = N \oplus N_0$, donde $N_0 = \oplus_{i \in J_0}M_i$. Entonces:
\[ N \simeq \frac{M}{N_0} = \frac{\oplus_{i \in I}M_i}{\oplus_{i \in J_0}M_i} \simeq \oplus_{i \in I \backslash J_0} M_i, \] de donde se sigue el resultado, con $J = I \backslash J_0$.
\end{proof*}
\begin{corolario}
Un módulo cociente $L$ de un módulo semisimple $M$ es isomorfo a un submódulo de $M$ y por lo tanto es también semisimple.
\end{corolario}
\begin{proof*}
Sea $L$ un módulo cociente de $M$, $\pi \colon M \to L$ el homomorfismo canónico y $N = \ker(\pi)$. Entonces, existe un submódulo $N'$ de $M$ tal que $M = N \oplus N'$ y por lo tanto $N'\simeq M/\ker(\pi) \simeq L$. Con esto, el resultado se sigue del corolario anterior.
\end{proof*}
\begin{definicion}
Un anillo $R$ es semisimple si como módulo $_RR$ es semisimple. 
\end{definicion}
Todo submódulo de $_RR$ es ideal izquierdo de un anillo $R$, así que $R$ es semisimple si y sólo si todo ideal izquierdo es un sumando directo.
\begin{teorema}
Sea $R$ un anillo. Entonces las siguientes proposicione son equivalentes:
\begin{bulletList}
\item\label{item:rssimple1} \hfil Todo $\modulo{R}$ es semisimple.
\item\label{item:rssimple2} \hfil $R$ es un anillo semisimple.
\item\label{item:rssimple3} \hfil $R$ es una suma directa de un número finito de ideales izquierdos minimales.
\end{bulletList}
\end{teorema}
\begin{proof*}~\newline
\indent (\ref{item:rssimple1}) $\implies$ (\ref{item:rssimple2}) es evidente.

(\ref{item:rssimple2}) $\implies$ (\ref{item:rssimple3}). Como los submódulos de $_RR$ son precisamente los ideales izquierdos minimales de $R$, se sigue del teorema \ref{teo:caracSemi} que $R$ se puede escribir como $R = \oplus_{i \in I}L_i$ donde cada $L_i, i \in I$ es un ideal izquierdo minimal. De esta manera sólo queda por demostrar que esta suma es finita.
En particular, como $R = \langle 1 \rangle$, el elemento $1 \in R$ se puede escribir como una suma finita; a saber: $1 = x_{i_1} + \cdots + x_{i_n}$, donde $x_{i_j} \in L_{i_j}$. Entonces para $r \in R$, se tiene que $r = r \cdot 1 = rx_{i_1} + \cdots +rx_{i_n}$, donde $rx_{i_j} \in L_{i_j}, 1\leq j \leq n$. Esto demuestra que $R \subset L_{i_1} \oplus \cdots \oplus L_{i_n}$, de donde $R = L_{i_1} \oplus \cdots \oplus L_{i_n}$.
Nótese que el teorema \ref{teo:caracSemi} demuestra inmediatamente que (\ref{item:rssimple3})~ $\implies$(\ref{item:rssimple2}), así que la demostración estará completa si se demuestra que (\ref{item:rssimple2}) $\implies$~(\ref{item:rssimple1}). 

Supóngase que $R$ es semisimple y sea $M$ un $\modulo{R}$, entonces de la proposición \ref{prop:bimodulos} se sabe que $M$ es una imagen epimórfica de un $\modulo{R}$ libre $F$. Entonces $F$ se puede escribir como $F = \oplus_{i}Ra_i$ donde $Ra_i \simeq R$ es semisimple. Por esto, $F$ es semisimple y por lo tanto $M$ lo es. 
\end{proof*}
\begin{teorema}\label{teo:idealIdem}
Sea $R$ un anillo. Entonces $R$ es semisimple si y sólo si todo ideal izquierdo $L$ de $R$ es de la forma $L = Re$, donde $e \in R$ es un idempotente.
\end{teorema}
\begin{proof*}
Supóngase que $R$ es semisimple y sea $L$ un ideal  izquierdo de $R$. Entonces $L$ es un sumando directo de $R$, entonces existe un ideal izquierdo $L'$ tal que $R = L \oplus L'$. De esa cuenta, se puede escribir $1 = x + y$ donde $x \in L$ y $y \in L'$. Entonces $x = x \cdot 1 = x^2 + xy$. De la igualdad anterior se deduce que $xy = x - x^2 \in L$ y como $L'$ es un ideal izquierdo se tiene que $xy \in L'$. De la definición de suma directa se sabe que $L \cap L' = (0)$ por lo tanto $xy = x - x^2 = 0$, de donde $x = x^2$ es un idempotente. Es obvio que $Rx \subset L$. Además dado $a  \in L$ se tiene que $a = a \cdot 1 = ax + ay$, de esto se obtiene $a -ax = ay\in L\cap L'=(0)$ y así $a = ax \in Rx$ demostrando que $L = Rx$. 

Ahora bien, supóngase que los ideales izquierdos de $R$ son de la forma propuesta en el enunciado. Dado esto, se requiere demostrar que todo ideal izquierdo de $R$ es sumando directo. Por hipótesis $L = Re$ donde $e \in R$ es idempotente. Sea $L'= R(1-e)$. Entonces es claro que $L'$ es un ideal izquierdo y dado un elemento $x \in R$ se puede escribir $x = xe + x(1-e)$, así que $R = Re + R(1-e)$. Además si $x \in Re \cap R(1-e)$ se tiene que cumplir que $x = re = s(1-e)$, con $r,s \in R$. Entonces $xe = s(1-e)e = 0$, de donde $x = 0$.  
\end{proof*}
\begin{teorema}\label{teo:familiaIdempotentes}
Sea $R = \oplus_{i = 1}^t L_i$ una descomposición de un anillo semisimple como suma directa de ideales izquierdos minimales. Entonces, existe una familia $\{e_1, \dots, e_t \}$ de elementos de $R$ tal que:
\begin{bulletList}
\item\label{item:orto1}\hfil $e_i \neq 0, 1 \leq i \leq t$ es idempotente.
\item\label{item:orto2}\hfil Si $i \neq j$ entonces $e_ie_j = 0$.
\item\label{item:orto3}\hfil $1 = e_1 + \cdots + e_t$.
\item\label{item:orto4} $e_i$ no se puede escribir como $e_i = e_i'+ e_i''$, donde $e_i', e_i''$ son idempotentes tales que $e_i, e_i'' \neq 0$ y $e_i'e_i''=0, 1\leq i \leq t$.
\end{bulletList}
Además, si existe una familia de idempotentes $\{e_1, \dots, e_t \}$ que satisface las cuatro condiciones anteriores, entonces la familia de ideales izquierdos minimales $L_i = Re_i$ es tal que $R = \oplus_{i=1}^tL_i$.
\end{teorema}
\begin{proof*}
Supóngase que $R = \oplus_{i = 1}^tL_i$ es una descomposición del anillo $R$ como suma directa de ideales izquierdos minimales. Con esta descomposición se puede escribir $1 = e_1 + \cdots + e_t$ donde $e_i \in L_i$. Entonces se deduce, como en el teorema anterior, que $e_i$ es idempotente tal que $L_i = Re_i, 1 \leq i \leq t$. Si $i \neq j$ entonces $e_ie_j = 0$. Por último, si para algún índice $i$ se puede escribir $e_i = e_i'+ e_i''$, donde $e_i', e_i''$ son idempotentes tales que $e_i, e_i'' \neq 0$ y $e_i'e_i''=0$ entonces de nuevo, como en el teorema anterior, se obtiene que $L_i = Re_i'\oplus Re_i''$ con $Re_i', Re_i'' \neq 0$, lo cual contradice la minimalidad de $L_i$. 

Para el converso, supóngase que existe una familia de idempotentes $\{ e_1, \dots, e_t \}$ que satisfacen las condiciones dadas. Se demostrará en primera instancia, que $L_i = Re_i$ es minimal. Para ello supóngase por el absurdo que no lo es, entonces existe un ideal izquierdo $J$ tal que $J \subset L_i$, pero como $_RR$ es semisimple entonces $L_i$ también lo es, por lo tanto existe $J'$ tal que $L_i = J\oplus J'$. Esto implica que se puede escribir $e_i = e_i'+ e_i''$, donde $e_i', e_i''$ son idempotentes tales que $e_i, e_i'' \neq 0$, lo cual es una contradicción.
$R = L_1 + L_2 + \cdots + L_t$ se deduce fácilmente del hecho que $1 = e_1 + \cdots + e_t$. Ahora para demostrar que la suma es directa, tómese $x \in L_j \cap \left( \sum_{i \neq j}L_i \right)$. Entonces se puede escribir $x = r_je_j = \sum_{i \neq j}r_ie_i$. Multiplicando por $e_j$ por la derecha la ecuación anterior se obtiene $r_je_je_j = x = \sum_{i \neq j}r_ie_ie_j = 0$.
\end{proof*}
\begin{definicion}
Sea $R$ un anillo. Una familia de idempotentes $\{ e_1, \dots, e_t \}$ que satisfacen las condiciones \ref{item:orto1}, \ref{item:orto2} y \ref{item:orto3} del teorema anterior es llamada una famlia completa de idempotentes ortogonales. Un idempotente que satisface la condición \ref{item:orto4} se llama primitivo.
\end{definicion}
\begin{lema}\label{lema:multiModulo}
Sea $L$ un ideal izquierdo minimal de un anillo semisimple $R$ y sea $M$ un $\modulo{R}$. Entonces $LM \neq (0)$ si y sólo si $L \simeq M$ como $R\mbox{-módulos}$. En este caso $LM = M$.
\end{lema}

%------------Inicio Subsección: teorema de Wedderburn-Artin
\subsection{\hskip 1.1em El teorema de Wedderburn-Artin}
Este teorema y los que sirven de base para su demostración son de mucha importancia, ya que revelan la estructura de los anillos semisimples.
\begin{lema}
Sea $L$ un ideal izquierdo minimal de un anillo semisimple $R$. Entonces la suma de todos los ideales izquierdos de $R$ isomorfos a $L$ es un ideal bilateral de $R$.
\end{lema}
\begin{proof*}
Sea $A = \sum_{J \simeq L}J$. Es evidente que $A$ es un ideal izquierdo. Se desea demostrar que $A$ es también un ideal derecho. Como $R$ es semisimple se puede escribir $R = \oplus_{i = 1}^tL_i$ como suma directa de ideales izquierdos minimales. Entonces $AR = \sum_{J \simeq L}JR = \sum_{J \sim L}\sum_{i=1}^{t}JL_i$, pero $JL_i = (0)$ o $JL_i = L_i$. Por el lema \ref{lema:multiModulo} se demuestra que la última alternativa sólo es posible cuando $J \simeq L_i$, lo que implica que $L_i \subset A$. De esta manera se demuestra que $AR \subset A$.
\end{proof*}
\begin{lema}
Sea $I$ un ideal que contiene a un ideal izquierdo minimal $L$ de un anillo semisimple. Entonces $I$ contiene a todos los ideales izquierdos isomorfos a $L$.
\end{lema}
\begin{proof*}
Sea $ L \subset I$ un anillo izquierdo minimal y sea $J$ un ideal izquierdo isomorfo a $L$. Entonces, del lema \ref{lema:multiModulo}, se tiene que $J = LJ \subset I$. 
\end{proof*}
\begin{proposicion}
Sea $L$ un ideal izquierdo minimal de un anillo simisimple $R$ y $B$ la suma de todos los ideales de $R$ isomorfos a $L$. Entonces $B$ es un ideal bilateral minimal de $R$.
\end{proposicion}
\begin{proof*}
Sea $B_1$ un ideal de $R$ contenido en $B$ y $L_1$ un ideal izquierdo minimal de $R$ contenido en $B_1$. Si $L_1 \not\simeq L$, entonces se tiene que $L_1J = (0)$, para todo $J \simeq L$. Así, $L_1B = (0)$ lo cual implica, en particular que $L_1L_1 = (0)$. Esto no es posible porque el teorema \ref{teo:idealIdem}  implica que $L_1$ contiene a un elemento idempotente. Este argumento implica que $L_1 \simeq L$ y aplicando  el lema anterior se obtiene que $B_1 = B$. 
\end{proof*}
Dada una descomposición de un anillo semisimple $R$ como suma directa de ideales izquierdos minimales, se puede agrupar los ideales izquierdos isomorfos de la siguiente manera:
\[ R = \underbrace{L_{11}\oplus \cdots \oplus L_{1r_1}} \oplus \underbrace{L_{21}\oplus \cdots \oplus L_{2r_2}} \oplus \cdots  \oplus \underbrace{L_{s1}\oplus \cdots \oplus L_{sr_s}}. \]
Con la notación anterior, $L_{ij} \simeq L_{ik}$ y $L_{ij}L_{kh} = (0)$ si $i \neq k$, por el lema \ref{lema:multiModulo}. 
\begin{teorema}
Con la notación anterior, sea $A_i$ la suma de todos los ideales izquierdos isomorfos a $L_{i1}, 1\leq i \leq s$. Entonces:
\begin{bulletList}
\item\label{item:Restructura1}\hfil Cada $A_i$ es un ideal minimal de $R$.
\item\label{item:Restructura2}\hfil $A_iA_j = (0)$ si $i \neq j$.
\item\label{item:Restructura3} $R = \oplus_{i=1}^{s}A_i$ como anillos, donde $s$ es el número de clases isomorficas de ideales minimales de $R$.
\end{bulletList}
\end{teorema}
\begin{proof*}
(\ref{item:Restructura1}) se sigue directamente de la proposición anterior. Para demostrar (\ref{item:Restructura2}), se escribe \[ R = (L_{11}\oplus \cdots \oplus L_{1r_1}) \oplus (L_{21}\oplus \cdots \oplus L_{2r_2}) \oplus \cdots  \oplus (L_{s1}\oplus \cdots \oplus L_{sr_s}). \]Entonces todo elemento $x \in R$ se puede escribir en la forma $x = x_{11} + \cdots + x_{rr_1} + \cdots + x_{s1} +\cdots +x_{xr_s}$, con $x_{ij} \in L_{ij}$. Sea $y_i = x_{i1} + \cdots + x_{ir_i}, 1\leq i \leq s$. Entonces $y_i \in A_i, 1\leq i \leq s$ y $x = y_1 + \cdots + y_s$. Esto demuestra que $R = A_1 + \cdots + A_s$. Para terminar, se tiene que $A_i\cap A_j = (0)$ se sigue de la definición de $A_i$ y del lema~\ref{lema:multiModulo}. 
\end{proof*}
\begin{definicion}
Un anillo $R$ es simple si sus únicos ideales son $(0)$ y $R$.
\end{definicion}
Nótese que si $D$ es un anillo y $n$ un entero positivo, entonces $M_n(D)$ es un anillo simple.
\begin{corolario}
Los ideales $A_i, 1\leq i \leq s$, definidos previamente son simples.
\end{corolario}
\begin{proposicion}\label{prop:unicidadDescomposicion}
Sea $R = \oplus_{i=1}^s A_i$ la descomposición de un anillo semisimple $R$ como suma directa de ideales minimales. Entonces:
\begin{bulletList}
\newItem Todo ideal $I$ de $R$ se puede escribir de la forma $I = A_{i_1} \oplus \cdots \oplus A_{i_t}$, donde $1\leq i_{i_1} < \cdots < i_{t} \leq s$.
\newItem Si $R = \oplus_{j = 1}^rB_i$ es otra descomposición de $R$ como suma suma directa de ideales minimales, entonces $s = r$ y (después de una posible ordenación de los índices) $A_i = B_i$ para todo $i$.
\end{bulletList}
\end{proposicion}
\begin{proof*}
Sea $I$ un ideal de $R$. Entonces $I = \oplus_{i = 1}^{s}(A_i \cap I)$. Como los $A_i$ son minimales la primera propiedad queda probada. Por la misma razón cada $B_j$ es igual a algún $A_i$ y viceversa. 
\end{proof*}
\begin{definicion}
Los únicos ideales minimales de un anillo semisimple $R$, son llamados las componentes simples de $R$. 
\end{definicion}
\begin{teorema}
Sea $R = \oplus_{i = 1}^sA_i$ una descomposición de un anillo semisimple como suma directa de ideales minimales. Entonces existe una familia $\{e_1, \dots, e_s \}$ de elementos de $R$ tal que:
\begin{bulletList}
\newItem $e_i \neq 0, 1\leq i  \leq t$ es un idempotente central.
\newItem Si $i \neq j$ entonces $e_ie_j = 0$.
\newItem $1 = e_1 + \cdots + e_t$.
\newItem $e_i$ no puede ser escrito como $e_i = e_i'+e_i''$ donde $e_i', e_i''$ son idempotentes centrales tales que $e_i',e_i'' \neq 0$ y $e_i'e_i'' = 0, 1\leq i \leq t$.
\end{bulletList}
\end{teorema}
\begin{proof*}
La demostración es análoga a la del teorema \ref{teo:familiaIdempotentes}. La única diferencia es que $e_i, 1\leq i \leq s$ son centrales. Entonces para $x \in R$ se tiene de la tercera condición que $x = \sum_{i=1}^{t}xe_i = \sum_{i=1}^{t}e_ix$. Como los $A_i$ son ideales y la suma es directa se concluye que $xe_i = e_ix$.
\end{proof*}
\begin{definicion}
Los elementos $\{ e_1, \dots, e_s \}$ del  teorema anterior son llamados los idempotentes centrales primitivos de $R$.
\end{definicion}
\begin{lema}\label{lema:estructuraSemi}
Sea $R$ un anillo, $M = M_1 \oplus \cdots \oplus M_r$ y $N = N_1 \oplus \cdots \oplus \cdots \oplus N_s$ dos $R\mbox{-módulos}$ escritos como sumas directas de submódulos. Sea $\epsilon_j \colon M_j \to M$ la inclusión de $M_j$ en $M$ y $\pi_i \colon N \to N_i$ el homomorfismo natural de $N$ hacia sus componentes.
\begin{bulletList}
\newItem Supóngase que para cualquier par de índices $i,j$ existe un homomorfismo $\phi_{ij} \in \hom_R(M_j, N_i)$. Entonces,  la aplicación $\phi \colon M \to N$ definida por: \[\phi(m_1 + \cdots + m_r) = \begin{pmatrix}
\phi_{11} & \cdots & \phi_{1r} \\
\vdots & \ddots & \vdots \\
\phi_{s1} & \cdots & \phi_{sr}
\end{pmatrix} 
\begin{pmatrix}
m_1 \\
\vdots \\
m_r
\end{pmatrix}\]
\[ = \underset{\in N_1}{\underbrace{\phi_{11}(m_1) + \cdots + \phi_{1r}(m_r)}} +\cdots+\underset{\in N_s}{\underbrace{\phi_{s1}(m_1) + \cdots + \phi_{sr}(m_r)}}, \] es un homomorfismo. Para indicar que $\phi$ es de la forma previamente descrita se  escribe $\phi = (\phi_{ij})$. El converso también es cierto, es decir, si $\phi$ es de la forma descrita en el inciso anterior, entonces $\phi_{ij} = \pi_i\circ\phi\circ\epsilon_j \in \hom_R(M_j, N_i)$ y $\phi = (\phi_{ij})$.
\newItem Para $\phi = (\phi_{ij})$ y $\psi = (\psi_{ij})$ se tiene que $\phi + \psi = (\phi_{ij} +  \psi_{ij})$.
\newItem $\hom_R(M^{(n)}, M^{(n)}) \simeq M_n(\hom_R(M,M))$ como anillos. 
\end{bulletList} 
\end{lema}
\begin{lema}
Sea $R$ un anillo, $M$ un $\modulo{R}$ semisimple y $B= \hom_R(M,M)$. Entonces $M$  admite una estructura de $\modulo{B}$ dada por $\phi \cdot m = \phi(m),$ para todo $\phi \in B, m \in M$. Más aún para cada $m \in M$ y $f \in \hom_B(M,M)$ existe un elemento $a \in R$ tal que $f(m) = am$.  
\end{lema}
\begin{proof*}
La primera aseveración es evidente. Para demostrar la segunda, sea $m \in M$ y considérese el submódulo $Rm$. Como $M$ es semisimple, entonces existe un submódulo $W$ tal que $M = RM \oplus W$. Si se denota  la proyección hacia $R_m$ como $\phi \colon M \to M$ se tiene que $\pi \in \hom_R(M,M) = B$. Dado un elemento $f \in \hom_B(M,M)$, se tiene:\[ f(m) = f(\pi(m)) = \pi(f(m)) \in Rm. \] Así, existe un elemento $a \in R$ tal que $f(m) = am$. 
\end{proof*}
\begin{teorema}[teorema de densidad de Jacobson]
Sea $M$ un $\modulo{R}$ semisimple, $B= \hom_R(M,M)$ y $f \in  \hom_B(M,M)$. Si $\{ m_1, \dots, m_n \}$ es un conjunto arbitrario de elementos de $M$, entonces existe un elemento $a \in R$ tal que $f(m_i) = am_i$, para todo $1\leq i \leq n$.
\end{teorema}
\begin{proof*}
Dada $f \in \hom_B(M,M)$ se define $f^{(n)} \colon M^{(n)} \to M^{(n)}$ por: \[ f^{(n)}(x_1 + \cdots + x_n) = f(x_1) + \cdots+ f(x_n), \ x_1, \dots, x_n \in M. \]
Sea $B' = \hom_R(M^{(n)}, M^{(n)})$. Se asegura que $f^{(n)} \in \hom_{B'}(M^{(n)}, M^{(n)})$. En efecto, dado $\phi \in B'$, por lema \ref{lema:estructuraSemi} se puede escribir $\phi = (\phi_{ij}) \in \hom_R(M_j,M_i)$. Se tiene
\begin{eqnarray*}
f^{(n)} \circ \phi(m_1+\cdots +m_n) &=& f^{(n)}(\phi_{11}(m_1) +\cdots+\phi_{1n}(m_n)+\cdots\\ 
& &\cdots +\phi_{n1}(m_1)+\cdots+\phi_{nn}(m_n))  \\
&=& \phi_{11}(f(m_1)) + \cdots + \phi_{1n}(f(m_n)) + \cdots \\ 
& &\cdots + \phi_{n1}(f(m_1 )) + \cdots + \phi_{nn}f((m_n))   \\
&=& \phi(f(m_1) + \cdots + f(m_n)  \\
&=&\phi \circ f^{(n)}(m_1 + \cdots + m_n).
\end{eqnarray*}

Además, debido al lema anterior, existe un elemento $a \in R$ tal que $f^{(n)}(m_1+\cdots+m_n) = a(m_1+\cdots+m_n)$, por lo tanto $f(m_i) = am_i, 1\leq i \leq n$.
\end{proof*}
\begin{lema}[Lema de Schur]
Sea $R$ un anillo, $M,N$ $R\mbox{-módulos}$ simples y $f \colon M \to N$ un homomorfismo no nulo. Entonces $f$ es un isomorfismo.
\end{lema}
\begin{proof*}
Dado que $\Ima(f)$ es un submódulo de un módulo simple $N$ y no es igual a $(0)$, entonces $\Ima(f) = N$, así que $f$ es epimorfismo. De manera similar $\ker(f)$ es un submódulo de un módulo simple $N$ y no es igual a $M$ entonces $\ker(f) = (0)$. De esto $f$ es un monomorfismo y por lo tanto $f$ es isomorfismo.
\end{proof*}
\begin{corolario}
Sea $R$ un anillo y $M,N$ $R\mbox{-módulos}$ simples. Entonces:
\begin{bulletList}
\newItem Si $M \not\simeq N$ entonces $\hom_R(M,N) = (0)$.
\newItem $\hom_R(M,M)$ es un anillo de división. 
\end{bulletList}
\end{corolario}
\begin{teorema}[Wedderburn-Artin]
Un anillo $R$ es semisimple si y sólo si es una suma directa de álgebras de matrices sobre anillos de división:
\[ R \simeq M_{n_1}(D_1)\oplus \cdots \cdots M_{n_s}(D_s).  \]
\end{teorema}
Para la demostración de este importante resultado, se sugiere ver \cite[200]{bib:AlgebraPostGrado}.
\begin{teorema}
Sea $R$ un anillo semisimple y supóngase que 
\[ R \simeq M_{n_1} \oplus \cdots \oplus M_{n_s} \simeq M_{m_1}(D_1')\oplus \cdots \oplus M_{m_r}(D_r'), \] donde $D_i, D_j', 1 \leq i \leq s, 1 \leq j \leq r$ son anillos de división. Entonces $s = r$. Además bajo un posible reordenamiento de los índices, se tiene que $n_i = m_i, D_i \simeq D_i'$. 
\end{teorema} \newpage
\begin{proof*}
Como los anillos de matrices sobre anillos de división son simples, se tiene por la proposición \ref{prop:unicidadDescomposicion} que $s = r$ y existe una biyección entre los dos conjuntos de ideales tal que los correspondientes ideales son iguales. Sólo falta demostrar que si $M_n(D)\simeq M_m(D')$, donde $D$ y $D'$ son anillos de división, entonces $n = m$ y $D \simeq D'$.

Sea $E = M_n(D)$, $E' = M_m(D')$ y 
\[ L = \begin{pmatrix} 
D &  0 & \cdots & 0 \\
D &  0 & \cdots & 0 \\
\vdots &\vdots& \ddots &\vdots \\
 D &  0 & \cdots & 0
\end{pmatrix} , L'=\begin{pmatrix}
D' & 0 & \cdots & 0\\
D' & 0 & \cdots & 0\\
\vdots &\vdots& \ddots &\vdots \\
D' & 0 & \cdots & 0\\
\end{pmatrix}.\]
\indent Entonces $L = eE, L'= fE'$ donde $e$ y $f$ son las correspondientes matrices idempotentes con 1 en la posición $(1,1)$ y ceros en cualquier otra posición. Bajo el isomorfismo $E \to E'$, $e$ tiene imagen $e'$ un idempotente tal que $e'L'$ es un ideal izquierdo minimal. Cambiando la base de $E'$ se tiene el isomorfismo $E \to E'$ tal que $e \mapsto f$ y $L \to L$. Entonces \[ D \simeq eEe \to fE'f \simeq D' \] y contando las dimensiones se llega a que $n = m$.  
\end{proof*}
