\begin{thebibliography}{99}
    \addcontentsline{toc}{chapter}{BIBLIOGRAFÍA}
    % LETRA B
    \bibitem{bib:libroLosGrandes} BELL, Eric. \textit{Los grandes matemáticos.} Argentina: Editorial Losada, 1948. 100~ p. 
    \bibitem{bib:codeBook} BLAKE, Ian.\textit{The mathematical theory of coding}. Estados Unidos: Academic Press Inc, 1975. 363 p.
    \bibitem{bib:burnside} BURNSIDE, William. \textit{The theory of groups of finite order}. 	2a ed. Cambridge: Cambridge University Press, 1911. 509 p.
        
    
    %LETRA C
     \bibitem{bib:Cauchy} CAUCHY, Augustin-Louis. Oeuvres complètes. Cambridge: Cambridge University Press, 2009. 524 p.
     
     %LETRA D
     \bibitem{bib:Deskins} DESKINS, Eugene. Finite abelian groups with isomorphic group algebras. \textit{Duke Mathematical Journal}. 1956, vol 23, núm. 1, p. 35-40.
     
     %LETRA F
      \bibitem{bib:solubilidad} FEIT, Walter, et al. The solvability of groups of odd order. \textit{Pacific J. Math}. 1963, vol 13, núm 3, p. 775-1029.
      
    %LETRA G
    \bibitem{bib:grupsfact}  GOLDSHMIDT, David. A group theoretic proof of the $p^aq^b$ theorem for odd primes. \textit{Mathematische Zeitschrift}. 1970, vol 113, núm. 5, p. 373-375.
    
    %LETRA H
     \bibitem{bib:historia} HAWKINS, Thomas. The origins of the theory of group characters. \textit{Archive for History of Exact Sciences}. 1971, vol 7, núm. 2, p. 142-170.
    \bibitem{bib:herstein} HERSTEIN, Nathain. \textit{Topics in algebra}. 2a ed.  New York: Macmillah, 1986.~ 400~ p. 
    
   %LETRA I
 	\bibitem{bib:Connell} IAN, Connell. On the group ring. \textit{Canad. J. Math}. 1963, vol 15, núm 1, p. 650-685.
 	
   \bibitem{bib:AlgebraPostGrado}ISAACS, Martin. \textit{Algebra: a graduate course}. Estados Unidos: Editorial Pacific Grove, 1940. 516 p.  
   
   %LETRA J
   \bibitem{bib:Jeanings} JENNINGS, Arthur. The structure of the group ring of a p-group over a modular field. \textit{Transactions of the American mathematical society}. 1941, vol. 50, núm 1, p. 175-185.
   
   
   %LETRA L    
    \bibitem{bib:lang}LANG, Serge. \textit{Linear algebra}. 3a ed. Nueva York: Springer-Verlag, 2004. 308~ p.
    
    %LETRA M
    \bibitem{bib:moser} MOSER, Claude. Représentation de -1 comme somme de carrés dans un corps cyclotomique quelconque. \textit{Journal of Number Theory}. 1973, vol 5, núm. 2, p. 138-141.
    
    %LETRA P
    \bibitem{bib:passman} PASSMAN, Donald. \textit{The algebraic structure of group rings}. New York: Wiley-Interscience, 1977. 550 p.
    \bibitem{bib:PerlisWalker} PERLIS, Sam; WALKER, Gordon. Abelian group algebras of finite order. \textit{Transactions of The American Mathematical Society}. 1950, vol 68, núm 3, p. 420-426.
    \bibitem{bib:main} POLCINO, César; SEHGAL, Sudarshan. \textit{An introduction to group rings}.  Dordrecht: Kluwer Academic Publishers, 2002. 371 p.
    
    %LETRA S
    \bibitem{bib:libroGuti} STEWART, Ian. \textit{De aquí al infinito -- Las matemáticas de hoy.} España: Editorial Crítica, 1998. 304 p.
    \bibitem{bib:Sehgal} SUDARSHAN, Sehgal.  \textit{Topics in group rings}. New York: Marcel Dekker, 1978. 233 p. 
    
    
    %LETRA Z
    
	\bibitem{bib:groupBook} ZASSENHAUS, Hans. \textit{The theory of groups}. Estados Unidos: Chelsea publishing company, 1937. 288 p. 
	
		
		
\end{thebibliography}
