\chapter{INTRODUCCIÓN}
Los grupo-anillos son una estructura muy interesante por sus propiedades algebraicas y su importancia surgió aparentemente, después de los trabajos de T. Molien, G. Frobenius, I. Schur y H. Maschke en los inicios del siglo \lsc{xx}. La importancia de esta estructura en la teoría de la representación de grupos fue establecida por E. Noether y R. Brauer y a partir de ese punto los grupo-anillos comenzaron a ser estudiados como materia aparte por derecho propio. 

El estudio de los grupo-anillos involucra el conocimiento de diversas ramas de la matemática (teoría de campos, el álgebra lineal y la teoría algebraica de números), además de estar ampliamente relacionados con la topología algebraica, el álgebra homológica y la $K$-teoría algebraica. En la última década, también se ha encontrado que los grupo-anillos tienen aplicaciones en la teoría algebraica de codificación.

Dentro de la teoría de códigos es muy importante el estudio de los códigos correctores, los cuales están explícitamente diseñados para evitar la pérdida de información debido a problemas de ruido en la transmisión. En este sentido los primeros diseños de códigos correctores fueron los de bloque, los cuales construyen bloques de información para luego ser transformados en otro tipo de bloques mediante una aplicación llamada diccionario. 

El coste computacional de la codificación es excesivo, así que se hace necesario introducir una estructura algebraica que permita simplicar los procesos de codificación. 

\newpage De esta forma los grupo-anillos juegan un papel importante en la teoría algebraica de la codificación, permitiendo conocer el código cíclico y su estructura sin necesidad de una implementación.

