\begin{thebibliography}{99}
    \addcontentsline{toc}{chapter}{BIBLIOGRAFÍA}
    \bibitem{bib:historia} T. Hawkins. \textit{The origins of the Theory of Group Characters}, Archive Hist. Exact Sci. $7$ (1970-71). p. 142-170.
    \bibitem{bib:burnside} WILLIAM, Burnside, \textit{The theory of Groups of Finite Order}. 	2da ed. Cambridge: Cambridge University Press, 1911.
    \bibitem{bib:grupsfact}  GOLDSHMIDT, David. \textit{A group theoretic proof of the $p^aq^b$ theorem for odd primes}. Math. Z. 13 (1970). p 373-375.
    \bibitem{bib:solubilidad} WALTER, Feit y JHON, Thompson. \textit{The solvability of groups of odd order}. Pacific J. Math. 15 (1963). p 775-1029.
    \bibitem{bib:lang}SERGE, Lang. \textit{Linear Algebra}. 3ra ed. Nueva York: Springer-Verlag, 2004. 308 p.
    \bibitem{bib:passman} DONALD, Passman. \textit{The algebraic structure of group rings}. New York: Wiley-Interscience, 1977. 550 p.
    \bibitem{bib:moser} CLAUDE, Moser. \textit{Representation de -1 comme somme de carres dans un corps cyclotomique quelconque}. J. Number Theory 5 (1973), 138-141.
    \bibitem{bib:Sehgal} SUDARSHAN, Sehgal.  \textit{Topics in Group Rings}.New York: Marcel Dekker, 1978. 233 p. 
    \bibitem{bib:herstein} NATHAN, Herstein. \textit{Topics in Algebra}. 2nda ed.  New York: Macmillah, 1986 . 381 p. 
    \bibitem{bib:libroGuti} Stewart, Ian. \textbf{De aquí al infinito -- Las matemáticas de hoy.} España: Editorial Crítica (1998).
    \bibitem{bib:libroLosGrandes} Bell, E.T. \textbf{Los grandes matemáticos.} Argentina: Editorial Losada (1948).
    \bibitem{bib:Cauchy} LOUIS, Cauchy. \textit{Oeuvres complètes de A.L. Cauchy]}, 26 vols., Gauthier-Villars, Paris, 1882-1938.

\end{thebibliography}
