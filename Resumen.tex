\chapter*{RESUMEN}
\addcontentsline{toc}{chapter}{RESUMEN}
En el siguiente trabajo de graduación se hace un estudio detallado de la teoría básica de los grupo-anillos, necesaria para el desarrollo de la teoría de códigos , dando énfasis en la relación que tienen con la teoría de grupos y la teoría de anillos, ambas materias de estudio de un pregrado en Matemática.

El trabajo está estructurado en seis capítulos cuyo contenido se describe a continuación:

El primer capítulo contiene todo el bagaje matemático que sirve de cimiento para un  estudio adecuado de los grupo-anillos. 

En el segundo capítulo se da la definición de un grupo-anillo y una grupo-álgebra, caso especial del anterior. Posteriormente, se establecen las condiciones necesarias y suficientes para que un grupo-anillo sea semisimple. 

En el tercer capítulo se estudia la teoría de representación de grupos y su relación con los módulos de los grupo-anillos. 

En el cuarto capítulo se estudian algunos elementos algebraicos de un grupo-anillo como los elementos nilpotentes, los idempotentes y las unidades de torsión.

En el quinto capítulo se da un breve introducción al estudio de las unidades de un grupo-anillo, mostrando algunas construcciones de unidades no triviales para los mismos.

Finalmente en el sexto capítulo se da una introducción a la teoría de códigos correctores, dando relevancia a los códigos cíclos y mostrando que dichos códigos tienen una fuerte conexión con las grupo-álgebras.




