%-----------------------Inicio Preámbulo-----------------------------------------------
\documentclass[letterpaper,12pt]{thesis}
\raggedbottom
\usepackage{layout}
\usepackage[utf8]{inputenc}
\usepackage[pdftex]{graphicx}
\usepackage{tikz}
\usepackage{titlesec}
\usepackage{parskip}
\usepackage{amsfonts,amsmath,amssymb,amsthm,mathrsfs}
\usepackage{latexsym,wasysym}
%manejando los margenes
\usepackage[letterpaper, textwidth=5.94095 in, textheight=21.3 cm, footskip = 0.815cm, left=1.5748 in, top=1.55480315in,twoside ]{geometry}
%\usepackage[paper=letterpaper, marginparsep=0in,footskip = 0.7505cm, marginparwidth= 0in, headsep= 0in, headheight= 0in, top=4cm, bottom=2.5cm,left=4cm,right=2.5cm,twoside]{geometry}
\usepackage{textcomp}
\usepackage[all]{xy}
\usepackage{longtable}
\usepackage[spanish]{babel}
\addto\captionsspanish{\renewcommand{\listfigurename}{ÍNDICE DE ILUSTRACIONES}}
\usepackage{bibthesis,color, enumerate, indentfirst}
\usepackage{hyperref,url,breakurl}
\usepackage{ dsfont }
%para manejar las imagenes (titulos y subtitulos de)
\usepackage{caption}
\usepackage{subcaption}
%para cambiar letra de subcaption a 10 pt
\usepackage{etoolbox}
\usepackage[nodisplayskipstretch]{setspace}
%Paquete para el manejo del Indice
\usepackage[titles]{tocloft}




%---------------------------Fin Preambulo------------------------------------------------
%------------Nueva Definicion para ámbito proof.
\makeatletter
\newenvironment{proof*}[1][\proofname]{\par
  \pushQED{\qed}%
  \normalfont \partopsep=\z@skip \topsep=\z@skip
  \trivlist
  \item[\hskip\labelsep
        \itshape
    #1\@addpunct{.}]\ignorespaces
}{%
  \popQED\endtrivlist\@endpefalse
}
\makeatother


\makeatletter
\g@addto@macro\normalsize{%
  \setlength\abovedisplayskip{1.81em}
  \setlength\belowdisplayskip{1.81em}
  \setlength\abovedisplayshortskip{1.81em}
  \setlength\belowdisplayshortskip{1.81em}
}
%-------------Implementacion distancia titulos, subtitulos, etc -------------
\titlespacing{\section}{0pt}{*0.1}{*0.1} % <------  Separación ente el titulo de seccion y parrafo y seccion y titulo.
\titlespacing{\subsection}{1.76cm}{*0.1}{*0.1} % <------  Separación ente el subtitulo y el inicio de párrafo.



%------------Sección de comandos-----------
\decimalpoint %<--------- para manejar punto decimal en lugar de coma decimal
\setlength{\parindent}{1cm}%<-------------------------sangría
%\setlength{\parskip}{7mm}
\setlength{\parskip}{1.37\baselineskip}%<---------------distancia entre párrafos
\setstretch{1.45}    
%\renewcommand\baselinestretch{1.5}               %<-----------------interlineado
\setlength\LTleft{0pt} \setlength\LTright{0pt} %<------------- parametros para tablas largas








% -------- Modificaciones a Citas (sistema lancasteriano) y direcciones web -------------
\bibpunct[--]{[}{]}{,}{n}{,}{;}
\setlength{\bibsep}{21pt}               % distancia entre items en la bibliograf�a
%\urlstyle{rm}                          % fuente en direcciones web

% ---- Sección de parametrización del Indice General y de ilustraciones -----------------
%Agregar lineas punteadas del capitulo al numero de pagina
\renewcommand{\cftchapleader}{\cftdotfill{\cftchapdotsep}}
\renewcommand{\cftchapdotsep}{\cftdotsep}
%quitar las negrillas para los titulos de los capitulos
\renewcommand{\cftchapfont}{\normalfont}
\renewcommand{\cftchappagefont}{\normalfont}
%Espacio vertical entre capítulos
\setlength{\cftbeforechapskip}{-.01cm}
%Espacio vertical entre secciones y capítulo
\setlength{\cftbeforesecskip}{-0.01cm}
%Espacio vertical entre subsecciones y secciones
\setlength{\cftbeforesubsecskip}{-.01cm}
%Espacio vertical entre subsubsecciones y subsecciones
\setlength{\cftbeforesubsubsecskip}{-.01cm}
% Espacio vertical entre figuras
\setlength{\cftbeforefigskip}{-.01cm}
% Espacio vertical entre tablas
\setlength{\cftbeforetabskip}{-.01cm}
% Espacio del margen al capítulo
\setlength{\cftchapindent}{0cm}
% Espacio entre número y capítulo
\setlength{\cftchapnumwidth}{1.36cm}
% Espacio del margen a la sección
\setlength{\cftsecindent}{1.21cm}
% Espacio entre número y sección
\setlength{\cftsecnumwidth}{1.27cm}
% Espacio del margen a la subsección
\setlength{\cftsubsecindent}{2.93cm}
% Espacio entre número y la subsección
\setlength{\cftsubsecnumwidth}{1.48cm }
% Espacio del margen a la subsección
\setlength{\cftsubsubsecindent}{4.92cm}
% Espacio entre número y la subsubsección
\setlength{\cftsubsubsecnumwidth}{2.25cm}
% Espacio entre número y figura
\setlength{\cftfignumwidth}{1.3cm}
% Espacio del margen a la figura
\setlength{\cftfigindent}{0cm}
% Espacio entre número y tabla
\setlength{\cfttabnumwidth}{1.3cm}
% Espacio del margen a la tabla
\setlength{\cfttabindent}{0cm}
%Para alinear los numeros romanos
\cftsetpnumwidth{2em}


%----------------Definicion de listas----------
%Para listados enumerados
\newcounter{ContadorLista}
\newenvironment{bulletList}{
\begin{list}
{\centering\arabic{ContadorLista}.}
{
\usecounter{ContadorLista}
\setlength{\labelsep}{1.125cm}% Separación entre número y texto
\setlength{\labelwidth}{1cm}
\setlength\listparindent{0cm}
\setlength\leftmargin{1.4cm} % Distancia entre borde y número
\setlength{\parsep}{.0cm}% Separación entre elementos
}
}{
\end{list}
}

%-----------------Definicion de estilos para los teoremas--------------------------------
\newtheoremstyle{ejem} %
  {\baselineskip} % Space above
  {\topsep} % Space below
  {} % Body font
  {} % Indent amount
  {\bfseries} % Theorem head font
  {.} % Punctuation after theorem head
  {.5em} % Space after theorem head
  {} % Theorem head spec (can be left empty, meaning `normal')
\newtheoremstyle{resto} %
  {\baselineskip} % Space above
  {0mm} % Space below
  {\itshape} % Body font
  {} % Indent amount
  {\bfseries} % Theorem head font
  {.} % Punctuation after theorem head
  {.5em} % Space after theorem head
  {} % Theorem head spec (can be left empty, meaning `normal')
  
\theoremstyle{resto}\newtheorem{teorema}{Teorema}[section]
\theoremstyle{resto}\newtheorem{lema}{Lema}[section]
\theoremstyle{resto}\newtheorem{proposicion}{Proposición}[section]
\theoremstyle{resto}\newtheorem{propiedad}{Propiedad}[section]
\theoremstyle{resto}\newtheorem{corolario}{Corolario}[section]
\theoremstyle{ejem}\newtheorem{nota}{Nota}[section]
\newtheorem{solucion}{Solución}[section]
\newtheorem{problema}{Problema}[section]
\theoremstyle{ejem}\newtheorem{definicion}{Definición}[section]
\theoremstyle{ejem}\newtheorem{notacion}{Notación}[section]
\theoremstyle{ejem}\newtheorem{consideracion}{Consideración}[section]
\theoremstyle{ejem}\newtheorem{ejercicio}{Ejercicio}[section]
\theoremstyle{ejem} \newtheorem{ejemplo}{Ejemplo}[section]
\numberwithin{equation}{chapter}
\renewcommand{\theequation}{\thechapter.\arabic{equation}}
%para manejar el estilo de los ejemplos

%---------------- Operadores y funciones -------------------------------------------
\DeclareMathOperator{\car}{car}
\DeclareMathOperator{\rango}{rango}
\DeclareMathOperator{\sop}{sop}
\DeclareMathOperator{\Ima}{Im}
\DeclareMathOperator{\tr}{tr}
\DeclareMathOperator{\grado}{deg}
\makeatother
\newcommand{\srg}[3]{\sum_{#1 \in #2} #3_{#1} #1 }
\newcommand{\modulo}[1]{R\mbox{-módulo}}
\newcommand{\newItem}{\item[$\bullet$]}

%-------palabras claves para el documento --------------------------
\hypersetup{pdftitle = {tituloDePDF}, pdfkeywords = {poner un listado de palabras clave}, pdfauthor = {\textcopyright\ 2012, Hugo Allan García Monterrosa (hugoallangm@gmail.com)}, pdfsubject = {Trabajo
de graduación para obtener la Licenciatura en Matemática Aplicada},
pdfstartview = {FitH}, plainpages = {false}, bookmarksnumbered =
{true}}


\begin{document}
\captionsetup[figure]{labelformat=simple, labelsep=period}
% Datos de la portada
\newcommand{\Nomb}{Hugo Allan García Monterrosa} 		%Nombre del estudiante
\newcommand{\TiTes}{Teoría de los grupo-anillos y sus aplicaciones}	                %Título de tesis
\newcommand{\Carrera}{Licenciatura en Matemática aplicada }	                        %Carrera
\newcommand{\Ases}{Ph.D. Sergio Roberto López Permouth}                   %Nombre del asesor
\newcommand{\Fec}{mayo de 2014}                	%Fecha: mes y año.
\newcommand{\Grado}{Licenciado en Matemática Aplicada }              	%Grado.
\newcommand{\Esc}{Ciencias} 				%Escuela.
\newcommand{\FecP}{noviembre de 2012}                     %Fecha de aceptación del protocolo.

% Miembros de la Junta Directiva
\newcommand{\Dec}{Ing. Murphy Olympo Paiz Recinos}     	
\newcommand{\VocI}{Ing. Alfredo Enrique Beber Aceituno}  	
\newcommand{\VocII}{Ing. Pedro Antonio Aguilar Polanco}
\newcommand{\VocIII}{Inga. Elvia Miriam Ruballos Samayoa}
\newcommand{\VocIV}{Br. Walter Rafael Véliz Muñoz}
\newcommand{\VocV}{Br. Sergio Alejandro Donis Soto}
\newcommand{\Sec}{Ing. Hugo Humberto Rivera Pérez}

% Terna
\newcommand{\ExaI}{Dra. Mayra Virginia Castillo Montes }
\newcommand{\ExaII}{Lic. Francisco Bernardo Raúl De La Rosa }
\newcommand{\ExaIII}{Lic. José Ricardo Rodrigo Vásquez Bianchi }
  	



\newpage
\thispagestyle{empty}
%%%%%%%%%%%%%%%%%%%%%%%%%%%%%%%%%%%%%%%%%%%%%%%%%%%%%%%%%%%%%%%%%%%%%%%%%%%%%%%%%%%
%                          Primera Página
%%%%%%%%%%%%%%%%%%%%%%%%%%%%%%%%%%%%%%%%%%%%%%%%%%%%%%%%%%%%%%%%%%%%%%%%%%%%%%%%%%%

\setbox0\vbox{\noindent
\hspace{.7cm}
\includegraphics[width=2.2cm]{Pictures/escudo.png}}
\ht0=13pt\box0
\hangindent=35mm\hangafter=-3\vspace{-1.55cm}
% aquí está el ecabezado
\noindent Universidad de San Carlos de Guatemala\\
Facultad de Ingeniería\\
Escuela de \Esc  \\ 
\vspace{5.2cm}
\begin{center}
% Título
\textbf{TEORÍA DE LOS GRUPO-ANILLOS} \\
\textbf{Y SUS APLICACIONES} \\ 
\vspace{5.8cm}
% Autor
\textbf{\Nomb}\\ 
Asesorado por: \Ases  \\
\hskip 7.5em Lic. William Roberto Gutiérrez Herrera\\
\vspace{1.5cm}
% Lugar y fecha
Guatemala,  \Fec
\end{center}


\newpage
\thispagestyle{empty}
\mbox{}
\newpage
\thispagestyle{empty}
%%%%%%%%%%%%%%%%%%%%%%%%%%%%%%%%%%%%%%%%%%%%%%%%%%%%%%%%%%%%%%%%%%%%%%%%%%%%%%%%%%%
%                          Segunda Página
%%%%%%%%%%%%%%%%%%%%%%%%%%%%%%%%%%%%%%%%%%%%%%%%%%%%%%%%%%%%%%%%%%%%%%%%%%%%%%%%%%%
\begin{center}
UNIVERSIDAD DE SAN CARLOS DE GUATEMALA \\[3mm]
\includegraphics[width=4cm]{Pictures/escudo.png} \\ [1mm]
FACULTAD DE INGENIERÍA\\
\vspace{1.4cm}
\textbf{TEORÍA DE LOS GRUPO-ANILLOS}\\
\textbf{Y SUS APLICACIONES} 

TRABAJO DE GRADUACIÓN 

PRESENTADO A LA JUNTA DIRECTIVA DE LA \\ 
FACULTAD DE INGENIERÍA \\ 
POR\\
\vspace{1.4cm}
\textbf{\MakeUppercase{\Nomb}} \\
ASESORADO POR \MakeUppercase{\Ases}\\
\hskip 11em LIC. WILLIAM ROBERTO GUTIÉRREZ HERRERA 
	
AL CONFERÍRSELE EL TÍTULO DE 

\textbf{\MakeUppercase{\Grado}} \\ 
%\vspace{1.3cm}
\vfill
GUATEMALA, \MakeUppercase{\Fec}
\end{center}

\newpage
\thispagestyle{empty}
\mbox{}
\newpage
\thispagestyle{empty}
%%%%%%%%%%%%%%%%%%%%%%%%%%%%%%%%%%%%%%%%%%%%%%%%%%%%%%%%%%%%%%%%%%%%%%%%%%%%%%%%%%%
%                          Tercera Página
%%%%%%%%%%%%%%%%%%%%%%%%%%%%%%%%%%%%%%%%%%%%%%%%%%%%%%%%%%%%%%%%%%%%%%%%%%%%%%%%%%%

\begin{center}

UNIVERSIDAD DE SAN CARLOS DE GUATEMALA \\
FACULTAD DE INGENIERÍA 

\includegraphics[width=4cm]{Pictures/escudo.png}\\ [17mm]

\textbf{NÓMINA DE JUNTA DIRECTIVA} 	

\begin{tabular}{ll}
    DECANO     & \Dec \\ 
    VOCAL I    & \VocI \\ 
    VOCAL II   & \VocII \\ 
    VOCAL III  & \VocIII \\ 
    VOCAL IV   & \VocIV \\ 
    VOCAL V    & \VocV \\ 
    SECRETARIO & \Sec  
\end{tabular}\\
\vspace{1.6cm}
\textbf{TRIBUNAL QUE PRACTICÓ EL EXAMEN GENERAL PRIVADO} 

\begin{tabular}{ll}
    DECANO     & \Dec \\
    EXAMINADOR & \ExaI \\
    EXAMINADOR & \ExaII \\
    EXAMINADOR & \ExaIII \\
    SECRETARIO & \Sec
\end{tabular}
\end{center}

\newpage
\thispagestyle{empty}
\mbox{}
\newpage
\thispagestyle{empty}
%%%%%%%%%%%%%%%%%%%%%%%%%%%%%%%%%%%%%%%%%%%%%%%%%%%%%%%%%%%%%%%%%%%%%%%%%%%%%%%%%%%
%                          Cuarta Página
%%%%%%%%%%%%%%%%%%%%%%%%%%%%%%%%%%%%%%%%%%%%%%%%%%%%%%%%%%%%%%%%%%%%%%%%%%%%%%%%%%%
\begin{center}
  \textbf{HONORABLE TRIBUNAL EXAMINADOR}
\end{center} 
\vspace{1.0cm}
En cumplimiento con los preceptos que establece la ley de la Universidad de
San Carlos de Guatemala, presento a su consideración mi trabajo de
graduación titulado: 
\vspace{.4cm}
\begin{center}
\textbf{ \MakeUppercase{\TiTes}} 
\end{center}
\vspace{1.0cm}
Tema que me fuera asignado por la Dirección de la Escuela de \Esc , con fecha \FecP. 
\vspace{.4cm}
\begin{figure}[!ht]
\begin{flushright}
\begin{tabular}{c}
%\includegraphics[scale=.5]{Pictures/Firma.jpg}\\
\Nomb
\end{tabular}
\end{flushright}
\end{figure}
\newpage
\thispagestyle{empty}

%Para incluir cartas
\newpage
\thispagestyle{empty}
\mbox{}
%\includepdf[pages={1}]{Cartas/NormativoGeneral_Evaluacion_y_Promocion.pdf}
\newpage
\thispagestyle{empty}
\mbox{}
%\includepdf[pages={2}]{Cartas/NormativoGeneral_Evaluacion_y_Promocion.pdf}
\newpage
\thispagestyle{empty}
\mbox{}

%Para incluir  dedicatoria y agradecimientos
\chapter*{ACTO QUE DEDICO A:}
\vspace{.5cm}
\thispagestyle{empty}
\begin{flushleft}
\renewcommand{\arraystretch}{1} % Espacio entre elementos 
\begin{longtable}{@{}l@{\extracolsep{2.8cm}}  p{4.0in}@{} l@{}}
\textbf{Persona 1} & Dedicatoria.\\
&\\
\textbf{Persona 2}& Dedicatoria.
\end{longtable}
\end{flushleft}
\newpage
\thispagestyle{empty}


\chapter*{AGRADECIMIENTOS A:}
\vspace{.5cm}
\thispagestyle{empty}
\begin{flushleft}
\renewcommand{\arraystretch}{1} % Espacio entre elementos 
\begin{longtable}{@{}l@{\extracolsep{2.8cm}}  p{4.0in}@{} l@{}}
\textbf{Persona 1 } & Agradecimiento.\\
&\\
\textbf{Persona 2}& Agradecimiento
\end{longtable}
\end{flushleft}
\newpage
\thispagestyle{empty}




%---------------Inicio del front matter -----------
\frontmatter     	
%Índice general
\tableofcontents
% Para añadir el Índice de ilustraciones
%Listado de figuras
\listoffigures
%Listado de tablas
%\listoftables
\chapter{LISTA DE SÍMBOLOS}
\vskip 1.38em
\begin{longtable}{@{}l@{\extracolsep{1.8cm}} p{4.75in} @{}}
  \textbf{Símbolo} & \textbf{Significado}\\[12pt]
  \endfirsthead
%LETRA C

    {\boldmath $a \mid b$} & $a$ divide a $b$\\[3pt]
    {\boldmath $a \equiv b$} & $a$ es congruente con $b$\\[3pt]
    {\boldmath $A \simeq B$} & $A$ es isomorfo a $B$\\[3pt]
    {\boldmath $A \triangleleft B$} & $A$ es subgrupo normal de $B$\\[3pt]
    {\boldmath $A \to B$} & $A$ se mapea en $B$\\[3pt]
    {\boldmath $a \nmid b$} & $a$ no divide a $b$\\[3pt]
    {\boldmath $Ann_d(X)$} & Aniquilador del conjunto $X$ por la derecha\\[3pt]
	{\boldmath $Ann_i(X)$} & Aniquilador del conjunto $X$ por la izquierda\\[3pt]
	{\boldmath $\car(K)$} & Característica del campo $K$\\[3pt]
	{\boldmath $\hom_K(A,A)$} & Conjunto de homomorfismos de $A$ en $A$ como $K\mbox{-módulos}$\\[3pt]
	{\boldmath $\mathds{Z}$} & Conjunto de números enteros\\[3pt]
	{\boldmath $\mathds{Q}$} & Conjunto de números racionales\\[3pt]
	{\boldmath $\mathds{R}$} & Conjunto de números reales\\[3pt]
	{\boldmath $\emptyset$} & Conjunto vacío\\[3pt]
	{\boldmath $[a,b]$} & Conmutador de a y b\\[3pt]
	{\boldmath $\cos$} & Función trigonométrica coseno\\[3pt]
	{\boldmath $\sin$} & Función trigonométrica seno\\[3pt]
	{\boldmath $\grado$} & Grado de una representación de un polinomio\\[3pt]
	{\boldmath $\implies$} & Implicación\\[3pt]
	{\boldmath $\infty$} & Infinito\\[3pt]
	{\boldmath $\cap$} & Intersección de conjuntos\\[3pt]
	{\boldmath $\ker(f)$} & Kernel de la función $f$	\\[3pt]
	{\boldmath $\neq$} & No igual a \\[3pt]
	{\boldmath $\notin$} & No pertenece a\\[3pt]
	{\boldmath $\|a\|$} & Norma del vector $a$\\[3pt]
	{\boldmath $\circ(p)$} & Orden del elemento $p$\\[3pt]
	{\boldmath $\in$} & Pertenencia \\[3pt]
	{\boldmath $\subseteq$} & Subconjunto\\[3pt]
	{\boldmath $\subset$} & Subconjunto propio\\[3pt]
	{\boldmath $\oplus$} & Suma directa\\[3pt]
	{\boldmath $\sum$} & Sumatoria\\[3pt]
	{\boldmath $\colon$} & Tal que\\[3pt]
	{\boldmath $\tr(A)$} & Traza de la matriz $A$\\[3pt]
	{\boldmath $\cup$} & Unión de conjuntos\\[3pt]
	{\boldmath $x \mapsto y$} & $x$ se mapea en $y$\\[3pt]
	
\end{longtable}
  		
\chapter{GLOSARIO}
%\addcontentsline{toc}{chapter}{GLOSARIO}
\vspace{.5cm}
\begin{flushleft}
\renewcommand{\arraystretch}{1} % Espacio entre elementos 
\begin{longtable}{@{}l@{\extracolsep{1.6cm}}  p{4.0in}@{} l@{}}
\textbf{Aplicación} & Se refiere a una regla que asigna a cada elemento de un primer conjunto, un único elemento de un segundo conjunto. \\
&\\
\multirow{2}{3cm}{\textbf{Campo ciclotómico}} & Es un cuerpo numérico que se obtiene al añadir una raíz primitiva de la unidad compleja a el campo de los números racionales.\\
&\\
\textbf{Cubierta} & Es una colección de subconjuntos $A$ de un conjunto $X$, tal que la unión de los elementos de la colección $A$ es igual a $X$. \\
&\\
\multirow{2}{2cm}{\textbf{Grupo soluble}} & Es un grupo para el cual existe una cadena finita de subgrupos $\{G_i\}_{i=1}^{n}\subset G$ tal que
 $\{1_G\}=G_0\subseteq G_1 \subseteq \dots \subseteq G_n = G$ donde para cada $i=0,1,\dots,n-1$ se cumple que $G_i$  es subgrupo normal en $G_{i+1}$ y
 el grupo cociente  $G_{i+1}/G_i$  es abeliano.\\
&\\
\multirow{2}{3cm}{\textbf{Inducción matemática}} & Es un razonamiento que permite demostrar una infinidad de proposiciones, o una proposición que depende de un parámetro $n$, que toma una infinidad de valores enteros.\\
&\\
\textbf{Norma} & Es la función que determina el tamaño de un elemento de un espacio vectorial.\\
&\\
\multirow{2}{3.5cm}{\textbf{Registro de desplazamiento}} &  Es un circuito digital secuencial consistente en una serie de biestables, generalmente de tipo D, conectados en cascada que basculan de forma sincrónica con la misma señal de reloj.\\
&\\
\textbf{Tupla} & Es una secuencia ordenada de objetos, esto es, una lista con un número limitado de objetos.\\
&\\
\end{longtable}
\end{flushleft}



		
\chapter{RESUMEN}
En el siguiente trabajo de investigación se hace un estudio detallado de la teoría básica de los grupo-anillos, necesaria para el desarrollo de la teoría de códigos\textbf{}, dando énfasis en la relación que tienen con la teoría de grupos y la teoría de anillos, ambas materias de estudio de un pregrado en Matemática.

El trabajo está estructurado en seis capítulos, cuyo contenido se describe a continuación:

El primer capítulo contiene todo el bagaje matemático que sirve de cimiento para un  estudio adecuado de los grupo-anillos. 

En el segundo capítulo se da la definición de un grupo-anillo y una grupo-álgebra, caso especial del anterior. Posteriormente, se establecen las condiciones necesarias y suficientes para que un grupo-anillo sea semisimple. 

En el tercer capítulo se estudia la teoría de representación de grupos y su relación con los módulos de los grupo-anillos. 

En el cuarto capítulo se estudian algunos elementos algebraicos de un grupo-anillo como los elementos nilpotentes, los idempotentes y las unidades de torsión.

En el quinto capítulo se da una breve introducción al estudio de las unidades de un grupo-anillo, mostrando algunas construcciones de unidades no triviales para los mismos.

Finalmente en el sexto capítulo se da una introducción a la teoría de códigos correctores, dando relevancia a los códigos cíclos y mostrando que dichos códigos tienen una fuerte conexión con las grupo-álgebras.




		
\chapter{OBJETIVOS}


% Respetar los espacios 
\vskip 0.34cm

\noindent\textbf{General}

Describir las características fundamentales de los grupo-anillos y su  relación con la teoría de representación de grupos.

\noindent\textbf{Específicos}
\vspace{-.4cm}
\begin{bulletList}

\item Identificar los elementos fundamentales que dan paso al estudio de los grupo-anillos.

\item Identificar las condiciones necesarias y suficientes para la semisimplicidad.

\item Mostrar ejemplos de unidades en los grupo-anillos en casos particulares.

\item Documentar algunas aplicaciones de los grupo-anillos en la teoría algebraica de códigos.

\end{bulletList}
     	
\chapter{INTRODUCCIÓN}

Introducción de la tesis (intr.tex)
  	
% Modificación para tabla de contenidos
\addtocontents{toc}{\protect\addvspace{1.8em}}


% -------------- Inicio del cuerpo de la tesis -----
\mainmatter
\addtocontents{toc}{\protect\addvspace{1.8em}}
\include{Chapters/chapter1}
\addtocontents{toc}{\protect\addvspace{1.8em}}
\include{Chapters/chapter2}
\addtocontents{toc}{\protect\addvspace{1.8em}}
\chapter{TEORÍA DE REPRESENTACIÓN DE GRUPOS}

\section{\hskip 1em Definición y Ejemplos}

Como se mencionó en el capítulo 1 \footnote{ponerlo ejemplos en esta parte de la definición de grupos} el concepto de \textbf{grupo de permutaciones} fue dado explícitamente por primera vez en las memorias de Galois en  1830, aunque la primera definición de grupo abstracto fue dado hasta en 1854 por Cayley, aunque pasó inadvertidamente por un tiempo, hasta que dicha definición fue dada nuevamente en repetidas ocasiones por varios matemáticos, a saber: Leopold Kronecker en 1870, Heinrich Martin Weber en 1882 y Ferdinand Georg Frobenius en 1887. De esa forma los grupos fueron considerados por mucho tiempo como objetos concretos antes de llegar a ser estudiados como estructuras algebraicas abstractas.

En este contexto histórico es natural hacer la pregunta: Dado un grupo abstracto ¿cómo se puede saber que grupo es? Es decir, ¿se puede decir cuando es un grupo de permutaciones, un grupo lineal o un grupo de transformaciones proyectivas (sólo por citar algunos ejemplos)?

En 1879, durante las lecturas de un coloquio matemático realizado en Evanston, Illinois, Felix Klein planteó la posibilidad de representar un grupo abstracto dado como un grupo de transformaciones lineales (véase \cite{bib:historia}).

Siguiendo estas ideas, Theodor Molien, Georg Frobenius, Issai Schur, William Burnside y  Heinrich Maschke desarrollaron la teoría básica de la representación de grupos al inicio del siglo XX y Burnside presentó la primera exposición  sistemática de este tema en su libro \cite{bib:burnside}, que actualmente es considerado un libro clásico en este tema. 

La teoría de la representación se volvió mas importante a medida que se fueron obteniendo nuevos resultados. Uno de los resultados mas importantes es el famoso teorema que establece que si $p$ y $q$ son números enteros primos y  $a$, $b$ enteros positivos, entonces cualquier grupo de orden $p^aq^b$ es soluble. Este teorema fue demostrado en 1904 por William Burnside usando la teoría de representación de grupos y, como dato curioso, la primera demostración que no utiliza dicha teoría fue proporcionada por John Griggs Thompson mas de 60 años después (ver \cite{bib:grupsfact}).

William Burnside también conjeturó que todo grupo de orden impar es soluble. Esta conjetura fue un problema abierto hasta que Walter Feit y John Thompson dieron una demostración de esta conjetura en 1963 (ver \cite{bib:solubilidad}), usando para ello teoría de la representación. Luego de hacer énfasis en la importancia histórica que tiene la teoría de representación de grupos, se entra a estudiar algunas definiciones de la misma.
\begin{definicion}
Sean $G$ un grupo, $R$ un anillo conmutativo y $V$ un $R\mbox{-módulo}$ libre de rango finito. Una \textbf{representación} de $G$ sobre $R$, con espacio de representación $V$, es un homomorfismo de grupos $T \colon G \to GL(V)$, donde $GL(V)$ es el grupo de automorfismo de $V$. El rango de $V$ es llamado \textbf{grado} de la representación $T$ y se denotará como $\grado(T)$.
\end{definicion}

Para $g \in G$ se denotará con $T_g \colon V \to V$ al automorfismo correspondiente bajo $T$. Así, si $g, h \in G$, se tiene que $T_{gh} = T_g \circ T_h$ y $T_1 = I$.

El caso en el que $R$ es un campo es de particular importancia. Históricamente, este fue el primer caso que se estudió y es en ese contexto donde se obtuvieron la mayor parte de resultados. 

Si se escoge una $R\mbox{-base}$ de $V$, se puede definir un isomorfismo $\phi$ de $GL(V)$ al grupo $GL(n,R)$ de matrices invertibles $n\times n$ con coeficientes en $R$, asignándole a cada automorfismo $T \in GL(V)$ su matriz respecto a la base dada. Esto da paso a la siguiente definición:

\begin{definicion}
Sea $G$ un grupo y $R$ un anillo conmutativo. Una representación matricial de $G$ sobre $R$ de grado $n$ es un homomorfismo de grupos $T \colon G \to GL(n,R)$.
\end{definicion} 

Si $T \colon G \to GL(V)$ es una representación de $G$ sobre $R$ con espacio de representación $V$ y se considera el isomorfismo $\phi \colon GL(V) \to GL(n,R)$ asociada a alguna $R-\mbox{base}$, entonces $\phi \circ T \colon G \to GL(n, R)$ es una representación matricial de $G$. De manera similar, dada una representación matricial $T \colon G \to GL(n,R)$, entonces $\phi^{-1}\circ T \colon G \to GL(V)$ es una representación de $G$ sobre $R$. Debido a este hecho, no se hará distinción entre representación y representación matricial.
\begin{ejemplo}
Dado un grupo $G$ y un anillo conmutativo $R$, la función $T \colon G \to GL(n,R)$ tal que a cada elemento $G$ le asocia la matriz identidad de $GL(n,R)$ es una representación matricial de $G$. A esta función  se le llama \textbf{representación trivial} de $G$ sobre $R$ de grado $n$. 
\end{ejemplo}
\begin{ejemplo}
Sea $G$ el grupo de Klein de cuatro elementos,es decir, $G = \{ 1, a, b, ab\}$. Este grupo tiene tres elementos de orden dos. Entonces $T \colon G \to GL(2, \mathds{Z})$ es la función tal que:
\begin{equation*} \mathsf{T(1)} = \begin{pmatrix}
1 & 0 \\
0 & 1
\end{pmatrix}, \quad \mathsf{T(a)} = \begin{pmatrix}
1 & 0 \\
0 & -1
\end{pmatrix} \end{equation*}
\begin{equation*} \mathsf{T(b)} = \begin{pmatrix}
-1 & 0 \\
0 & 2
\end{pmatrix}, \quad \mathsf{T(ab)} = \begin{pmatrix}
-1 & 0 \\
0 & -1
\end{pmatrix}. \end{equation*}
\end{ejemplo}
\begin{ejemplo}
Sea $S_n$ el grupo de simetrías de $n$ símbolos y $R$ un anillo conmutativo. Sea $V$ un $R-\mbox{módulo}$ libre de rango $n$ con base $\{v_1, v_2, \cdots, v_n\}$. Para facilitar la comprensión de este ejemplo, se sugiere al lector imaginar que $V = \underset{n}{\underbrace{\mathds{R} \oplus \cdots \oplus \mathds{R}}}$ con su base canónica. 

Por otra parte, considérese la función $f \colon S_n \to GL(V)$ de la siguiente manera: a cada elemento $\sigma 
\in S_n$, se le asigna la función $T_{\sigma} \in GL(V)$, que actúa, de manera natural, como:
\[T_{\sigma}(v_i) = v_{\sigma(i)}.\]
Como $T_{\sigma}$ deja a la base intacta (salvo permutaciones), es claro que $T_\sigma$ es un isomorfismo. 

 
Es claro que $T$ es un isomorfismo, por su definición, y por lo tanto una representación de $S_n$.

Como se puede apreciar una representación por si sóla puede ser poca descriptiva, por lo tanto se considera de mas utilidad conocer la representación matricial. Para este caso en particular, considérese $A(\sigma)$, la matriz asociada a $T_{\sigma}$, que se obtiene al calcular $T_{\sigma}(v_j)$ como combinación lineal de la base. Como $T_{\sigma} (v_j) = v_{\sigma(j)}$, entonces los coeficientes de la matriz anterior son cero en todas sus entradas excepto en $(\sigma(j),j)$, en la cual la entrada vale uno. De esta manera es fácil notar que $A(\sigma)$ es una matriz que tiene exactamente una entrada igual a uno en cada fila y columna y las demás iguales a cero. Dicha matriz se conoce como la \textbf{matriz de permutación}.
\end{ejemplo}
\begin{ejemplo}[La representación Regular]
Sea $G$ un grupo finito de orden $n$ y $R$ un anillo conmutativo. Se requiere definir una representación de $G$ sobre $R$, para ello se considerará  como espacio de representación a $RG$, es decir, a el grupo-anillo de $G$ sobre $R$. 

Considérese la función $T \colon G \to GL(RG)$ de la siguiente manera: a cada elemento $g \in G$ se le asigna la función lineal $T_g$ que transforma a los elementos de la base por medio de multiplicación por la izquierda, esto es, $T_g(g_i) = gg_i$. Es claro que $T$ es una representación de $G$, debido a que:
\begin{equation*} T_{gh}(y) = (gh)y = g(h(y)) = T_gT_h(y).  \end{equation*}

 
En este caso hay que recordar que $G$ es una base de $RG$ sobre $R$ y se pueden enumerar, en algún orden, los elementos de $G$ como sigue:
\begin{equation*} 
G = \{ 1=g_1, g_2, \cdots, g_n \}, 
\end{equation*}  por lo tanto es fácil notar que en la correspondiente representación matricial con respecto a la base $G$ de $RG$, la imagen de cualquier elemento $g \in G$ es una matriz de permutación, debido a la cerradura del producto en $G$. 
\end{ejemplo}
La representación anterior usualmente es llamada la \textbf{representación regular de $G$ sobre $R$ por la izquierda}. Para ilustrar de mejor manera a continuación se muestra un ejemplo:
\begin{ejemplo}
Sea $G = \{ 1,a,a^2 \}$ un grupo cíclico de orden tres. Enumérese los elementos de $G$ como $g_1 = 1$, $g_2 = a$, $g_3 = a^2$. Para encontrar la representación regular de $a$, basta con multiplicar por $a$ los elementos de $G$ por la izquierda:
\[ag_1 = g_2, \quad ag_2 = g_3, \quad ag_3 = g_1 
\]
entonces se tiene:
\[T_a(g_1) = g_2, \quad T_a(g_2) = g_3, \quad T_a(g_3) = g_1, \] 
por lo tanto la matriz asociada con $a$ en la base dada es:
\begin{equation*} \mathsf{\rho(a)} =  \begin{pmatrix}
0 & 0 & 1 \\
1 & 0 & 0 \\
0 & 1 & 0
\end{pmatrix},  \end{equation*} que no es más que una matriz de permutación.
\end{ejemplo}
\begin{ejemplo}
Considérese, de nuevo, el grupo de Klein de cuatro elementos, $G = \{ 1, a, b, ab\}$ con la numeración: $g_1 = 1$, $g_2 = a$, $g_3 = b$, $g_4 = ab$.

Para conocer la representación regular de $a$, se procede a multiplicar por  la izquierda por $a$ a los elementos de $G$:
\begin{equation*} ag_1 = g_2, \quad ag_2 = g_1, \quad ag_3 = g_4, \quad ag_4 = g_3, \end{equation*} entonces 
\begin{equation*} T_a(g_1) = g_2 , \quad T_a(g_2) = g_1, \quad T_a(g_3) = g_4, \quad T_a(g_4) = g_3 \end{equation*} y como en el ejemplo anterior, se puede obtener la representación matricial de $a$:
\begin{equation*}  \mathsf{\rho(a)} = \begin{pmatrix}
0 & 1 & 0 & 0 \\
1 & 0 & 0 & 0 \\
0 & 0 & 0 & 1 \\
0 & 0 & 1 & 0
\end{pmatrix} .\end{equation*}

De manera similar se obtiene la representación matricial de los elementos restantes de $G$:
\begin{equation*} \mathsf{\rho(b)} = \begin{pmatrix}
0 & 0 & 0 & 0\\
0 & 0 & 0 & 1\\
1 & 0 & 0 & 1\\
0 & 1 & 0 & 0
\end{pmatrix}, \quad \mathsf{\rho(ab)} = \begin{pmatrix}
0 & 0 & 0 & 1\\
0 & 0 & 1 & 0\\
0 & 1 & 0 & 0\\
1 & 0 & 0 & 0
\end{pmatrix} , \quad \mathsf{\rho(1)} = \begin{pmatrix}
1 & 0 & 0 & 0\\
0 & 1 & 0 & 0 \\
0 & 0 & 1 & 0\\
0 & 0 & 0 & 1
\end{pmatrix} \end{equation*}

\end{ejemplo}

\begin{nota}
Ya se mencionó que $\rho(g)$ con $g \in G$ es una matriz de permutación, pero es importante hacer notar que si se toma $ 1 \neq g \in G$, entonces para cualquier $g_i \in G$ se tiene que $ gg_i \neq g_i $. Esto implica que para cualquier elemento $g_i$ de la base se cumple que $T_g(g_i) \neq g_i$ y por ende los elementos de la diagonal de $\rho(g)$ son todos iguales a cero. Más aún, de lo anteriormente expuesto, se deduce que si $g \neq 1$ entonces $tr(\rho(g)) = 0$ si $g \neq 1$ y $tr(\rho(g)) = |G|$ si $g = 1$. Este resultado elemental es de mucha importancia cuando se está trabajando con la representación regular.
\end{nota}
\begin{ejemplo}\label{rciclica}(Algunas representaciones de grupos cíclicos)
Considérese el grupo cíclico $G = \{1,a, \cdots, a^{m-1} \}$ y sea $K$ un campo. Si se desea construir una representación matricial $A \colon G \to GL(n,K)$ es necesario escoger la matriz $A(a)$, ya que por ser $A$ un homomorfismo, las matrices de representación de los restantes elementos del grupo están determinadas por $A(a^r) = (A(a))^r $. Además para demostrar que $A$ es un homomorfismo de grupos, basta con probar que $(A(a))^r = I$, para algún $r \in \mathds{Z}$.

Supóngase que $\\car(K) \nmid m$ y que $K$ contiene una raíz primitiva de la unidad de orden $m$,  $\xi$. Entonces 
\begin{equation*} A \colon G \to GL(1,K) \end{equation*}  tal que, $A(a)  = \xi$ es una representación, ya que $(A(a))^r = \xi^r = 1$ para algún $r$.  Además, si $\{ \xi_1, \cdots, \xi_m \}$ es un conjunto de todas las raíces de la unidad unidad de orden $m$ que son distintas a pares entonces la función $B \colon G \to GL(m,K)$ dada por 
\begin{equation*} \mathsf{B(a)} = \begin{pmatrix}
\xi_1 &  &\dots &  0 \\
0 & \xi_2 & \dots & 0 \\
 & & \dots &  \\
 0 & 0 & \dots & \xi_m
\end{pmatrix} \end{equation*} es una representación de $G$ sobre $K$ de grado $m$, ya que $\xi_i^r = 1$ para algún $r \in \mathds{Z}$, entonces 
\begin{equation*} \mathsf{(B(a))^r} = \begin{pmatrix}
\xi_1^r &  &\dots &  0 \\
0 & \xi_2^r & \dots & 0 \\
 & & \dots &  \\
 0 & 0 & \dots & \xi_m^r
\end{pmatrix} = I. \end{equation*}

Nótese que esta representación es distinta a la representación regular, que en el caso de $a$, está dada por 
\begin{equation*} \mathsf{\Gamma}(a) = \begin{pmatrix}
0 & 0 & \dots & 0 & 1 \\
1 & 0 & \dots & 0 & 0\\
0 & 1 & \dots & 0 & 0\\
 &  &  \dots &  &  \\
 0 & 0 & \dots & 1 & 0
\end{pmatrix}. \end{equation*}
Además si $\car(K) \mid m$ entonces se propone la representación $C \colon G \to GL(2,K)$, dada por
\begin{equation*}
\mathsf{C(a)} = \begin{pmatrix}
1 & 1\\
0 & 1
\end{pmatrix} 
\end{equation*}
 ya que $\mathsf{(C(a))^r} = \begin{pmatrix}
1 & r\cdot 1\\
0 & 1
\end{pmatrix} = I$ para $r \in \mathds{Z}$, dado que $\\car(K) < \infty $.
\end{ejemplo} 
\begin{ejemplo}[Representación de $D_4$]\label{ejem:diedrico}
Considérese el grupo de simetrías de un cuadrado. Este grupo de 8 elementos, a saber, las reflexiones a través de los ejes $r_1, r_2, r_3, r_4$ (véase la Figura \ref{fig:ejemplo}) y las rotaciones con ángulos $\frac{\pi}{2}$, $\pi$, $\frac{3\pi}{2}$ y $2\pi$ alrededor del centro . 



\begin{figure}[t]
  \centering
  \caption{\hskip 2em \textbf{Forma gráfica del grupo $D_4$}}
  	  \vskip -0.59em
  	  \includegraphics{cuadrado}
	  \captionsetup[figure]{textfont = normal, labelformat=empty, labelsep=period}
	  \vskip 1.80em
  	  \caption*{Fuente: elaboración propia con programa para computadora geogebra.}
  \label{fig:ejemplo}
\end{figure}
Sea $a$ la rotación de ángulo $\frac{\pi}{2}$ y $b$ la reflexión a través del eje $r_2$. Es fácil ver, bajo consideraciones geométricas, que cualquier otro elemento de este grupo se puede obtener por medio de $a$ y $b$.

De manera mas abstracta, este grupo --que es llamado grupo diédrico de orden ocho y usualmente denotado por $D_4$ -- puede ser definido con dos generadores que satisfacen las relaciones
\begin{equation*}
a^4 = 1, \quad b^2 = 1 , \quad baba = 1.  
\end{equation*}
Por lo tanto este grupo puede ser descrito  como 
\begin{equation*}
D_4 = \{ 1, a, a^2, a^3, b, ab, a^2b, a^3b \}. 
\end{equation*}
Como todas los elementos de este grupo están en terminos de $a$ y $b$, entonces para encontrar una representación matricial $A \colon D_4 \to GL(n,K)$ sobre el campo $K$, será suficiente encontrar matrices $A(a)$, $B(b)$ tales que $A(a)^4 = I$, $A(b)^2 = I$, $A(b)A(a)A(b)A(a) = I.$

Es fácil determinar representaciones de grado uno para $D_4$ en un campo $K$ de característica diferente a dos, de la siguiente manera:
\begin{equation*}
\begin{aligned}
A(a) & = 1\\
B(a) & = 1 \\ 
C(a) & = -1 \\
D(a) & = -1
\end{aligned}
\qquad 
\begin{aligned}
A(b) & = 1\\
B(b) & =  -1 \\
C(b) & = 1 \\
D(b) & = -1.
\end{aligned}
\end{equation*}
Pensando en el significado geométrico de $a$ y $b$, como dos funciones del plano al plano, se puede calcular sus matrices con respecto a la base canónica para obtener otra representación matricial de $D_4$:
\begin{equation*}
\begin{aligned}
\mathsf{W(a)} = \begin{pmatrix}
0 & -1 \\
1 & 0
\end{pmatrix}
\end{aligned}
\qquad \text{,} \qquad
\begin{aligned}
\mathsf{W(b)} &= \begin{pmatrix}
 0 & 1 \\
 1 & 0
\end{pmatrix}. 
\end{aligned}
\end{equation*}
\end{ejemplo}

\begin{ejemplo}[Suma directa de representaciones]
Sean $T \colon G \to GL(V)$ y $S \colon G \to GL(W)$ dos representaciones de un grupo $G$ sobre un anillo conmutativo $R$. Se puede definir una nueva representación $V \oplus W$, que es llamada la \textbf{suma directa} de dos representaciones dadas y se denota como $T \oplus S$, de la siguiente manera: \begin{equation*} (T \oplus S)_g = T_g \oplus S_g, \quad \mbox{para cada } g \in G. \end{equation*}

Si se eligen bases $\{v_1, \dots, v_n\}$ y $\{ w_1, \dots, w_m \}$ de $V$ y $W$ respectivamente y se denota por $g \mapsto A(g)$ y $g \mapsto B(g)$ las correspondientes representaciones matriciales en las bases dadas, entonces la representación matricial asociada a $T \oplus S$ con respecto a la base $\{ (v_1,0), \dots , (v_n), (0,w_1), \dots, (0,w_n)  \}$ de $V \oplus W$, viene dada por
\begin{equation*} g \mapsto \begin{pmatrix}
A(g) & 0 \\
0 & B(g)
\end{pmatrix}. \end{equation*}
\end{ejemplo}
Los ejemplos anteriormente expuestos sirven de motivación para introducir algunos conceptos de teoría de la representación. En este trabajo se restringirán las representaciones al caso en el cual $R$ es un campo, debido a que con este caso se logra ilustrar la relación de teoría de representación con los problemas de grupo-anillos.

Primero considérese $T \colon G \to GL(V)$ una representación de un grupo $G$ sobre un campo $K$ y asúmase que $\phi \colon V \to W$ es un isomorfismo de espacios vectoriales sobre $K$. Entonces se puede definir una nueva representación $\overline{T} \colon G \to GL(W)$ por medio de $\overline{T_g} \colon \phi \circ T_g \circ \phi^{-1}$ para todo $g \in G$. Esto es, esencialmente, una copia de $T$. La relación entre estas dos representaciones está ilustrada en el diagrama de la figura \ref{fig:ejemploSumaRepresentaciones}, 
\begin{figure}[t]
  \centering
  \caption{\hskip 2em \textbf{Diagrama conmutativo para representaciones equivalentes}}
  	  \vskip -0.59em 
  	  \[\xymatrix { V \ar[r]^{T_g} 
  	  \ar[d]_{\phi}
  	   & V \ar[d]^{\phi} \\
  	  W \ar[r]_{\overline{T_g}} & W}\]
	  \captionsetup[figure]{textfont = normal, labelformat=empty, labelsep=period}
	  \vskip 0.20em
  	  \caption*{Fuente: elaboración propia con programa para computadora \textbf{xymatrix}.}
  \label{fig:ejemploSumaRepresentaciones}
\end{figure}
lo cual sugiere lo siguiente:
\begin{definicion}
Dos representaciones $T \colon G \to GL(V)$ y $\overline{T} \colon G \to GL(W)$ de un grupo $G$ sobre el mismo campo $K$ se dicen que son \textbf{equivalentes} si existe un isomorfismo $\phi \colon V \to W$ tal que $\overline{T_g} = \phi T_g \phi^{-1} $ para cualquier $g \in G$.
\end{definicion}
\begin{definicion}
Dos representaciones matriciales $A \colon G \to GL(n,K)$ y $B \colon G \to GL(n,K)$ de un grupo $G$ sobre un campo $K$ se dicen equivalentes si existe una matriz invertible $U \in GL(n,K)$ tal que $A(g) = UB(g)U^{-1}$ para cualquier $g \in G$.
\end{definicion}

\begin{ejemplo}\label{ejemciclico}
Sea $G$ un grupo cíclico de orden $m$ y $K$ un campo que contiene a $\{ \xi_1, \xi_2, \dots, \xi_m \}$, el conjunto de todas las raíces distintas de la unidad de orden $m$. Entonces, si se consideran las representaciones $B$ y $\Gamma$ dadas en el ejemplo \ref{rciclica} con 
\begin{equation*} \mathsf{U} = \begin{pmatrix}
\xi_1 & \xi_1^2 & \cdots & \xi_1^m \\
\xi_2 & \xi_2^2 & \cdots & \xi_2^m \\
 & & \cdots & \\
 \xi_m & \xi_m^2 & \cdots & \xi_m^m \\
 
\end{pmatrix}, \quad \mathsf{U} \in GL(n,K)\footnote{Esto es evidente, ya que $\mathsf{U}$ es una matriz de Vandermonde con $det(\mathsf{U}) = \prod_{1 \leq i < j \leq m}(\xi_i - \xi_j) \neq 0$.} \end{equation*} entonces, calculando por un lado se tiene 
\begin{equation*} \mathsf{B(a)U} = \begin{pmatrix}
\xi_1 & 0 & \cdots & 0\\
0 & \xi_2 & \cdots & 0\\
 & & \cdots & \\
 0 & 0 & \cdots & \xi_m
\end{pmatrix} \begin{pmatrix}
\xi_1 & \xi_1^2 & \cdots & \xi_1^m \\
\xi_2 & \xi_2^2 & \cdots & \xi_2^m \\
 & & \cdots & \\
 \xi_m & \xi_m^2 & \cdots & \xi_m^m
\end{pmatrix} = \begin{pmatrix}
\xi_1^2 & \xi_1^3 & \cdots & \xi_1 \\
\xi_2^2 & \xi_2^3 & \cdots & \xi_2 \\
 & & \cdots & \\
\xi_m^2 & \xi_m^2 & \cdots & \xi_m
\end{pmatrix}, \end{equation*} similarmente \begin{equation*} 
\mathsf{U\Gamma(a) } = 
\begin{pmatrix}
\xi_1 & \xi_1^2 & \cdots & \xi_1^m \\
\xi_2 & \xi_2^2 & \cdots & \xi_2^m \\
 & & \cdots & \\
 \xi_m & \xi_m^2 & \cdots & \xi_m^m \\
\end{pmatrix}
\begin{pmatrix}
0 & 0 & \cdots & 1 \\
1 & 0 & \cdots & 0 \\
 & & \cdots & \\
0 & 0 & \cdots & 1 \\
\end{pmatrix} = \begin{pmatrix}
\xi_1^2 & \xi_1^3 & \cdots & \xi_1 \\
\xi_2^2 & \xi_2^3 & \cdots & \xi_2 \\
 & & \cdots & \\
\xi_m^2 & \xi_m^2 & \cdots & \xi_m
\end{pmatrix}
\end{equation*} con lo que se ha demostrado que $\mathsf{A(g)} = \mathsf{UB(g)U^{-1}}$, para cualquier $g \in G$ y concluye que $\mathsf{B}$ y $\mathsf{\Gamma}$ son equivalentes. 
\end{ejemplo}

Considérese $T \colon G \to GL(V)$ una representación de un grupo $G$ sobre el campo $K$, con espacio de representación $V$ y supóngase que $V$ contiene un subespacio $W$ que es invariable bajo $T$, esto es, un subespacio tal que $T_g(W) \subset W$, para cualquier $g \in G$. Entonces se puede considerar el homomorfismo de grupos que asigna a cada elemento $g \in G$ la restricción de $T_g$ al subespacio $W$. Por ser $T_g$ la restricción, es claro que el homomorfismo anterior es una representación de $G$ sobre $K$, con espacio de representación $W$.

Con el afán de dar una representación matricial de este hecho, considérese una base $\{ w_1, w_2, \dots, w_t \}$ de $W$ y extiéndase a una base $ \{ w_1, \cdots, w_t, v_{t+1}, \cdots, v_n \}$ de $V$. Entonces la matriz asociada a cada función $T_g$, $g \in G$ con respecto a esa base es de la forma
\begin{equation*} \begin{pmatrix}
\mathsf{A(g)} & \mathsf{B(g)} \\
0 & \mathsf{C(g)}
\end{pmatrix} \end{equation*} donde $\mathsf{A(g)} \in GL(t,K), C(g) \in GL(n-t,K)$ y $\mathsf{B(g)}$ es una matriz de $t \times (n-t)$. Estas consideraciones sugieren lo siguiente:
\begin{definicion}
Una representación $T \colon G \to GL(V)$ de un grupo $G$ sobre un campo $K$ es llamada \textbf{irreducible} si los únicos subespacios propios de $V$ que son invariantes bajo $T$ son los triviales, es decir, $V$ y $\{ 0 \}$.
\end{definicion}

La representación es llamada \textbf{reducible} si $V$ contiene  subespacios no triviales que son invariantes bajo $T$. 

\begin{definicion}
Una representación matricial $M \colon G \to GL(n,K)$ es llamada reducible si existe una matriz $\mathsf{U} \in GL(n,K)$ tal que para cualquier $g \in G$, se tiene que la matriz $\mathsf{U}M(g)U^{-1}$ es de la forma
\begin{equation*}
\mathsf{UM(g)U^{-1}} = \begin{pmatrix}
\mathsf{A(g)} & \mathsf{B(g)} \\
0 & C(g)
\end{pmatrix}. 
\end{equation*}  
\end{definicion}

El ejemplo \ref{ejemciclico} muestra que la representación regular de un grupo cíclico de orden $m$, sobre un campo $K$ que contiene raíces de la unidad de orden $m$ es reducible. De hecho, cualquier representación regular de un grupo finito $G \neq \{ 1 \}$ sobre cualquier campo es reducible. En efecto, nótese que si en el espacio de representación $RG$ se toma el elemento $\hat{G} = \sum_{g \in G}g$ entonces $ T_g(\hat{G}) = \hat{G}$ por lo tanto el subespacio generado por $\hat{G}$ es invariante bajo $T$ y $(\hat{G}) \neq RG.$  

\begin{definicion}
Una representación $T \colon G \to GL(V)$ de un grupo $G$ sobre un campo $K$ es llamado completamente irreducible si para todo subespacio $W$ que es invariante bajo $T$ existe un subespacio invariante $W'$ tal que $V = W \oplus W'$.
\end{definicion}
Para entender de mejor manera esta definición se dará una interpretación en términos de matrices.

Sea $\{ w_1, w_2, \dots, w_t \}$ y $\{ w_{t+1}, \dots, w_n\}$ bases dadas para $W$ y $W'$ respectivamente, entonces $\{ w_1, w_t, w_{t+1}, \dots, w_n \}$ es una base de $V$ y para cualquier $g \in G$ la matriz de $T_g$ con respecto a esta base es de la forma
\begin{equation*} \begin{pmatrix}
\mathsf{A(g)} & 0 \\
0 & \mathsf{B(g)}
\end{pmatrix} \end{equation*} donde $\mathsf{A(g)}$ y $\mathsf{B(g)}$ son las matrices de representación de $T_g$ en $W$ y $W'$ con respecto a las bases dadas. 

\begin{definicion}
Una representación matricial $M \colon G \to GL(n,K)$ es llamada completamente reducible si cualquier representación matricial $M$ de la forma 
\begin{equation*} \begin{pmatrix}
\mathsf{A(g)} & \mathsf{B(g)} \\
0 & \mathsf{C(g)}
\end{pmatrix} \end{equation*} es equivalente a una representación matricial de la forma
\begin{equation*} \begin{pmatrix}
\mathsf{A(g)} & 0 \\
0 & \mathsf{D(g)} 
\end{pmatrix}. \end{equation*}
\end{definicion}


\section{\hskip 1em Representación y Módulos.}
 En este sección se estudiará la conexión que hay entre módulos y representaciones. Dicha conexión se establece usando el concepto de grupo-anillo.
 
 \begin{proposicion}\label{prop:biyeccionModulos}
 Sea $G$ un grupo y $R$ un anillo conmutativo con unidad. Entonces, existe una biyección entre representaciones de $G$ sobre $R$ y $RG\mbox{-módulos}$ libres y de rango finito. 
 \end{proposicion}
 \begin{proof*}
 Dada una representación $T \colon G \to GL(V)$ de $G$ sobre $R$, se asocia a ella el $RG\mbox{-módulo}$ construido a partir de $V$ manteniendo la misma estructura aditiva y definiendo el producto de un elemento $v \in V$ por un escalar $\alpha = \srg{g}{G}{a} \in RG$ como 
 \begin{equation}\label{defproducto}
 \alpha v = \left( \srg{g}{G}{a}  \right)v = \sum_{g \in G}a(g)T_g(v).
 \end{equation}
 
 Usando está definición de producto se verifica:
 \begin{bulletList}
 \newItem Distributividad de la suma de escalares respecto al producto por escalar \begin{eqnarray*}
  (\alpha + \beta)v &=& \left( \sum_{g \in G}(a_g + b_g)g  \right)v \\
   &=& \sum_{g \in G}(a_g + b_g)T_g(v) \\
   &=& \sum_{g \in G}a_gT_g(v) + \sum_{g \in G}b_gT_g(v) \\
    &=& \alpha v + \beta v .
 \end{eqnarray*}
 \newItem Distributividad de la  suma de elementos del módulo respecto al producto por escalar
 \begin{eqnarray*}
 \alpha(v + w) &=& \left(\srg{g}{G}{a}(v + w)  \right) \\
  &=& \sum_{g \in G}a_gT_g(v+w) \\
  &=& \sum_{g \in G}a_gT_g(v) + \sum_{g \in G}a_gT_g(w) \\
  &=& \alpha v + \alpha w .
 \end{eqnarray*}
 \newItem Para la asociatividad, por un lado se tiene
 \begin{eqnarray*}
 \alpha(\beta v) &=& \left( \sum_{h \in G}a(g)h \right)\left( \sum_{g \in G}b(g)T_g(v) \right) \\
  &=& \sum_{h \in G}a(h)T_h\left( \sum_{g \in G}b(g)T_g(v)  \right) \\
  &=& \sum_{h, g \in G}a(h)b(g)T_{hg}(v).
 \end{eqnarray*}
 
 Por otro lado se tiene
 \begin{eqnarray*}
 (\alpha\beta)v &=& \left( \sum_{h,g \in G}a(h)b(g)hg  \right)(v)\\
  &=& \sum_{h,g \in G}a(h)b(g)T_{hg}(v)
 \end{eqnarray*}
 con lo que se comprueba que $\alpha(\beta v) = (\alpha\beta)v$.
 \newItem Considérese $\alpha = 1_{G}$, entonces
 \begin{eqnarray*}
 \alpha v &=& T_{1_G}(v)\\
  &=& I(v) \\
  &=& v.
 \end{eqnarray*}
 \end{bulletList}
  Por lo expuesto anteriormente es fácil notar que la multiplicación por escalar definida en la ecuación \ref{defproducto} induce un $RG\mbox{-módulo}$.
 
 Al converso, si $M$ es un $RG\mbox{-módulo}$ de rango finito sobre $R$, se define la representación de $G$ sobre $R$ asignando a cada elemento $g \in G$ el $R\mbox{-automorfismo}$ $T_g \colon M \to M$ dado por $T_g(m) = gm$.
 
 Nótese que dado $T \colon G \to GL(V)$ una representación de $G$ sobre $R$ y $M$ su $RG\mbox{-módulo}$ inducido, se tiene que $S$, la representación inducida por $M$, es tal que $S_g(m) = gm = \alpha m \simeq T_g(m)$, donde $\alpha$ es la imagen de la inmersión de $G$ en $RG$ dada en el teorema \ref{inmersion}.
 
 De manera similar, dado $M$ un $RG\mbox{-módulo}$ y $S \colon G \to GL(M)$ su representación inducida, entonces su $RG\mbox{-módulo}$ inducido por la ecuación \ref{defproducto} deja invariante a $M$. Por lo tanto se ha demostrado que las aplicaciones construidas anteriormente son inversas la una de la otra. 
 \end{proof*}
  Como ejemplo considérese  un grupo finito $G$ y $RG$ como un módulo sobre sí mismo, de esta forma $RG$ es de rango finito $|G|$ sobre $R$. Entonces, dado un elemento $x \in G$, la representación $T_x \colon RG \to RG$ viene dada por: \begin{equation*}T_{x}\left( \sum_{g \in G}a(g)g\right) = x\left( \sum_{g \in G}a(g)g\right) = \left( \sum_{g \in G}a(g)gxg\right).\end{equation*}
  
  Esto significa que $x \in G$ actúa en los elementos de la base $G = \{ g_1, \dots, g_n \}$ multiplicándolos por la izquierda. En otras palabras, la representación asociada al $RG\mbox{-módulo}$ $RG$ es precisamente la representación regular de $G$. 
\begin{lema}
  Sea $T \colon G \to GL(V)$ una representación de un grupo $G$ sobre un campo $K$, con espacio de representación $V$, entonces un subespacio $W \subset V$ es invariante bajo $T$ si y sólo si $W$ es un $KG\mbox{-módulo}$ de $V$. 
  \end{lema}
  \begin{proof*}
  Se procede a demostrar este hecho en dos partes:
  \begin{bulletList}
  \newItem Sea $W \subset V$ invariante bajo $T$, entonces $T_{g}(W) = W$ para cualquier $g \in G$. Sean $w_1, w_2 \in W$ se tiene que $T_{g}(w_1 + w_2) = T_{g}(w_1) + T_{g}(w_2) \in W$, así $T_{g}^{-1}(T_{g}(w_1 + w_2)) \in W$. Por otra parte, si se considera $\alpha = \sum_{g \in G}a(g)g$ entonces $\alpha w = \sum_{g \in G}a(g)T_g(w) \in W$ y por lo tanto $W$ es un $KG\mbox{-módulo}$ de $V$.
  \newItem Sea $W$ subespacio de $V$, $W \subset V$ y $W$ un $KG\mbox{-módulo}$ de $V$, entonces para $w \in W$ y $g \in RG$ se tiene $gw = 1\cdot T_g(w) \in W$. \qedhere
  \end{bulletList} 
  \end{proof*}
  
  \begin{teorema}\label{teo:relacionTG}
  Sea $G$ un grupo y $K$ un campo. Entonces:
  \begin{bulletList}
  \newItem Dos representaciones $T$ y $T'$ de $G$ sobre $R$ son equivalentes si y sólo si los correspondientes $RG\mbox{-módulos}$ son isomorfos.
  \newItem Una representación es irreducible (o completamente reducible) si y sólo si el correspondiente $RG\mbox{-módulo}$ es irreducible (o completamente reducible).
  \end{bulletList}
  \end{teorema}
  \begin{proof*}
  Se procede a demostrar este lema por incisos:
  \begin{bulletList}
  \newItem Supóngase que $T$ y $T'$ son representaciones de $G$ sobre $K$ equivalentes, entonces existe $\phi \colon V \to W$ isomorfismo, donde $V$ y $W$ son los espacios de representación de $T$ y $T'$ respectivamente, tal que $T'_g = \phi T_g \phi^{-1}$ para cualquier $g \in G$. Entonces se probará que $\phi$ es isomorfismo de $RG\mbox{-módulos}$ también.  En efecto
  \begin{eqnarray*}
   \phi(\alpha v) &=& \phi\left( \sum_{g \in G}a(g)T_g(v) \right) \\ 
   &=&  \sum_{g \in G}a(g)\phi(T_g(v)) \\
    & =& \sum_{g \in G}a(g)T'_g(\phi(v))\\ 
    &=& \alpha \phi(v).
  \end{eqnarray*}
   \newItem Supóngase que $M$ y $N$ son $RG\mbox{-módulos}$ isomorfos, entonces existe $f \colon M \to N$ isomorfismo de $RG\mbox{-módulos}$. Sean $T$ y $T'$ las representaciones inducidas por $M$ y $N$ respectivamente, entonces:
   \begin{eqnarray*}
   (fT_gf^{-1})(n) &=& 
    f\left(T_g\left(f^{-1}(n)\right)\right)\\
    &=& f\left(gf^{-1}(n)\right) \\
    &=& gf\left(f^{-1}(n)\right) \\
    &=& gn \\
    &=& T'_g(n)
   \end{eqnarray*} con lo que se comprueba que $T$ y $T'$ son equivalentes. 
   
   \newItem Si una representación $T$ es irreducible, entonces los únicos subespacios de $V$ que son invariantes bajo $T$ son los triviales y ,por el lema anterior, los únicos submódulos de $M$, el módulo inducido por $T$, son los los triviales. De manera análoga se puede demostrar el converso. \qedhere
   \end{bulletList}
  \end{proof*}
  También es posible notar que si un $RG\mbox{-módulo}$ $M$ admite una descomposición como suma directa de submódulos $M = \oplus_{i = 1}^{t}M_i$ y si $T$ y $T_i$ denota las representaciones correspondientes a estos módulos, entonces $T = \oplus_{i = 1}^tT_i$. 
 
 En lo que sigue, se mostrará como la información que ya se conoce a cerca de los grupo-anillos se puede trasladar a términos de representaciones de grupos. 
 
 El lector deberá recordar que en el corolario \ref{cor:car}, como consecuencia directa del teorema de Maschke, se demostró que si $K$ es un campo tal que $\car(K) \nmid |G|$, entonces $KG$ es un anillo semisimple. Además, se demostró en el teorema \ref{teo:wa} que en este caso todo $KG\mbox{-módulo}$ es simple. Por lo tanto, en particular, se sigue inmediatamente que todo $KG\mbox{-módulo}$ finito dimensional sobre $K$ se puede escribir como suma directa de módulos irreducibles.
 
 En términos de representaciones, esto significa que bajo estas condiciones, toda representación de $G$ sobre $K$ es la suma directa de representaciones irreducibles. Así, para determinar todas las representaciones de $G$ sobre $K$, mediante equivalencia, es suficiente determinar todos los $KG\mbox{módulos}$ irreducibles, salvo isomorfismos. 
 
 Ahora, es necesario hacer uso del teorema de Wedderburn-Artin aplicado a grupo-anillos (teorema \ref{teo:wa}), el cual establece que el número de $KG\mbox{-módulos}$ irreducibles que no son isomorfos entre sí, es precisamente el número de componentes simples de $KG$ y estas están determinadas exclusivamente por la estructura de $KG$. En particular, es importante recordar que si se escribe $KG$ en la forma
 \begin{equation*} KG \simeq \oplus_{i = 1}^{r}M_{n_i}(D_i) \end{equation*} donde $D_i, 1\leq i \leq r$, son anillos de división que contienen a $K$ en sus centros, y si se calcula la dimensión en ambos lados de la ecuación, se tiene
 \begin{equation*} |G| = \sum_{i = 1}^{r}n_{i}^2[D_i:K]. \end{equation*}
 
 Por otro lado, se sabe que el módulo irreducible $I_i$ correspondiente a la componente simple $M_{n_i}(D_i)$ es isomorfo a $D_{i}^{n_i}$. Como el grado de la correspondiente representación $T_i$ viene dado por la dimensión de este módulo sobre $K$, se obtiene que
 \begin{equation*}  \grado (T_i) = [D_{i}^{n_i}:K] = n_i[D_i:K] \end{equation*} así, se puede escribir \begin{equation*} |G| = \sum_{i = 1}^{r}n_i\grado(T_i). \end{equation*} 
 
 \begin{ejemplo}
 Se mostró  en el ejemplo \ref{ejem:orden7} que si $G = <a>$ denota al grupo cíclico de orden siete, entonces
 \begin{equation*} \mathds{Q}G \simeq \mathds{Q}\oplus\mathds{Q}(\zeta), \end{equation*} donde $\zeta$ denota una raíz primitiva de la unidad de orden siete. De lo anterior, las componentes simples de $\mathds{Q}G$ son anillos de matrices de $1\times 1$ sobre los anillos $\mathds{Q}$ y $\mathds{Q}(\zeta)$ respectivamente y por ende existen solamente dos representaciones irreducibles que no son equivalentes, $S$ y $T$ de $G$ sobre $\mathds{Q}$, con grados
 \begin{equation*} \grado(S) = [\mathds{Q}:\mathds{Q}] = 1, \quad \grado(T) = [\mathds{Q}:\mathds{Q}] = 6. \end{equation*} 
 
 Como las representaciones $1\mbox{-dimensionales}$ son equivalentes si y sólo si son iguales y como cualquier grupo admite la representación trivial $S \colon G \to GL(1,\mathds{Q})$ dada por $S_g = 1$, para cada $g \in G$, entonces la representación $1\mbox{-dimensional}$ de $G$ sobre $\mathds{Q}$ es la trivial.
 
 Para determinar $T_a$, de acuerdo a las consideraciones anteriores, se debe considerar el $\mathds{Q}G\mbox{-módulo}$ irreducible $I_2 = D_2^{n_2}$ correspondiente a la segunda componente simple de $\mathds{Q}$. Entonces, la representación $T \colon G \to GL(I_2)$ está dada por $T_a(v) = av$, para cada $v \in I_2$. En este caso, $n_2 = 1$ y $D_2 = \mathds{Q}(\zeta)$, así que $I_2 = \mathds{Q}(\zeta)$, donde la multiplicación por un elemento $\alpha = (\alpha_1, \alpha_2) \in \mathds{Q}G$ está dada por $\alpha v = \alpha_2v$, para todo $v \in \mathds{Q}(\zeta)$. Recordando que el elemento $a \in \mathds{Q}(\zeta)$ le corresponde, vía isomorfismo, el elemento $(1,\zeta) \in \mathds{Q}\oplus\mathds{Q}(\zeta)$ se tiene
 \begin{equation*} T_a(v) = av = \zeta v, \quad v 
 \in \mathds{Q}(\zeta). \end{equation*} Por lo tanto, si se toma $\{ 1, \zeta, \zeta^2, \dots, \zeta^5 \}$ como una base de $\mathds{Q}(\zeta)$ sobre $\mathds{Q}$, entonces la correspondiente matriz está dada por
 \begin{equation*} \mathsf{A(a)} = \begin{pmatrix}
 0 & 0 & 0 & 0 & 0 & -1 \\
 1 & 0 & 0 & 0 & 0 & -1 \\
 0 & 1 & 0 & 0 & 0 & -1 \\
 0 & 0 & 1 & 0 & 0 & -1 \\
 0 & 0 & 0 & 1 & 0 & -1 \\
 0 & 0 & 0 & 0 & 1 & -1
 \end{pmatrix}. \end{equation*}
 \end{ejemplo}
 \begin{ejemplo}[Representaciones del grupo diédrico de orden ocho.]
 Se ha probado en ele ejemplo \ref{ejem:diedrico} que el grupo $D_4$ admite cuatro representaciones distintas de grado uno y una representación $W$ de grado dos sobre $\mathds{Q}$, por lo tanto existen cuatro componentes simples isomorfas a $\mathds{Q}$. Sean $M_n(D)$ la componente simple correspondiente a la representación de grado dos. Como $2 = \grado(W) = n[D:\mathds{Q}]$, entonces $n = 1$ y $[D:\mathds{Q}]$ o $n = 2$ y $[D:\mathds{Q}] = 1$.
 
 Para el primer caso, se puede observar que $\mathds{Q}D_4$ debe ser de la forma \begin{equation*} \mathds{Q}D_4 \simeq \mathds{Q} \oplus \mathds{Q} \oplus \mathds{Q}\oplus\mathds{Q}\oplus D\oplus D',\end{equation*} donde $D'$ es una anillo de división con $[D':\mathds{Q}] = 2$. Es fácil notar que un anillo de división de dimensión dos sobre un campo tiene que ser conmutativo, entonces $\mathds{Q}D_4$ es conmutativo, lo cual es una contradicción, ya que $D_4$ no es abeliano. 
 
 En consecuencia, se debe tener que $n = 2$ y $D = \mathds{Q}$. De esta forma \begin{equation*}  \mathds{Q}D_4 \simeq \mathds{Q}\oplus\mathds{Q}\oplus\mathds{Q}\oplus\mathds{Q}\oplus M_{2}(\mathds{Q}). \end{equation*}
 \end{ejemplo}
 


\addtocontents{toc}{\protect\addvspace{1.8em}}
\chapter{ELEMENTOS ALGEBRAICOS}

\section{\quad Generalidades y definiciones}
En este capítulo será de especial interés estudiar algunos elementos algebraicos en grupo-álgebras usando la representación regular que puede ser definida para un álgebra finito dimensional con unidad sobre un campo $K$ de la siguiente manera.

\begin{definicion}
Sea $T \colon A \to \hom_K(A,A)$ tal que $a \mapsto T_a \in \hom_K(A,A)$, definida mediante multiplicación por la izquierda por $a$. Esto es, $T_a$ es una aplicación tal que $T_a(x) = ax$, para cualquier $x \in A$.  
\end{definicion}

Se puede observar a partir de la definición que
\begin{eqnarray*}
T_{a+b} &=& T_a + T_b \\
T_{ab} &=& T_aT_b \\
T_{ka} &=& kT_a
\end{eqnarray*}
para todo $a,b \in A,$  $k\in K$. Mas aún, la aplicación $a \mapsto T_a$ es inyectiva debido a que $T_a(1) = a$. Eligiendo una base $\{ a_1, \dots, a_n \}$ de $A$ sobre $K$ se puede representar a $T_a$ con una matriz $\rho(a)$, con lo que  se obtiene la representación matricial:
\begin{equation*}
a \mapsto \rho(a) \in M_n(K).
\end{equation*}
Si $a$ es un elemento algebraico de $A$, esto es, si existe un polinomio no nulo $f(X) \in K[X]$ tal que $f(a) = 0$, entonces los valores propios de la matriz $\rho(a)$ también anulan a $f(X)$, debido al teorema de Cayley-Hamilton (véase \cite[241]{bib:lang}). 

De esta manera, por ejemplo, si $a$ es un elemento nilpotente entonces los valores propios de $\rho(a)$ son todos cero. Si $a$ es de orden multiplicativo finito, es decir, si $a^m = 1$ para algún $m$ entero positivo, entonces los valores propios de $\rho(a)$ son raíces de la unidad de orden $m$. 
\begin{lema}\label{lem:traza}
Sea $G$ un grupo finito y $K$ un campo. Sea $p$ la representación regular de $KG$ y $\gamma = \sum_{g \in G}\gamma(g)g \in KG$. Entonces la traza de $\rho(\gamma)$ viene dada por
\begin{equation*}
\tr\rho(\gamma) = |G|\gamma(1).
\end{equation*}
\end{lema} 
\begin{proof*}
Se sabe que $\tr\rho(\gamma)$ es independiente de la base elegida, así que se elige $G=\{g_1, \dots, g_n \}$ como $K$-base para $KG$ y se asume que $g_1=1$.
Entonces
\begin{equation*}
\rho(\gamma) = \rho\left( \sum_{g \in G}\gamma(g)g \right) = \sum_{g \in G}\gamma(g)\rho(g) .
\end{equation*}
Para un elemento $ 1 \neq g  \in G$, se tiene $gg_i \neq g_i$, para $1\leq i\leq n$, de donde se sigue que los elementos de la diagonal de la matriz $\rho(g)$ son todos nulos si $g \neq 1$. Así $\tr\rho(g) = 0$ si $g \neq 1$. Más aún, como $\rho(1)$ es la matriz identidad, se tiene que $\tr\rho(1) = n$. Entonces
\begin{equation*}
\tr\rho(\gamma) = \sum_{g \in G}\gamma(g)\tr(g) = \gamma(1)\tr\rho(1) = \gamma(1)|G|.\qedhere
\end{equation*} 
\end{proof*}

\begin{lema}\label{lem:BH}
Sea $\gamma = \sum_{g \in G}\gamma(g)g$ una unidad de orden finito en el grupo-anillo entero  $\mathds{Z}G$ con $G$ un grupo finito y asúmase que $\gamma(1)\neq 0$. Entonces $\gamma = \gamma(1) = \pm 1.$
\end{lema}
\begin{proof*}
Sea $|G| = n$ y supóngase que $\gamma^{m} = 1$ para algún entero positivo $m$. Si se considera la representación regular $\rho$ del grupo-álgebra $\mathds{C}G$ y a $\mathds{Z}G$ como un subanillo de la misma, se tiene que $\tr\rho(\gamma) = n\gamma(1)$.
Como $\gamma^{m} = 1$, entonces $\left( \rho(\gamma) \right)^m = \rho(\gamma^{m}) = I$, de esto se sigue que $\rho(\gamma)$ es raíz del polinomio $X^m-1$, cuyas raíces son todas distintas. Esto implica, por el teorema espectral (véase \cite[214]{bib:lang}), que existe una base de $\mathds{C}G$ donde la matriz de $\rho(\gamma)$ es diagonal de la forma
\begin{equation*}
\mathds{A} = \begin{pmatrix}
\xi_1 & & & &\\
 & \xi_2 & & \\
 & & \ddots & \\
  & & & \xi_n
\end{pmatrix}, \quad \xi_i^m = 1.
\end{equation*}
 Entonces $\tr\rho(\gamma) = \sum\limits_{i = 1}^{n}\xi_i$ y así
 \begin{equation*}
 n\gamma(1) = \sum\limits_{i =1}^{n}\xi_i.
 \end{equation*}
 
 Por lo tanto, aplicando valor absoluto,
 \begin{equation*}
  |n\gamma(1)| = \left| \sum_{i = 1}^{n}\xi_i \right| \leq \sum_{i = 1}^{n}|\xi| = n.
 \end{equation*}
 Como $|n\gamma(1)| = n|\gamma(1)| \leq n$, entonces $|\gamma(1)| = 1$ y $\left| \sum_{i = 1}^{n}\xi_i \right| = \sum\limits_{i = 1}^{n}| \xi_i|$, lo cual sucede si y sólo si $\xi_1=\xi_2=\cdots = \xi_n$.
 
 Así $n\gamma(1) = n\xi_1$ y $\gamma(1) = \xi_1=\pm 1$. Se concluye que $\rho(\gamma) = \pm I$, de donde, $\gamma = \pm 1$.
\end{proof*}
\begin{corolario}
Supóngase que $\gamma = \sum_{g \in G}\gamma(g)g$ es una unidad central en el grupo-anillo entero $\mathds{Z}G$ con $G$ un grupo finito de orden finito. Entonces $\gamma$ es de la forma $\gamma = \pm g$ con $g \in  \mathcal{Z}(G)$.
\end{corolario}
\begin{proof*}
Sea $\gamma = \sum_{g \in G}\gamma(g)g$ una unidad central de orden $m$. Supóngase que $\gamma(g_0) \neq 0$, para algún $g_0 \in G$. Entonces $\gamma g_0^{-1}$ es también una unidad de orden finito en $\mathds{Z}G$. Mas aún, el coeficiente de $1$ en la expresión de $\gamma g_0^{-1}$ es $\gamma(g_0) \neq 0$, de donde $\gamma g_0^{-1} = \pm 1$ y por lo tanto $\gamma = \pm g_0$.
\end{proof*}
Una consecuencia inmediata del corolario anterior es el siguiente
\begin{teorema}\label{teo:Graham-Higman}
Sea $A$ un grupo abeliano finito. Entonces, el grupo de torsión de las unidades del grupo-anillo entero $\mathds{Z}A$ es igual $\pm A$.
\end{teorema}

Ahora se desea hacer un estudio de los elementos idempotentes. Es evidente que en cualquier anillo con unidad el $0$ y el $1$ son elementos idempotentes, estos elementos son llamados \textbf{idempotentes triviales} de un anillo. Se verá a continuación que los elementos idempotentes $e$ en un grupo-álgebra están fuertemente influenciados por su primer coeficiente $e(1)$.

\begin{teorema}\label{teo:idme}
Sea $G$ un grupo finito y $K$ un campo de característica cero. Supóngase que $e \in KG$ y $e$ es idempotente. Entonces:
\begin{bulletList}
\newItem $e(1) \in \mathds{Q}$.
\newItem $0 \leq e(1) \leq 1$.
\newItem $e(1) = 0 \Leftrightarrow e = 0$ y $e(1) = 1 \Leftrightarrow e=1$.   
\end{bulletList} 
\end{teorema}
\begin{proof*}
Considérese la representación regular de $KG$ escrita con respecto a la base $G$ de $KG$. Entonces, por el lema \ref{lem:traza}, se tiene que $\tr\rho = |G| e(1)$. Más aún, como $e^2 = e$, $\rho(e)$ satisface el polinomio $X^2 - X = X(X-1)$ y por lo tanto $\rho(e)$ puede ser diagonalizada. Los valores propios de $\rho(e)$ son $0$ o $1$ ya que $\rho(e)$ es idempotente. Debido a que la traza es la suma de los donde valores propios, se tiene \ que  $\tr\rho(e) = r$,  donde  $r$  es el número de valores propios iguales a $1$ y por lo tanto también es el  rango de $\rho(e)$. Por lo expuesto anteriormente se puede afirmar que $0\leq e(1) \leq 1$.

Nótese que $e(1) = 0 $ si y sólo si el rango de $\rho(e)$ es $0$ y eso pasa sólo si $e = 0$. Similarmente $e(1) = 1$ si y sólo si el rango de $\rho(e)$ es $\mid G \mid $, lo cual pasa sólo si $\rho(e)$ es la matriz identidad, es decir, si $e = 1$. 
\end{proof*} 
\section{\hskip 1em  Elementos idempotentes}
Se ha demostrado en el teorema \ref{teo:idme} que si $K$ es un campo de característica cero y $G$ es un grupo finito, entonces cualquier elemento idempotente $e \in KG$ cumple que $e(1) \in \mathds{Q}$. Se dará, en esta sección, un resultado análogo a este resultado, donde $K$ tiene característica $p > 0$.

\begin{teorema}\label{teo:genidem}
Sea $K$ un campo de característica $p > 0$ y sea $G$ cualquier grupo. Supóngase que $e \in KG$ es un idempotente. Entonces $e(1) \in F_p$, donde $F_p$ es el subcampo primo de $K$. 
\end{teorema}

La demostración de este resultado está fuera del alcance de este trabajo, pero se recomienda al lector consultar \cite{bib:passman}. 

En el teorema \ref{teo:idme} se demostró el teorema anterior cuando la característica del campo es cero con la condición de que el grupo sea finito, pero dicho resultado es válido aún cuando el grupo es infinito, pero su demostración requiere el uso de resultados previos de teoría de números. Para la demostración del siguiente resultado, se sugiere al lector consultar \cite{bib:passman}.

\begin{teorema}\label{teo:genidem23}
Sea $G$ un grupo cualquiera y $K$ un campo de característica cero. Supóngase que $e = e^2 = \sum e(g)g \in KG$. Entonces:
\begin{bulletList}
\newItem $e(1) \in \mathds{Q}$.
\newItem $0\leq e(1) \leq 1$.
\newItem $e(1) = 0 \Leftrightarrow e= 0 $ y $e(1) = 1 \Leftrightarrow e = 1$. 
\end{bulletList}
\end{teorema} 

Supóngase que $e = e^2 \in \mathds{Z}G$, como $e(1)$ es un entero, se sigue del teorema anterior que $e = 0$ o $e = 1$. Así, se obtiene el siguiente
\begin{corolario}
El grupo-anillo entero $\mathds{Z}G$ sólo contiene idempotentes triviales, para cualquier grupo $G$.
\end{corolario}
\section{\hskip 1em Unidades de torsión}
Se demostró en el lema \ref{lem:BH} que si $G$ es un grupo finito, $\gamma \in \mathds{Z}G$ es una unidad de orden finito y $\gamma(1) \neq 0$ entonces $\gamma = \pm 1$. Dicho resultado es válido también cuando $G$ es un grupo infinito. 
\begin{teorema}\label{teo:Passman-Bass}
Sea $\gamma = \sum\gamma(g)g \in \mathds{Z}G$ que satisface $\gamma^n = 1$, para algún entero positivo $n$. Si $\gamma(1) \neq 0$ entonces $\gamma = \pm 1$.
\end{teorema} 
\begin{proof*}
Sea $\mathds{C}[X]$ el anillo de polinomios con coeficientes en $\mathds{C}$. Considérese el homomorfismo $\phi \colon \mathds{C}[X] \to \mathds{C}[\gamma]$ dada por $X \mapsto \gamma$. El kernel de este homomorfismo es el ideal $\langle f(X) \rangle$ generado por el polinomio minimal $f(X)$ de $\gamma$. Entonces $f(X)$ divide a $X^n - 1$ y por lo tanto tiene sus raíces distintas. 
Así, se tiene 
\begin{equation*}
\mathds{C}[\gamma] \simeq \frac{\mathds{C}[X]}{\langle (X) \rangle} \simeq \mathds{C}\oplus\mathds{C}\oplus\cdots\oplus\mathds{C} \simeq \oplus_i\mathds{C}e_i,
\end{equation*}
donde los $e_i$ son idempotentes ortogonales primitivos de $\mathds{C}[\gamma]$. De lo anterior, se puede escribir $\gamma = \sum_{i}\xi_ie_i$ donde $\xi_i\in \mathds{C}$, $\xi_i^n = 1$ y $e_ie_j = \delta_{ij}e_j$, con $\delta_{ij}$ la función delta de Kronecker. 

Calculando el primer coeficiente en ambos lados de la ecuación y usando el teorema \ref{teo:genidem23} se obtiene
\begin{equation*}
\gamma(1) = \sum\xi_ie_i(1) = \sum\xi_i\frac{r_i}{s}, \quad \mbox{con } r_i, s \in \mathds{Z}, \quad r_i,s \geq 0.
\end{equation*}
Entonces, $s\gamma(1) = \sum\xi_ir_i$. De igual manera; como $\sum e_i = 1$ se tiene que $1 = \sum \frac{r_i}{s}$, así $\sum r_i = s$. De donde
\begin{equation*}
|s\gamma(1)| = \left| \sum\xi_ir_i  \right| \leq \sum |\xi||r_i| = \sum |r_i|=s.
\end{equation*}
Como $|s\gamma(1)|\leq s$, se tiene $|\gamma(1)|= 1$ y también $\left| \sum \xi_ir_i \right| = \sum|\xi_i||r_i|$. Se sigue que todos los $\xi_i$ son iguales y $\gamma = \sum \xi_1e_1 = \xi_1 = \gamma(1) \in \mathds{Z}$. 
\end{proof*}

Este último resultado tiene bastantes consecuencias útiles. El lector deberá recordar que, como se mostró en la proposición \ref{prep:conjugacion}, hay una involución estándar en $\mathds{Z}G$ dada por 
\begin{equation*}
\gamma = \sum\gamma(g)g \mapsto \gamma* = \sum\gamma(g)g^{-1},
\end{equation*}
tal que
\begin{eqnarray*}
(\gamma^*)^* &=& \gamma \\
(\gamma + \mu)^* &=& \gamma^* + \mu^*\\
(\gamma\mu)^* &=& \mu^*\gamma^*\\
(c\gamma)^* &=& c\gamma^*,
\end{eqnarray*}
para todo $\gamma, \mu \in \mathds{Z}G$ y $c \in \mathds{Z}$. Más aún, 
\begin{equation*}
(\gamma\gamma^*)(1) = \sum(\gamma(g))^2
\end{equation*}
lo cual implica que $\gamma\gamma^* = 0 $ si y sólo si $\gamma = 0$.

\begin{corolario}
Supóngase que $\gamma \in \mathds{Z}G$ tiene la propiedad de conmutar con $\gamma^*$. Si $\gamma$ es una unidad central de orden finito, entonces $\gamma = \pm g_0$ para algún $g_0 \in G$.
\end{corolario}
\begin{proof*}
Por hipótesis $\gamma^n = 1$ para algún entero positivo $n$ y $\gamma\gamma^* = \gamma^*\gamma$, por lo tanto $(\gamma\gamma^*) = 1$. Más aún, $(\gamma\gamma^*)(1) = \sum \gamma(g)^2 \neq 0$. Entonces, por el teorema anterior, $\gamma\gamma^* = 1$. De esta manera, existe un único coeficiente $\gamma(g_0)$ que es distinto de cero. Se concluye entonces que $\gamma = \pm g_0$. 
\end{proof*}
\begin{corolario}
Todas las unidades centrales de orden finito en $\mathds{Z}G$ son triviales.
\end{corolario}
\begin{corolario}
Si $A$ es un grupo abeliano cualquiera, entonces todas las unidades de torsión de $\mathds{Z}A$ son triviales.
\end{corolario}
\section{\hskip 1em Elementos nilpotentes}
Ahora se desea clasificar los grupo-álgebras $KG$ de un grupo finito $G$ sobre un anillo $K$ tal que $KG$ no tiene elementos nilpotentes no triviales. Es posible observar que si $\car(K) = p > 0$ y $G$ contiene un elemento $g$ tal que $g^{p^{n}} = 1$ para algún entero positivo $n$, entonces $(g - 1 )^{p^{n}} = g^{p^{n}} - 1 = 0$. De esto se sigue el resultado
\begin{proposicion}\label{prop:nilpotentes}
Si $K$ es un campo de característica $p > 0 $ y $G$ contiene $p$-elementos, entonces $KG$ contiene elementos nilpotentes.  
\end{proposicion}
A partir de este punto se asumirá que $G$ es finito, $p\geq 0$ y que si $p >0 $ entonces $G$ no tiene $p$-elementos. Supóngase que $KG$ no contiene elementos nilpotentes a excepción de los triviales y sea $e \in KG$ un idempotente. Entonces, para cada $x \in KG$, el idempotente $\eta = ex(1-e)$ satisface $\eta^2 = 0$ y por lo tanto $ ex = exe $. Similarmente si $\eta = (1 - e)xe$ entonces $\eta^2 = 0$ y $xe = exe$, con lo que se comprueba que $e$ es central. Ahora para cualquier $g \in G$, el elemento $e = \frac{\hat{g}}{\circ (g)} = \frac{\sum_{i = 1}^{\circ(g)}g^i}{\circ(g)}$ es idempotente y por lo tanto es central. Esto significa que el subgrupo $\langle g \rangle$ es normal para cualquier $g \in G$. Se sigue del teorema \ref{teo:abelianohamiltoniano} que $G$ es abeliano o hamiltoneano. 

En el caso que $G$ sea hamiltoniano, $G = K_8 \times E \times A$, donde $K_8$ es el grupo de cuaterniones de orden ocho, $E^2 = 1$ y $A$ es un grupo abeliano de orden impar. Para obtener más información de este caso es necesario hacer un estudio mas profundo del grupo-álgebra $FK_8$, donde $F$ es un campo.

\begin{proposicion}\label{prop:AH}
Sea $K$ un campo de característica $p \geq 0$ y sea $G$ un grupo finito. Si $KG$ no tiene elementos nilpotentes entonces todos los idempotentes de $KG$ son centrales y $G$ es abeliano o hamiltoniano.
\end{proposicion}

En lo que sigue el lector deberá recordar el concepto de números cuaterniones. Los números cuaterniones, con coeficientes racionales, se escriben como sumas directas de espacios vectoriales:
\begin{equation*}
H_{\mathds{Q}} = \mathds{Q}\oplus\mathds{Q}i\oplus\mathds{Q}j\oplus\mathds{Q}k,
\end{equation*}
con su ya conocida multiplicación, véase \cite[ 31]{bib:herstein}. Se puede definir formalmente una estructura similar sobre cualquier campo $F$. Considérese el espacio vectorial
\begin{equation*}
H(F) = F\oplus Fi\oplus Fj\oplus Fk 
\end{equation*} 
y defínase la multiplicación distributivamente con $i^2 = j^2 = k^2 = -1$, $ij = k = -ji$, $jk = i = -kj$ y $ki = j = -ik$. De esta forma $H(F)$ es un anillo no conmutativo.

Para un elemento $\alpha = a + bi + cj + dk \in H(F)$ se define:
\[
\overline{\alpha} = a - bi - cj -dk \mbox{ , } \alpha'= a - bi +cj + dk.
\]
Luego de hacer algunos cálculos sencillos se obtiene
\begin{lema}\label{lem:propq}
Sea $\alpha \in H(F)$. Entonces:
\begin{bulletList}
\newItem $\alpha\overline{\alpha} = a^2 + b^2 + c^2 + d^2$.
\newItem $\alpha'\alpha = (a^2 + b^2 - c^2 -d^2) + (2ac + 2bd)j + (2ad -2bc)k$.
\end{bulletList}
Al escalar $N(\alpha) := \alpha\overline{\alpha}$ se le llama la norma de $\alpha$.
\end{lema}
\begin{proposicion}
El álgebra de los cuaterniones $H(F)$ tiene divisores de cero si y sólo si la ecuación $X^2 + Y^2 = -1$ tiene solución en $F$.
\end{proposicion}
\begin{proof*}
Supóngase que existen elementos $a,b \in F$ tal que $a^2 + b^2 = -1$. Entonces, para $\alpha = a + bi + j$ se tiene $N(\alpha) = \alpha\overline{\gamma} = 0$; $\alpha$ es divisor de cero en $H(F)$.

Falta demostrar que si $H(F)$ tiene divisores de cero, entonces la ecuación $X^2 + Y^2 = -1$ tiene soluciones en $F$. Cuando $F$ es de característica dos se tiene $1 + 0 = -1$ entonces se asumirá que $\car (F) \neq 2$. Supóngase que existen elementos $\alpha = a + bi + cj + dk \neq 0$ y $\beta \neq 0 \in H(F)$ tal que $\alpha\beta = 0 $. Entonces, $\overline{\alpha}\alpha\beta = N(\alpha)\beta = 0 $ de donde $N(\alpha) = a^2 +b^2 + c^2 + d^2 = 0$. Si alguno de los coeficientes de $\alpha$ es cero, se tiene que la ecuación $X^2 + Y^2 =-1$ tiene solución en $F$. Si, por el contrario, todos los coeficientes de $\alpha$ son distintos de cero se puede considerar $\alpha'\alpha\beta = 0$ y del lema \ref{lem:propq} se sabe que $\gamma = \alpha'\alpha$ a lo sumo tiene tres coeficientes no nulos y por lo tanto $N(\gamma)\beta = 0$ implica que $X^2 + Y^2 = -1$ tiene solución en $F$. Por último falta considerar el caso en que $\gamma = 0$, en cuyo caso se tiene, por el lema \ref{lem:propq}, que $a^2 + b^2 -c^2 -d^2 = 0 $ y de $N(\alpha) = 0 $ se sigue  que $a^2 + b^2 + c^2 + d^2 = 0 $ entonces $a^2 + b^2 = 0$, de donde $\left( \frac{a}{b} \right)^2 + 0 = -1$.
\end{proof*}
Este resultado se puede ampliar de la siguiente manera.
\begin{proposicion}
Las siguientes proposiciones son equivalentes:
\begin{bulletList}
\newItem El álgebra de los cuaterniones $H(F)$ no tiene divisores de cero.
\newItem La ecuación $X^2 + Y^2 = -1$ no tiene solución en $F$.
\newItem $H(F)$ es un anillo de división.
\end{bulletList}
\end{proposicion}

\begin{proof*}
Para la demostración de esta proposición solo falta probar que si $H(F)$ no tiene divisores de cero entonces es un anillo de división. Sea $0 \neq \alpha \in H(F)$ entonces $\overline{\alpha}\alpha = N(\alpha) \neq 0$ y por lo tanto $\alpha\left( \frac{\overline{\alpha}}{N(\alpha)}\right) = 1$.
\end{proof*}

\begin{teorema}\label{teo:caracterizacion}
Supóngase que $\car(F) \neq 2$. Entonces el álgebra de los cuaterniones $H(F)$ es un anillo de división o isomorfo a $M_2(F)$, el anillo de matrices de $2\times 2$ sobre el campo $F$. La última opción sucede únicamente si $X^2 + Y^2 = -1$ tiene solución en~$F$.
\end{teorema}

\begin{proof*}
Se debe demostrar que si $H(F)$ no es un anillo de división entonces es isomorfo a $M_2(F)$. En efecto, supóngase que existen $x,y \in F$ tal que $x^2 + y^2 = -1$. Considérese la aplicación $\theta \colon H(F) \to M_2(F)$ dada por:
\begin{eqnarray*}
\theta(i) &=& \begin{pmatrix}
x & y\\
y & -x
\end{pmatrix}\\
\theta(j) &=& \begin{pmatrix}
0 & 1\\
-1 & 0
\end{pmatrix}\\
\theta(k) &=& \begin{pmatrix}
-y & x \\
x & y 
\end{pmatrix} \\
\theta(1) &=& \begin{pmatrix}
1 & 0 \\
0 & 1
\end{pmatrix}
\end{eqnarray*}
y por extensión lineal en $F$.
Para demostrar que $\theta$ es biyectiva, basta demostrar que las cuatro matrices dadas anteriormente son linealmente independientes en $F$, es decir, si existen $a, b, c, d \in F$ tal que 
\begin{equation*}
a\begin{pmatrix}
1 & 0\\
0 & -1
\end{pmatrix} + b\begin{pmatrix}
x & y \\
y & -x
\end{pmatrix} + c\begin{pmatrix}
0 & 1 \\
-1 & 0
\end{pmatrix} + d\begin{pmatrix}
-y & x\\
x & y
\end{pmatrix} = 0
\end{equation*}
entonces $a = b = c = d = 0$. En efecto, de la ecuación anterior se obtiene
\begin{eqnarray*}
a + bx - dy &=& 0\\
by + c + dx &=& 0 \\
by -c + dx &=& 0 \\
a -bx + dy &=& 0
\end{eqnarray*}
un sistema de ecuaciones homogéneo en $a, b, c,d$ con determinante $4(x^2 + y^2) \neq 0$, entonces la única solución de dicho sistema es la trivial, de donde $H(F) \simeq M_2(F)$.
\end{proof*}

Por lo expuesto en esta sección, el lector podrá intuir que es esencial poder trabajar con álgebras de cuaterniones sobre campos ciclótomicos de la forma $F = \mathds{Q}(\xi)$, donde $\xi$ es una raíz primitiva de la unidad. 

\begin{teorema}
Sea $F = \mathds{Q}(\xi_m)$ un campo ciclotómico, con $\xi_m$ una raíz primitiva de la unidad de orden $m$, donde $m$ es un entero impar mayor que $1$. Entonces, la ecuación $X^2 + Y^2 = -1$ tiene solución en $F$ si y sólo si el orden multiplicativo de $2$ módulo $m$ es par.
\end{teorema}

Para la demostración de dicho teorema y una lectura más profunda de este tema el lector puede consultar \cite{bib:moser}. Como consecuencia de este teorema se tiene
\begin{lema}
Sea $F = \mathds{Q}(\xi_m)$ como en el teorema anterior. Entonces $H(F)$ tiene divisores de cero si y sólo si el orden multiplicativo de $2$ módulo $m$ es par.
\end{lema}

Ahora se describe la estructura algebraica del grupo-álgebra $F(K_8)$.

\begin{lema}\label{lem:K8}
Sea $F$ un campo de característica distinta de $2 $. Entonces,
\begin{equation*}
FK_8 \simeq 4F\oplus H(F).
\end{equation*}
\end{lema}
\begin{proof*}
Se escribe $K_8$ como
\begin{equation*}
K_8 = \langle a, b \colon a^4 = 1, a^2 = b^2, bab^{-1} = a^{-1} \rangle
\end{equation*}
y como se demostró en el ejercicio \ref{ejer:klein}, $\overline{K_8} = \frac{K_8}{K_8'}$ es el grupo de Klein de cuatro elementos, se tiene
\begin{equation*}
F\overline{K_8} \simeq F \oplus F \oplus F \oplus F.
\end{equation*}
Por otro lado, como existe $\phi \colon FK_8 \to H(F)$ endomorfismo dada por $a \mapsto i$, $b \mapsto j$, se sigue que $H(F)$ es isomorfo a algún sumando simple de $FK_8$ y contando las dimensiones se obtiene el resultado. 
\end{proof*}

Ahora se tiene la capacidad de clasificar las grupo-álgebras $FG$ con la propiedad que $FG$ no contenga elementos nilpotentes.
\begin{teorema}\label{teo:caracCarEntera}
Sea $F$ un campo de característica $p > 0$ y sea $G$ un grupo finito. Entonces $FG$ no tiene elementos nilpotentes si y sólo si $G$ es un $p'$-grupo abeliano. 
\end{teorema}
\begin{proof*}
Supóngase que $FG$ no tiene elementos nilpotentes. Entonces de las proposiciones \ref{prop:nilpotentes} y \ref{prop:AH} se tiene que $G$ es un $p'$-grupo y que $G$ es abeliano o hamiltoniano. Supóngase que $G$ es hamiltoneano. Entonces $p \neq 2$. Más aún, siempre se puede resolver $X^2 + Y^2 = -1$ en un campo con $p$ elementos (y por lo tanto también en $F$). Entonces $FK_8$ tiene elementos nilpotentes por el teorema \ref{teo:caracterizacion} y el lema \ref{lem:K8}. Así, $G$ debe ser abeliano.
Para el converso basta notar que $FG$, siendo semisimple y conmutativo, es suma directa de campos. 
\end{proof*}

Existe una caracterización cuando el campo tiene característica cero, a continuación se presentan los resultados sin su demostración, pero se recomienda al lector consultar \cite{bib:Sehgal}.
\begin{teorema}\label{teo:caracCar0}
Sea $G$ un grupo finito de orden $2^km$ con $(2,m) = 1$. Entonces $\mathds{Q}G$ no tiene elementos nilpotentes si y sólo si $G$ es abeliano o hamiltoniano con la propiedad de que el orden de $2$ módulo $m$ sea impar. 
\end{teorema}

\begin{teorema}
Sea $G$ un grupo nilpotente finitamente generado. El grupo-anillo $\mathds{Z}G$ no tiene elementos nilpotentes si y sólo si cada subgrupo finito de $G$ es normal y sucede alguna de las siguientes condiciones:
\begin{bulletList}
\newItem $T(G)$, el conjunto de los elementos de torsión de $G$, es un subgrupo abeliano.
\newItem $T(G) = K_8 \times E \times A$, donde $E$ es un $2$-grupo elemental abeliano y $A$ es un grupo abeliano de orden impar $m$ tal que el orden multiplicativo de $2$ módulo $m$ es impar.  
\end{bulletList}
\end{teorema}
\addtocontents{toc}{\protect\addvspace{1.8em}}
\chapter{UNIDADES DE LOS GRUPO-ANILLOS}\label{chap:unidades}
\section{\hskip 1em Algunas formas de construir unidades}
Sea $R$ un anillo. El conjunto $\mathcal{U}(R) = \{x \in R: \mbox{ existe } y\in R: xy=yx=1\}$ es el grupo de unidades de un anillo. 
En particular, dado un grupo $G$ y un anillo $R$, $\mathcal{U}(RG)$ denota al grupo de unidades del grupo-anillo $RG$. Como la función de aumento $\mathcal{E} \colon RG \to R$, dada por $\mathcal{E} \left( \sum a(g)g \right) = \sum a(g)$, es un homomorfismo de anillos, se tiene que $\mathcal{E}(u) \in \mathcal{U}(R)$, para todo $u \in \mathcal{U}(RG).$ Se denotará como $\mathcal{U}_1(RG)$ el subgrupo de unidades de aumento $1$ en $\mathcal{U}(RG)$, a saber
\[\mathcal{U}_1(RG) = \{u \in \mathcal{U}(RG): \mathcal{E}(u) = 1 \}.\]

 Para una unidad $u$ del grupo-anillo integral $\mathds{Z}G$ se tiene que $\mathcal{E}(u) = \pm1$, entonces es claro que \[ \mathcal{U}(\mathds{Z}G) = \pm \mathcal U_1(\mathds{Z}G) .\]
De la misma manera, para un anillo $R$ arbitrario se tiene que \[ \mathcal{U}(RG) = \mathcal{U}(R) \times \mathcal{U}_1(RG). \]
No se conocen muchas formas para construir unidades. La mayoría de las construcciones conocidas son antiguas y elementales. A lo largo de este capítulo, se mostrará y describirá algunas de estas construcciones, donde se trabajará principalmente con grupo-álgebras $KG$ sobre un campo $K$ y con el grupo-anillo entero $\mathds{Z}G$.
\begin{ejemplo}[Unidades Triviales]
Un elemento de la forma $rg$, donde $r \in \mathcal{U}(R)$ y $g \in G$, tiene inverso $r^{-1}g^{-1}$. Los elementos de esta forma son llamados \textbf{unidades triviales} de $RG$. De esta manera, por ejemplo, los elementos $\pm g$, $g \in G$ son las unidades triviales del grupo-anillo entero $\mathds{Z}G$. Si $K$ es un campo, entonces las unidades triviales de $KG$ son los elementos de la forma $kg, k \in K, k \neq 0, g \in G$. Hablando de manera general, los grupo-anillos contienen unidades no triviales.
\end{ejemplo}
\begin{ejemplo}\label{ejem:unipotentes}
Sea $\eta \in R$ tal que $\eta^2 =0$, entonces se tiene $(1+\eta)(1-\eta)=1$. De este hecho, tanto $1+\eta$ como $1-\eta$ son unidades de $R$. De la misma manera, si $\eta \in R$ es tal que $\eta ^k =0$ para algún entero positivo $k$, entonces se tiene que
\begin{equation*}
(1-\eta)(1+\eta+\eta^2+\cdots+\eta^{k-1}) =  1-\eta^{k} = 1,
\end{equation*}
\vskip-6em
\begin{equation*}
(1+\eta)(1-\eta+\eta^2+\cdots\pm \eta^{k-1}) =  1 \pm \eta^{k} = 1.
\end{equation*}
Así, $1\pm \eta$ son unidades de $R$. Estas unidades son llamadas \textbf{unidades unipotentes} de $R$. En un grupo-álgebra $KG$ sobre un campo de característica $p>0$ se puede iniciar la búsqueda de unidades unipotentes investigando a los elementos nilpotentes. Si $g \in G$ es de orden $p^n$, entonces $(1-g)^{p^n} = 0$, de esta forma se demuestra que $\mu = 1-g$ es nilpotente.

En este caso $1-\eta = g$ es trivial, pero $1+\eta = 2-g$ es no trivial, a menos que $\car(K)=2$. Nótese que $g-g^2=g(1-g)$ también es nilpotente, entonces $1+g-g^2$ es una unidad no trivial si $g^2 \neq 1$.

En el teorema \ref{teo:caracCar0} y \ref{teo:caracCarEntera} se clasificaron  todos los grupos finitos tal que el grupo-álgebra $KG$ no tiene elementos nilpotentes. Se vera entonces que las grupo-álgebras de grupos finitos casi siempre tienen unidades no triviales.
\end{ejemplo}

\begin{proposicion}\label{prop:UnidadesTriviales}
Sea $G$ un grupo tal que no es libre de elementos de torsión y $K$ un campo de característica $p\geq 0$. Entonces $KG$ sólo tiene unidades triviales si y sólo si se cumple alguna de las siguientes condiciones
\begin{bulletList}
\newItem $K=F_2$ y $G=C_2$ o $C_3$.
\newItem $K=F_3$ y $G=C_2$.
\end{bulletList}
\end{proposicion}
\begin{proof*}
Supóngase que todas las unidades de $KG$ son triviales. Considérese $N=\langle a \rangle$ subgrupo finito de $G$ de orden $n$. Si no existe $b \in G$ que normalize a $N$, entonces $\eta = (a-1)(1+a+\cdots + a^{n-1})$ es no nulo, pero $\eta^2 = (a-1)b(1+a+\cdots+a^{n-1})(a-1)b(1+a+\cdots+a^{n-1})=0$, de esa cuenta, $\eta +1$ es unidad no trivial de $KG$, proposición que contradice la hipótesis, de donde se concluye que todo subgrupo finito de $G$ es normal.

Sea $H$ un subgrupo finito propio de $G$ y considérese $\hat{H} = \sum_{h \in H}h$. Es fácil notar que  $\hat{H}$ es central y $\hat{H}^2 = |H|\hat{H} $. Tómese $g \in G-H$ fijo. Si $|H| =0$ en $K$ entonces $\hat{H}^2 = 0$ y $g + \hat{H}$ es una unidad no trivial de $KG$ con inverso $g^-1(1-g^-1\hat{H})$. Si $|H| \neq 0$ en $K$, entonces $e = \frac{1}{\hat{H}}\hat{H}$ es idempotente central y $e +g(1-e)$ es una unidad no trivial con inverso $e+g^-1(1-e)$. En ambos casos se llega a una contradicción, por lo que se concluye que $G = \langle a \rangle$ es de orden primo.

Si $\car (K) = p$ entonces $1+c\hat{G}, c \in K$ es una unidad no trivial, a menos que $p=2$ y $K=F_2$. 

Por otro lado, si $\car(K) \neq p$ entonces, del hecho que $K\langle a \rangle$ es semisimple y conmutativo, $K\langle a \rangle$ es suma directa de campos, a saber
\[ K \langle a \rangle \simeq K\oplus K(\zeta) \oplus K(\theta) \oplus + \cdots \] donde $\zeta, \theta, \dots$ son raíces de la unidad de orden $p$. Bajo este isomorfismo, se tiene $a \mapsto (1,\zeta,\theta,\dots)$, por lo que una unidad trivial $ka^ i, 0\neq k \in K$ tiene imagen $(k,k\zeta^i,k\theta^i,\dots)$. Nótese que si la descomposición de $K\langle a\rangle$ tuviera más de dos componentes se tendrían unidades de la forma $(1,\zeta,1,\dots)$ que no corresponden a unidades triviales de $K\langle a \rangle$.
Entonces se debe tener \[ K\langle a \rangle \simeq K \oplus E\mbox{, } E = K(\zeta)\mbox{, } |K|=q\mbox{, } |E|=q^r\mbox{, } \circ(a) = p. \] Al contar el número de unidades y de elementos se tiene \[ p(q-1) = (q-1)(q^r-1), p^q = q\cdot q^r. \] De la condición anterior, se tiene que $q^p = q(p-1)$ y $q^{p-1} = p+1$, lo cual sólo es posible para $q=2$ y $p=3$ o $q=3$ y $p=2$. Con lo que se demuestra que $K=F_2$ y $G=C_3$ o $K=F_3$ y $G=C_2$.

Para el converso, una simple inspección demuestra que $F_2C_2, F_3C_3 \simeq F_2\oplus F_4$ y $F_3C_2 \simeq F_3 \oplus F_3$ tiene dos, tres y cuatro unidades triviales, lo cual coincide con el número de unidades triviales en cada caso. 
\end{proof*}

Se ha llegado al punto en el que se desea clasificar los grupos de torsión $G$ de tal forma que el grupo-anillo entero $\mathds{Z}G$ tenga solo unidades triviales.

\begin{ejemplo}
En el ejemplo anterior se dio la construcción de unidades unipotentes a partir de elementos nilpotentes. Ahora se verán elementos nilpotentes en particular que también poseen esa característica. 

Supóngase que $R$ tiene divisores de cero, es decir, se pueden encontrar elementos $x,y \in R$ no nulos tales que $xy = 0$. Si $t$ es algún otro elemento de $R$ entonces $\eta =ytx$ es no nulo tal que $\eta ^2 = (ytx)(ytx) = ytxytx = 0$, así $1+\eta$ es una unidad. En el caso especial cuando $R =\mathds{Z}G$ es un grupo-anillo entero, una manera sencilla de obtener un divisor de cero es considerar un elemento $a \in G$ de orden finito $ n >1$, entonces $a-1$ es divisor de cero, ya que $(a-1)(1+a+\cdot+a^{n-1})=0$. De esa manera, tomando cualquier elemento $b \in G$, se puede construir una unidad de la forma 
\begin{equation}\label{eqn:biciclicas}
\mu_{a,b} = 1+(a-1)b\hat{a}, \mbox{ con } \hat{a} = 1+a+\cdots+a^{n-1}.
\end{equation}
\end{ejemplo}
\begin{definicion}
Sean $a \in G$ un elemento de orden finito $n$ y $b$ cualquier otro elemento de $G$. La unidad $\mu_{a,b}$ dada por la ecuación~\eqref{eqn:biciclicas} es llamada unidad bicíclica del grupo-anillo $\mathds{Z}G$. Se denotará por $\mathcal{B}_2$ el subgrupo de $\mathcal{U}(\mathds{Z}G)$ generado por todas las unidades bicíclicas de $\mathds{Z}G$.
\end{definicion}

Es claro que si $a,b \in G$ conmutan, entonces $\mu_{a,b} = 1$. Se desea saber para que casos $\mu_{a,b}$ es una unidad trivial de $\mathds{Z}G$.

\begin{proposicion}\label{prop:unidadesb}
Sean $g,h$ elementos de un grupo $G$ con $\circ(g) = n < \infty$. Entonces, la unidad bicíclica $\mu_{g,h}$ es trivial si y sólo si $h$ normaliza a $\langle g \rangle$, en cuyo caso $\mu_{g,h} = 1$.
\end{proposicion}
\begin{proof*}
Supóngase que $h$ normaliza a $\langle g \rangle$, entonces $h^{-1}gh = g^j$, para algún entero positivo $j$. De esto se tiene $gh = g^jh$ y como $g^j\hat{g} = \hat{g}$, se tiene $gh\hat{g} = h\hat{g}$. Haciendo los cálculos $\mu_{g,h} = 1+(g-1)h\hat{g}= 1+gh\hat{g}-h\hat{g} =1$.

Para el converso, supóngase que $\mu_{g,h}$ es trivial, entonces, del hecho que $\mathcal{E}(\mu_{g,h})=1$, existe $x \in G$ tal que $\mu_{g,h}=x$. De esta cuenta, se tiene
\[ 1+(1-g)h\hat{g} = x \] y de esta ecuación se infiere que \[ 1+ h(1+g+g^2+\cdots + g^{n-1}) = x +gh(1+g+g^2+\cdots+g^{n-1}).  \] Si $x=1$ se tiene que $h=ghg^i$ para algún entero positivo $i$. Si $x \neq 1$ entonces $h \notin \langle g \rangle$, pero $1$ aparece en el lado izquierdo de la ecuación, por lo que también debe aparecer en el lado derecho, esto es, existe $k$ entero positivo tal que $ghg^k = 1$ entonces $h = g^{-1}g^{-k}= g^{-(k+1)}$ y por lo tanto $h \in \langle g \rangle$, lo cual es una contradicción. 
\end{proof*}
Como consecuencia inmediata se tiene el resultado:
\begin{proposicion}
Sea $G$ un grupo finito. El grupo $\mathcal{B}_2$ es trivial si y sólo si todo subgrupo de $G$ es normal.
\end{proposicion}
\begin{proposicion}
Toda unidad bicíclica $\mu_{g,h} \neq 1$ de $\mathds{Z}G$ es orden infinito. 
\end{proposicion}
\begin{proof*}
Dado $\mu_{g,h} = 1 +(g-1)h\hat{g}$ se tiene
\[  \mu_{g,h}^s = (1+(g-1)h\hat{g}^s = 1+s(g-1)h\hat{g})  \] entonces $\mu_{g,h}^s = 1$ si y sólo si $(g-1)h\hat{g} = 0$, lo cual sucede solo si $\mu_{g,h} = 1$.
\end{proof*}
Se desea explorar que pasa cuando se trabaja con grupos conmutativos finitos. El lector deberá recordar la definición de la función totiente de Euler $\phi$. 
\begin{definicion}
Sea $g$ un elemento de orden $n$ en un grupo $G$. Una unidad cíclica \nopagebreak[0] de Bass\footnote{Hyman Bass (5 de octubre, 1932) es un matemático estadounidense conocido por sus trabajos en álgebra y en matemática educativa.  } es un elemento del grupo-anillo $\mathds{Z}G$ de la forma:
\[ \mu_{i} = (1+g+\cdots+g^{i-1})^{\phi(n)} + \frac{1-i^{\phi{n}}}{n}\hat{g}  \] donde $i$ es un entero tal que $1<i<n-1$ y $(i,n) = 1$.
\end{definicion}
Como es natural, se debe mostrar que $\mu_i$ es una unidad. Es claro que, para $g \in G, \mu_i$ pertenece al grupo-anillo $\mathds{Q} \langle g \rangle$. Se vió en el ejemplo \ref{ejem:descomposicionRacional} que $\mathds{Q}\langle g \rangle \simeq \oplus_{d \mid  n} \mathds{Q}(\zeta_d)$ donde $\zeta_d$ es una raíz primitiva de la unidad de orden $d$. Más aún, bajo este isomorfismo, la proyección de $g$ en cada componente es la respectiva raíz de la unidad, así que un elemento de la forma $(1+g+\cdots+g^{i-1})$ proyecta, en cada componente, un elemento de la forma: \[ 1+\zeta_d + \cdots +\zeta_d^{i-1} \in \mathds{Z}[\zeta_n].  \] Si $\zeta_d \neq 1,$ entonces el elemento $\zeta_d$ es invertible en $\mathds{Z}[\zeta_d]$ y es llamada unidad ciclotómica. De lo anterior, el inverso de $\alpha_d$ es \[ \alpha_d^{-1} = \frac{\zeta_d-1}{\zeta_d^i-1} = \frac{\zeta_d^{ik}}{\zeta_d^{i}-1} = 1+\zeta_d^i + \cdots + \zeta_d^{i(k-1)}, \] donde $k$ es cualquier entero tal que $ik \equiv 1 \pmod{n} $. Es claro que $\alpha_d^{-1} \in \mathds{Z}[\zeta_d] \in \mathds{Z}[\zeta_n]$.

Para la primer componente las cosas cambian, ya que la proyección es precisamente el valor $i$, que no es invertible. Ahora bien, como $(i,n) = 1$ y aplicando el teorema de Euler, se tiene que $i^{\phi(n)} = 1 + tn$ para algún $t \in \mathds{Z}$. Considérese el elemento \[  (1+g+\cdots +g^{i-1})^{\phi(n)} -t\hat{g} \] y nótese que $\hat{g}$ es cero en cualquier componente $\mathds{Q}(\zeta_d)$, con $\zeta_d \neq 1$, por lo que la proyección de $\mu_i$ e cada una de estas componentes es una unidad. Ahora, analizando el caso de la primera componente de nuevo, se puede observar que dicha proyección es $i^{\phi(n)}-tn = 1$, con lo cual se prueba que la proyección sobre dicha componente también es unidad. Más aún, se obtuvo que $-t = \frac{1-i^{\phi(n)}}{n}$, por lo que el elemento $(1+g+\cdots+g^{i-1})^{\phi(n)} -t\hat{g})$ considerado anteriormente es precisamente $\mu_i$. 

De esta forma se ha demostrado que la proyección de $\mu_i$ en todas las componente de $\oplus_{d \mid n}\mathds{Z}[\zeta_d]$ es una unidad. Si se denota por $R$ la preimagen de este anillo bajo el isomorfismo, se tiene que $\mu_i$ es unidad en $R$.
\begin{proposicion}
Sea $g$ un elemento de orden finito en un grupo $G$. Entonces, el elemento \[  \mu_i = (1+g+\cdots+g^{i-1})^{\phi(n)} + \frac{1-i^{\phi(n)}}{n}\hat{g}, \] donde $i$ es un entero tal que $1<i<n-1$ y $(i,n) = 1$, es invertible y su inversa es \[  \mu_i^{-1} = (1+g^i + \cdots + g^{i(k-1)})^{\phi(n)} + \frac{1-k^{\phi(n)}}{n}\hat{g},\] donde $k$ es cualquier entero tal que $ik \equiv 1 \pmod
{n}$.
\end{proposicion}

\begin{proposicion}\label{prop:basscyclic}
Sea $g$ un elemento de orden finito $n$ en un grupo $G$ y sea $l$ un entero tal que $1<l<n-1$ y $(l,n) = 1$. Entonces, la unidad cíclica de Bass \[  \mu_l  = (1+g+\cdots+g^{l-1})^{\phi(n)} + \frac{1-l^{\phi(n)}}{n}\hat{g} \] es de orden infinito. 
\end{proposicion}
\begin{proof*}
Se sabe que \[ \mathds{Q}\langle g \rangle \simeq \mathds{Q} \oplus \cdot \oplus \mathds{Q}(\zeta^d) \oplus \cdots \oplus \mathds{Q}(\zeta),\] donde $\zeta$ es una raíz primitiva de la unidad de orden $n$ y $d$ representa a los divisores de $n$. Más aún, en el isomorfismo se tiene \[ g \mapsto (1, \dots , \zeta^d, \dots , \zeta).\] Sea $\mu_l$ como en la proposición. Se requieren demostrar que la proyección $\mu_l(\zeta)$ en la última componente es de orden infinito. Primero nótese que dicha proyección  es de la forma $\mu_l(\zeta) = (1+\zeta+\cdots+\zeta^{l-1})^{\phi(n)}$, de esa cuenta, si $(1+\cdots + \zeta^{l-1})^{\phi(n)}$ fuera de orden finito, entonces se tendría que $(1+\cdots + \zeta^{l-1})$ sería de orden finito. Como $\{ \pm \zeta^t \colon 0\leq t\leq n-1 \}$ son todas raíces de la unidad de $\mathds{Q}(\zeta)$, se tendría que $(1+\cdots +\zeta^{l-1}) = \pm \zeta^s$ para algún entero positivo $s$. Multiplicando la última ecuación por $(1-\zeta)$ se observa que $1-\zeta^l = \pm \zeta^s(1-\zeta)$. Así, tomando valores absolutos, se obtiene $\mid 1 - \zeta^l \mid = \mid 1-\zeta \mid$. Escribiendo $\zeta = \cos \theta + i \sin \theta$, por el teorema de DeMoivre, se tiene $\zeta^l = \cos(l\theta) + i \sin(l\theta)$, de donde se deduce que $\mid 1 - \zeta \mid^2 = \mid 1 - (\cos\theta + i \sin\theta) \mid ^2 = 2(1-\cos\theta)$ y $\mid 1-\zeta^l \mid^2 = \mid 1-(\cos(l\theta) + i\sin(l\theta)  \mid^2 = 2(1-\cos(l\theta))$, de esa cuenta, $\cos\theta = \cos (l\theta)$, lo cual implica que $l\theta = \theta$ o $l\theta = 2\pi-\theta$, por lo tanto $\zeta^l = \zeta$ o $\zeta^l = \zeta^{-1}$. En cualquiera de los dos casos se obtiene una contradicción, lo cual demuestra que $\mu_l$ tiene orden infinito.
\end{proof*}
\begin{nota}
En la definición de unidad cíclica de Bass $\mu_l$, $l$ está en el rango $1<l<n-1$. Si se toma $l=n-1$ se tiene \[ \mu_l= (1+g+\cdots+g^{n-2})^{\phi(n)}+\frac{1-l {\phi(n)}}{n}\hat{g}.\] La proyección de $\mu_l$ sobre cualquier componente es $(-g^{-1})^{\phi(n)}$ y $\mu_l = (-g^{-1})^{\phi(n)}$ es trivial. Así mismo, de la restricción $1<l<n-1$, se tiene que $n\geq 5$ para que $\mu_l$ esté definida.
\end{nota}
\begin{nota}
Lo proposición anterior demuestra que $\mu_l$ es una unidad no trivial.
\end{nota}
\begin{ejemplo}
Ahora considérese $g \in G$ un elemento de orden impar, $n \neq 1$ y el elemento \[ \mu = 1-g+g^2-\cdots +g^{c-1},\] donde $(c,2n)=1$. Entonces la proyección en cada componente de $\mathds{Q}\langle g \rangle$ es una unidad ciclótomica y como la proyección en la primera componente es $1$, se tiene que $\mu$ es una unidad en $\mathds{Z}\langle g \rangle$. Esta unidad es llamada una \textbf{unidad alternante}.
\end{ejemplo}

% % % % % % % % % %Inicio de Sección % % % %
\section{\hskip 1em Unidades Triviales}
En el capítulo anterior se demostró que si $G$ es un grupo abeliano, entonces todas las unidades de torsión de $\mathds{Z}G$ son triviales. Ahora en esta sección se hará un breve estudio de los grupos $G$ que hacen que todas las unidades de $\mathds{Z}G$ sean triviales.

El lector deberá recordar que una unidad trivial de $\mathds{Z}G$ es un elemento de la forma $\pm g$, $ g \in G$. Así, si todas las unidades de $\mathds{Z}G$ son triviales, entonces se tiene que $\mathcal{U}(\mathds{ZG}) = \pm G$. Esta condición se traduce, en términos de unidades normalizadas, como $\mathcal{U}_1(\mathds{Z}G) = G.$
\begin{lema}
Sea $G$ un grupo de torsión tal que $\mathcal{U}_1(\mathds{Z}G) = G.$ Entonces todo subgrupo de $G$ es normal. 
\end{lema}
\newpage
\begin{proof*}
Para demostrar este lema, es suficiente demostrar que todo subgrupo cíclico de $G$ es normal.  De esta forma, supóngase que existe un subgrupo cíclico $\langle g \rangle$ de $G$ que no es normal, es decir, existe $h \in G$, tal que $h^{-1}gh \notin \langle g \rangle$ y se sigue de la proposición \ref{prop:unidadesb} que la unidad bicíclica $u = 1 + (1-g)h\hat{g}$ es no trivial. 
\end{proof*}
Es sabido que si $G$ es un grupo abeliano, entonces sus subgrupos son normales. Además, se recuerda al lector que todo grupo de torsión no abeliano $G$ tal que todos sus subgrupos son normales es llamado un grupo hamiltoniano, este grupo tiene la forma \[ G = K_8\times E \times A,\] donde $E$ es un 2-grupo abeliano elemental, es decir, todo elemento $a \neq 1$ en $E$ es de orden 2, $A$ es un grupo abeliano donde todos sus elementos son de orden impar y $K_8$ es el grupo de los cuaterniones de orden ocho: \[ K_8 = \langle a,b \colon a^4 =1 , a^2 = b^2, bab^{-1} = a^{-1} \rangle. \]
\begin{proposicion}
Sea $G$ un grupo de torsión tal que $\mathcal{U}_1(\mathds{Z}G) = G$. Entonces $G$ es abeliano de exponente igual a $1,2,3,4$ o $6$, o bien, $G$ es un 2-grupo hamiltoniano.
\end{proposicion}
\begin{proof*}
Del lema anterior se sigue que $G$ es abeliano o bien  $G$ es hamiltoniano. Primero supóngase que $G$ es abeliano. Si su exponente es diferente de $1,2,3,4$ ó $6$ entonces $G$ contiene un elemento de orden $n$, con $n = 5$ o $n>6$. En ambos casos, se tiene que $\phi(n) > 2$ (ya que $\phi(n) \equiv \pmod{2}$)  y la proposición \ref{prop:basscyclic} demuestra que $G$ contiene una unidad cíclica de Bass que es no trivial.

De manera análoga, si $G$ es hamiltoniano pero no es un 2-grupo, entonces $G$ contiene un elemento $x \in A$ de orden $p>2$. Entonces, el elemento $g = ax$ tiene orden $n=4p$ y, de nuevo, $\phi(n) > 2$, por lo que $G$ contiene una unidad cíclica de Bass. 
\end{proof*}
La condición dada en la proposición anterior también es suficiente, pero su demostración no es tan trivial. Se demostrará este hecho a través de una serie de lemas.
\begin{lema}\label{lema:primerLema}
Sea $G$ un grupo tal que las unidades de $\mathds{Z}G$ son triviales y $C_2$ un grupo cíclico de orden 2. Entonces las unidades de $\mathds{Z}(G \times C_2)$ también son triviales. 
\end{lema}
\begin{proof*}
Sea $C_2 = \langle a:a^2 = 1 \rangle$. Como $\mathds{Z}(G\times C_2) \simeq (\mathds{Z}G)C_2$, un elemento $u \in \mathds{Z}(G\times C_2)$ puede ser escrito de la forma $u = \alpha + \beta a$ donde $\alpha,\beta \in \mathds{Z}G$. Debido a que $u$ es unidad, tiene que existir otro elemento $u^{-1} = \gamma +\delta a$ tal que \[  (\alpha + \beta a)(\gamma + \delta a) = (\alpha \gamma + \beta \delta)+(\alpha\delta + \beta\gamma)a =1. \] Entonces 
\begin{eqnarray*}
\alpha\gamma +\beta\delta &=& 1 \\ 
\alpha\delta + \beta\gamma  &=& 0.
\end{eqnarray*} Así, se tiene
\[ (\alpha + \beta)(\gamma + \delta) = \alpha\gamma + \beta\delta + \alpha\delta + \beta\gamma = 1  \]
\vskip -6em
\[ (\alpha - \beta)(\gamma - \delta) = \alpha\gamma + \beta\delta -( \alpha\delta + \beta\gamma) = 1  \] lo cual demuestra que $(\alpha + \beta)$ y $(\alpha -\beta)$ son unidades en $\mathds{Z}G$ y por lo tanto son unidades triviales. Entonces, existen $g_1,g_2 \in G$ tales que \[ \alpha + \beta = \pm g_1, \quad \alpha -\beta = \pm g_2.\] 
De estas últimas igualdades, se sigue que $\alpha = \dfrac{1}{2}(\pm g_1 \pm g_2)$, pero como los coeficientes de $\alpha$ deben ser enteros, tiene que ser cierto que $g_1 = \pm g_2$.
De esta manera, se tienen dos opciones: \[ \alpha +\beta = \alpha - \beta = \pm g_1 \quad \mbox{o} \quad \alpha +\beta = -(\alpha -\beta) = \pm g_1. \]
Para el primer caso, se obtiene $\alpha = \pm g_1$ y $\beta = 0$, mientras que para el segundo caso $\alpha =0$ y $\beta = \pm g_1$. En ambos casos se obtiene que $u$ es trivial. 
\end{proof*}
\begin{lema}\label{lema:segundoLema}
Las unidades del grupo-anillo $\mathds{Z}K_8$ son triviales. 
\end{lema} 
\begin{proof*}
En este punto, vale la pena recordar que \[ K_8 = \{ 1,a,b,ab,a^2,a^3,a^2b,ab^3  \}.\] Entonces, todo elemento $\alpha \in \mathds{Z}K_8$ es de la forma \[ \alpha = x_0 + x_1a + x_2b + x_3ab + y_0a^2 + y_1a^3 + y_2a^2b + y_3ab^3.\] Ahora, téngase en consideración al anillo de cuaterniones enteros, esto es, el anillo \[ H = \{ m_0 + m_1i+ m_2j +m_3k \colon m_0, m_1, m_2, m_3 \in \mathds{Z} \}.\] Es fácil ver que las únicas unidades de $H$ son $\pm 1, \pm i, \pm j, \pm k.$ Ahora considérese el epimorfismo $\phi \colon \mathds{Z}K_8 \to H$ dado por \[ \alpha \mapsto (x_0 - y_0) + (x_1 - y_1)i + (x_2 - y_2)j +(x_3-y_3)k. \]

Por ser un morfismo, si $\alpha$ es unidad en $\mathds{Z}K_8$ entonces $\phi(\alpha)$ es unidad de $H$; por lo tanto, para algún índice $r, \quad 0 \leq r \leq 3,$ se debe cumplir que 
\begin{eqnarray*}
x_r - y_r &=&1 \\  
x_s - y_s &=& 0 \mbox{ si } s \neq r .
\end{eqnarray*}
Por otro lado, es fácil notar que $a^2$  es central y que $\frac{K_8}{\langle a^2\rangle} \simeq C_2 \times C_2.$ Si se denota como $\bar{g}$ la clase de un elemento $g \in K_8$ bajo el cociente y como $\psi \colon \mathds{Z}K_8 \to \mathds{Z}\left(\dfrac{K_8}{\langle a^2\rangle}\right)$, la extensión de la proyección canónica $K_8 \to \left( \dfrac{K_8}{\langle a^2\rangle} \right)$ hacia $\mathds{Z}K_8$, se tiene que \[ \psi(\alpha) = (x_0 + y_0) +(x_1 + y_1) \bar{a} + (x_2 + y_2)\bar{b} + (x_3+y_3)\bar{ab}.\] Se sigue del lema anterior que las unidades de $\mathds{Z}(C_2\times C_2)$ son triviales. Así, para algún índice $h$, $0\leq h \leq 3,$ se tiene 
 \begin{eqnarray*}
 x_h + y_h &=& \pm 1 \\
 x_k + y_k &=& 0, \mbox{ si } h \neq k.
 \end{eqnarray*}
 Es fácil notar que $r = h$ y \[ x_r = \mp 1, y_r =0, x_s = y_s =0, \mbox{ si } s \neq r, \] o 
 \[ x_r =0 , y_r = \pm 1, x_s = y_s = 0 \mbox{ si } s \neq r.  \] En ambos casos se llega a que $\alpha$ es unidad trivial de $\mathds{Z}K_8$.
\end{proof*}
\begin{lema}\label{lema:tercerLema}
Sea $\zeta$ una raíz primitiva de la unida de orden 3 ó 4. Entonces, las unidades del anillo ciclotómico $\mathds{Z}[\zeta]$ son simplemente $\{\pm \zeta^{i} \}$.
\end{lema}
\begin{proof*}
Se considerará primero el caso en que $\zeta$ es una raíz cúbica de la unidad. Recordemos que el polinomio minimal de $\zeta$ es $X^2 + X + 1,$ así que todo elemento $\alpha \in \mathds{Z}[\zeta]$ es de la forma $\alpha = a+b\zeta$, con $a,b \in \mathds{Z}$. Supóngase que $\alpha$ es una unidad de $\mathds{Z}[\zeta]$. Dado que la aplicación $f \colon \mathds{Z}[\zeta]\to \mathds{Z}[\zeta]$ dada por $f(x+y) = x+y\zeta^2$ es un automorfismo, se sigue que $\alpha^{'} = a + b\zeta^2$ es también una unidad y así \[ \alpha\alpha^{'} = (a + b\zeta)(a + b\zeta^2) = a^2 + b^2 + ab(\zeta  + \zeta^2) = a^2 + b^2 -ab \] es también una unidad, pero $\alpha\alpha^{'} \in \mathds{Z}$, así que $a^2 + b^2 -ab = \pm 1$. Supóngase, sin pérdida de generalidad, que $\mid a \mid \geq \mid b \mid$. Si $b \neq 0$, se sigue $a^2 + b^2 > ab \pm 1,$ lo cual es una contradicción. 

Si $b = 0$, entonces $\alpha = a \in \mathds{Z}$ es una unidad y $\alpha = \pm 1$. Si $b = 1,$ se tiene que $a^2 + 1 = a \pm 1$ lo cual implica que $a^2 = a$ o  $a^2-a+2 = 0$. Para el primer caso se tiene $a = 0$  o $a=1$ y para el segundo caso no se tiene solución en los enteros. Si $a=0$ se tiene $\alpha = b\zeta$ y como $\mid \alpha \mid = 1$, se sigue que $\alpha = \pm \zeta$. Finalmente, si $a = b = 1$ se tiene que $\alpha =1 + \zeta = -\zeta^2$.
El caso en que $\zeta$ es raíz primitiva de la unidad de orden cuatro es aún más fácil, ya que $\zeta = i$ y los elementos en $\mathds{Z}[i]$ son de la forma $\alpha = a + bi, a,b \in \mathds{Z}$, es decir que $\alpha \in \mathds{C}$ y por lo tanto $a = \pm 1$ y $b = 0$ o $a = 0$ y $b = \pm 1$
\end{proof*}
\begin{teorema}[Higman]\label{teo:Higman}
Sea $G$ un grupo de torsión. Entonces, todas las unidades de $\mathds{Z}G$ son triviales si y sólo si $G$ es un grupo abeliano de exponente igual a $1,2,3,4$ o $6$ o $G$ es un 2-grupo hamiltoniano. 
\end{teorema}
\begin{proof*}
La condición necesaria ya se ha demostrado. Para probar la condición suficiente, considérese el caso en que $G$ es un grupo abeliano de exponente igual a $1,2,3,4$ o $6$ y supóngase que $G$ es finito. En este caso, el teorema \ref{teo:Perlis-Walker} asegura que \[ \mathds{Q}G \simeq \oplus_{d \mid n}a_d\mathds{Q}(\zeta_d)  \] donde $\zeta_d$ denota a las raíces primitivas de la unidad de orden $d$ y $a_d = \dfrac{\eta_d}{\mid K(\zeta_d : K \mid)}$. En esta fórmula, $\eta_d$ denota el número de elementos de orden $d$ en $G$. En otra palabras, solamente las raíces de la unidad cuyos órdenes son iguales a los órdenes de los elementos en $G$ aparecen en la descomposición.
Sea $R$ la preimagen, bajo el isomorfismo, del orden \[ M = \oplus_{d \mid n}a_d \mathds{Z}[\zeta_d].\] Nótese que si $G$ es como se propuso al inicio, entonces $G$ es de la forma 
\begin{eqnarray*}
G &\simeq& C_2 \times \cdots \times C_2, \\
G &\simeq& C_3 \times \cdots \times C_3, \\
G &\simeq& C_4 \times \cdots \times C_4,  \\
G &\simeq& C_2 \times \cdots \times C_2 \times C_3 \times \cdots \times C_3, \\
G &\simeq& C_2 \times \cdots \times C_2 \times C_4 \times \cdots \times C_4.
\end{eqnarray*}
Sin embargo, por el lema \ref{lema:primerLema}, se puede asumir que $G$ es del segundo o del tercer tipo. En ambos casos, se sigue del lema \ref{lema:tercerLema} que todas las unidades de $R$ son triviales y por lo tanto de orden finito. 
Como $\mathcal{U}(\mathds{Z}G)$ está contenido en $R$, también sus unidades son de orden finito, como $G$ es abeliano, se sigue que estas deben ser triviales.
En el caso en que $G$ es un 2-grupo hamiltoniano la conclusión se sigue directamente de los lemas \ref{lema:primerLema} y \ref{lema:tercerLema}
\end{proof*}

\addtocontents{toc}{\protect\addvspace{1.8em}}
\chapter{APLICACIONES}
En este capítulo se darán los conceptos básicos de teoría de códigos. Se empezará dando una descripción del sistema de comunicación como lo propuso Claude E. Shannon\footnote{Claude Elwood Shannon (Míchigan, 30 de abril de 1916 -- 24 de febrero de 2001) fue un ingeniero electrónico y matemático estadounidense, recordado como el padre de la teoría de la información.} en 1948. En esta parte también se introducirán todos los conceptos básicos del sistema de comunicación como canal, codificador, decodificador y código, además de hacer un breve estudio de los códigos lineales, para terminar este capítulo con una clase de códigos en particular, los cíclicos. 

\section{\quad Sistema de comunicación}
La figura \ref{fig:sistemaComunicacion} muestra un sistema de comunicación de una fuente a un destino mediante un canal. La comunicación puede ser en el dominio del espacio (es decir, de un punto a otro) o en el dominio del tiempo (al guardar información en algún punto en el tiempo para ser recuperada posteriormente).
\begin{figure}
\centering
\caption{\hskip 2em Sistema de comunicación propuesto por Shannon}
\vskip 1em
\includegraphics[angle=-90,width=0.9\linewidth]{Pictures/sistema.pdf}
\vskip 1.80em
\caption*{Fuente: elaboración propia, con software Dia.}
\label{fig:sistemaComunicacion}
\end{figure}
La codificación de la fuente tiene doble propósito. Primero, servir como traductor entre la salida de la fuente y la entrada al canal. Por ejemplo, si la información transmitida de la fuente al destino está en señal análoga y el canal espera recibir señal digital, se necesitará una conversión, de análoga a digital en la fase de codificación y un convertidor de señal digital a análoga en la fase de decodificación. 

Como segunda función se podría requerir que el codificador de la fuente comprima la salida de la fuente para economizar en la longitud de la transmisión, eso significa que en el otro extremo, el decodificador de la fuente necesitará descomprimir la señal.

Algunas aplicaciones necesitan que el decodificador restaure la información para que sea idéntica a la original, en cuyo caso se dice que la compresión es sin pérdidas.

Otras aplicaciones, como la mayoría de transmisiones de audio e imágenes, permiten una diferencia controlada o distorsión entre la información original y la restaurada, así que esta posibilidad es usada para lograr mayor compresión. En este caso se dice que la compresión es con pérdidas.

Los canales no son perfectos debido a limitaciones físicas y de ingeniería, es decir, su salida puede diferir de su entrada debido al ruido o a defectos de fabricación.

Más aún, en algunos casos el diseño requiere que el formato de la información de salida del canal difiera del formato de entrada. Además hay aplicaciones tales como los medios de almacenamiento masivo magnético y óptico, donde no se permiten ciertos patrones en el flujo de bits a transmitir. Dado esto, el rol principal del codificador del canal, es superar estas limitaciones y hacer el canal tan transparente  como sea posible, tanto desde el punto de vista de la fuente como del destino. 

Es así como entran a participar los códigos, estos fueron inventados para corregir errores en los canales de comunicación debido al ruido. Por ejemplo, supóngase que hay un cable telegráfico desde la ciudad de Guatemala hasta la ciudad de Panamá, mediante el cual se pueden transmitir unos y ceros. Usualmente cuando un cero es enviado se recibe un cero, pero ocasionalmente un cero puede ser recibido como un uno o un uno como un cero. Supóngase que en promedio, 1 de cada 100 símbolos se recibe de forma errónea, es decir, por cada símbolo hay una probabilidad $p=1/100$ de que ocurra un error en el canal. A esto se le llama un canal binario simétrico y se denota como BSC por sus siglas en inglés.
\vskip 1em 
\begin{figure}[h!]
\centering
\caption{\hskip 2em Canal binario simétrico}
\vskip 1em
% Graphic for TeX using PGF
% Title: /mnt/DataWin/Dropbox/Tesis/Tesis Actual/Pictures' Source/CBS.dia
% Creator: Dia v0.97.2
% CreationDate: Tue Feb 11 10:34:04 2014
% For: hugo
% \usepackage{tikz}
% The following commands are not supported in PSTricks at present
% We define them conditionally, so when they are implemented,
% this pgf file will use them.
\ifx\du\undefined
  \newlength{\du}
\fi
\setlength{\du}{15\unitlength}
\begin{tikzpicture}
\pgftransformxscale{0.800000}
\pgftransformyscale{0.800000}
\definecolor{dialinecolor}{rgb}{0.000000, 0.000000, 0.000000}
\pgfsetstrokecolor{dialinecolor}
\definecolor{dialinecolor}{rgb}{1.000000, 1.000000, 1.000000}
\pgfsetfillcolor{dialinecolor}
\pgfsetlinewidth{0.100000\du}
\pgfsetdash{}{0pt}
\pgfsetdash{}{0pt}
\pgfsetbuttcap
{
\definecolor{dialinecolor}{rgb}{0.000000, 0.000000, 0.000000}
\pgfsetfillcolor{dialinecolor}
% was here!!!
\pgfsetarrowsstart{to}
\definecolor{dialinecolor}{rgb}{0.000000, 0.000000, 0.000000}
\pgfsetstrokecolor{dialinecolor}
\draw (20.100000\du,3.050000\du)--(3.100000\du,3.100000\du);
}
\pgfsetlinewidth{0.100000\du}
\pgfsetdash{}{0pt}
\pgfsetdash{}{0pt}
\pgfsetbuttcap
{
\definecolor{dialinecolor}{rgb}{0.000000, 0.000000, 0.000000}
\pgfsetfillcolor{dialinecolor}
% was here!!!
\pgfsetarrowsstart{to}
\definecolor{dialinecolor}{rgb}{0.000000, 0.000000, 0.000000}
\pgfsetstrokecolor{dialinecolor}
\draw (20.005100\du,13.919000\du)--(3.005150\du,13.969000\du);
}
\pgfsetlinewidth{0.100000\du}
\pgfsetdash{}{0pt}
\pgfsetdash{}{0pt}
\pgfsetbuttcap
{
\definecolor{dialinecolor}{rgb}{0.000000, 0.000000, 0.000000}
\pgfsetfillcolor{dialinecolor}
% was here!!!
\pgfsetarrowsstart{to}
\definecolor{dialinecolor}{rgb}{0.000000, 0.000000, 0.000000}
\pgfsetstrokecolor{dialinecolor}
\draw (20.050000\du,13.800000\du)--(3.200000\du,3.100000\du);
}
\pgfsetlinewidth{0.100000\du}
\pgfsetdash{}{0pt}
\pgfsetdash{}{0pt}
\pgfsetbuttcap
{
\definecolor{dialinecolor}{rgb}{0.000000, 0.000000, 0.000000}
\pgfsetfillcolor{dialinecolor}
% was here!!!
\pgfsetarrowsstart{to}
\definecolor{dialinecolor}{rgb}{0.000000, 0.000000, 0.000000}
\pgfsetstrokecolor{dialinecolor}
\draw (20.000000\du,3.050000\du)--(3.100000\du,13.850000\du);
}
% setfont left to latex
\definecolor{dialinecolor}{rgb}{0.000000, 0.000000, 0.000000}
\pgfsetstrokecolor{dialinecolor}
\node[anchor=west] at (2.000000\du,3.200000\du){$0$};
% setfont left to latex
\definecolor{dialinecolor}{rgb}{0.000000, 0.000000, 0.000000}
\pgfsetstrokecolor{dialinecolor}
\node[anchor=west] at (2.000000\du,14.250000\du){$1$};
% setfont left to latex
\definecolor{dialinecolor}{rgb}{0.000000, 0.000000, 0.000000}
\pgfsetstrokecolor{dialinecolor}
\node[anchor=west] at (9.700000\du,1.650000\du){$1-p$};
% setfont left to latex
\definecolor{dialinecolor}{rgb}{0.000000, 0.000000, 0.000000}
\pgfsetstrokecolor{dialinecolor}
\node[anchor=west] at (10.550000\du,14.700000\du){$1-p$};
% setfont left to latex
\definecolor{dialinecolor}{rgb}{0.000000, 0.000000, 0.000000}
\pgfsetstrokecolor{dialinecolor}
\node[anchor=west] at (9.650000\du,6.450000\du){$p$};
% setfont left to latex
\definecolor{dialinecolor}{rgb}{0.000000, 0.000000, 0.000000}
\pgfsetstrokecolor{dialinecolor}
\node[anchor=west] at (9.400000\du,10.550000\du){$p$};
% setfont left to latex
\definecolor{dialinecolor}{rgb}{0.000000, 0.000000, 0.000000}
\pgfsetstrokecolor{dialinecolor}
\node[anchor=west] at (20.400000\du,3.150000\du){$0$};
% setfont left to latex
\definecolor{dialinecolor}{rgb}{0.000000, 0.000000, 0.000000}
\pgfsetstrokecolor{dialinecolor}
\node[anchor=west] at (21.350000\du,14.000000\du){};
% setfont left to latex
\definecolor{dialinecolor}{rgb}{0.000000, 0.000000, 0.000000}
\pgfsetstrokecolor{dialinecolor}
\node[anchor=west] at (20.250000\du,14.150000\du){$1$};
\end{tikzpicture}

\vskip 1.80em
\caption*{Fuente: elaboración propia, con software Dia.}
\label{fig:CBS}
\end{figure} 
\hskip 1cm Además supóngase que se enviarán muchos mensajes por ese cable y se necesita enviarlos de manera rápida y segura. Los mensajes ya se encuentran escritos como cadenas de ceros y unos, producidos, quizás, por alguna computadora.

Se van a codificar estos mensajes para darles una protección en contra del ruido del canal. Un bloque de $k$ símbolos del mensaje $u = u_1\dots u_k,\ u_i = 0 \mbox{ o } 1$, será codificado como una palabra-código $x = x_1 \dots x_n, x_i = 0 \mbox{ o } 1$ donde $n \geq k$ (véase la figura \ref{fig:codificacion}). Estas palabra-códigos forman un código.
\vskip 1em 
\captionsetup[figure]{labelformat=simple, labelsep=period}
\begin{figure}[h!]
\caption{\hskip 2em Proceso de codificación}
\centering
\vskip 1em
% Graphic for TeX using PGF
% Title: /mnt/DataWin/Dropbox/Tesis/Tesis Actual/codi.dia
% Creator: Dia v0.97.2
% CreationDate: Wed Dec 11 21:23:59 2013
% For: hugo
% \usepackage{tikz}
% The following commands are not supported in PSTricks at present
% We define them conditionally, so when they are implemented,
% this pgf file will use them.
\ifx\du\undefined
  \newlength{\du}
\fi
\setlength{\du}{15\unitlength}
\begin{tikzpicture}
\pgftransformxscale{0.700000}
\pgftransformyscale{-1.000000}
\definecolor{dialinecolor}{rgb}{0.000000, 0.000000, 0.000000}
\pgfsetstrokecolor{dialinecolor}
\definecolor{dialinecolor}{rgb}{1.000000, 1.000000, 1.000000}
\pgfsetfillcolor{dialinecolor}
\pgfsetlinewidth{0.100000\du}
\pgfsetdash{}{0pt}
\pgfsetdash{}{0pt}
\pgfsetmiterjoin
\definecolor{dialinecolor}{rgb}{1.000000, 1.000000, 1.000000}
\pgfsetfillcolor{dialinecolor}
\fill (3.731981\du,6.945629\du)--(3.731981\du,13.482575\du)--(9.631981\du,13.482575\du)--(9.631981\du,6.945629\du)--cycle;
\definecolor{dialinecolor}{rgb}{0.000000, 0.000000, 0.000000}
\pgfsetstrokecolor{dialinecolor}
\draw (3.731981\du,6.945629\du)--(3.731981\du,13.482575\du)--(9.631981\du,13.482575\du)--(9.631981\du,6.945629\du)--cycle;
% setfont left to latex
\definecolor{dialinecolor}{rgb}{0.000000, 0.000000, 0.000000}
\pgfsetstrokecolor{dialinecolor}
\node[anchor=west] at (4.006981\du,9.767031\du){Fuente del };
% setfont left to latex
\definecolor{dialinecolor}{rgb}{0.000000, 0.000000, 0.000000}
\pgfsetstrokecolor{dialinecolor}
\node[anchor=west] at (4.006981\du,11.089242\du){Mensaje};
\pgfsetlinewidth{0.100000\du}
\pgfsetdash{}{0pt}
\pgfsetdash{}{0pt}
\pgfsetmiterjoin
\definecolor{dialinecolor}{rgb}{1.000000, 1.000000, 1.000000}
\pgfsetfillcolor{dialinecolor}
\fill (13.386981\du,8.142031\du)--(13.386981\du,12.282031\du)--(21.636981\du,12.282031\du)--(21.636981\du,8.142031\du)--cycle;
\definecolor{dialinecolor}{rgb}{0.000000, 0.000000, 0.000000}
\pgfsetstrokecolor{dialinecolor}
\draw (13.386981\du,8.142031\du)--(13.386981\du,12.282031\du)--(21.636981\du,12.282031\du)--(21.636981\du,8.142031\du)--cycle;
% setfont left to latex
\definecolor{dialinecolor}{rgb}{0.000000, 0.000000, 0.000000}
\pgfsetstrokecolor{dialinecolor}
\node[anchor=west] at (14.561981\du,10.357031\du){Codificador};
\pgfsetlinewidth{0.100000\du}
\pgfsetdash{}{0pt}
\pgfsetdash{}{0pt}
\pgfsetbuttcap
{
\definecolor{dialinecolor}{rgb}{0.000000, 0.000000, 0.000000}
\pgfsetfillcolor{dialinecolor}
% was here!!!
\pgfsetarrowsstart{to}
\definecolor{dialinecolor}{rgb}{0.000000, 0.000000, 0.000000}
\pgfsetstrokecolor{dialinecolor}
\draw (13.386981\du,10.212031\du)--(9.631981\du,10.214102\du);
}
\pgfsetlinewidth{0.100000\du}
\pgfsetdash{}{0pt}
\pgfsetdash{}{0pt}
\pgfsetbuttcap
{
\definecolor{dialinecolor}{rgb}{0.000000, 0.000000, 0.000000}
\pgfsetfillcolor{dialinecolor}
% was here!!!
\pgfsetarrowsstart{to}
\definecolor{dialinecolor}{rgb}{0.000000, 0.000000, 0.000000}
\pgfsetstrokecolor{dialinecolor}
\draw (26.770935\du,10.256598\du)--(21.636981\du,10.212031\du);
}
% setfont left to latex
\definecolor{dialinecolor}{rgb}{0.000000, 0.000000, 0.000000}
\pgfsetstrokecolor{dialinecolor}
\node[anchor=west] at (9.731981\du,11.042031\du){Mensaje};
% setfont left to latex
\definecolor{dialinecolor}{rgb}{0.000000, 0.000000, 0.000000}
\pgfsetstrokecolor{dialinecolor}
\node[anchor=west] at (9.731981\du,11.842031\du){$u_1\dots u_k$};
% setfont left to latex
\definecolor{dialinecolor}{rgb}{0.000000, 0.000000, 0.000000}
\pgfsetstrokecolor{dialinecolor}
\node[anchor=west] at (21.381981\du,11.292031\du){Palabra-código};
% setfont left to latex
\definecolor{dialinecolor}{rgb}{0.000000, 0.000000, 0.000000}
\pgfsetstrokecolor{dialinecolor}
\node[anchor=west] at (21.881981\du,12.092031\du){$x_1\dots x_n$};
\pgfsetlinewidth{0.100000\du}
\pgfsetdash{}{0pt}
\pgfsetdash{}{0pt}
\pgfsetmiterjoin
\definecolor{dialinecolor}{rgb}{1.000000, 1.000000, 1.000000}
\pgfsetfillcolor{dialinecolor}
\fill (26.820292\du,7.014162\du)--(26.820292\du,13.551108\du)--(32.720292\du,13.551108\du)--(32.720292\du,7.014162\du)--cycle;
\definecolor{dialinecolor}{rgb}{0.000000, 0.000000, 0.000000}
\pgfsetstrokecolor{dialinecolor}
\draw (26.820292\du,7.014162\du)--(26.820292\du,13.551108\du)--(32.720292\du,13.551108\du)--(32.720292\du,7.014162\du)--cycle;
% setfont left to latex
\definecolor{dialinecolor}{rgb}{0.000000, 0.000000, 0.000000}
\pgfsetstrokecolor{dialinecolor}
\node[anchor=west] at (28.299923\du,10.484211\du){Canal};
% setfont left to latex
\definecolor{dialinecolor}{rgb}{0.000000, 0.000000, 0.000000}
\pgfsetstrokecolor{dialinecolor}
\node[anchor=west] at (28.168743\du,15.632869\du){Ruido};
\pgfsetlinewidth{0.100000\du}
\pgfsetdash{}{0pt}
\pgfsetdash{}{0pt}
\pgfsetbuttcap
{
\definecolor{dialinecolor}{rgb}{0.000000, 0.000000, 0.000000}
\pgfsetfillcolor{dialinecolor}
% was here!!!
\pgfsetarrowsstart{to}
\definecolor{dialinecolor}{rgb}{0.000000, 0.000000, 0.000000}
\pgfsetstrokecolor{dialinecolor}
\draw (29.770292\du,13.551108\du)--(29.802718\du,15.192715\du);
}
\end{tikzpicture}

\vskip 1.80em
\caption*{Fuente: elaboración propia, con software Dia y exportado a \TeX}
\label{fig:codificacion}
\end{figure}

La primera parte de la palabra-código consiste en el mensaje mismo: \[ x_1 = u_1,\ x_2 = u_2,\ \dots ,\ x_k = u_k, \] seguido de $n-k$ símbolos de comparación $x_{k+1}, \dots , x_n$. 
Los símbolos de comparación son elegidos de tal forma que las palabra-códigos satisfagan 
\[ H \begin{pmatrix}
x_1 \\ 
x_2 \\
\vdots \\
x_n
\end{pmatrix} = Hx^{t} = 0, \] donde la matriz $H$ de $(n-k)\times k$ es la matriz de comparación de paridad del código, dada por: 
\begin{equation}\label{ecu:defCodigo}
H = [A \mid I_{n-k}],
\end{equation}  
donde $A$ es una matriz fija de $(n-k)\times k$ de ceros y unos e: \[ I = \begin{pmatrix}
1 & 0 & \dots & 0 \\
0 & 1 & \dots & 0 \\
\vdots & \vdots & \ddots & \vdots \\
0 & 0 & \dots & 1
\end{pmatrix} \] es la matriz identidad de $(n-k) \times (n-k)$. La aritmética en la ecuación~ \eqref{ecu:defCodigo} se hace en módulo 2, es decir que se está trabajando con el campo $\mathds{Z}_2$. 
\begin{ejemplo}
La matriz de comparación de paridad 
\[ H = \left(\begin{array}{ccc|ccc}
0 & 1 & 1 & 1 & 0 & 0 \\
1 & 0 & 1 & 0 & 1 & 0 \\
1 & 1 & 0 & 0 & 0 & 1
\end{array}\right) \] define un código con $k=3$ y $n=6$. Para este código: 
\[ A = \begin{pmatrix}
0 & 1 & 1\\
1 & 0 & 1 \\
1 & 1 & 0
\end{pmatrix}.  \]
El mensaje $u_1u_2u_3$ es codificado en la palabra-código $x = x_1x_2x_3x_4x_5x_6$ que empieza con el propio mensaje:
\[ x_1 = u_1, x_2 = u_2, x_3 = u_3, \] seguido de tres símbolos de comparación $x_4x_5x_6$ tales que $Hx^t = 0$, es decir, se cumple:
\begin{equation}\label{ecu:comparacion}
	\begin{split}
	x_2 + x_3 + x_4 &= 0,  \\
	x_1 + x_3 + x_5 &= 0,\\
	x_1 + x_2 + x_6 &= 0. 	
	\end{split}
\end{equation}
Si el mensaje es $u = 011$, entonces $x_1= 0,\  x_2 = 1, \ x_3 =1$ y los símbolos de comparación son:
\begin{eqnarray*}
x_4 & = & 0 \\
x_5 &=& 1 \\
x_6 &=& 1
\end{eqnarray*}
por lo que la palabra-código es $x = 011011$.
\end{ejemplo}
Las ecuaciones dadas por~\eqref{ecu:comparacion} son llamadas de comparación de paridad del código. Las ecuaciones de paridad dicen que el cuarto, quinto y sexto símbolo siempre deben sumar cero módulo 2, es decir, su suma siempre es par. 

Es fácil notar que el número de palabra-códigos posibles para esta matriz es $2^3 = 8$. Estas son:
\[ \begin{array}{ccc}
000000 & 011011 & 110110 \\
001110 & 100011 & 111000 \\
010101 & 101101 & 
\end{array}. \]
En general en un código hay $2^k$ palabra-códigos, si el alfabeto es $\{0,1\}$.

Luego de haber visto de manera intuitiva un código lineal, se da la definición formal:
\begin{definicion}
Un código de bloques lineal sobre GF(q) de longitud $n$ y dimensión $k$, es un subespacio de $V_n(q)$ de dimensión $k$, donde $V_n(q)$ es un espacio vectorial de dimensión $n$ sobre GF(q). Se denota a este como código lineal $(n,k)$.
\end{definicion}
El término código de bloques se refiere al hecho que toda palabra-código tiene la misma longitud, es decir, es una $n\mbox{-upla}$. La palabra lineal se deriva del simple hecho que las palabra-códigos forman un subespacio. 
También existen los códigos de árbol, de los cuales los códigos de convolución son un caso especial. Dichos códigos no dividen el mensaje en bloques. En la práctica de la ingeniería estos códigos son bastante importantes, pero no se hablará de estos en este trabajo.
\begin{definicion}
La distancia de Hamming $d(u,v)$ entre dos $n\mbox{-uplas}$ $u$ y $v$ de $V_n(q)$ está definida como el número de coordenadas en el cual difieren. El peso de Hamming $\omega(u)$ de una $n\mbox{-upla } u \in V_n(q)$ es el número de coordenadas no nulas de $u$.
\end{definicion}
La mayor parte del trabajo en teoría de códigos se desarrolla en base a la distancia de Hamming. Otra métrica con la que también se trabaja es la de Lee.

Como se dijo anteriormente, en la transmisión de una palabra-código en algún canal, se puede producir errores debido al ruido. Por error, se entiende que puede ocurrir un cambio en cualquiera de las coordenadas de la $n\mbox{-upla}$ trasmitida. Justamente el punto de codificar es que, bajo ciertas circunstancias, estos errores se pueden corregir.

\section{\hskip 1em Códigos cíclicos} 

Ya que se ha hecho la presentación de los códigos lineales, se procede a estudiar una clase muy particular e importante de ellos; los códigos cíclicos.

\begin{definicion}
Un código es cíclico si es lineal y además todo desplazamiento cíclico de las coordenadas de una palabra-código es también una palabra-código. 
\end{definicion}
Los códigos cíclicos figuran entre los primeros que aparecieron para el uso práctico, estos eran implementados usando registros de desplazamientos. 
Para poder hacer un buen diseño de códigos es necesario conocer la estructura algebraica de los mismos. En esta sección se abordará la estructura algebraica de los códigos cíclicos.
El lector deberá notar que los códigos cíclicos no dependen de la linealidad de los mismos, de hecho, también existen los códigos cíclicos no lineales.

Resulta muy conveniente identificar a los códigos cíclicos con polinomios, esto es, a cada palabra-código:
\[ a = (a_0, a_1, \dots , a_{n-1}) \in \mathds{F}^n \] se le asocia el polinomio: \[ a(x) = a_0 + a_1x+ \cdots + a_{n-1}x^{n-1} \in \mathds{F}[x]_n .\] 

Si $c$ es una palabra-código del código $\mathcal{C}$, entonces $c(x)$ es su polinomio-código asociado. Con esta identificación, la palabra-código con corrimiento $\tilde{c}$ tiene el polinomio-código asociado: \[ \tilde{c}(x) = c_{n-1} + c_0x + c_1x^2 + \cdots + c_{n-2}x^{n-2}. \] 

Así, es tentador pensar que $\tilde{c}(x) $ es casi igual al producto $xc(x)$. Más aún, 
\[ \tilde{c}(x) = xc(x) - c_{n-1}(x^n -1)  \] de donde se deduce que: \[ \tilde{c}(x) = xc(x) (\mbox{mód } x^n-1) .\] 

Dicho de otra manera $\tilde{c}(x)$y $xc(x)$ son iguales en el anillo de polinomios $\mathds{F}[x]$ con multiplicación módulo $x^n-1$. Como se esta trabajando con códigos, según la definición $\mathds{F} = GF(q) $ y $GF(q)[x]$ es el anillo de polinomios sobre $GF(q)$ en la variable $x$. Entonces, por lo descrito en el párrafo anterior, resulta claro que hay un isomorfismo natural entre $GF(q)[x]/(x^n-1)$ y los polinomios de grado menor que $n$ con multiplicación definida módulo $x^n-1$. Se denota al anillo $GF(q)[x]/(x^n-1)$ como $A_n$. Es claro que $A_n$ es también un álgebra sobre $GF(q)$. Más aún, si $C_n = \langle x \rangle,\ x^n =1$, es un grupo cíclico de orden $n$, es evidente que existe un isomorfismo entre $GF(q)C_n$ y $GF(q)[x]/(x^n-1) = A_n$, donde se ha utilizado el símbolo $x$ para ambas álgebras con el único propósito de hacer énfasis en esta identificación.

Lo anterior quiere decir que $A_n \simeq GF(q)C_n$, pero $GF(q)C_n$ es un grupo-álgebra, con lo cual se ha demostrado que los códigos cíclicos son un grupo-álgebra y por lo tanto su estructura algebraica está plenamente identificada. 
\begin{teorema}
Un código lineal $(n,k) \ \mathcal{C}$ sobre $GF(q)$ es cíclico si y sólo si dicho código es un ideal de $A_n$.
\end{teorema}
\begin{proof*}
Sea $\mathcal{C}$ un ideal de $A_n$. Es claro que $\mathcal{C}$ es un subespacio de $A_n$ como espacio vectorial, por lo que solamente falta verificar que es un subespacio cíclico, lo cual se sigue del hecho que $\mathcal{C}$ es ideal y por lo tanto cerrado bajo multiplicación por $x$.
Para el converso, supóngase que $\mathcal{C}$ es un subespacio cíclico, eso quiere decir que es cerrado bajo la suma y el producto por $x$ y por lo tanto es cerrado bajo el producto de cualquier elemento de $A_n$, entonces $\mathcal{C}$ es un ideal.
\end{proof*}
Como el lector podrá notar, el estudio de códigos cíclicos, se reduce a estudiar los ideales de $A_n$, que es un grupo-álgebra. Para ello será provechoso conocer cuando $A_n$ es semisimple o no, tarea que se realizó en el capítulo 3. 
Así por el corolario \ref{cor:car} se sabe que $A_n$ es semisimple si y solo si $\car(K) \nmid |G|$. En el caso de tener un campo binario, esta condición se reduce a exigir que $|G|$ sea un número impar. Supóngase que $\car(K) \nmid |G|$ entonces aplicando el teorema \ref{teo:idealIdem} se deduce que todo ideal de $A_n$ es de la forma $L=A_ne$, donde $e \in A_n$ es un elemento idempotente. De esto se obtiene el siguiente:
\begin{teorema}
Si $\car(K)\nmid|G|$ entonces el estudio de los grupos cíclicos es equivalente al estudio de ideales en grupo-álgebras generadas por elementos idempotentes.
\end{teorema}
Para el estudio de los ideales de $A_n$ es importante conocer la factorización de $x^n-1$. En general se desea que los factores de $x^n-1$ no sean de multiplicidad mayor a 1. Recordando que $GF(q)$ es el campo extensión de Galois con $p^n$ elementos entonces se puede verificar que $x^n-1$ no posee factores aparte de los lineales si y solo si $(n,q)=1$.
\begin{teorema}
El único polinomio mónico $g(x)$ de grado mínimo en un ideal $A$ de $A_n$ es un generador de $A$ y divide a $x^n-1$. La dimensión de $A$ es $n -\grado(g)$. Más aún, si $g(x)$ es un divisor de $x^n-1$ entonces también es un generador de un ideal $A$ en $A_n$.
\end{teorema}
Para la demostración del  teorema anterior, se sugiere consultar \cite[52]{bib:codeBook}. De la misma fuente se puede consultar la demostración del siguiente:
\begin{teorema}
Sean $A_1$ y $A_2$ dos ideales del grupo-álgebra $GF(q)C_n$ con polinomios generadores $g_1(x)$ y $g_2(x)$ respectivamente. Entonces
\begin{bulletList}
\newItem $A_1 \cup A_2$ es generado por $(g_1(x),g_2(x)$.
\newItem $A_1 \cap A_2$ es generado por $[g_1(x),g_2(x)]$.
\newItem $A_1A_2$ es generado por $(g_1(x)g_2(x),x^n-1)$.
\end{bulletList}
\end{teorema}
Un código $(n,k)$ cíclico sobre $GF(q)$ corresponde a un ideal $A_n$ que es generado por $g(x)$, un divisor de $x^n-1$. De esta forma, siendo $\sigma_q$ la función definida como $\sigma_{q^n-1} \equiv iq \pmod{q^n-1}$, se tiene según \cite[ 31]{bib:codeBook}, que el número de factores de $x^n-1$ es $\sigma_q(n)$ y de esta cuenta el número de códigos cíclicos de longitud $n$ sobre $GF(q)$,  $(n,q)=1$ es $2^{\sigma_q(n)}$, lo cual incluye los códigos triviales, a saber: el código $(0)$ y el grupo-álgebra. Esto complementa la teoría desarrollada en la sección \ref{sec:codigos} en la cual se demuestra que el grupo-álgebra $GF(q)C_n$ es suma directa de extensiones ciclotómicas de $GF(q)$.
Ahora bien, algunos de estos códigos podrían ser equivalentes, así que es de interés saber cuando los ideales del grupo-álgebra son equivalentes para lo cual se sugiere consultar \cite[43-45]{bib:codeBook} 
\newpage
\begin{ejemplo}
Considérese el grupo-álgebra del grupo cíclico $C_3$ sobre $GF(2)$ cuyos elementos son:\[ GF(2)C_3 = \{O, l,g,g^2, 1 +g, 1 +g^2,g+g^2, 1 +g+g^2\} \]
y como $C_3$ es abeliano, de la sección \ref{sec:codigos}, se tiene que los ideales del grupo-álgebra deben ser bilaterales. Esto implica que los ideales no pueden ser isomorfos a pares y además cada ideal contiene un idempotente central que los genera (véase teorema~\ref{teo:idealIdem}). Para esta álgebra en particular existen dos ideales no triviales, que son:
\begin{align*}
A_1 &= \{O, 1 + g + g2\}, \\
A_2 &=\{O,g+g^2, 1 +g^2, 1 +g\}
\end{align*}
con idempotentes generadores:
\begin{align*}
e_1 &= 1 + g + g^2\\
 e_2 &= g + g^2
\end{align*}
respectivamente. Además se sigue que: \[ GF(2)C_3 = A_1 \oplus A_2.\]
\end{ejemplo} 






%----------------Inicio del apartado Final de la tesis ------------------------------
% Modificación para tabla de contenidos
%\addtocontents{toc}{\protect\addvspace{+.8cm}}
     %En el apartado final, las páginas son numeradas como en apartado principal (la numeración continúa) 
     %pero los capítulos no son numerados. 
     \backmatter     		
     % Modificación para tabla de contenidos
     \addtocontents{toc}{\protect\addvspace{1.8em}}
     \chapter{CONCLUSIONES}


\begin{bulletList}
\item Para el estudio de los grupo-anillos es importante conocer la estructura de los grupos abelianos y hamiltonianos así como la teoría de módulos y el teorema de Wedderburn-Artin.
\item Las condiciones necesarias y suficientes para que un grupo-anillo sea semisimple vienen dadas por el teorema de Maschke.
\item Toda representación de un anillo conmutativo sobre un grupo dado corresponde a un módulo del grupo-anillo correspondiente.
\item En general no es fácil encontrar unidades no triviales en grupo-anillos, pero es posible construir algunas usando elementos idempotentes.
\item Las grupo-álgebras dan estructura matemática a los códigos correctores conocidos como cíclicos. 
\item Cuando la característica del campo no divide al orden del grupo el estudio de los códigos cíclicos se reduce a estudiar los ideales de grupo-álgebras generadas por elementos idempotentes.
\end{bulletList}

     \chapter*{RECOMENDACIONES}
\addcontentsline{toc}{chapter}{RECOMENDACIONES}


\begin{bulletList}

\item Usar el primer capítulo como guía de temas para el desarrollo de un curso de álgebra moderna de pregrado. 

\item Para el estudio de los códigos cíclicos se puede utilizar los resultados del capítulo 2, en especial los de las secciones 2.3 y 2.4.

\item Estudiar la teoría de códigos correctores desde el punto de vista algebraico, permitiendo así dictaminar la capacidad correctora de los mismos.

\item Para la lectura de los capítulos 2 y 3 es importante tener conocimientos previos de teoría de grupos y anillos. 

\item Implementar el estudio de teoría de módulos en el curso de álgebra 2 de la licenciatura en matemática aplicada de la USAC.
\end{bulletList}

     \begin{thebibliography}{99}
    \addcontentsline{toc}{chapter}{BIBLIOGRAFÍA}
    % LETRA B
    \bibitem{bib:libroLosGrandes} BELL, Eric. \textit{Los grandes matemáticos.} Argentina: Editorial Losada, 1948. 100~ p. 
    \bibitem{bib:codeBook} BLAKE, Ian.\textit{The mathematical theory of coding}. Estados Unidos: Academic Press Inc, 1975. 363 p.
    \bibitem{bib:burnside} BURNSIDE, William. \textit{The theory of groups of finite order}. 	2a ed. Cambridge: Cambridge University Press, 1911. 509 p.
        
    
    %LETRA C
     \bibitem{bib:Cauchy} CAUCHY, Augustin-Louis. Oeuvres complètes. Cambridge: Cambridge University Press, 2009. 524 p.
     
     %LETRA D
     \bibitem{bib:Deskins} DESKINS, Eugene. Finite abelian groups with isomorphic group algebras. \textit{Duke Mathematical Journal}. 1956, vol 23, núm. 1, p. 35-40.
     
     %LETRA F
      \bibitem{bib:solubilidad} FEIT, Walter, et al. The solvability of groups of odd order. \textit{Pacific J. Math}. 1963, vol 13, núm 3, p. 775-1029.
      
    %LETRA G
    \bibitem{bib:grupsfact}  GOLDSHMIDT, David. A group theoretic proof of the $p^aq^b$ theorem for odd primes. \textit{Mathematische Zeitschrift}. 1970, vol 113, núm. 5, p. 373-375.
    
    %LETRA H
     \bibitem{bib:historia} HAWKINS, Thomas. The origins of the theory of group characters. \textit{Archive for History of Exact Sciences}. 1971, vol 7, núm. 2, p. 142-170.
    \bibitem{bib:herstein} HERSTEIN, Nathain. \textit{Topics in algebra}. 2a ed.  New York: Macmillah, 1986.~ 400~ p. 
    
   %LETRA I
 	\bibitem{bib:Connell} IAN, Connell. On the group ring. \textit{Canad. J. Math}. 1963, vol 15, núm 1, p. 650-685.
 	
   \bibitem{bib:AlgebraPostGrado}ISAACS, Martin. \textit{Algebra: a graduate course}. Estados Unidos: Editorial Pacific Grove, 1940. 516 p.  
   
   %LETRA J
   \bibitem{bib:Jeanings} JENNINGS, Arthur. The structure of the group ring of a p-group over a modular field. \textit{Transactions of the American mathematical society}. 1941, vol. 50, núm 1, p. 175-185.
   
   
   %LETRA L    
    \bibitem{bib:lang}LANG, Serge. \textit{Linear algebra}. 3a ed. Nueva York: Springer-Verlag, 2004. 308~ p.
    
    %LETRA M
    \bibitem{bib:moser} MOSER, Claude. Représentation de -1 comme somme de carrés dans un corps cyclotomique quelconque. \textit{Journal of Number Theory}. 1973, vol 5, núm. 2, p. 138-141.
    
    %LETRA P
    \bibitem{bib:passman} PASSMAN, Donald. \textit{The algebraic structure of group rings}. New York: Wiley-Interscience, 1977. 550 p.
    \bibitem{bib:PerlisWalker} PERLIS, Sam; WALKER, Gordon. Abelian group algebras of finite order. \textit{Transactions of The American Mathematical Society}. 1950, vol 68, núm 3, p. 420-426.
    \bibitem{bib:main} POLCINO, César; SEHGAL, Sudarshan. \textit{An introduction to group rings}.  Dordrecht: Kluwer Academic Publishers, 2002. 371 p.
    
    %LETRA S
    \bibitem{bib:libroGuti} STEWART, Ian. \textit{De aquí al infinito -- Las matemáticas de hoy.} España: Editorial Crítica, 1998. 304 p.
    \bibitem{bib:Sehgal} SUDARSHAN, Sehgal.  \textit{Topics in group rings}. New York: Marcel Dekker, 1978. 233 p. 
    
    
    %LETRA Z
    
	\bibitem{bib:groupBook} ZASSENHAUS, Hans. \textit{The theory of groups}. Estados Unidos: Chelsea publishing company, 1937. 288 p. 
	
		
		
\end{thebibliography}

     
\end{document}




