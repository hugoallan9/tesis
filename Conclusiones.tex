\chapter{CONCLUSIONES}


\begin{finalList}
\item Para el estudio de los grupo-anillos es importante conocer la estructura de los grupos abelianos y hamiltonianos, así como la teoría de módulos y el teorema de Wedderburn-Artin.
\item Las condiciones necesarias y suficientes para que un grupo-anillo sea semisimple, vienen dadas por el teorema de Maschke.
\item Toda representación de un anillo conmutativo sobre un grupo dado, corresponde a un módulo del grupo-anillo correspondiente.
\item En general no es fácil encontrar unidades no triviales en grupo-anillos, pero es posible construir algunas usando elementos idempotentes.
\item Las grupo-álgebras dan estructura matemática a los códigos correctores conocidos como cíclicos. 
\item Cuando la característica del campo no divide al orden del grupo el estudio de los códigos cíclicos se reduce a estudiar los ideales de grupo-álgebras generadas por elementos idempotentes.
\end{finalList}
