\chapter{TEORÍA DE REPRESENTACIÓN DE GRUPOS}

\section{Definición y Ejemplos}

Como se mencionó en el capítulo 1 \footnote{ponerlo ejemplos en esta parte de la definición de grupos} el concepto de \textbf{grupo de permutaciones} fue dado explícitamente por primera vez en \footnote{poner el libro: Memorias de Galois} 1830, aunque la primera definición de grupo abstracto fue dado hasta en 1854 por Cayley, aunque pasó inadvertidamente por un tiempo, hasta que dicha definición fue dada nuevamente en repetidas ocasiones por varios matemáticos, a saber: Leopold Kronecker en 1870, Heinrich Martin Weber en 1882 y Ferdinand Georg Frobenius en 1887. De esa forma los grupos fueron considerados por mucho tiempo como objetos concretos antes de llegar a ser estudiados como estructuras algebraicas abstractas.

En este contexto histórico es natural hacer la pregunta: Dado un grupo abstracto ¿ Cómo se puede saber que grupo es -en  particular - ? Es decir, ¿ Se puede decir cuando es un grupo de permutaciones, un grupo lineal o un grupo de transformaciones proyectivas  - sólo por citar algunos ejemplos- ?

En 1879, durante las lecturas de un coloquio matemático realizado en Evanston, Illinois, Felix Klein planteó la posibilidad de representar un grupo abstracto dado como un grupo de transformaciones lineales \footnote{hacer referencia al libro T Hawkins, Hypercomplex number Lie groups and the creation of group representation theory, Archive Hist. Exact Sci. 8 (1972), pagina 269}.

Siguiendo estas ideas, Theodor Molien, Georg Frobenius, Issai Schur, William Burnside y  Heinrich Maschke desarrollaron la teoría básica de la representación de grupos al inicio del siglo XX y Burnside presentó la primera exposición  sistemática de este tema en su libro \footnote{referencia al libro}, que actualmente es considerado un libro clásico en este tema. 

La teoría de la representación se volvió mas importante a medida que se fueron obteniendo nuevos resultados.

Uno de los resultados mas importantes es el famoso teorema que establece que si $p$ y $q$ son números enteros primos y  $a$, $b$ enteros positivos, entonces cualquier grupo de orden $p^aq^b$ es soluble. \footnote{poner esto en el glosario o en donde corresponde: Un grupo G es soluble si hay una cadena de subgrupos ${e} = H_0 \subset H_1 \subset \cdots \subset H_n \subset H_n = G$ tal que para cada $i$, el subgrupo $H_i$ es normal en $H_{i+1}$ y el grupo cociente $H_{i+1}/H_i$ es abeliano.} Este teorema fue demostrado en 1904 por William Burnside usando la teoría de representación de grupos y, como dato curioso, la primera demostración que no utiliza dicha teoría fue proporcionada por John Griggs Thompson mas de 60 años después (ver \footnote{poner la bibliografia 48})

William Burnside también conjeturó que todo grupo de orden impar es soluble. Esta conjetura fue un problema abierto hasta que Walter Feit y John Thompson dieron una demostración de esta conjetura en 1963 \footnote{ver 39}, usando para ello teoría de la representación. 

Luego de hacer énfasis en la importancia histórica que tiene la teoría de representación de grupos, se entra a estudiar algunas definiciones de la misma.

\begin{definicion}
Sea $G$ un grupo, $R$ un anillo conmutativo y $V$ un $R\mbox{-módulo}$ libre de rango finito. Una \textbf{representación} de $G$ sobre $R$, con espacio de representación $V$, es un homomorfismo de grupos $T \colon G \to GL(V)$, donde $GL(V)$ es el grupo de automorfismos de $V$. El rango de $V$ es llamado \textbf{grado} de la representación $T$ y se denotará como $grad(T)$.
\end{definicion}

Para $g \in G$ se denotará como $T_g \colon V \to V$ al automorfismo correspondiente bajo $T$. Así, si $g, h \in G$, se tiene que $T_{gh} = T_g \circ T_h$ y $T_1 = I$.

El caso en el que $R$ es un campo es de particular importancia. Históricamente, este fue el primer caso que se estudió y es en ese contexto donde se obtuvieron la mayor parte de resultados. 

Si se escoge una $R\mbox{-base}$ de $V$, se puede definir un isomorfismo $\phi$ de $GL(V)$ al grupo $GL(n,R)$ de matrices invertibles $n\times n$ con coeficientes en $R$, asignándole a cada automorfismo $T \in GL(V)$ su matriz respecto a la base dada. Esto da paso a la siguiente definición:

\begin{definicion}
Sea $G$ un grupo y $R$ un anillo conmutativo. Una representación matricial de $G$ sobre $R$ de grado $n$ es un homomorfismo de grupos $T \colon G \to GL(n,R)$.


\end{definicion} 


Si $T \colon G \to GL(V)$ es una representación de $G$ sobre $R$ con espacio de representación $V$ y se considera el isomorfismo $\phi \colon GL(V) \to GL(n,R)$ asociada a alguna $R-\mbox{base}$, entonces $\phi \circ T \colon G \to GL(n, R)$ es una representación matricial de $G$. De manera similar, dada una representación matricial $T \colon G \to GL(n,R)$, entonces $\phi^{-1}\circ T \colon G \to GL(V)$ es una representación de $G$ sobre $R$. Debido a este hecho, no se hará distinción entre representación y representación matricial.

Para ilustrar lo que se expuso anteriormente, se ha considerado necesario, exponer algunos ejemplos sencillos.

\begin{ejemplo}
Dado un grupo $G$ y un anillo conmutativo $R$, la función $T \colon G \to GL(n,R)$ tal que a cada elemento $G$ le asocia la matriz identidad de $GL(n,R)$ es una representación matricial de $G$. A esta función  se le llama \textbf{representación trivial} de $G$ sobre $R$ de grado $n$. 
\end{ejemplo}


\begin{ejemplo}
Sea $G$ el grupo de Klein de cuatro elementos,es decir, $G = \{ 1, a, b, ab\}$. Este grupo tiene tres elementos de orden dos. Entonces $T \colon G \to GL(2, \mathds{Z})$ es la función tal que:

\[ \mathsf{T(1)} = \begin{pmatrix}
1 & 0 \\
0 & 1
\end{pmatrix}, \quad \mathsf{T(a)} = \begin{pmatrix}
1 & 0 \\
0 & -1
\end{pmatrix} \]

\[ \mathsf{T(b)} = \begin{pmatrix}
-1 & 0 \\
0 & 2
\end{pmatrix}, \quad \mathsf{T(ab)} = \begin{pmatrix}
-1 & 0 \\
0 & -1
\end{pmatrix} \]
\end{ejemplo}

\begin{ejemplo}
Sea $S_n$ el grupo de simetrías de $n$ símbolos y $R$ un anillo conmutativo. Sea $V$ un $R-\mbox{módulo}$ libre de rango $n$ con base $\{v_1, v_2, \cdots, v_n\}$. Para facilitar la comprensión de este ejemplo, se sugiere al lector imaginar que $V = \underset{n}{\underbrace{\mathds{R} \oplus \cdots \oplus \mathds{R}}}$ con su base canónica. 

Por otra parte, considérese la función $f \colon S_n \to GL(V)$ de la siguiente manera: a cada elemento $\sigma 
\in S_n$, se le asigna la función $T_{\sigma} \in GL(V)$, que actúa, de manera natural, como:

\[ T_{\sigma}(v_i) = v_{\sigma(i)}. \]

Como $T_{\sigma}$ deja a la base intacta (salvo permutaciones), es claro que $T_\sigma$ es un isomorfismo. 

Es claro que $T$ es un isomorfismo, por su definición, y por lo tanto una representación de $S_n$.

Como se puede apreciar una representación por si sóla puede ser poca descriptiva, por lo tanto se considera de mas utilidad conocer la representación matricial. Para este caso en particular, considérese $A(\sigma)$, la matriz asociada a $T_{\sigma}$, que se obtiene al calcular $T_{\sigma}(v_j)$ como combinación lineal de la base. Como $T_{\sigma} (v_j) = v_{\sigma(j)}$, entonces los coeficientes de la matriz anterior son cero en todas sus entradas excepto en $(\sigma(j),j)$, en la cual la entrada vale uno. De esta manera es fácil notar que $A(\sigma)$ es una matriz que tiene exactamente una entrada igual a uno en cada fila y columna y las demás iguales a cero. Dicha matriz se conoce como la \textbf{matriz de permutación}.

\end{ejemplo}

\begin{ejemplo}[La representación Regular]
Sea $G$ un grupo finito de orden $n$ y $R$ un anillo conmutativo. Se requiere definir una representación de $G$ sobre $R$, para ello se considerará  como espacio de representación a $RG$, es decir, a el grupo-anillo de $G$ sobre $R$. 

Considérese la función $T \colon G \to GL(RG)$ de la siguiente manera: a cada elemento $g \in G$ se le asigna la función lineal $T_g$ que transforma a los elementos de la base por medio de multiplicación por la izquierda, esto es, $T_g(g_i) = gg_i$. Es claro que $T$ es una representación de $G$, debido a que:

\[ T_{gh}(y) = (gh)y = g(h(y)) = T_gT_h(y).  \] 

En este caso hay que recordar que $G$ es una base de $RG$ sobre $R$ y se pueden enumerar, en algún orden, los elementos de $G$ como sigue:

\[ G = \{ 1=g_1, g_2, \cdots, g_n \}, \]  por lo tanto es fácil notar que en la correspondiente representación matricial con respecto a la base $G$ de $RG$, la imagen de cualquier elemento $g \in G$ es una matriz de permutación, debido a la cerradura del producto en $G$. 


\end{ejemplo}


La representación anterior usualmente es llamada la \textbf{representación regular de $G$ sobre $R$}. Es importante notar que esta representación se construyó a partir de la multiplicación por la izquierda, así que sería mas apropiado llamarla representación regular por la izquierda de $G$ sobre $R$.

Para ilustrar de mejor manera a continuación se muestra un ejemplo:

\begin{ejemplo}
Sea $G = \{ 1,a,a^2 \}$ un grupo cíclico de orden tres. Enúmerese los elementos de $G$ como $g_1 = 1$, $g_2 = a$, $g_3 = a^2$. Para encontrar la representación regular de $a$, basta con multiplicar por $a$ los elementos de $G$ por la izquierda:

\[  ag_1 = g_2, \quad ag_2 = g_3, \quad ag_3 = g_1 \] entonces se tiene:

\[  T_a(g_1) = g_2, \quad T_a(g_2) = g_3, \quad T_a(g_3) = g_1, \] por lo tanto la matriz asociada con $a$ en la base dada es:

\[ \mathsf{\rho(a)} =  \begin{pmatrix}
0 & 0 & 1 \\
1 & 0 & 0 \\
0 & 1 & 0
\end{pmatrix},  \] que no es más que una matriz de permutación.



\end{ejemplo}


\begin{ejemplo}
Considérese, de nuevo, el grupo de Klein de cuatro elementos, $G = \{ 1, a, b, ab\}$ con la numeración: $g_1 = 1$, $g_2 = a$, $g_3 = b$, $g_4 = ab$.

Para conocer la representación regular de $a$, se procede a multiplicar por  la izquierda por $a$ a los elementos de $G$:

\[ ag_1 = g_2, \quad ag_2 = g_1, \quad ag_3 = g_4, \quad ag_4 = g_3, \] entonces 

\[ T_a(g_1) = g_2 , \quad T_a(g_2) = g_1, \quad T_a(g_3) = g_4, \quad T_a(g_4) = g_3 \] y como en el ejemplo anterior, se puede obtener la representación matricial de $a$:

\[  \mathsf{\rho(a)} = \begin{pmatrix}
0 & 1 & 0 & 0 \\
1 & 0 & 0 & 0 \\
0 & 0 & 0 & 1 \\
0 & 0 & 1 & 0
\end{pmatrix} .\]

De manera similar se obtiene la representación matricial de los elementos restantes de $G$:

\[ \mathsf{\rho(b)} = \begin{pmatrix}
0 & 0 & 0 & 0\\
0 & 0 & 0 & 1\\
1 & 0 & 0 & 1\\
0 & 1 & 0 & 0
\end{pmatrix}, \quad \mathsf{\rho(ab)} = \begin{pmatrix}
0 & 0 & 0 & 1\\
0 & 0 & 1 & 0\\
0 & 1 & 0 & 0\\
1 & 0 & 0 & 0
\end{pmatrix} , \quad \mathsf{\rho(1)} = \begin{pmatrix}
1 & 0 & 0 & 0\\
0 & 1 & 0 & 0 \\
0 & 0 & 1 & 0\\
0 & 0 & 0 & 1
\end{pmatrix} \]

\end{ejemplo}

\begin{nota}
Ya se mencionó que $\rho(g)$ con $g \in G$ es una matriz de permutación, pero es importante hacer notar que si se toma $ 1 \neq g \in G$, entonces para cualquier $g_i \in G$ se tiene que $ gg_i \neq g_i $. Esto implica que para cualquier elemento $g_i$ de la base se cumple que $T_g(g_i) \neq g_i$ y por ende los elementos de la diagonal de $\rho(g)$ son todos iguales a cero. Más aún, de lo anteriormente expuesto, se deduce que si $g \neq 1$ entonces $tr(\rho(g)) = 0$ si $g \neq 1$ y $tr(\rho(g)) = |G|$ si $g = 1$. Este resultado elemental es de mucha importancia cuando se está trabajando con la representación regular.
\end{nota}

\begin{ejemplo}\label{rciclica}[Algunas representaciones de grupos cíclicos]
Considérese el grupo cíclico $G = \{1,a, \cdots, a^{m-1} \}$ y sea $K$ un campo. Si se desea construir una representación matricial $A \colon G \to GL(n,K)$ es necesario escoger la matriz $A(a)$, ya que por ser $A$ un homomorfismo, las matrices de representación de los restantes elementos del grupo están determinadas por $A(a^r) = (A(a))^r $. Además para demostrar que $A$ es un homomorfismo de grupos, basta con probar que $(A(a))^r = I$, para algún $r \in \mathds{Z}$.

Súpongase que $car(K) \nmid m$ y que $K$ contiene una raíz primitiva de la unidad de orden $m$,  $\xi$. Entonces 
\[ A \colon G \to GL(1,K) \]  tal que, $A(a)  = \xi$ es una representación, ya que $(A(a))^r = \xi^r = 1$ para algún $r$.  Además, si $\{ \xi_1, \cdots, \xi_m \}$ es un conjunto de todas las raíces de la unidad unidad de ordem $m$ que son distintas a pares entonces la función $B \colon G \to GL(m,K)$ dada por 

\[ \mathsf{B(a)} = \begin{pmatrix}
\xi_1 &  &\dots &  0 \\
0 & \xi_2 & \dots & 0 \\
 & & \dots &  \\
 0 & 0 & \dots & \xi_m
\end{pmatrix} \] es una represetanción de $G$ sobre $K$ de grado $m$, ya que $\xi_i^r = 1$ para algún $r \in \mathds{Z}$, entonces 

\[ \mathsf{(B(a))^r} = \begin{pmatrix}
\xi_1^r &  &\dots &  0 \\
0 & \xi_2^r & \dots & 0 \\
 & & \dots &  \\
 0 & 0 & \dots & \xi_m^r
\end{pmatrix} = I. \]

Nótese que esta representación es distinta a la representación regular, que en el caso de $a$, está dada por 

\[ \mathsf{\Gamma}(a) = \begin{pmatrix}
0 & 0 & \dots & 0 & 1 \\
1 & 0 & \dots & 0 & 0\\
0 & 1 & \dots & 0 & 0\\
 &  &  \dots &  &  \\
 0 & 0 & \dots & 1 & 0
\end{pmatrix}. \]
Finalmente si $car(K) \mid m$ entonces se propone la representación $C \colon G \to GL(2,K)$, dada por

\[ \mathsf{C(a)} = \begin{pmatrix}
1 & 1\\
0 & 1
\end{pmatrix} \] ya que $\mathsf{(C(a))^r} = \begin{pmatrix}
1 & r\cdot 1\\
0 & 1
\end{pmatrix} = I$ para $r \in \mathds{Z}$, esto por que $car(K) < \infty $.
\end{ejemplo} 

\begin{ejemplo}[Representación de $D_4$]
Considérese el grupo de simetrías de un cuadrado. Este grupo de 8 elementos, a saber, las reflecciones a través de los ejes $r_1, r_2, r_3, r_4$ y las rotaciones con ángulos $\frac{\pi}{2}$, $\pi$, $\frac{3\pi}{2}$ y $2\pi$ alrededor del centro.  

Sea $a$ la rotación de ángulo $\frac{\pi}{2}$ y $b$ la reflexión a través del eje $r_2$. Es fácil ver, bajo consideraciones geométricas, que cualquier otro elemento de este grupo se puede obtener por medio de $a$ y $b$.

De manera mas abstracta, este grupo --que es llamado grupo dihédrico de orden ocho y usualmente denotado por $D_4$ -- puede ser definido con dos generadores que satisfacen las relaciones
\[ a^4 = 1, \quad b^2 = 1 , \quad baba = 1.  \]
Por lo tanto este grupo puede ser descrito  como 
\[  D_4 = \{ 1, a, a^2, a^3, b, ab, a^2b, a^3b \}. \]
Como todas los elementos de este grupo están en terminos de $a$ y $b$, entonces para encontrar una representación matricial $A \colon D_4 \to GL(n,K)$ sobre el campo $K$, será suficiente encontrar matrices $A(a)$, $B(b)$ tales que $A(a)^4 = I$, $A(b)^2 = I$, $A(b)A(a)A(b)A(a) = I.$

Es fácil determinar representaciones de grado uno para $D_4$ en un campo $K$ de característica diferente a dos, de la siguiente manera:

\begin{equation*}
\begin{aligned}
A(a) & = 1\\
B(a) & = 1 \\ 
C(a) & = -1 \\
D(a) & = -1
\end{aligned}
\qquad 
\begin{aligned}
A(b) & = 1\\
B(b) & =  -1 \\
C(b) & = 1 \\
D(b) & = -1.
\end{aligned}
\end{equation*}

Pensando en el significado geométrico de $a$ y $b$, como dos funciones del plano al plano, se puede calcular sus matrices con respecto a la base canónica para obtener otra representación matricial de $D_4$:
\[ \mathsf{W(a)} = \begin{pmatrix}
0 & -1 \\
1 & 0
\end{pmatrix}, \quad
\mathsf{W(b)} = \begin{pmatrix}
 0 & 1 \\
 1 & 0
\end{pmatrix}. \]

\end{ejemplo}

\begin{ejemplo}[Suma directa de representaciones]
Sean $T \colon G \to GL(V)$ y $S \colon G \to GL(W)$ dos representaciones de un grupo $G$ sobre un anillo conmutativo $R$. Se puede definir una nueva representación $V \oplus W$, que es llamada la \textbf{suma directa} de dos representaciones dadas y se denota como $T \oplus S$, de la siguiente manera: \[ (T \oplus S)_g = T_g \oplus S_g, \quad \mbox{para cada } g \in G. \]

Si se eligen bases $\{v_1, \dots, v_n\}$ y $\{ w_1, \dots, w_m \}$ de $V$ y $N$ respectivamente y se denota por $g \mapsto A(g)$ y $g \mapsto B(g)$ las correspondientes representaciones matriciales en las bases dadas, entonces la representación matricial asociada a $T \oplus S$ con respecto a la base $\{ (v_1,0), \dots , (v_n), (0,w_1), \dots, (0,w_n)  \}$ de $V \oplus W$, viene dada por
\[ g \mapsto \begin{pmatrix}
A(g) & 0 \\
0 & B(g)
\end{pmatrix}. \]

\end{ejemplo}

Los ejemplos anteriormente expuestos sirven de motivación para introducir algunos conceptos de teoría de la representación. En este trabajo se restringirán las representaciones al caso en el cual $R$ es un campo, debido a que con este caso se logra ilustrar la relación de teoría de representación con los problemas de grupo-anillos.

Primero considérese $T \colon G \to GL(V)$ una representación de un grupo $G$ sobre un campo $K$ y asúmase que $\phi \colon V \to W$ es un isomorfismo de espacios vectoriales sobre $K$. Entonces se puede definir una nueva representación $\overline{T} \colon G \to GL(W)$ por medio de $\overline{T_g} \colon \phi \circ T_g \circ \phi^{-1}$ para todo $g \in G$. Esto es, escensialmente, una copia de $T$. La relación entre estas dos representaciones está ilustrada en el siguiente diagrama:
\[\xymatrix { V \ar[r]^{T_g} 
\ar[d]_{\phi}
 & V \ar[d]^{\phi} \\
W \ar[r]_{\overline{T_g}} & W}.\]
Lo cual sugiere lo siguiente:
\begin{definicion}
Dos representaciones $T \colon G \to GL(V)$ y $\overline{T} \colon G \to GL(W)$ de un grupo $G$ sobre el mismo campo $K$ se dicen que son \textbf{equivalentes} si existe un isomorfismo $\phi \colon V \to W$ tal que $\overline{T_g} = \phi T_g \phi^{-1} $ para cualquier $g \in G$.
\end{definicion}
\begin{definicion}
Dos representaciones matriciales $A \colon G \to GL(n,K)$ y $B \colon G \to GL(n,K)$ de un grupo $G$ sobre un campo $K$ se dicen equivalentes si existe una matriz invertible $U \in GL(n,K)$ tal que $A(g) = UB(g)U^{-1}$ para cualquier $g \in G$.
\end{definicion}

\begin{ejemplo}\label{ejemciclico}
Sea $G$ un grupo cíclico de orden $m$ y $K$ un campo que contiene a $\{ \xi_1, \xi_2, \dots, \xi_m \}$, el conjunto de todas las raíces distintas de la unidad de orden $m$. Entonces, si se consideran las representaciones $B$ y $\Gamma$ dadas en el ejemplo \ref{rciclica} con 
\[ \mathsf{U} = \begin{pmatrix}
\xi_1 & \xi_1^2 & \cdots & \xi_1^m \\
\xi_2 & \xi_2^2 & \cdots & \xi_2^m \\
 & & \cdots & \\
 \xi_m & \xi_m^2 & \cdots & \xi_m^m \\
 
\end{pmatrix}, \quad \mathsf{U} \in GL(n,K)\footnote{Esto es evidente, ya que $\mathsf{U}$ es una matriz de Vandermonde con $det(\mathsf{U}) = \prod_{1 \leq i < j \leq m}(\xi_i - \xi_j) \neq 0$.} \] entonces, calculando por un lado se tiene 
\[ \mathsf{B(a)U} = \begin{pmatrix}
\xi_1 & 0 & \cdots & 0\\
0 & \xi_2 & \cdots & 0\\
 & & \cdots & \\
 0 & 0 & \cdots & \xi_m
\end{pmatrix} \begin{pmatrix}
\xi_1 & \xi_1^2 & \cdots & \xi_1^m \\
\xi_2 & \xi_2^2 & \cdots & \xi_2^m \\
 & & \cdots & \\
 \xi_m & \xi_m^2 & \cdots & \xi_m^m
\end{pmatrix} = \begin{pmatrix}
\xi_1^2 & \xi_1^3 & \cdots & \xi_1 \\
\xi_2^2 & \xi_2^3 & \cdots & \xi_2 \\
 & & \cdots & \\
\xi_m^2 & \xi_m^2 & \cdots & \xi_m
\end{pmatrix} \] similarmente \[ 
\mathsf{U\Gamma(a) } = 
\begin{pmatrix}
\xi_1 & \xi_1^2 & \cdots & \xi_1^m \\
\xi_2 & \xi_2^2 & \cdots & \xi_2^m \\
 & & \cdots & \\
 \xi_m & \xi_m^2 & \cdots & \xi_m^m \\
\end{pmatrix}
\begin{pmatrix}
0 & 0 & \cdots & 1 \\
1 & 0 & \cdots & 0 \\
 & & \cdots & \\
0 & 0 & \cdots & 1 \\
\end{pmatrix} = \begin{pmatrix}
\xi_1^2 & \xi_1^3 & \cdots & \xi_1 \\
\xi_2^2 & \xi_2^3 & \cdots & \xi_2 \\
 & & \cdots & \\
\xi_m^2 & \xi_m^2 & \cdots & \xi_m
\end{pmatrix}
\] con lo que se ha demostrado que $\mathsf{A(g)} = \mathsf{UB(g)U^{-1}}$, para cualquier $g \in G$ y concluye que $\mathsf{B}$ y $\mathsf{\Gamma}$ son equivalentes. 
\end{ejemplo}

Considérese $T \colon G \to GL(V)$ una representación de un grupo $G$ sobre el campo $K$, con espacio de representación $V$ y asúmase que $V$ contiene un subespacio $W$ que es invariable bajo $T$, esto es, un subespacio tal que $T_g(W) \subset W$, para cualquier $g \in G$. Entonces se puede considerar el homomorfismo de grupos que asigna a cada elemento $g \in G$ la restricción de $T_g$ al subespacio $W$. Por ser $T_g$ la restricción se tiene entonces es claro que el homomorfismo anterior es una representación de $G$ sobre $K$, con espacio de representación $W$.

Con el afán de dar una representación matricial de este hecho, considérese una base $\{ w_1, w_2, \dots, w_t \}$ de $W$ y extiéndase a una base $ \{ w_1, \cdots, w_t, v_{t+1}, \cdots, v_n \}$ de $V$. Entonces la matriz asociada a cada función $T_g$, $g \in G$ con respecto a esa base es de la forma
\[ \begin{pmatrix}
\mathsf{A(g)} & \mathsf{B(g)} \\
0 & \mathsf{C(g)}
\end{pmatrix} \] donde $\mathsf{A(g)} \in GL(t,K), C(g) \in GL(n-t,K)$ y $\mathsf{B(g)}$ es una matriz de $t \times (n-t)$. Estas consideraciones sugieren lo siguiente
\begin{definicion}
Una representación $T \colon G \to GL(V)$ de un grupo $G$ sobre un campo $K$ es llamada irreducible si los únicos subespacios propios de $V$ que son invariantes bajo $T$ son los triviales, es decir, $V$ y $\{ 0 \}$
\end{definicion}

La representación es llamada \textbf{reducible} si $V$ contiene  subespacios no triviales que son invariantes bajo $T$. 

\begin{definicion}
Una representación matricial $M \colon G \to GL(n,K)$ es llamada reducible si existe una matriz $\mathsf{U} \in GL(n,K)$ tal que para cualquier $g \in G$, se tiene que la matriz $\mathsf{U}M(g)U^{-1}$ es de la forma
\[  \mathsf{UM(g)U^{-1}} = \begin{pmatrix}
\mathsf{A(g)} & \mathsf{B(g)} \\
0 & C(g)
\end{pmatrix} \]
\end{definicion}