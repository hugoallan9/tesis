\documentclass[letterpaper,12pt]{thesis}
\usepackage[utf8]{inputenc}
\usepackage[pdftex]{graphicx}
\usepackage{tikz}
\usepackage{titlesec}
\titlespacing{\section}{0pt}{*3}{*0}
\usepackage{parskip}
\usepackage{amsfonts,amsmath,amssymb,amsthm,mathrsfs}
\usepackage{latexsym,wasysym}
%manejando los margenes
\usepackage[paper=letterpaper, marginparsep=0in, marginparwidth= 0in, headsep= 0in, headheight= 0in, top=4cm, bottom=2.5cm,left=4cm,right=2.5cm,twoside]{geometry}
\usepackage{textcomp}
\usepackage[all]{xy}
\usepackage{longtable}
\usepackage[spanish]{babel}
\usepackage{bibthesis,color, enumerate, indentfirst}
\usepackage{hyperref,url,breakurl}
\usepackage{ dsfont }
%para manejar las imagenes (titulos y subtitulos de)
\usepackage{caption}
\usepackage{subcaption}
%para cambiar letra de subcaption a 10 pt
\usepackage{etoolbox}
\usepackage[nodisplayskipstretch]{setspace}


%\usepackage[scaled]{helvet}
%\renewcommand*\familydefault{\sfdefault}
%\usepackage[T1]{fontenc}



%control de espacio en ecuaciones
%\BeforeBeginEnvironment{equation}{\begin{singlespace}}
%\AfterEndEnvironment{equation}{\end{singlespace}\noindent\ignorespaces}
%\BeforeBeginEnvironment{align}{\begin{onehalfspace}}
%\AfterEndEnvironment{align}{\end{onehalfspace}\noindent\ignorespaces}

\decimalpoint
\newenvironment{myproof}[1][\proofname]
 {\par\pushQED{\qed}%
  \normalfont \topsep6\p@\@plus6\p@\relax
  \trivlist
  \item[\hskip\labelsep\itshape#1\@addpunct{.}]\par\noindent}
 {\popQED\endtrivlist\@endpefalse}
\makeatother
\makeatletter
\patchcmd{\caption@@@make}% <cmd>
  {\ifcaption@star}% <search>
    {\ifcaption@star\footnotesize}% <replace>
    {}{}% <success><failure>
\makeatother

\hypersetup{pdftitle = {tituloDePDF}, pdfkeywords = {poner un listado de palabras clave}, pdfauthor = {\textcopyright\ 2012, Hugo Allan García Monterrosa (hugoallangm@gmail.com)}, pdfsubject = {Trabajo
de graduación para obtener la Licenciatura en Matemática Aplicada},
pdfstartview = {FitH}, plainpages = {false}, bookmarksnumbered =
{true}}


\setlength{\parindent}{1cm}              % sangr�a
\setlength{\parskip}{1.3\baselineskip}              % distancia entre p�rrafos

\setstretch{1.3}                   % interlineado

\setlength\LTleft{0pt} \setlength\LTright{0pt} % par�metros para tablas largas

% Citas (sistema lancasteriano) y direcciones web ------------------
\bibpunct[--]{[}{]}{,}{n}{,}{;}
\setlength{\bibsep}{21pt}               % distancia entre items en la bibliograf�a
%\urlstyle{rm}                          % fuente en direcciones web

% Protecci�n de la tabla de contenidos
\addtocontents{toc}{\protect\setlength{\parskip}{1pt}}





\newtheoremstyle{ejem} %
  {\baselineskip} % Space above
  {\topsep} % Space below
  {} % Body font
  {} % Indent amount
  {\bfseries} % Theorem head font
  {.} % Punctuation after theorem head
  {.5em} % Space after theorem head
  {} % Theorem head spec (can be left empty, meaning `normal')
\newtheoremstyle{resto} %
  {\baselineskip} % Space above
  {\topsep} % Space below
  {\itshape} % Body font
  {} % Indent amount
  {\bfseries} % Theorem head font
  {.} % Punctuation after theorem head
  {.5em} % Space after theorem head
  {} % Theorem head spec (can be left empty, meaning `normal')
  
\theoremstyle{resto}\newtheorem{teorema}{Teorema}
\theoremstyle{resto}\newtheorem{lema}{Lema}
\theoremstyle{resto}\newtheorem{proposicion}{Proposición}
\theoremstyle{resto}\newtheorem{propiedad}{Propiedad}
\theoremstyle{resto}\newtheorem{corolario}{Corolario}
\theoremstyle{resto}\newtheorem{nota}{Nota}
\newtheorem{solucion}{Solución}
\newtheorem{problema}{Problema}
\theoremstyle{resto}\newtheorem{definicion}{Definición}
\newtheorem{notacion}{Notación}
\newtheorem{consideracion}{Consideración}
\newtheorem{ejercicio}{Ejercicio}
\theoremstyle{ejem} \newtheorem{ejemplo}{Ejemplo}
\numberwithin{equation}{chapter}
\renewcommand{\theequation}{\thechapter.\arabic{equation}}
%para manejar el estilo de los ejemplos

% Operadores y funciones -------------------------------------------
\DeclareMathOperator{\car}{car}
\DeclareMathOperator{\tr}{tr}
\DeclareMathOperator{\grado}{deg}
\makeatother
\newcommand{\srg}[3]{\sum_{#1 \in #2} #3_{#1} #1 }
%\theoremstyle{\plain}



\begin{document}
\captionsetup[figure]{labelformat=simple, labelsep=period}
{ %\setlength{\baselineskip}{1.3\baselineskip}

%% Datos de la portada
\newcommand{\Nomb}{Hugo Allan García Monterrosa} 		%Nombre del estudiante
\newcommand{\TiTes}{Teoría de los grupo-anillos y sus aplicaciones}	                %Título de tesis
\newcommand{\Carrera}{Licenciatura en Matemática aplicada }	                        %Carrera
\newcommand{\Ases}{Ph.D. Sergio Roberto López Permouth}                   %Nombre del asesor
\newcommand{\Fec}{mayo de 2014}                	%Fecha: mes y año.
\newcommand{\Grado}{Licenciado en Matemática Aplicada }              	%Grado.
\newcommand{\Esc}{Ciencias} 				%Escuela.
\newcommand{\FecP}{noviembre de 2012}                     %Fecha de aceptación del protocolo.

% Miembros de la Junta Directiva
\newcommand{\Dec}{Ing. Murphy Olympo Paiz Recinos}     	
\newcommand{\VocI}{Ing. Alfredo Enrique Beber Aceituno}  	
\newcommand{\VocII}{Ing. Pedro Antonio Aguilar Polanco}
\newcommand{\VocIII}{Inga. Elvia Miriam Ruballos Samayoa}
\newcommand{\VocIV}{Br. Walter Rafael Véliz Muñoz}
\newcommand{\VocV}{Br. Sergio Alejandro Donis Soto}
\newcommand{\Sec}{Ing. Hugo Humberto Rivera Pérez}

% Terna
\newcommand{\ExaI}{Dra. Mayra Virginia Castillo Montes }
\newcommand{\ExaII}{Lic. Francisco Bernardo Raúl De La Rosa }
\newcommand{\ExaIII}{Lic. José Ricardo Rodrigo Vásquez Bianchi }
  	



\newpage
\thispagestyle{empty}
%%%%%%%%%%%%%%%%%%%%%%%%%%%%%%%%%%%%%%%%%%%%%%%%%%%%%%%%%%%%%%%%%%%%%%%%%%%%%%%%%%%
%                          Primera Página
%%%%%%%%%%%%%%%%%%%%%%%%%%%%%%%%%%%%%%%%%%%%%%%%%%%%%%%%%%%%%%%%%%%%%%%%%%%%%%%%%%%

\setbox0\vbox{\noindent
\hspace{.7cm}
\includegraphics[width=2.2cm]{Pictures/escudo.png}}
\ht0=13pt\box0
\hangindent=35mm\hangafter=-3\vspace{-1.55cm}
% aquí está el ecabezado
\noindent Universidad de San Carlos de Guatemala\\
Facultad de Ingeniería\\
Escuela de \Esc  \\ 
\vspace{5.2cm}
\begin{center}
% Título
\textbf{TEORÍA DE LOS GRUPO-ANILLOS} \\
\textbf{Y SUS APLICACIONES} \\ 
\vspace{5.8cm}
% Autor
\textbf{\Nomb}\\ 
Asesorado por: \Ases  \\
\hskip 7.5em Lic. William Roberto Gutiérrez Herrera\\
\vspace{1.5cm}
% Lugar y fecha
Guatemala,  \Fec
\end{center}


\newpage
\thispagestyle{empty}
\mbox{}
\newpage
\thispagestyle{empty}
%%%%%%%%%%%%%%%%%%%%%%%%%%%%%%%%%%%%%%%%%%%%%%%%%%%%%%%%%%%%%%%%%%%%%%%%%%%%%%%%%%%
%                          Segunda Página
%%%%%%%%%%%%%%%%%%%%%%%%%%%%%%%%%%%%%%%%%%%%%%%%%%%%%%%%%%%%%%%%%%%%%%%%%%%%%%%%%%%
\begin{center}
UNIVERSIDAD DE SAN CARLOS DE GUATEMALA \\[3mm]
\includegraphics[width=4cm]{Pictures/escudo.png} \\ [1mm]
FACULTAD DE INGENIERÍA\\
\vspace{1.4cm}
\textbf{TEORÍA DE LOS GRUPO-ANILLOS}\\
\textbf{Y SUS APLICACIONES} 

TRABAJO DE GRADUACIÓN 

PRESENTADO A LA JUNTA DIRECTIVA DE LA \\ 
FACULTAD DE INGENIERÍA \\ 
POR\\
\vspace{1.4cm}
\textbf{\MakeUppercase{\Nomb}} \\
ASESORADO POR \MakeUppercase{\Ases}\\
\hskip 11em LIC. WILLIAM ROBERTO GUTIÉRREZ HERRERA 
	
AL CONFERÍRSELE EL TÍTULO DE 

\textbf{\MakeUppercase{\Grado}} \\ 
%\vspace{1.3cm}
\vfill
GUATEMALA, \MakeUppercase{\Fec}
\end{center}

\newpage
\thispagestyle{empty}
\mbox{}
\newpage
\thispagestyle{empty}
%%%%%%%%%%%%%%%%%%%%%%%%%%%%%%%%%%%%%%%%%%%%%%%%%%%%%%%%%%%%%%%%%%%%%%%%%%%%%%%%%%%
%                          Tercera Página
%%%%%%%%%%%%%%%%%%%%%%%%%%%%%%%%%%%%%%%%%%%%%%%%%%%%%%%%%%%%%%%%%%%%%%%%%%%%%%%%%%%

\begin{center}

UNIVERSIDAD DE SAN CARLOS DE GUATEMALA \\
FACULTAD DE INGENIERÍA 

\includegraphics[width=4cm]{Pictures/escudo.png}\\ [17mm]

\textbf{NÓMINA DE JUNTA DIRECTIVA} 	

\begin{tabular}{ll}
    DECANO     & \Dec \\ 
    VOCAL I    & \VocI \\ 
    VOCAL II   & \VocII \\ 
    VOCAL III  & \VocIII \\ 
    VOCAL IV   & \VocIV \\ 
    VOCAL V    & \VocV \\ 
    SECRETARIO & \Sec  
\end{tabular}\\
\vspace{1.6cm}
\textbf{TRIBUNAL QUE PRACTICÓ EL EXAMEN GENERAL PRIVADO} 

\begin{tabular}{ll}
    DECANO     & \Dec \\
    EXAMINADOR & \ExaI \\
    EXAMINADOR & \ExaII \\
    EXAMINADOR & \ExaIII \\
    SECRETARIO & \Sec
\end{tabular}
\end{center}

\newpage
\thispagestyle{empty}
\mbox{}
\newpage
\thispagestyle{empty}
%%%%%%%%%%%%%%%%%%%%%%%%%%%%%%%%%%%%%%%%%%%%%%%%%%%%%%%%%%%%%%%%%%%%%%%%%%%%%%%%%%%
%                          Cuarta Página
%%%%%%%%%%%%%%%%%%%%%%%%%%%%%%%%%%%%%%%%%%%%%%%%%%%%%%%%%%%%%%%%%%%%%%%%%%%%%%%%%%%
\begin{center}
  \textbf{HONORABLE TRIBUNAL EXAMINADOR}
\end{center} 
\vspace{1.0cm}
En cumplimiento con los preceptos que establece la ley de la Universidad de
San Carlos de Guatemala, presento a su consideración mi trabajo de
graduación titulado: 
\vspace{.4cm}
\begin{center}
\textbf{ \MakeUppercase{\TiTes}} 
\end{center}
\vspace{1.0cm}
Tema que me fuera asignado por la Dirección de la Escuela de \Esc , con fecha \FecP. 
\vspace{.4cm}
\begin{figure}[!ht]
\begin{flushright}
\begin{tabular}{c}
%\includegraphics[scale=.5]{Pictures/Firma.jpg}\\
\Nomb
\end{tabular}
\end{flushright}
\end{figure}
\newpage
\thispagestyle{empty}
 \textcolor[rgb]{1,1,1}{.} \thispagestyle{empty}   % Portada

%\pdfbookmark[111]{IDENTIFICACIÓN}{idn}

\thispagestyle{empty}

\begin{center}
UNIVERSIDAD DE SAN CARLOS DE GUATEMALA
\end{center}

\begin{figure}[h]
  \begin{center}
    \includegraphics[width=4.15cm]{escudo}\\
  \end{center}
\end{figure}

\begin{center}
\vspace{-7.5mm}FACULTAD DE INGENIERÍA\\[1.5cm]

\textbf{TITULO DE TU TESIS (identi.tex)}\\[1.15cm]

TRABAJO DE GRADUACIÓN \\ PRESENTADO A LA JUNTA DIRECTIVA DE LA \\
FACULTAD DE INGENIERÍA
\\POR \\[1.15cm]
\textbf{HUGO ALLAN GARCÍA MONTERROSA} \\
ASESORADO POR EL \textsc{Lic.}~WILLIAM ROBERTO GUTIERREZ HERRERA \\[1cm]
AL CONFERÍRSELE EL TÍTULO\\
\textbf{LICENCIADO EN MATEMÁTICA APLICADA}\\[2.5cm]
GUATEMALA, FECHA

\end{center}
 \textcolor[rgb]{1,1,1}{.} \thispagestyle{empty}    % Identificaci�n

%\pdfbookmark[112]{NÓMINA DE JUNTA DIRECTIVA}{ndj}
\thispagestyle{empty}

\begin{center}
UNIVERSIDAD DE SAN CARLOS DE GUATEMALA\\ FACULTAD DE INGENIERÍA
\end{center}

\begin{figure}[h]
  \begin{center}
    \includegraphics[width=4.15cm]{escudo}\\
  \end{center}
\end{figure}

\begin{center}
\textbf{NÓMINA DE JUNTA DIRECTIVA}
\end{center}

\begin{tabbing}
  SECRETARIOXXXX \= Ing.~Murphy Olimpo Paiz Recinosxxxxx  \kill
  DECANO \> Ing.~Murphy Olympo Paiz Recinos \\
  VOCAL I \> Ing. Alfredo Enrique Beber Aceituno\\
  VOCAL II \> Ing. Pedro Antonio Aguilar Polanco\\
  VOCAL III \> Ing. Miguel Angel Dávila Calderón\\
  VOCAL IV \> Br. Juan Carlos Molina Jiménez \\
  VOCAL V \> Br. Mario Maldonado Muralles  \\
  SECRETARIO \> Ing.~Hugo Humberto Rivera Pérez
\end{tabbing}


\begin{center}
\textbf{TRIBUNAL QUE PRACTICÓ EL EXAMEN GENERAL PRIVADO (Ver nomina.tex)}
\end{center}

\begin{tabbing}
  SECRETARIOXXXX \= Ing.~Murphy Olimpo Paiz Recinosxxxxxx  \kill
  DECANO \> Ing.~Murphy Olympo Paiz Recinos \\
  EXAMINADORA \> Dra.~Mayra Virginia Castillo Montes \\
  EXAMINADOR \> Lic.~William Roberto Gutiérrez Herrera \\
  EXAMINADOR \> Lic.~Francisco Bernardo Ral De La Rosa\\
  SECRETARIO \> Ing.~Hugo Humberto Rivera Pérez
\end{tabbing}

 \textcolor[rgb]{1,1,1}{.} \thispagestyle{empty}    % N�mina de Junta Directiva

%\pdfbookmark[113]{HOJA DE PROTOCOLO}{hdp}
\thispagestyle{empty}

\begin{center}
{\large \textbf{HONORABLE TRIBUNAL EXAMINADOR}}
\end{center}

\vspace{60pt}

\noindent Cumpliendo con los preceptos que establece la ley de la
Universidad de San Carlos de Guatemala, presento a su consideración
mi trabajo de graduación titulado:

\vspace{24pt}

\begin{center}
\textbf{\large TEORÍA DE LOS GRUPO-ANILLOS Y SUS APLICACIONES}
\end{center}

\vspace{24pt}

\noindent tema que me fuera asignado por la Coordinación de la
Carrera de Licenciatura en Matemática Aplicada, el (Fecha).

\vspace{54pt}

\begin{flushright}
Hugo Allan García Monterrosa
\end{flushright}
 \textcolor[rgb]{1,1,1}{.} \thispagestyle{empty}    % Hoja de protocolo

% si las cartas est�n en un fichero PDF, ser� f�cil adjuntarlas con el paquete: pdfpages
%\include{letter1} %\textcolor[rgb]{1,1,1}{.} \thispagestyle{empty}   % Carta del asesor
%\include{letter2} %\textcolor[rgb]{1,1,1}{.} \thispagestyle{empty}   % Carta del revisor
%\include{letter3} %\textcolor[rgb]{1,1,1}{.} \thispagestyle{empty}   % Carta del Director de Escuela
%\include{letter4} \textcolor[rgb]{1,1,1}{.} \thispagestyle{empty}   % Orden de impresi�n

%\pdfbookmark[114]{AGRADECIMIENTOS}{agr}
\begin{center}
{\textbf{\Large AGRADECIMIENTOS A:}}
\end{center}

\vspace{10pt}   \thispagestyle{empty}

\noindent
\begin{tabular*}{\textwidth}{@{}l@{\extracolsep{\fill}} p{3.7in}@{}}
	\textbf{Dios} & Por permitirme culminar mis estudios de pregrado, brindando fortaleza y ayuda en todo momento.\\[18pt]
	\textbf{Dedicatoria2} & Ver agrade.tex.\\[18pt]
	\end{tabular*}
	
 \textcolor[rgb]{1,1,1}{.} \thispagestyle{empty}    % Agradecimientos

%\include{dedica} \thispagestyle{empty}                              % Dedicatoria

\par}

\frontmatter    % --------------------------------------------------  Hojas preliminares

\tableofcontents    % �ndice general vinculado

{     
%\onehalfspacing 
%\setlength{\baselineskip}{1.3\baselineskip}
%\include{simbolos}  % Lista de s�mbolos

%\chapter{RESUMEN}

Resumen de tesis (obje.tex)

% -------------------------------------------------------------------------------

\chapter{OBJETIVOS}

{\Large \textbf{General}} \label{objetivog}
\begin{itemize}
    \item Solo un objetivo general (obje.tex)
\end{itemize}

\vspace{18pt}

{\Large \textbf{Específicos}}

\begin{enumerate}
    \item Al menos un objetivo específico (obje.tex)
\end{enumerate}
      % Resumen y objetivos

%\chapter{INTRODUCCIÓN}

Introducción de la tesis (intr.tex)
      % Introducci�n

\mainmatter

\chapter{CONCEPTOS PRELIMINARES}
En este capítulo se presentará la teoría básica del álgebra abstracta necesaria para la comprensión del contenido a desarrollar más adelante. Dicha exposición no pretende ser una guía de estudios del álgebra, más bien refresca resultados básicos de teoría de grupos, anillos y álgebras. En la medida de lo posible se evitará dar demostraciones de los resultados de estos conceptos, a menos que no sean materia de estudio de una licenciatura en Matemática.
\section{Antecedentes}
La \textit{teoría de grupos} como la conocemos actualmente tiene sus orígenes en los trabajos de Ruffini, Abel, Lagrange y Galois a inicios siglo \lsc{xix}, quienes trabajaron con el concepto de \textbf{permutación} ( en su tiempo Cauchy las llamaba \textsf{sustituciones}, ver \cite[p.104]{bib:libroGuti} ). Con Cayley \cite[p. 104]{bib:libroLosGrandes} se formalizó el concepto de \textbf{grupo} y además se dieron muchos avances significativos que impulsaron la investigación de este tema. 

Entre los avances hechos por Cayley figuran:
\begin{bulletList}
\newItem Definición formal de grupo usando la notación de multiplicación.
\newItem Utilizar una \textit{tabla} para mostrar como actúa una operación.
\newItem Demostración de existencia de dos grupos no isomorfos de orden cuatro, dando ejemplos explícitos. 
\newItem Demostración de existencia de dos grupos no isomorfos de orden seis, uno de los cuales es conmutativo y el otro es isomorfo a $\mathcal{S}_3$, el grupo de permutaciones de tres elementos.
\newItem Demostración de que el orden de todo elemento es divisor del orden del grupo, cuando éste es finito. 
\end{bulletList}
%-----------------------> teoría de grupos
\section{Teoría de grupos}
\begin{definicion}
Un \textbf{grupo} es un conjunto no vacío $G$ junto con una operación binaria, denotada como $\cdot$, tal que para cada $a,b \in G$ se cumplen las siguientes condiciones:
\begin{bulletList}
\item $(a\cdot b)\cdot c = a \cdot (b\cdot c)$,
\item Existe un elemento único $1\in G$, tal que $a\cdot 1 = 1 \cdot a = a$,
\item Para cada $a\in G$ existe un elemento único $a^{-1} \in G$, tal que $a\cdot a^{-1} =a^{-1}\cdot a =1$.
\end{bulletList}
Si, además de las tres propiedades anteriores, se cumple que \[ a\cdot b = b \cdot a , \mbox{ para cada } a,b \in G \] entonces se dice que el grupo es \textbf{abeliano}  o \textbf{conmutativo}. 
Si el conjunto $G$ es finito, entonces el número de elementos de $G$ es llamado el \textbf{orden} de $G$ y se denota como $\circ(G)$. 
\end{definicion}
\begin{ejemplo}\label{ejemplo:simetrias}
Sea $M$ un conjunto finito. El lector deberá recordar que una aplicación biyectiva de $M$ a $M$ es llamada \textit{permutación} de $M$. Es claro entonces que la aplicación identidad de $M$ a $M$ es una permutación, que la composición de dos permutaciones es una permutación y la inversa de una permutación también es permutación. A partir de estos hechos, es evidente que dado un conjunto $M$ se puede construir conjunto de permutaciones y que este constituye un grupo respecto a la composición de funciones. Este grupo usualmente es denotado como $\mathcal{S}_M$ y es llamado el \textbf{grupo de permutaciones de $M$}. 
Si $M = \{  1,2,\dots, n \}$ entonces $\mathcal{S}_M$ es llamado el \textit{grupo de simetrías de grado $n$} y se denotada como $\mathcal{S}_n$. Dado un elemento $\psi \in \mathcal{S}_n$, si se elige que $i_k = \psi (k), \ 1 \leq k \leq n$, entonces se puede representar $\psi$ en la forma:
\[ \psi = \begin{pmatrix}
1 & 2 & 3 & \cdots & n \\
i_1 & i_2 & i_3 & \cdots & i_n
\end{pmatrix}, \]
la cual es una notación introducida por Cauchy en 1845 \cite[vol 1, p. 64-90]{bib:Cauchy}. Usando esta notación, la inversa de $\psi$ se representa como \[ \psi^{-1} = \begin{pmatrix}
i_1 & i_2 & i_3 & \cdots & i_n \\
1 & 2 & 3 & \cdots & n
\end{pmatrix}.
 \]
 Dadas, por ejemplo, \[ \phi = \begin{pmatrix}
 1 & 2 & 3 & 4 & 5 \\
 3 & 5 & 2 & 4 & 1
 \end{pmatrix} \mbox{ y } \psi = \begin{pmatrix}
 1 & 2 & 3 & 4 & 5 \\
  2 & 1 & 4 & 5 & 3
 \end{pmatrix}, \]
 se tiene que $(\phi \circ \psi)(1) = \phi(2) = 5$. Haciendo el cálculo para el resto de los números se obtiene \[ \phi \circ \psi = \begin{pmatrix}
 1 & 2 & 3 & 4 & 5 \\
  5 & 3 & 4 & 1 & 2
 \end{pmatrix}. \]
 De la misma manera \[ \psi \circ \phi = \begin{pmatrix}
 1 & 2 & 3 & 4 & 5 \\
  4 & 3 & 1 & 5 & 2
 \end{pmatrix}. \]
 Este simple cálculo demuestra que, en general, $\mathcal{S}_n$ no es conmutativo. De hecho, es fácil demostrar que $\mathcal{S}_n$ es conmutativo si y sólo si $n \leq 2$. 
\end{ejemplo}
\begin{definicion}
Un subconjunto no vacío $H$ de un grupo $G$ es llamado \textbf{subgrupo de $G$} si es cerrado bajo la operación de $G$ y $H$, con la restricción de la operación de $G$, es un grupo por sí mismo.
\end{definicion}
\begin{ejemplo}[Subgrupos cíclicos]
Sea $a$ un elemento del grupo $G$. Para un exponente entero se definen las potencias de $a$ como
\[ a^n = \left\{ \begin{array}{lr}
\underset{\mbox{n veces}}{\underbrace{a \cdot a \cdots a}} & \mbox{ si } n >0\\
\underset{\mid n \mid \mbox{ veces }}{\underbrace{a^{-1}\cdot a^{-1}\cdots a^{-1}}} & \mbox{ si } n<0 \\
1 & \mbox{ si } n = 0.
\end{array} \right.  \]
Como $a^m \cdot a^n = a^{m+n}$, se sigue que el conjunto
\[\langle a \rangle = \{ a^n : n \in \mathds{Z} \} \]
es un subgrupo de $G$, llamado \textbf{subgrupo cíclico de $G$} generado por $a$.
Si este grupo es finito, entonces existen enteros positivos $n,m$ distintos tales que $a^n = a^m$, de esta cuenta, se tiene $a^{n-m} = a^{m-n} = 1$. El entero positivo más pequeño $n$  tal que $a^n = 1$ se le llama \textbf{orden de $a$} y se denota como $\circ(a)$. Si $\langle a \rangle$ es infinito se dice que $a$ es de \textbf{orden infinito}.
Si existe un elemento $a$ en $G$ tal que $G = \langle a \rangle$, entonces se dice que $G$ es un \textbf{grupo cíclico} y que $a$ es un \textbf{generador} de $G$. Nótese que $\circ(a) = \mid \langle a \rangle \mid $.
\end{ejemplo}
\begin{ejemplo}
Sea $X$ un subconjunto no vacío de un grupo $G$. Se define el \textbf{subgrupo generado por $X$} como la intersección de todos los subgrupos de $G$ que contienen a $X$. Nótese que esta familia de subgrupos es no vacía, ya que por lo menos $G$ pertenece a ella. Es fácil demostrar que esta intersección definida previamente es un subgrupo de $G$. Este subgrupo es denotado como $\langle X \rangle$. Se propone como ejercicio al lector, demostrar que 
\[ \langle X \rangle = \{ x_1^{\epsilon_1} \cdots x_k^{\epsilon_k} \colon x_i \in X, \ \epsilon_i = \pm 1, \ k \geq 1 \} \cup \{1\} .\]
Si $\langle X \rangle = G$ se dice que $X$ es un \textbf{conjunto de generadores de $G$}. Si $X$ es finito, entonces se dice que $G$ es un \textbf{grupo finitamente generado}.
\end{ejemplo}
 \begin{lema}
 Un subconjunto no vacío $H$ de un grupo $G$ es un subgrupo de $G$ si y sólo si para cualesquiera $x,y \in H$ se tiene que $x^{-1}y \in H$.
 \end{lema}
 \begin{definicion}
 El \textbf{centro} de un grupo $G$ es el subgrupo \[ \mathcal{Z}(G) = \{ a \in G \colon ax=xa, \mbox{ para cada }  x \in G \}. \]
\end{definicion}
Dado un subgrupo $H$ de un grupo $G$, se puede definir una \textit{partición} de $G$, es decir una cubierta de $G$ hecha de subconjuntos disjuntos. 
\begin{definicion}
Sea $H$ un subgrupo de un grupo $G$. Dado un elemento $a \in G$, los subconjuntos de la forma 
\begin{eqnarray*}
aH &=& \{ ah \colon h \in H \}, \\
Ha &=& \{ ha \colon h \in H \} 
\end{eqnarray*}
son llamados  clases lateral izquierda y derecha del subgrupo $H$ determinadas por $a$, respectivamente.
\end{definicion}
Algunas propiedades elementales de las clases laterales son:
\begin{proposicion}
Sea $H$ un subgrupo de un grupo $G$ y $a,b$ elementos arbitrarios de $G$. Entonces se cumple:
\begin{bulletList}
\item Si $b \in aH$ entonces $bH = aH$.
\item Si $b \notin aH$ entonces $aH \cap bH = \emptyset$.
\end{bulletList}
\end{proposicion}
\begin{corolario}
Sea $H$ un subgrupo de un grupo $G$. Dados $a,b \in G$ se cumple que $b \in aH$ si y sólo si $aH = bH$.
\end{corolario}
Todo elemento en una clase lateral es un \textbf{representante} de la misma. Un conjunto completo de representantes de un clase lateral izquierda ( derecha ) es llamado \textbf{transversal izquierdo ( derecho ) de $H$ en $G$}.
\begin{definicion}
Sea $H$ un subgrupo de un grupo $G$. Si el numero de clases izquierdas(derechas) de $H$ en $G$ es finito, entonces este número es llamado  \textbf{índice} de $H$ en $G$ y se denota como $(G:H)$.
\end{definicion}
\begin{teorema}[Lagrange]
Sea $H$ un subgrupo de un grupo finito $G$. Entonces, el orden de $H$ divide a el orden de $G$. Más aún, de manera más formal, se tiene
\[ \mid G \mid = (G:H) \mid H \mid. \]
\end{teorema}
\begin{corolario}
Sea $a$ un elemento de un grupo finito $G$. Entonces $\circ(a)$ es un divisor de $\mid G \mid$.
\end{corolario}
\begin{ejemplo}
Considérese nuevamente a $\mathcal{S}_3$, el grupo de simetrías de grado tres. Se sabe que $\mid \mathcal{S}_3 \mid = 6$. Explícitamente, este grupo se expresa como \[ \mathcal{S}_3 = \{ I, (12), (13), (23), (123), (132) \}, \mbox{ donde } I = (1) .\]
Sea $H = \{ I, (12) \} $ y $\alpha = (123)$. Entonces \[ \alpha H = \{ (123), (13) \}   \mbox{ y } H\alpha = \{ (123), (23) \} ,\] con esto se demuestra que, en general, las clases laterales derechas e izquierdas determinadas por el mismo elemento no son iguales.
\end{ejemplo}
Los subgrupos cuyas clases laterales derechas e izquierdas generadas por el mismo elemento son iguales son de especial importancia. Nótese que para un elemento $a$ y un subgrupo $H$ de un grupo $G$, se tiene que $aH = Ha$ si y sólo si $a^{-1}Ha=H$. Esto sugiere la siguiente
\begin{definicion}
Sea $H$ un subgrupo de un grupo $G$. Se dice que $H$ es \textbf{normal} en $G$, y se escribe $H \triangleleft G$ si $a^{-1}Ha = H$ para cualquier $a \in G$.
\end{definicion}

%-------------subsección --------------------------
\subsection{Homomorfismos y grupos cocientes}
El concepto de \textit{homomorfismo} es, quizás, uno de los conceptos más importantes en álgebra y en particular para este trabajo.
\begin{definicion}
Sean $G_1$ y $G_2 $ grupos. Una aplicación $f \colon G_1 \to G_2$ es llamada un \textbf{homomorfismo de grupos} si para cada $g, h \in G$ se cumple que \[ f(g \cdot h) = f(g) \cdot f(h) . \]
\end{definicion}
\begin{definicion}
Sea $f \colon G_1 \to G_2$ un homomorfismo de grupos. Entonces, la \textbf{imagen de $f$} es el conjunto \[ \Ima(f) = \{ y \in G_2 \colon \mbox{ existe }x \in G_1, f(x) =y \}. \]
El \textbf{kernel de $f$} es el conjunto \[ \ker(f) =  \{ x \in G_1 \colon f(x) = 1 \}.\] 
\end{definicion}
\begin{definicion}
Un homomorfismo de grupos $f \colon G_1 \to G_2$ es llamado un \textbf{epimorfismo} si es sobreyectivo. Se llama a $f$ un \textbf{monomorfismo} si es inyectivo. Por último, se dice que $f$ es un  \textbf{isomorfismo} si es sobreyectivo e inyectivo. Dados dos grupos $G_1$ y $G_2$, se dice que son isomorfos, y se denota como $G_1 \simeq G_2$ si existe un isomorfismo $f \colon G_1 \to G_2$. 
\end{definicion} 
Un homomorfismo de un grupo $G$ en sí mismo es llamado un \textbf{endomorfismo} y si a su vez es un isomorfismo se llama  \textbf{automorfismo} de $G$. 
El siguiente resultado se debe al famoso matemático británico Arthur Cayley, el cual demuestra la relevancia de los grupos de permutación en la teoría de grupos.
\begin{teorema}[Cayley]
Todo grupo $G$ es isomorfo a un grupo de permutaciones.
\end{teorema}
\begin{definicion}
Sea $H$ un subgrupo normal de un grupo $G$ y $a, b \in G$. Se dice que $a\equiv b(\mod{H})$ si $b^{-1}a \in H$. Es fácil demostrar que esta relación es de equivalencia. Para un elemento $a \in G$ se denota su clase de equivalencia como \[ \bar{a} = \{ x \in G \colon x \equiv a \pmod{H}  \} = \{ x \in G \colon a^{-1}x \in H \} = aH. | \]
Se denota como $G/H$ al conjunto de clases de equivalencia de los elementos de $G$. Se define el producto de elementos en $G/H$ como \[ \bar{a}\cdot \bar{b} = \overline{ab}. \] Esta operación es bien definida y $G/H$ es un grupo, llamado \textbf{grupo cociente}. 
\end{definicion}
Considérese la aplicación $\omega \colon G \to G/H$ dada por: \[G \ni a \mapsto \omega(a) = 	\bar{a} = aH. \]
Es evidente que $\omega$ es un epimorfismo de grupos, llamado \textbf{homomorfismo canónico} de $G$ hacia el grupo cociente $G/H$. Este homomorfismo satisface que $\omega(1) = 1H = H$ y $\ker(\omega) = H.$ 
\begin{teorema}[Primer teorema de isomorfía de grupos]
Sea $f \colon G_1  \to G_2$ un homomorfismo de grupos, $\omega$ el homomorfismo canónico de $G_1$ hacia el grupo cociente $X = G_1/\ker(f)$ e $i$ la inclusión de $\Ima(f)$ en $G_2$. Entonces existe un homomorfismo único $\bar{f} \colon G_1/\ker(f) \to \Ima(f)$ tal que $f = i \circ \bar{f}\circ \omega$, es decir que diagrama  de la figura \ref{fig:primerTeoremaIsomofia} conmuta.
\begin{figure}
\caption{\quad \textbf{Primer teorema de isomorfía de Grupos}}
\centering
$\xymatrix { G_1 \ar[r]^f 
\ar[d]_{\omega}
 & G_2\\
X \ar[r]_{\bar{f}} & \Ima(f) \ar[u]_i }$
\caption*{Fuente: elaboración propia con paquete \textbf{xymatrix} para \LaTeX.}
\label{fig:primerTeoremaIsomofia}
\end{figure}
Además $\bar{f}$ es un isomorfismo. 
\end{teorema}
\begin{corolario}
Sea $f \colon G_1 \to G_2$ un epimorfismo. Entonces \[ G_1/\ker(f) \simeq G_2. \]
\end{corolario}
\begin{lema}
Sea $H$ un subgrupo normal de $G$. Entonces
\begin{bulletList}
\item Para cada subgrupo $K$ de $G$ que contiene a $H$, el conjunto $K/H = \{ xH \colon x \in K \}$ es un subgrupo de $G/H$ que es normal si y sólo si $K$ es normal.
\item Si $\mathcal{K}$ es un subgrupo de $G/H$, entonces la preimagen $K = \{ x \in G \colon xH \in \mathcal{K} \}$ es un subgrupo de $G$ que contiene a $H$, tal que $\mathcal{K} = K/H$.
\end{bulletList}
\end{lema}
\begin{teorema}
Sea $f \colon G_1 \to G_2$ un epimorfismo de grupos, entonces existe un biyección entre el conjunto de subgrupos de $G_2$ y el conjunto de subgrupos de $G_1$ que contienen a $\ker(f)$.
\end{teorema}
\begin{teorema}[Segundo teorema de isomorfía de grupos]
Sea $H$ y $K$ subgrupos de un grupo $G$ y supóngase que $K$ es normal. Entonces: \[ \frac{H}{H \cap K} \simeq \frac{HK}{K}, \] donde $HK = \{ hk \colon h \in H, \ k \in K \}$.
\end{teorema}
\begin{teorema}[Tercer teorema de isomorfía de grupos]
Sean $H \subset K$ subgrupos normales de un grupo $G$. Entonces:
\[\frac{G/H}{K/H} \simeq \frac{G}{K}. \]
\end{teorema}
\subsection{Productos directos}
\begin{definicion}
Sean $H,\ K$ subgrupos de un grupo $G$. Se dice que $G$ es el \textbf{producto directo interno} de $H$ y $K$ y se escribe $G = H \times K$ si se cumplen las siguientes condiciones:
\begin{bulletList}
\item $G=HK$,
\item $H \cap K = \{1\}$,
\item $H \triangleleft G$ y $K \triangleleft G$.
\end{bulletList}
\end{definicion}
La definición anterior se puede extender a familias arbitrarias de subgrupos normales.
\begin{definicion}
Sea $\{H_i\}_{i \in I}$ un familia de subgrupos normales de un grupo $G$. Entonces $G$ es llamado el \textbf{producto directo interno} de los subgrupos $\{H_i\}_{i \in I}$ si se cumplen las siguientes condiciones:
\begin{bulletList}
\item $G =\langle H_i \colon i \in I \rangle$, es decir que cada elemento $g$ de $G$ se puede escribir como producto de un número finito de elementos de los subgrupos $\{H_i\}_{i \in I}$.
\item $H_i \cap \langle H_j \colon j \in I, \ j \neq i \rangle = \{1\}$ para todo índice $i \in I$.
\end{bulletList}
\end{definicion} 
Si $G_1, \dots, G_n$ son una familia de grupos y $G = G_1\dot{\times}\cdots\dot{\times}G_n$ su \textbf{producto directo externo}, entonces los conjuntos
\[ H_1=\{(x,1,\dots,1)\colon x \in G_1\},\dots,H_n=\{(1,1,\dots,x)\colon x \in G_n \} \] son subgrupos normales de $G$ tales que $G_i \simeq H_i,1\leq i \leq n$ y $G$ es también el producto directo interno de los subgrupos $H_1, \dots, H_n$.
De manera similar, si $G$ es el producto directo interno de una familia de subgrupos normales $H_1,\dots,H_n$ y se construye el producto directo externo $\bar{G} = H_1\dot{\times}\cdots\dot{\times}H_n$, entonces se tiene que $\bar{G} \simeq G$. Debido a este hecho, no se hace distinción entre producto interno y externo.
%-------------subsección de grupos abelianos
\subsection{Grupos Abelianos}
Los grupos abelianos resultan de mucho interés para el desarrollo del capítulo \ref{chap:unidades}, así que se remarca sus importancia.
Sea $G$ un grupo abeliano. Un elemento de $G$ es llamado un \textbf{elemento de torsión} si este es de orden finito. Si dos elementos $g,h \in G$ son de torsión, de órdenes $m$ y $n$ respectivamente, entonces es inmediato que $(g^{-1}h)^{mn} = 1$, lo cual demuestra que el conjunto de elementos de torsión de $G$ es un subgrupo de $G$. Nótese que este hecho también demuestra que dado un número primo $p$, el conjunto de elementos de $G$ cuyos órdenes son potencias de $p$ también constituyen un subgrupo de $G$.
\begin{definicion}
Sea $G$ un grupo abeliano. Entonces, el subgrupo 
\[T(G) = \{ g \in G \colon \circ(g) < \infty \}  \]
es llamado el \textbf{subgrupo de torsión} de $G$ y el subgrupo \[ G(p) = \{g \in G \colon \circ(g) \mbox{ es una potencia de } p  \} \] es llamado la \textbf{componente $p$-primaria} de $G$.
\end{definicion}
Se dice que $G$ es un grupo \textbf{libre de elementos de torsión} si $T(G) = 1$.
Un grupo abeliano que es producto directo de grupos cíclicos infinitos se llama \textbf{abeliano libre}. Al número de factores directos se le llama \textbf{rango} del grupo abeliano libre. Si dicho número no es finito, se dice que tiene \textbf{rango infinito}.
\begin{teorema}
Un grupo $G$ que es abeliano, finitamente generado y libre de elementos de torsión debe ser libre.
\end{teorema}
\begin{teorema}
Sea $G$ un grupo abeliano finitamente generado. Entonces $T(G)$ es finito, $G/T(G)$ es libre de rango finito y \[ G \simeq T(G)\times \frac{G}{T(G)}. \]
\end{teorema}
\begin{lema}
Sea $g$ un elemento de orden $\circ(g) = p_1^{n_1}\cdots p_t^{n_t}$ de un grupo $G$. Entonces se puede escribir $g = g_1\cdots g_t$ con $\circ(g_i) = p_i^{n_i}, i\leq i \leq t$. Más aún, los elementos $g_1, \dots, g_t$ que están determinados de manera única son potencias de $g$ y por lo tanto conmutan entre ellos. 
\end{lema}
Un elemento cuyo orden es potencia de un primo $p$ es llamado  \textbf{$p$-elemento}. Por otro lado, si $p$ no divide el orden del elemento, se dice que es un \textbf{$p'$-elemento}.
\begin{lema}
Sea $G$ un grupo abeliano finito de orden $\mid G \mid = p_1^{n_1}\cdots p_t^{n_t}$. Entonces \[ G = G(p_1)\times \cdots \times G(p_t). \]
\end{lema}
\begin{definicion}
Sea $p$ un primo. Un grupo finito $G$ es llamado \textbf{$p$-grupo} si su orden es una potencia de $p$.
Se dice que un grupo abeliano $G$ es un \textbf{abeliano elemental} si existe un primo $p$ tal que todos los elementos distintos al elemento identidad son de orden $p$.
\end{definicion}
Es interesante notar que los $p$-grupos abelianos elementales también pueden ser vistos como espacios vectoriales.
\begin{lema}
Sea $G$ un $p$-grupo abeliano elemental. Entonces $G$ es un espacio vectorial sobre $\mathds{Z}_p$. Más aún, si $G$ es finito entonces puede ser escrito como producto directo de de número finito de grupos cíclicos de orden $p$.
\end{lema}
Para un grupo $G$ se define su \textbf{exponente}, y se denota como $\exp(G)$, como el entero positivo más pequeño $m$, tal que $g^m=1$ para cualquier $g \in G$. Nótese que $G$ es un $p$-grupo abeliano elemental si y sólo $\exp(G)=p$ y que si $G$ es un grupo abeliano con $\exp(G) = p^m$, entonces $\exp(G^p) = p^{m-1}$.
\begin{teorema}\label{teo:estructuraAbelianos}
Sea $G$ un $p$-grupo abeliano finito. Entonces $G$ se puede escribir como producto directo de $p$-subgrupos cíclicos. Esta descomposición es única, en el sentido que si
\[ G = C_1 \times \dots \times C_t = D_1 \times \cdots \times D_s \]
donde $C_i,C_j, 1\leq i \leq t, 1\leq j \leq s,$ son $p$-grupos cíclicos de órdenes $p^{n_1} \geq \cdots \geq p^{n_t}>1$ y $p^{m_1}\geq \cdots \geq p^{m_s}>1$ respectivamente, entonces $t=s$ y $n_i = m_i, 1\leq i \leq t$.
\end{teorema} 
\begin{proposicion}
Sea $G$ un grupo abeliano finito de orden $n$. Entonces para cada divisor $d$ de $n$, el número de subgrupos cíclicos de $G$ de orden $d$ es igual al número de factores cíclicos de $G$ del mismo orden.
\end{proposicion}

%--------------Seccion Hamiltonianos
\subsection{Grupos Hamiltonianos}
Se define el \textbf{grupo de cuaterniones} de orden 8 como:\[ K_8 = \{ \langle a,b \colon a^4 =1, \ a^2 = b ^2, \ bab^{-1} = a^{-1} \rangle. \} \]
Los elementos de este grupo se enumeran de la siguiente manera:
\[ K_8 = \{ 1,a,a^2, a^3, b, ab, a^2b, a^3b \}. \]
\begin{definicion}
Dados dos elementos $x,y$ en un grupo $G$, el \textbf{conmutador} de $x$ y $y$ es el elemento $[x,y] = x^{-1}y^{-1}xy \in G$. Dados dos subconjuntos $H$ y $K$ de un grupo $G$, se denota como $[H,K]$ el subgrupo de $G$ generado por el conjunto:
\[\{(h,k) \colon h \in H, k \in K\}. \]
En particular, el grupo $G'=[G,G]$ es llamado \textbf{subgrupo conmutador} o \textbf{subgrupo derivado} de $G$.
\end{definicion}
Un cálculo directo demuestra que $K'_8 = \mathcal{Z}(K_8) = \{1, a^2\}$. Además, $a^2$ es el único elemento de $K_8$ de orden 2, así que si $H$ es un subgrupo cualquiera de $K_8$, entonces se tiene que $K'_8 = \{1,a^2\} \subset H$. Es fácil notar que cualquier subgrupo de $K_8$ es normal.
\begin{definicion}
El 4-grupo de Klein está definido por
\[ G = \{1,a,b,ab\}, \] donde sus elementos distintos de la identidad son de orden 2.
\end{definicion}
\begin{ejercicio}\label{ejer:klein}
Demostrar que $K_8/K'_8$ es el 4-grupo de Klein.
\end{ejercicio} 
\begin{solucion}
Se sabe que $K'_8 = \{1,a^2\}$, entonces calculando las clases de equivalencia se tiene:
\begin{eqnarray*}
\bar{1} &=& K'_8\\
\bar{a} &=& \{a, a^3\} \\
\bar{b} &=&  \{b, a^2b \} \\
\overline{ab} &=& \{ ab, a^3b \}
\end{eqnarray*}
lo cual genera una partición, por lo tanto $K_8/K'_8 = \{\bar{1},\bar{a},\bar{b}, \bar{ab}  \}$. Ahora bien, $\bar{a}\cdot\bar{a} =\bar{a^2}=\bar{1}, \ \bar{b}\bar{b} = \bar{b^2} = \bar{a^2} = \bar{1}$ y finalmente $\overline{ab}\cdot\overline{ab} = \overline{(ab)^2} = \bar{1}$ con lo que concluye la demostración. \qedsymbol
\end{solucion}
\begin{definicion}
Un grupo G es  \textbf{Hamiltoniano} si no es conmutativo y todos sus subgrupos son normales. 
\end{definicion}
Los grupos Hamiltonianos son bastante conocidos y jugarán un papel importante en este trabajo.
\begin{lema}
Todo grupo Hamiltoniano contiene un subgrupo isomorfo a $K_8$.
\end{lema}
\begin{teorema}
Un grupo $G$ es Hamiltoniano si y sólo si $G$ es producto directo de un grupo cuaterniones de orden 8, un 2-grupo abeliano elemental $E$ y un grupo abeliano $A$ en el cual todos sus elementos son de orden impar.
\end{teorema}
La demostración de este importante teorema se puede consultar en \cite[página 130]{bib:groupBook}.
%----------------> teoria de anillos, módulos  y algebras
\section{Anillos, Módulos y Álgebras}
En lo subsiguiente será de vital importancia conocer propiedades importantes de los anillos, módulos y álgebras, es por eso que se ha dedica esta sección para su estudio, en la mayoría de veces, sin demostraciones.
\subsection{Anillos}
Se comienza con la definición de anillo y algunos conceptos básicos.
\begin{definicion}
Un \textbf{anillo} es un conjunto no vacío $R$ junto con dos operaciones binarias llamadas suma y multiplicación y denotadas como $+$ y $\cdot$ respectivamente, tal que para todo $a,b\in R$ se cumplen las siguientes propiedades:
\begin{bulletList}
\item $a+(b+c) = (a+b)+c$
\item Existe un único elemento $0\in R$ tal que $a+0=0+a=a$
\item $a+b=b+a$
\item Existe un elemento $-a\in R$ tal que $a+(-a) = (-a)+a=0$
\item $a\cdot(b\cdot c) = (a\cdot b)\cdot c$
\item $a\cdot(b+c) = a\cdot b+ a\cdot c$
\item $(a+b)\cdot c= a\cdot c+b\cdot c $
\end{bulletList}
Además, si se satisface
\begin{bulletList}\addtocounter{ContadorLista}{7}
\item $a \cdot b = b\cdot a$
\end{bulletList}
se dice que el anillo es \textbf{conmutativo}.
Un anillo es llamado  \textbf{dominio} si satisface:
\begin{bulletList}\addtocounter{ContadorLista}{8}
\item $a\cdot b = 0$ implica que $a = 0$  o $b = 0$.
\end{bulletList}
\end{definicion}
Dos elementos $a,b$ distintos de cero de un anillo $R$ tales que $ab = 0$ son llamados \textbf{divisores de cero}. Así que un dominio es un anillo sin divisores de cero.
Un anillo $R$ que contiene un elemento $1\neq 0$ tal que \[ 1\cdot a = a \cdot 1 = a, \mbox{ para todo } a \in R \] es llamado \textbf{anillo con unidad}. Un anillo con unidad que es un dominio conmutativo se llama \textbf{dominio integral}.
\begin{definicion}
Un elemento $a$ de un anillo $R$ es \textbf{invertible} si existe un elemento $a^{-1} \in R$, conocido como su \textbf{inverso}, tal que $a\cdot a^{-1} = a^{-1} \cdot a = 1$. El conjunto \[ \mathcal{U}(R) = \{a \in R \colon a \mbox{ es invertible} \} \] se llama el \textbf{grupo de unidades} de $R$.
\end{definicion}
Un anillo se denomina \textbf{anillo de división} si todos los elementos distintos de cero son invertibles, dicho de otra manera, si $R\backslash \{0\} = \mathcal{U}(R)$. Un anillo de división conmutativo es un \textbf{campo}.
\begin{ejemplo}
El conjunto $\mathds{Z}_m = \{ \bar{0}, \bar{1}, \dots, \overline{m-1} \}$ de enteros módulo $m$ es un anillo conmutativo. Más aún $\mathds{Z}_m$ es un campo si y sólo si $m$ es un número primo.
\end{ejemplo}
\begin{ejemplo}
Sea $R$ un anillo. Entonces el conjunto $R[X]$ de todos los polinomios con coeficientes en $R$ e indeterminadas en $X$ con la operación usual de suma y multiplicación de polinomios es un anillo. El anillo de polinomios $R[X]$ es un dominio si y sólo si $R$ lo es. 
\end{ejemplo}
\begin{ejemplo}
Sea $R$ un anillo. Entonces, el conjunto $M_n(R)$ de todas la matrices de $n\times n$ con entrdas en $R$, con suma y producto usual de matrices es un anillo. Este anillo es llamado \textbf{anillo completo de matrices} de $n\times n$ sobre $R$.
\end{ejemplo}
\begin{ejemplo}
Sean $R_1, R_2, \dots, R_n$ anillos. El anillo \[ R_1\dot{\oplus}R_2\dot{\oplus}\cdots\dot{\oplus} R_n = \{ (a_1,a_2,\dots,a_n) \dots a_i \in R_i, 1\leq i \leq n \},  \] con la suma y el producto definido componente a componente es llamado la \textbf{suma directa} de anillos $R_1, R_2,\dots,R_n$.
\end{ejemplo}
El siguiente ejemplo es de particular importancia ya que fue el primer ejemplo de anillos no conmutativos en la historia de las matemáticas. 
\begin{ejemplo}
Sean $i,j,k$ símbolos dados y considérese el conjunto $\mathcal{H}$ de todas las expresiones de la forma $x_0 + x_1i+x_2j+x_3k$ donde los coeficientes $x_0,x_1,x_2,x_3$ son todos números reales.
Se define la suma de dos elementos de este conjunto como \[ (x_0 + x_1i+x_2j+x_3k)+(y_0 + y_1i+y_2j+y_3k) = (x_0 + y_0) + (x_1 + y_1)i + (x_2+y_2)j + (x_3+y_3)k.  \] La multiplicación está definida de manera distributiva, con las siguientes reglas:
\begin{equation*}
i^2 = j^2 = k^2 = -1
\end{equation*}
\begin{equation*}
ij = k = -ji
\end{equation*} \[ jk = i = -kj \] \[ ki = j = -ik .\] Por medio de un cálculo sencillo se demuestra que $\mathcal{H}$ es un anillo, llamado \textbf{anillo de cuaterniones reales}. Dado un cuaternión $\alpha = x_0 + x_1i + x_2j + x_3k$, se define el \textbf{conjugado} $\bar{\alpha}$ de $\alpha$ como \[ \bar{\alpha} = x_0-x_1i - x_2j-x_3k. \] La norma de $\alpha$ se define como \[ \lVert \alpha \rVert = \alpha\bar{\alpha} = x_0 ^2 + x_1^2 + x_ 2^2 + x_ 3^2. \]
Al igual que en el caso de los números complejos, se cumple que:
\begin{bulletList}
\item $\lVert \alpha\beta \rVert = \lVert \alpha \rVert \lVert \beta \rVert = \lVert \alpha\beta \rVert$
\item $\lVert \alpha \rVert \geq 0$ y $\lVert \alpha \rVert  = 0 $ si y sólo si $\alpha = 0 $
\end{bulletList}
Si $\alpha \in \mathcal{H}$ es distinto de cero, se tiene que $\lVert \alpha \rVert \neq 0$, entonces se puede definir $\alpha' = \bar{\alpha}/\lVert \alpha \rVert$. Entonces \[ \alpha\alpha= \alpha \frac{\bar{\alpha}}{\lVert \alpha \rVert} = 1 \] y de manera similar se demuestra que $\alpha'\alpha = 1$, de donde se sigue que $\alpha^{-1} = \alpha'$. Este argumento demuestra que $\mathcal{H}$ es un anillo de división.
Se define $\mathcal{H}_{\mathds{Q}}$ restringiendo $\mathcal{H}$ al campo de los números racionales. A este conjunto se le conoce como cuaterniones racionales.
Finalmente se definen los \textbf{cuaterniones enteros} como \[ \mathcal{H}_{\mathds{Z}} = \{ x_0 + x_1i + x_2j + x_3k \colon x_0, x_1, x_2, x_3 \in \mathds{Z} \}. \]
Es evidente que $\mathcal{H}_{\mathds{Z}}$ es un anillo de división.
Sea $\alpha \in \mathcal{U}(\mathcal{H}_{\mathds{Z}})$ entonces es inmediato, de la definición de norma, que $\lVert \alpha \rVert$ es un entero positivo, más aún, como $\alpha\alpha'= 1$ entonces $\lVert \alpha \rVert \lVert \alpha'\rVert = 1$ y consecuentemente $\lVert \alpha \rVert = 1$, es decir, $x_i = 1$ para algún índice $0\leq i \leq 3 $ y $x_j = 0 $ para $j \neq i$. De lo anterior se tiene \[ \mathcal{U}(\mathcal{H}_{\mathds{Z}}) = \{ \pm1, \pm i, \pm j, \pm k \}. \] 
\end{ejemplo} 
\begin{definicion}
Un subconjunto no vacío $S$ de un anillo $R$ es un \textbf{subanillo} de $R$ si es cerrado bajo las operaciones de $R$ y es un anillo respecto a estas operaciones.
\end{definicion}
\begin{definicion}
El \textbf{centro} de un anillo $R$ es el subanillo \[ \mathcal{Z}(R) = \{a \in R \colon ax = xa, \mbox{ para todo } x \in R \}. \]
\end{definicion}
\begin{definicion}
Un subconjunto no vacío $L$ de un anillo $R$ es un \textbf{ideal izquierdo} de $R$ si cumple las siguientes propiedades:
\begin{bulletList}
\item Si $x,y \in L$ entonces $x-y \in L$.
\item Si $x \in L$ y $a \in R$ entonces $ax \in L$.
\end{bulletList}
\end{definicion}
De manera similar se define un \textbf{ideal derecho} de un anillo $R$. Un subconjunto no vacío $L$ de un anillo $R$ se llama \textbf{ideal} de $R$ si es un ideal derecho e izquierdo de $R$. 
Los subconjuntos $\{0\}$ y $R$ de un anillo $R$ siempre son ideales de $R$. Un ideal $L$ de $R$ distinto de estos se llama \textbf{ideal propio}. Nótese que si un ideal $L$ de un anillo $R$ contiene un elemento invertible $a$ entonces $L=R$ no es ideal propio. 
\begin{definicion}
Sea $R$ un anillo y $a \in R$. El conjunto \[ RA = \{ xa \colon x \in R \} \] es llamado  el \textbf{ideal izquierdo generado por $a$}. Al elemento $a$ se le conoce como \textbf{generador} de este ideal. Los ideales generados por un elemento $a \in R$ se llaman \textbf{ideales principales} y se denotan como $(a)$.
\end{definicion}
\begin{proposicion}
Sea $D$ un anillo de división y $n$ un entero positivo. Entonces el anillo completo de matrices $M_n(D)$ no contiene ideales propios.
\end{proposicion}
\begin{definicion}
Sean $R, S$ anillos. Una aplicación $f \colon R \to S$ se llama \textbf{homomorfismo de anillos} si para cualesquiera $a,b \in R$ se cumple:
\begin{bulletList}
\item $f(a+b) = f(a) +f(b)$
\item $f(ab) = f(a)f(b)$.
\end{bulletList}
\end{definicion}
\begin{definicion}
Sea $f \colon R \to S$ un homomorfismo de anillos. Entonces la \textbf{imágen} de $f$ es el subanillo \[\Ima (f) = \{ y \in S \colon (\exists x \in R) f(x)=y \}. \]
El \textbf{kernel} de $f$ es el ideal\[ \ker(f) = \{ x \in R \colon f(x) = 0 \}. \]
\end{definicion}
\begin{definicion}
Un homomorfismo de anillos $f \colon R \to S$ es un \textbf{epimorfismo} si es sobreyectivo. Se dice que $f$ es un \textbf{monomorfismo} si es inyectivo. Finalmente, si $f$ es inyectivo y sobreyectivo entonces se dice que $f$ es un \textbf{isomorfismo}.
\end{definicion}
Un homomorfismo de un anillo $R$ en sí mismo es un \textbf{endomorfismo} y si también es un isomorfismo entonces se llama \textbf{automorfismo} de $R$.
\begin{definicion}
Sea $I$ un ideal de $R$. El grupo cociente aditivo $R/I$ es un anillo con la multiplicación definida como $\bar{r}\bar{s} = \overline{rs}$. Al anillo $R/I$ se le llama anillo cociente de $R$ por $I$. 
\end{definicion}

%----------------Inicio de algebras y módulos
\subsection{Módulos y álgebras}
\begin{definicion}
Sea $R$ un anillo. Un grupo aditivio abeliano $M$ es llamado un $R\mbox{-módulo}$ izquierdo si para todo $a \in R$ y $m \in M$ se cumple que $am \in M$ y 
\begin{bulletList}
\item $(a+b)m = am  + bm$
\item $a(m_1 + m_2) = am_1 + am_2$
\item $a(b)m = (ab)m$
\item $1m = m$ 
\end{bulletList}
para cualesquiera $a,b \in R$ y $m, m_1, m_2 \in M$.
\end{definicion}
De manera similar, dado un anillo $R$, se define un $R\mbox{-módulo}$ derecho considerando la multiplicación de elementos de $M$ por elementos de $R$ por la derecha. De acá en adelante, un $R\mbox{-módulo}$ izquierdo será abreviado como $R\mbox{-módulo}$. 
\begin{ejemplo}
Un ideal izquierdo $L$ de un anillo $R$ es un $R\mbox{-módulo}$, ya que el producto de elementos de $R$ por elementos de $L$ pertenece a $L$. Así mismo, los ideales derechos también son $R\mbox{-módulos}$ derechos.
En particular, un anillo siempre es un módulo sobre sí mismo. Cuando se considere un anillo como un $R\mbox{-módulo}$ izquierdo o derecho sobre sí mismo se denotará como $_{R}R$ y $R_R$ respectivamente.
\end{ejemplo}
\begin{ejemplo}
Sea $L$ un ideal derecho de un anillo $R$ y $R/L$ el grupo cociente aditivo. Entonces $R/L$ es un $R\mbox{-módulo}$ con \[  r(a+L) = ra +L, \mbox{ para todo } r, a \in R  \]
\end{ejemplo} 
\begin{definicion}
Sea $R$ un anillo conmutativo. Un $R\mbox{-módulo}$ $A$ es un $R\mbox{-álgebra}$ si existe una operación de multiplicación definida en $A$ tal que con su adición y esta multiplicación $A$ es un anillo que cumple: \[ r(ab) = (ra)b = a(rb), \] para todo $r \in R$ y $a,b \in A$.
\end{definicion}
\begin{definicion}
Sea $M$ un módulo sobre un anillo $R$. Un conjunto no vacío $N \subset M$ es llamado un $R\mbox{-submódulo}$ de $M$ si se cumplen las siguientes condiciones:
\begin{bulletList}
\item Para todos $x,y \in N$ se tiene $x + y \in N$
\item Para cualquier $r \in R$ y todo $n \in N$, $rn \in N$
\end{bulletList}
Si $R$ es conmutativo y $M$ es un $R\mbox{-álgebra}$, entonces  se dice que $N$ es un $R\mbox{-subálgebra}$ de $M$ si es un submódulo y un subanillo de $M$ al mismo tiempo. 
\end{definicion}
Todo módulo $M \neq \{0\}$ contiene al menos dos submódulos, a saber, $M$ y $\{0\}$, que son llamados \textbf{triviales}. Un submódulo no trivial es llamado \textbf{submódulo propio}. Un módulo, distinto al módulo que sólo contiene al $0$, que no contiene submódulos propios es un  \textbf{módulo simple}. 
Sea $N$ un submódulo de un $R\mbox{-módulo}$ $M$. Al igual que en el caso de los anillos, el grupo cociente aditivo $M/N$ es un $R\mbox{-módulo}$ con $r\bar{m} = \overline{rm}, r\in R, m \in M$. Este es el módulo cociente entre $M$ y $N$. 

%-----------------subsection modulos libres
\subsection{Módulos libres}
\begin{definicion}
Un conjunto $S = \{ s_i\}_{i \in I}$ de elementos de un $R\mbox{-módulo}$ $M$ es un \textbf{conjunto de generadores} de $M$ si $M = RS$, es decir, si todo elemtno de $M$ se puede escribir como un combinación lineal finita de elementos de $S$ con coeficientes en $R$.
\end{definicion}
\begin{definicion}
Un conjunto $S = \{ s_i\}_{i \in I}$ de elementos de un $R\mbox{-módulo}$ $M$ es  \textbf{linealmente independiente} o $R\mbox{-libre}$ si toda combinación lineal de elementos de $S$ con coeficientes en $R$ de la forma \[ r_{i_1}s_{i_1} + \cdots + r_{i_t}s_{i_t} = 0   \] implica que $r_{i_1} = \cdots = r_{i_t} = 0$. 
\end{definicion}
\begin{definicion}
Un conjunto $S = \{ s_i\}_{i \in I}$ de elementos de un $R\mbox{-módulo}$ $M$ es una \textbf{base} de $M$ sobre $R$ o una \textbf{$R\mbox{-base}$} si es linealmente independiente y un conjunto de generadores. 
\end{definicion}
\begin{definicion}
Un $R\mbox{-módulo}$ $M$ es \textbf{libre} si tiene una base.
\end{definicion}
\begin{definicion}
Sea $\{M_i\}_{i \in I}$ un familia de submódulos de un $R\mbox{-módulo}$ $M$. Se dice que $M$ es la \textbf{suma directa} de submódulos de esta familia y se escribe $M = \oplus_{i \in I}M_i$ si se cumplen las siguientes condiciones:
\begin{bulletList}
\item Para todo $i \in I$ se satisface que $M_i \cap \left( \sum_{j \neq i} M_j \right) = \emptyset$
\item $M = \sum_{i \in I}m_i$
\end{bulletList}
En particular, si $\{ m_i \}_{i \in I}$ es un $R\mbox{-base}$ de $M$, entonces $M$ es la suma directa de $M = \oplus_{i \in I}Rm_i$.
\end{definicion}
\begin{definicion}
Un submódulo $N$ de un $R\mbox{-módulo}$ $M$ es un \textbf{sumando directo} si existe otro módulo $N'$ tal que $M = N \oplus N'$. Un módulo que no contiene sumandos directos, a excepción de los triviales, se llama \textbf{indescomponible}. 
\end{definicion}
Caso contrario a los espacio vectoriales, no todo submódulo de un módulo dado es un sumando directo.
\begin{lema}
Sea $N$ un submódulo de un $R\mbox{-módulo}$ $M$. Entonces $N$ es un sumando directo de $M$ si y sólo si existe un endomorfismo $f \colon M \to M$ tal que $f \circ f = f$ e $\Ima (f) = N$.
\end{lema}
El homomorfismo $f \colon M \to M$ del lema anterior es la  \textbf{proyección} de $M$ en $N$. 
\begin{proposicion}\label{prop:bimodulos}
Sea $R$ un anillo. Todo $\modulo{R}$ $M$ es una imagen epimórfica de un $\modulo{R}$ libre. 
\end{proposicion}

% ------------Inicio subseccion semisimplicidad
\subsection{Semisimplicidad}
En álgebra lineal se demuestra que todo subespacio de espacio vectorial es un sumando directo. Esta aseveración no es válida en el caso de módulos sobre anillos arbitrarios, por ejemplo, $\mathds{Z}$ no es un sumando directo de $\mathds{Q}$ como $\mathds{Z}\mbox{-módulo}$. Será de interés conocer los módulos que cumplen la propiedad de tener submódulos que sean sumandos directos. 
\begin{definicion}
Un $R\mbox{-módulo}$ $M$ es \textbf{semisimple} si todo submódulo de $M$ es un sumando directo. 
\end{definicion}  
\begin{proposicion}
Sea $N \neq (0)$ un submódulo de un módulo semisimple $M$. Entonces $N$ es semisimple y contiene a un módulo simple.
\end{proposicion}
\begin{proof}
Sea $S$ un submódulo arbitario de $N$. Entonces $N$ es también submódulo de $M$, así que existe $S'$ tal que $M = S \oplus S'$. Se asegura que $N = S \oplus (S'\cup N)$. En efecto, por definición $S \cup (S'\cup N) \subset S\cup S' = (0)$. Por otro lado, dado un elemento $n \in N$ se puede escribir $n = x + y$ con $x \in S, y \in S'$, pero $y = n - x \in N$, entonces $y \in N \cup S'$, con lo que se demuestra que $N$ es semisimple. 
Para demostrar que $N$ contiene a un submódulo semisimple elíjase un elemento $x \in N, x \neq 0$. Considérese la familia de submódulos de $N$ que tienen a $x$ como elemento y nótese que dicha familia es no vacía, por lo que, usando el lema de Zorn, existe un elemento maximal $N_1$. Como $N$ es semisimple, existe $N_2$ submódulo de $N$ tal que $N = N_1 \oplus N_2$. Se requiere demostrar que $N_2$ es simple.
Si $N_2$ no fuera simple, entonces existe $W$ submódulo propio de $N_2$ tal que $N_2 = W \oplus W'$, con $W'$ submódulo de $N_2$. De esta manera, $N = N_1 \oplus W \oplus W'$ y $N_1 = (N_1 + W)\cup(N_1 + W')$. Como $x \notin N_1$ entonces $x \notin N_1 + W$ ni $x \not N_1 + W'$, lo cual contradice el hecho que $N_1$ es maximal.   
\end{proof}
\begin{teorema}\label{teo:caracSemi}
Sea $M$ un $\modulo{R}$. Entonces, las siguientes condiciones son equivalentes:
\begin{bulletList}
\item\label{item:ssimple1} $M$ es semisimple
\item\label{item:ssimple2} $M$ es suma directa de submódulos simples
\item\label{item:ssimple3} $M$ es suma - no necesariamente directa - de submódulos simples.
\end{bulletList}
\end{teorema}
\begin{proof}
\ref{item:ssimple1} $\implies$ \ref{item:ssimple2}. Sea $\mathcal{F}$ la colección de todos los submódulos de $M$ que se pueden escribir como suma directa de submódulos simples. La proposición anterior asegura la existencia de dichos submódulos. Se define un orden en $\mathcal{F}$ de la siguiente manera: dados dos elementos $\oplus_{i \in I}M_i$ y $\oplus_{i \in J}M_i$ de $\mathcal{F}$, se tiene que $\oplus_{i \in I}M_i  \prec \oplus_{i \in J}M_i$ si y sólo si $I \subset J$. 
Ahora como $(\mathcal{F, \prec})$ satisface las condiciones del lema de Zorn, existe un elemento maximal $M_0 \in \mathcal{F}$ que se puede escribir como $M_0 = \oplus_{i \in I}M_i$ con $M_i, i \in I$ simple.
Ahora sólo falta demostrar que $M_0 = M$. En efecto, si $M_0 \neq M$, entonces existe un submódulo $N$ de $M$ tal que $M = M_0 \oplus N$, pero, por la proposición anterior, $N$ contiene un submódulo simple $S$ y por lo tanto $M_0 \oplus S = \oplus_{i \in I}M_i \oplus S \supset M_0$, lo cual contradice la maximalidad de $M_0$.

\ref{item:ssimple2} $\implies$ $\ref{item:ssimple3}$ es trivial.

\ref{item:ssimple3} $\implies$ $\ref{item:ssimple1}$. Supóngase que $M = \sum_{i \in I}M_i$, donde cada componente $M_i, i \in I$ es simple. Sea $N$ un submódulo propio cualquiera de $M$. Se demostrará que $N$ es sumando directo. 
Considérese la familia \[ \mathcal{J} = \left\{ \sum_{i \in J} M_i \colon J \subset I, \left(\sum_{i \in J}M_i \right) \cap N = (0)  \right\} \] y nótese que si $N \cap M_i \neq (0)$ entonces $M_i \subset N$. Como $N \neq M$, se deduce que existe al menos un submódulo $M_i$ tal que $N \cap M_i = (0)$ y $\mathcal{J} \neq \emptyset$. Por el lema de Zorn se puede encontrar un submódulo maximal en $\mathcal{J}$, a saber $M_0 = \sum_{i \in \mathcal{J}_0}M_i$. Como $\left( \sum_{i \in \mathcal{J}_0}M_i  \right) \cap N = (0)$, sólo queda por demostrar que $M = M_0 + N$. Para ello se demostrará que $M_i \subset M_0 + N$, para todo $i \in I$. Supóngase por el absurdo que esto no es cierto, entonces existe un índice $i_{0}$ tal que $M_{i_0} \nsubseteq M_0 + N $. Ahora bien, $M_{i_0}$ es simple, se tiene que $M_{i_0} \cap (M_0 + N) = (0)$. Entonces $(M_{i_0} + M_0)\cap N =(0)$, lo que implica que $M_{i_0} + M_0 \in \mathcal{J}$, lo cual contradice la maximalidad de $M_0$.
\end{proof}
Si se conoce la descomposición de un módulo semisimple como suma directa de módulos simples, entonces se puede determinar la estructura de todos sus submódulos. 
\begin{corolario}
Sea $M = \oplus_{i \in I}M_i$ una descomposición de un módulo semisimple $M$ como suma directa de submódulos y sea $N$ un submódulo de $M$. Entonces, existe un subconjunto de índices $J \subset I $ tal que $N \simeq \oplus_{i \in J}M_i$.
\end{corolario}
\begin{proof}
Como se vió en la demostración de la última implicación del teorema anterior, dado un submódulo $N$ de $M$ se puede encontrar un subconjunto de índices $J_0 \subset I$  tal que $M = N \oplus N_0$, donde $N_0 = \oplus_{i \in J_0}M_i$. Entonces:
\[ N \simeq \frac{M}{N_0} = \frac{\oplus_{i \in I}M_i}{\oplus_{i \in J_0}M_i} \simeq \oplus_{i \in I \backslash} M_i, \] de donde se sigue el resultado, con $J = I \backslash J_0$.
\end{proof}
\begin{corolario}
Un módulo cociente $L$ de un módulo semisimple $M$ es isomorfo a un submódulo de $M$ y por lo tanto es también semisimple.
\end{corolario}
\begin{proof}
Sea $L$ un módulo cociente de $M$, $\pi \colon M \to L$ el homomorfismo canónico y $N = \ker(\pi)$. Entonces, existe un submódulo $N'$ de $M$ tal que $M = N \oplus N'$ y por lo tanto $N'\simeq M/\ker(\pi) \simeq L$. Con esto, el resultado se sigue del corolario anterior.
\end{proof}
\begin{definicion}
Un anillo $R$ es \textbf{semisimple} si como módulo $_RR$ es semisimple. 
\end{definicion}
Todo submódulo de $_RR$ es ideal izquierdo de un anillo $R$, así que $R$ es semisimple si y sólo si todo ideal izquierdo es un sumando directo.
\begin{teorema}
Sea $R$ un anillo. Entonces las siguientes proposicione son equivalentes:
\begin{bulletList}
\item\label{item:rssimple1} Todo $\modulo{R}$ es semisimple
\item\label{item:rssimple2} $R$ es un anillo semisimple
\item\label{item:rssimple3} $R$ es una suma directa de un número finito de ideales izquierdos minimales.
\end{bulletList}
\end{teorema}
\begin{proof}
\ref{item:rssimple1} $\implies$ \ref{item:rssimple2} es evidente.

\ref{item:rssimple2} $\implies$ \ref{item:rssimple3}. Como los submódulos de $_RR$ son precisamente los ideales izquierdos minimales de $R$, se sigue del teorema \ref{teo:caracSemi} que $R$ se puede escribir como $R = \oplus_{i \in I}L_i$ donde cada $L_i, i \in I$ es un ideal izquierdo minimal. De esta manera sólo queda por demostrar que esta suma es finita.
En particular, como $R = \langle 1 \rangle$, el elemento $1 \in R$ se puede escribir como una suma finita; a saber: $1 = x_{i_1} + \cdots + x_{i_n}$, donde $x_{i_j} \in L_{i_j}$. Entonces para $r \in R$, se tiene que $r = r \cdot 1 = rx_{i_1} + \cdots +rx_{i_n}$, donde $rx_{i_j} \in L_{i_j}, 1\leq j \leq n$. Esto demuestra que $R \subset L_{i_1} \oplus \cdots \oplus L_{i_n}$, de donde $R = L_{i_1} \oplus \cdots \oplus L_{i_n}$.
Nótese que el teorema \ref{teo:caracSemi} demuestra inmediatamente que \ref{item:rssimple3} $\implies$ \ref{item:rssimple2}, así que la demostración estará completa si se demuestra que \ref{item:rssimple2} $\implies$ \ref{item:rssimple1}. 
Supóngase que $R$ es semisimple y sea $M$ un $\modulo{R}$, entonces de la proposición \ref{prop:bimodulos} se sabe que $M$ es una imagen epimórfica de un $\modulo{R}$ libre $F$. Entonces $F$ se puede escribir como $F = \oplus_{i}Ra_i$ donde $Ra_i \simeq R$ es semisimple. Por esto, $F$ es semisimple y por lo tanto $M$ lo es. 
\end{proof}
\begin{teorema}\label{teo:idealIdem}
Sea $R$ un anillo. Entonces $R$ es semisimple si y sólo si todo ideal izquierdo $L$ de $R$ es de la forma $L = Re$, donde $e \in R$ es un idempotente.
\end{teorema}
\begin{proof}
Supóngase que $R$ es semisimple y sea $L$ un ideal  izquierdo de $R$. Entonces $L$ es un sumando directo de $R$ entonces existe un ideal izquierdo $L'$ tal que $R = L \oplus L'$. De esa cuenta, se puede escribir $1 = x + y$ donde $x \in L$ y $y \in L'$. Entonces $x = x \cdot 1 = x^2 + xy$. De la igualdad anterior se deduce que $xy = x - x^2 \in L$ y como $L'$ es un ideal izquierdo se tiene que $xy \in L'$. De la definición de suma directa se sabe que $L \cap L' = (0)$ por lo tanto $xy = x - x^2 = 0$, de donde $x = x^2$ es un idempotente. Es obvio que $Rx \subset L$. Además dado $a  \in L$ se tiene que $a = a \cdot 1 = ax + ay$, de esto se obtiene $a -ax = ay\in L\cap L'=(0)$ y así $a = ax \in Rx$ demostrando que $L = Rx$. 

Ahora bien, supóngase que los ideales izquierdos de $R$ son de la forma propuesta en el enunciado. Dado esto, se requiere demostrar que todo ideal izquierdo de $R$ es sumando directo. Por hipótesis $L = Re$ donde $e \in R$ es idempotente. Sea $L'= R(1-e)$. Entonces es claro que $L'$ es un ideal izquierdo y dado un elemento $x \in R$ se puede escribir $x = xe + x(1-e)$, así que $R = Re + R(1-e)$. Además si $x \in Re \cap R(1-e)$ se tiene que cumplir que $x = re = s(1-e)$, con $r,s \in R$. Entonces $xe = s(1-e)e = 0$, de donde $x = 0$.  
\end{proof}
\begin{teorema}\label{teo:familiaIdempotentes}
Sea $R = \oplus_{i = 1}^t L_i$ una descomposición de un anillo semisimple como suma directa de ideales izquierdos minimales. Entonces, existe una familia $\{e_1, \dots, e_t \}$ de elementos de $R$ tal que:
\begin{bulletList}
\item\label{item:orto1} $e_i \neq 0, 1 \leq i \leq t$ es idempotente
\item\label{item:orto2} Si $i \neq j$ entonces $e_ie_j = 0$
\item\label{item:orto3} $1 = e_1 + \cdots + e_t$
\item\label{item:orto4} $e_i$ no se puede escribir como $e_i = e_i'+ e_i''$, donde $e_i', e_i''$ son idempotentes tales que $e_i, e_i'' \neq 0$ y $e_i'e_i''=0, 1\leq i \leq t$
\end{bulletList}
Además, si existe una familia de idempotentes $\{e_1, \dots, e_t \}$ que satisface las cuatro condiciones anteriores entonces la familia de ideales izquierdos minimales $L_i = Re_i$ es tal que $R = \oplus_{i=1}^tL_i$.
\end{teorema}
\begin{proof}
Supóngase que $R = \oplus_{i = 1}^tL_i$ es una descomposición del anillo $R$ como suma directa de ideales izquierdos minimales. Con esta descomposición se puede escribir $1 = e_1 + \cdots + e_t$ donde $e_i \in L_i$. Entonces se deduce, como en el teorema anterior, que $e_i$ es idempotente tal que $L_i = Re_i, 1 \leq i \leq t$. Si $i \neq j$ entonces $e_ie_j = 0$. Por último, si para algún índice $i$ se puede escribir $e_i = e_i'+ e_i''$, donde $e_i', e_i''$ son idempotentes tales que $e_i, e_i'' \neq 0$ y $e_i'e_i''=0$ entonces de nuevo, como en el teorema anterior, se obtiene que $L_i = Re_i'\oplus Re_i''$ con $Re_i', Re_i'' \neq 0$, lo cual contradice la minimalidad de $L_i$. 

Para el converso, supóngase que existe una famlia de idempotentes $\{ e_1, \dots, e_t \}$ que satisfacen las condiciones dadas. Se demostrará en primera instacia, que $L_i = Re_i$ es minimal. Para ello supóngase por el absurdo que no lo es, entonces existe un ideal izquierdo $J$ tal que $J \subset L_i$, pero como $_RR$ es semisimple entonces $L_i$ también lo es, por lo tanto existe $J'$ tal que $L_i = J\oplus J'$. Esto implica que se puede escribir $e_i = e_i'+ e_i''$, donde $e_i', e_i''$ son idempotentes tales que $e_i, e_i'' \neq 0$, lo cual es una contradicción.
$R = L_1 + L_2 + \cdots + L_t$ se deduce fácilmente del hecho que $1 = e_1 + \cdots + e_t$. Ahora para demostrar que la suma es directa, tómese $x \in L_j \cap \left( \sum_{i \neq j}L_i \right)$. Entonces se puede escribir $x = r_je_j = \sum_{i \neq j}r_ie_i$. Multiplicando por $e_j$ por la derecha la ecuación anterior se obtiene $r_je_je_j = x = \sum_{i \neq j}r_ie_ie_j = 0$.
\end{proof}
\begin{definicion}
Sea $R$ un anillo. Una familia de idempotentes $\{ e_1, \dots, e_t \}$ que satisfacen las condiciones \ref{item:orto1}, \ref{item:orto2} y \ref{item:orto3} del teorema anterior es llamada una \textbf{famlia completa de idempotentes ortogonales}. Un idempotente que satisface la condición \ref{item:orto4} se llama \textbf{primitivo}.
\end{definicion}
\begin{lema}\label{lema:multiModulo}
Sea $L$ un ideal izquierdo minimal de un anillo semisimple $R$ y sea $M$ un $\modulo{R}$. Entonces $LM \neq (0)$ si y sólo si $L \simeq M$ como $R\mbox{-módulos}$. En este caso $LM = M$.
\end{lema}

%------------Inicio Subsección: teorema de Wedderburn-Artin
\subsection{El teorema de Wedderburn-Artin}
Este teorema y los que sirven de base para su demostración son de vital importancia, ya que revelan la estructura de los anillos semisimples.
\begin{lema}
Sea $L$ un ideal izquierdo minimal de un anillo semisimple $R$. Entonces la suma de todos los ideales izquierdos de $R$ isomorfos a $L$ es un ideal bilateral de $R$.
\end{lema}
\begin{proof}
Sea $A = \sum_{J \simeq L}J$. Es evidente que $A$ es un ideal izquierdo. Se desea demostrar que $A$ es también un ideal derecho. Como $R$ es semisimple se puede escribir $R = \oplus_{i = 1}^tL_i$ como suma directa de ideales izquierdos minimales. Entonces $AR = \sum_{J \simeq L}JR = \sum_{J \sim L}\sum_{i=1}^{t}JL_i$, pero $JL_i = (0)$ o $JL_i = L_i$. Por el lema \ref{lema:multiModulo} se demuestra que la última alternativa sólo es posible cuando $J \simeq L_i$, lo que implica que $L_i \subset A$. De esta manera se demuestra que $AR \subset A$.
\end{proof}
\begin{lema}
Sea $I$ un ideal que contiene a un ideal izquierdo minimal $L$ de un anillo semisimple. Entonces $I$ contiene a todos los ideales izquierdos isomorfos a $L$.
\end{lema}
\begin{proof}
Sea $ L \subset I$ un anillo izquierdo minimal y sea $J$ un ideal izquierdo isomorfo a $L$. Entonces, del lema \ref{lema:multiModulo}, se tiene que $J = LJ \subset I$. 
\end{proof}
\begin{proposicion}
Sea $L$ un ideal izquierdo minimal de un anillo simisimple $R$ y $B$ la suma de todos los ideales de $R$ isomorfos a $L$. Entonces $B$ es un ideal bilateral minimal de $R$.
\end{proposicion}
\begin{proof}
Sea $B_1$ un ideal de $R$ contenido en $B$ y $L_1$ un ideal izquierdo minimal de $R$ contenido en $B_1$. Si $L_1 \not\simeq L$, entonces se tiene que $L_1J = (0)$, para todo $J \simeq L$. Así, $L_1B = (0)$ lo cual implica, en particular, que $L_1L_1 = (0)$. Esto no es posible porque el teorema \ref{teo:idealIdem}  implica que $L_1$ contiene a un elemento idempotente. Este argumento implica que $L_1 \simeq L$ y por el lema anterior se obtiene que $B_1 = B$. 
\end{proof}
Dada una descomposición de un anillo semisimple $R$ como suma directa de ideales izquierdos minimales se puede agrupar los ideales izquierdos isomorfos de la siguiente manera:
\[ R = \underbrace{L_{11}\oplus \cdots \oplus L_{1r_1}} \oplus \underbrace{L_{21}\oplus \cdots \oplus L_{2r_2}} \oplus \cdots  \oplus \underbrace{L_{s1}\oplus \cdots \oplus L_{sr_s}}. \]
Con la notación anterior, $L_{ij} \simeq L_{ik}$ y $L_{ij}L_{kh} = (0)$ si $i \neq k$, por el lema \ref{lema:multiModulo}. 
\begin{teorema}
Con la notación anterior, sea $A_i$ la suma de todos los ideales izquierdos isomorfos a $L_{i1}, 1\leq i \leq s$. Entonces:
\begin{bulletList}
\item\label{item:Restructura1} Cada $A_i$ es un ideal minimal de $R$.
\item\label{item:Restructura2} $A_iA_j = (0)$ si $i \neq j$.
\item\label{item:Restructura3} $R = \oplus_{i=1}^{s}A_i$ como anillos, donde $s$ es el número de clases isomorficas de ideales minimales de $R$.
\end{bulletList}
\end{teorema}
\begin{proof}
\ref{item:Restructura1} se sigue directamente de la proposición anterior. Para demostrar \ref{item:Restructura2}, se escribe \[ R = (L_{11}\oplus \cdots \oplus L_{1r_1}) \oplus (L_{21}\oplus \cdots \oplus L_{2r_2}) \oplus \cdots  \oplus (L_{s1}\oplus \cdots \oplus L_{sr_s}). \]Entonces todo elemento $x \in R$ se pude escribir en la forma $x = x_{11} + \cdots + x_{rr_1} + \cdots + x_{s1} +\cdots +x_{xr_s}$, con $x_{ij} \in L_{ij}$. Sea $y_i = x_{i1} + \cdots + x_{ir_i}, 1\leq i \leq s$. Entonces $y_i \in A_i, 1\leq i \leq s$ y $x = y_1 + \cdots + y_s$. Esto demuestra que $R = A_1 + \cdots + A_s$. Para terminar el hecho que $A_i\cup A_j = (0)$ se sigue de la definición de $A_i$ y del lema \ref{lema:multiModulo}. 
\end{proof}
\begin{definicion}
Un anillo $R$ es \textbf{simple} si sus únicos ideales son $(0)$ y $R$.
\end{definicion}
Nótese que si $D$ es un anillo y $n$ un entero positivo, entonces $M_n(D)$ es un anillo simple.
\begin{corolario}
Los ideales $A_i, 1\leq i \leq s$, definidos previamente son simples.
\end{corolario}
\begin{proposicion}\label{prop:unicidadDescomposicion}
Sea $R = \oplus_{i=1}^s A_i$ la descomposición de un anillo semisimple $R$ como suma directa de ideales minimales. Entonces:
\begin{bulletList}
\item Todo ideal $I$ de $R$ se puede escribir de la forma $I = A_{i_1} \oplus \cdots \oplus A_{i_t}$, donde $1\leq i_{i_1} < \cdots < i_{t} \leq s$.
\item Si $R = \oplus_{j = 1}^rB_i$ es otra descomposición de $R$ como suma suma directa de ideales minimales, entonces $s = r$ y - después de una posible ordenación de los índices - $A_i = B_i$ para todo $i$.
\end{bulletList}
\end{proposicion}
\begin{proof}
Sea $I$ un ideal de $R$. Entonces $I = \oplus_{i = 1}^{s}(A_i \cap I)$. Como los $A_i$ son minimales la primera propiedad queda probada. Por la misma razón cada $B_j$ es igual a algún $A_i$ y viceversa. 
\end{proof}
\begin{definicion}
Los únicos ideales minimales de un anillo semisimple $R$ son llamados las \textbf{componentes simples de $R$}. 
\end{definicion}
\begin{teorema}
Sea $R = \oplus_{i = 1}^sA_i$ una descomposición de un anillo semisimple como suma directa de ideales minimales.Entonces existe una familia $\{e_1, \dots, e_s \}$ de elementos de $R$ tal que:
\begin{bulletList}
\item $e_i \neq 0, 1\leq i  \leq t$ es un idempotente central.
\item Si $i \neq j$ entonces $e_ie_j = 0$.
\item $1 = e_1 + \cdots + e_t$.
\item $e_i$ no puede ser escrito como $e_i = e_i'+e_i''$ donde $e_i', e_i''$ son idempotentes centrales tales que $e_i',e_i'' \neq 0$ y $e_i'e_i'' = 0, 1\leq i \leq t$.
\end{bulletList}
\end{teorema}
\begin{proof}
La demostración es idéntica a la del teorema \ref{teo:familiaIdempotentes}. La única diferencia es que $e_i, 1\leq i \leq s$ son centrales. Entonces para $x \in R$ se tiene de la tercera condición que $x = \sum_{i=1}^{t}xe_i = \sum_{i=1}^{t}e_ix$. Como los $A_i$ son ideales y la suma es directa se concluye que $xe_i = e_ix$.
\end{proof}
\begin{definicion}
Los elementos $\{ e_1, \dots, e_s \}$ del  teorema anterior son llamados los \textbf{idempotentes centrales primitivos de $R$}.
\end{definicion}
\begin{lema}\label{lema:estructuraSemi}
Sea $R$ un anillo, $M = M_1 \oplus \cdots \oplus M_r$ y $N = N_1 \oplus \cdots \oplus \cdots N_s$ dos $R\mbox{-módulos}$ escritos como sumas directas de submódulos. Sea $\epsilon_j \colon M_j \to M$ la inclusión de $M_j$ en $M$ y $\pi_i \colon N \to N_i$ el homomorfismo natural de $N$ hacia sus componentes.
\begin{bulletList}
\item Supóngase que para cualquier par de índices $i,j$ existe un homomorfismo $\phi_{ij} \in \hom_R(M_j, N_i)$. Entonces,  la aplicación $\phi \colon M \to N$ definida por: \[\phi(m_1 + \cdots + m_r) = \begin{pmatrix}
\phi_{11} & \cdots & \phi_{1r} \\
\cdots & \cdots & \cdots \\
\phi_{s1} & \cdots & \phi_{sr}
\end{pmatrix} 
\begin{pmatrix}
m_1 \\
\cdots \\
m_r
\end{pmatrix}\]
\[ = \underset{\in N_1}{\underbrace{\phi_{11}(m_1) + \cdots + \phi_{1r}(m_r)}} +\cdots+\underset{\in N_s}{\underbrace{\phi_{s1}(m_1) + \cdots + \phi_{sr}(m_r)}}, \] es un homomorfismo. Para indicar que $\phi$ es de la forma previamente descrita se se escribe $\phi = (\phi_{ij})$.
\item El converso también es cierto, es decir, si $\phi$ es de la forma descrita en el inciso anterior, esntonces $\phi_{ij} = \pi_i\circ\phi\circ\epsilon_j \in \hom_R(M_j, N_i)$ y $\phi = (\phi_{ij})$.
\item Para $\phi = (\phi_{ij})$ y $\phi = (\psi_{ij})$ se tiene que $\phi + \psi = (\phi_{ij} +  \psi_{ij})$.
\item $\hom_R(M^{(n)}, M^{(n)}) \simeq M_n(\hom_R(M,M))$ como anillos. 
\end{bulletList} 
\end{lema}
\begin{lema}
Sea $R$ un anillo, $M$ un $\modulo{R}$ semisimple y $B= \hom_R(M,M)$. Entonces $M$  admite una estructura de $\modulo{B}$ dada por $\phi \cdot m = \phi(m),$ para todo $\phi \in B, m \in M$. Más aún para cada $m \in M$ y $f \in \hom_B(M,M)$ existe un elemento $a \in R$ tal que $f(m) = am$.  
\end{lema}
\begin{proof}
La primera aseveración es evidente. Para demostrar la segunda, sea $m \in M$ y considérese el submódulo $Rm$. Como $M$ es semisimple, entonces existe un submódulo $W$ tal que $M = RM \oplus W$. Si se denota  la proyección hacia $R_m$ como $\phi \colon M \to M$ se tiene que $\pi \in \hom_R(M,M) = B$. Dado un elemento $f \in \hom_B(M,M)$, se tiene:\[ f(m) = f(\pi(m)) = \pi(f(m)) \in Rm. \] Así, existe un elemento $a \in R$ tal que $f(m) = am$. 
\end{proof}
\begin{teorema}[Teorema de densidad de Jacobson]
Sea $M$ un $\modulo{R}$ semisimple, $B= \hom_R(M,M)$ y $f \in  \hom_B(M,M)$. Si $\{ m_1, \dots, m_n \}$ es un conjunto arbitario de elementos de $M$, entonces existe un elemento $a \in R$ tal que $f(m_i) = am_i$, para todo $1\leq i \leq n$.
\end{teorema}
\begin{proof}
Dada $f \in \hom_B(M,M)$ se define $f^{(n)} \colon M^{(n)} \to M^{(n)}$ definida por: \[ f^{(n)}(x_1 + \cdots + x_n) = f(x_1) + \cdots f(x_n), \ x_1, \dots, x_n \in M. \]
Sea $B' = \hom_R(M^{(n)}, M^{(n)})$. Se asegura que $f^{(n)} \in \hom_{B'}(M^{(n)}, M^{(n)})$. En efecto, dado $\phi \in B'$, por lema \ref{lema:estructuraSemi} se puede escribir $\phi = (\phi_{ij}) \in \hom_R(M_j,M_i)$. Se tiene
\begin{eqnarray*}
f^{(n)} \circ \phi(m_1+\cdots +m_n) &=& \\ 
f^{(n)}(\phi_{11}(m_1) +\cdots+\phi_{1n}(m_n)+\cdots+\phi_{n1}(m_1)+\cdots+\phi_{nn}(m_n)) &=& \\
\phi_{11}(f(m_1)) + \cdots + \phi_{1n}(f(m_1)) + \cdots + \phi_{n1}(f(m_1 )) + \cdots + \phi_{nn}f((m_n)) &=&  \\
\phi(f(m_1) + \cdots + f(m_n) &=& \\
\phi \circ f^{(n)}(m_1 + \cdots + m_n).
\end{eqnarray*}
Por el lema anterior, existe un elemento $a \in R$ tal que $f^{(n)}(m_1+\cdots+m_n) = a(m_1+\cdots+m_n)$, por lo tanto $f(m_i) = am_i, 1\leq i \leq n$.
\end{proof}
\begin{lema}[Lema de Schur]
Sea $R$ un anillo, $M,N$ $R\mbox{-módulos}$ simples y $f \colon M \to N$ un homomorfismo no nulo. Entonces $f$ es un isomorfismo.
\end{lema}
\begin{proof}
Dado que $\Ima(f)$ es un submódulo de un módulo simple $N$ y no es igual a $(0)$, entonces $\Ima(f) = N$, así que $f$ es epimorfismo. De manera similar $\ker(f)$ es un submódulo de un módulo simple $N$ y no es igual a $M$ entonces $\ker(f) = (0)$. De esto $f$ es un monomorfismo y por lo tanto $f$ es isomorfismo.
\end{proof}
\begin{corolario}
Sea $R$ un anillo y $M,N$ $R\mbox{-módulos}$ simples. Entonces:
\begin{bulletList}
\item Si $M \not\simeq N$ entonces $\hom_R(M,N) = (0)$.
\item $\hom_R(M,M)$ es un anillo de división. 
\end{bulletList}
\end{corolario}
\begin{teorema}[Wedderburn-Artin]
Un anillo $R$ es semisimple si y sólo si es una suma directa de álgebras de matrices sobre anillos de división:
\[ R \simeq M_{n_1}(D_1)\oplus \cdots \cdots M_{n_s}(D_s).  \]
\end{teorema}
Para la demostración de este importante resultado se sugiere ver \cite[página 200]{bib:AlgebraPostGrado}.
\begin{teorema}
Sea $R$ un anillo semisimple y supóngase que 
\[ R \simeq M_{n_1} \oplus \cdots \oplus M_{n_s} \simeq M_{m_1}(D_1')\oplus \cdots \oplus M_{m_r}(D_r'), \] donde $D_i, D_j', 1 \leq i \leq s, 1 \leq j \leq r$ son anillos de división. Entonces $s = r$ y, bajo un posible reordenamiento de los índices, se tiene que $n_i = m_i, D_i \simeq D_i'$. 
\end{teorema} 
\begin{proof}
Como los anillos de matrices sobre anillos de división son simples se tiene por la proposición \ref{prop:unicidadDescomposicion} que $s = r$ y existe una biyección entre los dos conjuntos de ideales tal que los correspondientes ideales son iguales. Sólo falta demostrar que si $M_n(D)\simeq M_m(D')$, donde $D$ y $D'$ son anillos de división, entonces $n = m$ y $D \simeq D'$.

Sea $E = M_n(D)$, $E' = M_m(D')$ y 
\[ L = \begin{pmatrix} 
D &  0 & \cdots & 0 \\
D &  0 & \cdots & 0 \\
 & & \cdots & \\
 D &  0 & \cdots & 0
\end{pmatrix} , \begin{pmatrix}
D' & 0 & \cdots & 0\\
D' & 0 & \cdots & 0\\
 & & \cdots & \\
D' & 0 & \cdots & 0\\
\end{pmatrix}.\]
Entonces $L = eE, L'= fE'$ donde $e$ y $f$ son las correspondientes matrices idempotentes con 1 en la posición $(1,1)$ y ceros en cualquier otra posición. Bajo el isomorfismo $E \to E'$, $e$ tiene imagen $e'$ un idempotente tal que $e'L'$ es un ideal izquierdo minimal. Cambiando la base de $E'$ se tiene el isomorfismo $E \to E'$ tal que $e \mapsto f$ y $L \to L$. Entonces \[ D \simeq eEe \to fE'f \simeq D' \] y contando las dimensiones se llega a que $n = m$.  
\end{proof}


\chapter{GRUPO-ANILLOS}

%----------------->Definicion formal de los grupo-anillos, 
\section{Hechos Básicos De Los Grupo-Anillos}
En este capítulo se darán las definiciones formales matemáticas que dan paso al estudio de los grupo-anillos y se relacionará la teoría de grupos y anillos con esta nueva estructura matemática.

Considérese la siguiente construcción: Sea G un grupo cualquiera y R un anillo cualquiera. Entonces se define $RG:=\{\alpha | \alpha \colon  G \to R , |sop(\alpha)|< \infty \}$ donde $sop(\alpha):=\{g \in G: \alpha (g)\neq 0\}$, a el conjunto $sop(\alpha)$ se le llama el soporte de $\alpha$. Se puede observar entonces que los elementos de $RG$ son funciones con soporte finito. 

Como $RG$ es un conjunto de funciones, se puede considerar la suma usual de funciones para definir la operación suma en $RG$, a saber $+ \colon RG\times RG \to R $ de tal forma que si $\alpha, \beta \in RG$ entonces $(\alpha +\beta)(g):=\alpha(g)+\beta(g)$ para todo $g$ elemento de $G$ . Similarmente se puede definir la operación producto en $RG$ como $\cdot \colon RG \times RG \to R$ de tal forma que si $u\in G$ $(\alpha \cdot \beta )(u):= \sum_{gh=u}\alpha(g)\beta(h) $. Con estas nociones en mente   se procede a definir a un grupo-anillo. 
\begin{definicion}
El conjunto $RG$ con las operaciones $+$ y $\cdot$ mencionadas anteriormente es llamado el \textbf{grupo-anillo de G sobre R}. En el caso en que R es conmutativo a RG se le llama también el \textbf{grupo-algebra de G sobre R} 
\end{definicion}

Ahora se procede a mostrar dos teoremas que son básicos para el estudio de esta nueva estructura algebraica.

\begin{teorema}
Existe una copia de $G$ en $RG$, es decir, se puede encontrar $G_1 \subset RG$ tal que existe  un homomorfismo entre $G$ y $G_1$.  
\end{teorema}

\begin{proof}
Considérese la función $i \colon G \to RG$ tal que $x \mapsto \alpha$ donde $\alpha(x)=1$ y $\alpha(g)=0$ si $g\neq 0$. Con la identificación anterior es fácil notar que $i$ es una función inyectiva.
En efecto, si $x,y \in G$ entonces $i(x)=\alpha$ , $i(y)=\beta$, pero $\alpha \neq \beta$ si $x \neq y$, por definición.
Ahora se probará que $i$ es un homomorfismo de grupos. Nótese que $i(xy)=\gamma$ , donde $\gamma(xy)=1$ y $\gamma(g)=0$ si $g \neq xy$. Por otro lado, $i(x)i(y)=\alpha\beta$ donde $(\alpha\beta)(u)= \sum_{gh=u}\alpha(g)\beta(h)  $, pero el producto $ \alpha(g)\beta(h)  $ se anula a menos que  $g=x$ y $h=y$, en cuyo caso la función vale $1$, con lo que se ha demostrado que  $i(x)i(y)=i(xy)$. 
\end{proof}

Generalmente a $i$ se le llama la función de inclusión, así que será la forma en que se nombrará de aquí en adelante.

\begin{teorema}
Existe una copia de $R$ en $RG$.
\end{teorema}


\begin{proof}
Considérese la función $v \colon R \to RG$ tal que $v(r) = \beta$ con $\beta(g) = r$ si $ g = 1_G $ y $\beta(g)=0$ si $g \neq 1_G$. Es claro que $v$ es inyectiva y la demostración es exactamente igual que en el teorema anterior. Ahora falta probar que $v$ es un homomorfismo de anillos (con la aclaración que el hecho que RG es un anillo se probará mas adelante). En efecto, $v(sr)=\theta$ donde $\theta(g)=sr$ si $g=1_G$ y $\theta(g) = 0$ si $g \neq 1_G$. De manera similar se tiene que $v(s)v(r)=\gamma\beta$ donde $(\gamma\beta)(u)=\sum_{gh=u}\gamma(g)\beta(h) $ pero $\gamma $ y $\beta$ se anulan a menos que $g=h=1_G$ y en ese caso $u=1_G$, por lo que se ha probado que $v$ es un homomorfismo de anillos. \qedhere
\end{proof}

Con las identificaciones anteriores en mente es fácil probar la siguiente propiedad. 
\begin{propiedad}
Si $g \in G$ y $r \in R$ entonces $rg=gr$ en $RG$.
\end{propiedad}
 
\begin{proof}
Nótese que $r=\gamma$ y $x=\alpha$ y usando la definición del producto en $RG$ se ve que $rx=\gamma\alpha$ donde $(\gamma\alpha)(u)=\sum_{gh=u}\gamma(g) \alpha(h) $ pero por definición $\gamma$ y $\alpha$ se anulan en todas partes excepto en $g=1_G$ y $h=x$ respectivamente, por lo tanto $(\gamma\alpha)(u)=r$ cuando $u=x$ y $(\gamma\alpha)(u)=0$ para $u \neq x $ \\
Por otro lado $xr=\alpha\gamma$ dada por $(\alpha\gamma)(u)=\sum_{gh=u}\alpha(g)\gamma(h)$ de nuevo la función sólo existe cuando $g=x$ y $h=1_G$ de esa forma $(\alpha\gamma)(u)=r$ cuando $u=x$ y se anula en cualquier otro caso, con la cual concluye la demostración.  \qedhere
 \end{proof}

La definición de grupo-anillo que se presentó anteriormente es bastante rigurosa y además es bien definida, ya que se ha construido un espacio vectorial de funciones en el cual todas las operaciones tienen sentido, lo cual le brinda el soporte necesario para trabajar en álgebra. En algunas ocasiones resulta un poco tedioso y complicado estar trabajando sobre un espacio vectorial de funciones, así que se replanteará los grupo-anillos como \textit{R-combinaciones lineales}, es decir, a cada elemento de $RG$ se le asigna una combinación lineal de elementos de $G$ con coeficientes en $R$, de la siguiente manera

\begin{equation}
\alpha = \srg{g}{G}{a}
\end{equation}


donde $a_g \in R$ y $a_g \neq 0$ si $g \in sop(\alpha)$


\begin{nota}
Con la identificación anterior se verifica que la suma de $\alpha, \beta \in RG$ es componente a componente, es decir $\alpha + \beta= \srg{g}{G}{a}+\srg{g}{G}{b}= \sum_{g\in G} (a_g+b_g)g $ y el producto está dado por $\alpha\beta=\sum_{g,h\in G}a_gb_hgh$

\end{nota}


Ahora es práctico establecer los siguientes teoremas:

\begin{teorema}\label{grupo}
$RG$ es un grupo aditivo

\end{teorema}

\begin{proof}
Se procede por incisos:
\begin{itemize}
\item[i)]  Sean $\alpha, \beta, \gamma \in RG$ entonces $\alpha+(\beta+\gamma)=\srg{g}{G}{a}+\left(\srg{g}{G}{b}+ \srg{g}{G}{c} \right)= \srg{g}{G}{a}+\left(\sum_{g\in G}(b_g+c_g)g\right)=\sum_{g\in G}(a_g+b_g+c_g)g= \sum_{g\in G}((a_g+b_g)+c_g)g=\left(\sum_{g\in G}(a_g+b_g)g \right) +\srg{g}{G}{c}=(\alpha+\beta)+\gamma $

\item[ii)] Existe $0 \in RG$ tal que $0+\gamma=\gamma+0=\gamma$ para cualquier $\gamma \in RG$. A saber $0=\sum_{g \in G}0\cdot g$. Con esta identificación en mente se procede a hacer los cálculos: $\alpha + 0 = \sum_{g \in G}(a_g+0)g=\sum_{g\in G}(0+a_g)=\srg{g}{G}{a}=\alpha$

\item[iii)] Existe $-\alpha$ tal que $\alpha+(-\alpha)= (-\alpha)+\alpha =0$ para cualquier $\alpha \in RG$. En efecto $-\alpha = \srg{g}{G}{-a}$ y por lo tanto $\alpha+ (-\alpha)=\sum_{g\in G}(a_g+(-a_g) )g= \sum_{g\in G}((-a_g)+a_g)g = \sum_{g \in G}0\cdot g = 0 $

\item[iv)] $\alpha + \beta = \srg{g}{G}{a}+\srg{g}{G}{b}=\sum_{g \in G}(a_g+b_g)g= \sum_{g \in G}(b_g+a_g)g=\beta + \alpha $ \qedhere
\end{itemize}

La clausura de la operación $+$  se sigue directamente de la definición. Vale la pena notar que para realizar esta prueba se uso simplemente el hecho que $G$ es grupo y $R$ es un anillo y por lo tanto satisfacen propiedades algebraicas respecto de sus operaciones.
\end{proof}

Nótese que se ha probado que $(RG,+)$ es un grupo abeliano, lo cual será de utilidad para el siguiente teorema:

\begin{teorema}
$RG$ es un anillo con las operaciones $+$ y $\cdot$
\end{teorema}


\begin{proof}
Ya se ha probado que $(RG,+)$ es un grupo abeliano, por lo que a continuación se probará, de nuevo por incisos, que $(RG,\cdot)$ es asociativo y distributivo tanto por la derecha como por la izquierda:
\begin{itemize}
\item[v)] $\alpha(\beta\gamma) = \left(\srg{g}{G}{a}\right)\left[\left(\srg{g}{G}{b}\right)\left(\srg{g}{G}{c}\right)\right] = \left(\srg{g}{G}{a}\right)\left(\sum_{g,h\in G}b_gc_hgh\right) = \sum_{f,g,h \in G}a_f(b_gc_h)f(gh) = \sum_{f,g,h \in G} (a_fb_g)c_h(fg)h = (\alpha\beta)\gamma$
\item[vi)]  $\alpha(\beta + \gamma) = \left( \srg{g}{G}{a} \right) \left(\srg{g}{G}{b} + \srg{g}{G}{c} \right) = \srg{g}{G}{a} \left( \sum_{g \in G}(b_g+c_g)\right) = \sum_{g,h \in G}a_g(b_h+c_h)gh  = \sum_{g,h \in G} a_gb_hgh + \sum_{g,h \in G} a_gc_hgh = \alpha\beta + \alpha\gamma$  
\item[vii)] $(\alpha + \beta)\gamma = \left( \sum_{g\in G}(a_g+b_g)g\right) \left(\srg{g}{G}{c}\right) = \sum_{g,h \in G} (a_g+b_g)c_hgh = \sum_{g,h \in G}a_gc_hgh + \sum_{g,h\in G}b_gc_hgh = \alpha\gamma + \beta\gamma$ \qedhere


\end{itemize}
\end{proof}


Es de interés estudiar la estructura algebraica de $RG$, así que se introduce una operación mas sobre $RG$


\begin{definicion}
Sea $\lambda \in R$ entonces se define el producto por elementos del anillo como: 
\begin{equation}
\lambda \left(\srg{g}{G}{a}\right) = \srg{g}{G}{\lambda a}
\end{equation}
\end{definicion}

Con esta definición podemos proclamar el siguiente teorema

\begin{teorema}
$RG$ es un $R$ -módulo
\end{teorema}


\begin{proof}
Ya se estableció en el teorema \ref{grupo} que $(RG,+)$ es un grupo aditivo. De la definición anterior se sigue que $\lambda\gamma \in RG$. Ahora se procede por incisos: 
\begin{itemize}
\item[i)] $(\lambda_1+\lambda_2)\alpha = \sum_{g \in G} (\lambda_1+\lambda_2)a_gg = \srg{g}{G}{\lambda_1 a} + \srg{g}{G}{\lambda_2 a} = \lambda_1\alpha + \lambda_2\alpha$
\item[ii)] $\lambda(\alpha + \beta) = \lambda\sum_{a_g+b_g}g = \sum{g \in G}\lambda(a_g+b_g)g = \srg{g}{G}{\lambda a} + \srg{g}{G}{\lambda b} = \lambda\alpha + \lambda\beta$
\item [iii)] $\lambda_1(\lambda_2\alpha) = \lambda_1\srg{g}{G}{\lambda_2 a} = \sum_{g\in G}(\lambda_1(\lambda_2 a_g))g = \sum_{g\in G}((\lambda_1\lambda_2)a_g)g = \lambda_1\lambda_2\alpha$
\item[iv)] $1_R\alpha = \sum{g \in G}1_Ra_gg = \srg{g}{G}{a}$
\end{itemize}
Y con esto concluye la prueba. \qedhere

\end{proof}

Una extensión del resultado anteriormente presentado es que si $R$ es un anillo conmutativo entonces $RG$ es un álgebra sobre $R$. Se puede resaltar que si $R$ es conmutativo  entonces el rango de $RG$ como módulo libre sobre $R$ está bien definido, de hecho si $G$ es finito se tiene que $rango(RG) = |G|$

Ahora se establecerá un resultado de mucha importancia en los grupo-anillos, ya que relaciona a estos con los homomorfismos, que es uno de los objetivos del álgebra.

\begin{proposicion}\label{up}
Sea $G$ un grupo y $R$ un anillo. Dado cualquier anillo $A$ tal que $R \subset A$ y cualquier función $f \colon G \to A$ tal que $f(gh) = f(g)f(h)$ para cualquier $g,h \in G$, existe un único homomorfismo de anillos $f^* \colon  RG \to A$, que es $R$-lineal, tal que $f^*\circ i = f$, donde $i$ es la función de inclusión. Lo anterior se reduce a decir que el siguiente diagrama es conmutativo:
\[\xymatrix { G \ar[r]^f 
\ar[d]_i & A\\
RG \ar@{--}[ru]_{f^*} & }\]
\end{proposicion}

\begin{proof}
Considérese la función $f^* \colon RG \to A$ tal que $f^*(g)=\sum_{g\in G}a_gf(g)$. Ahora solo falta hacer los cálculos correspondiente para mostrar que $f^*$ es un homomorfismo de anillos. En efecto, $f^*(\alpha + \beta ) = \sum_{g \in G}(a_g + b_g)f(g) = \sum_{g \in G}a_gf(g) + \sum_{g \in G}b_gf(g) = f^*(\alpha) + f^*(\beta) $. \\ 
Similarmente $f^*(\alpha\beta)=\sum_{g,h\in G}a_gb_hf(gh)= \sum_{g,h\in G}a_gb_hf(g)f(h) = f^*(\alpha)f^*(\beta)$. Ahora sea $r \in R$ entonces $f^*(r\alpha)=\sum_{g\in G}ra_gf(g)=r\sum_{g\in G}a_gf(g)=rf^*(\alpha)$
Sea $x \in G$ entonces $i(x) = \srg{g}{G}{a}$ donde $a_g = 1$ si $g = x$ y $a_g= 0$ en cualquier otro caso, por lo tanto $f^*(i(g))= \sum_{g \in G} a_gf(g)=f(x)$. De los cálculos anteriores se sigue que $f^*\circ i = f$, con lo cual concluye la prueba. 
\qedhere
\end{proof}

De la proposición anterior se deriva un corolario que no es mas que un caso especial de la misma,  pero se establecerá por aparte porque será de utilidad en el desarrollo de este trabajo de graduación.

\begin{corolario}\label{aumento}
Sea $f \colon G \to H$ un homomorfismo de grupos. Entonces, existe un único homomorfismo de anillos $f^* \colon RG \to RH$ tal que $f^*(g) = f(g)$ para cualquier $g \in G$. Si $R$ es conmutativo, entonces $f^*$ es un homomorfismo de $R-$ álgebras, mas aún si $f$ es un epimorfismo (monomorfismo), entonces $f^*$ es también un epimorfismo (monomorfismso)
\end{corolario}

\begin{proof}
Usar el teorema anterior con $A=RH$ lo anterior se puede hacer porque $RH$ es un anillo que contiene a R y hay una copia de $H$ en $RH$, con lo cual se deriva que debe existir $f^*$ homomorfismo $R-$ lineal de anillos tal que $f^*(g)=f(g)$ para cualquier elemento $g \in G$. Con lo cual concluye la prueba.
\qedhere
\end{proof}




De hecho la proposición \ref{up} se puede utilizar como una definición de $RG$, como se sigue de la siguiente proposición:

\begin{proposicion}
Sea $G$ un grupo y $R$ un anillo. Sea $X$ un anillo conteniendo $R$ y $\nu \colon G \to X$ una función tal que $\nu (gh) = \nu(g)\nu(h)$ para todo $g,h \in G$ y tal que, para todo anillo $A$ que contiene a $R$ y cualquier función $f \colon G \to A$ que satisface $f(gh) = f(g)f(h)$ para todo $g, h \in G$, existe un único homomorfismo $R$-lineal $f^* \colon X \to A$ tal que el siguiente diagrama es conmutativo: 
\[\xymatrix { G \ar[r]^f 
\ar[d]_{\nu}
 & A\\
X \ar@{ --}[ru]_{f^*} & }\]
Entonces $X \simeq RG$
\end{proposicion}


\begin{proof}
La demostración es tan simple como notar que el siguiente diagrama conmuta con $I_X$

$$ \xymatrix @!0 @R=3pc @C=4.5pc { & X \ar@{--}[d]^{i*} \ar@{ --}^{I_X}[ddr] \\ & RG \ar@{--}_{\nu^*}[rd] & \\ G \ar[ruu]^{\nu} \ar[ru]^{i} \ar[rr]^{\nu}& & X}$$

\qedhere
\end{proof}

\begin{nota}
Si en el corolario \ref{aumento} se hace $H=\{1\}$ y se considera la función $m \colon G \to \{1\}$ entonces esta función induce un homomorfismo de anillos $\epsilon \colon RG \to R$ tal que $\epsilon\left(\srg{g}{G}{a}\right) = \sum_{g \in G} a(g)$. 
\end{nota}

\begin{definicion}
El homomorfismo $\epsilon \colon RG \to R$ dado por \[ \epsilon \left( \srg{g}{G}{a}\right) = \sum_{g \in G}a_g \] es llamado la \textbf{ función de aumento de RG} y su núcleo, denotado por $\Delta (G)$, es llamado el \textbf{el ideal de aumento} de $RG$
\end{definicion}


Ahora se puede decir algunas propiedades importantes del ideal de aumento de $RG$. Nótese que si un elemento $\alpha = \srg{g}{G}{a}$ pertenece al ideal de aumento entonces $ \epsilon \left( \srg{g}{G}{a}\right) = \sum_{g \in G}a_g = 0 $ por lo tanto se puede escribir $\alpha$ de la siguiente forma: 
\[\alpha = \srg{g}{G}{a} -\sum_{g \in G}a_g = \sum_{g \in G}a_g(g-1) \]
Por lo tanto es claro que cualquier elemento de la forma $g-1, g \in G$ pertenece a $\Delta(G)$, mas aún se acaba de probar que el conjunto $\{g-1 : g \in G, g \neq 1\}$ es un conjunto de generadores del ideal de aumento de $RG$. Por otro lado, de la definición de $RG$ se sigue que el conjunto anterior en linealmente independiente, con lo cual se ha probado la siguiente proposición:

\begin{proposicion} \label{gen}
El conjunto $\{ g-1 : g \in G , g \neq 1\}$ es base de $\Delta (G)$ sobre $R$. Es decir, se puede escribir
\begin{equation*}
\Delta (G) = \left\{ \sum_{g \in G} a_g(g-1) : g \in G , g \neq 1, a_g \in R \right\}
\end{equation*}
donde, como es usual, se debe asumir que solo un número finito de los coeficientes $a_g$ son distintos de cero.
\end{proposicion}

Nótese que, en particular, si $R$ es conmutativo y $G$ es finito, entonces $\Delta (G)$ es un módulo libre sobre $R$ con rango $|G|-1$

Se concluye esta sección mostrando que el grupo-anillo $RG$ donde $R$ es conmutativo es un anillo con \textbf{involución}

\begin{proposicion}
Sea $R$ un anillo conmutativo. La función $* \colon RG \to RG$ definida por 
\begin{equation}
\left(\sum_{g \in G}a(g)g\right)^* = \sum_{g \in G} a(g) g^{-1}
\end{equation}
satisface: 

\begin{itemize}
\item[(i)] $(\alpha + \beta)^* = \alpha^* + \beta^*$
\item[(ii)] $(\alpha\beta)^* = \beta^*\alpha^*$
\item[(iii)] $\alpha^* = \alpha$
\end{itemize}
\end{proposicion}


\begin{proof}
Se procede por incisos:
\begin{itemize}
\item[(i)] $\left( \sum_{g \in G} (a_g +b_g)g\right)^* =  \sum_{g \in G} (a_g +b_g)g^{-1} = \alpha^* + \beta^*$
\item[(ii)]  $\left( \sum_{g, h \in G} (a_gb_h)gh\right)^* = \sum_{g,h \in G} a_gb_hh^{-1}g^{-1} = \sum_{g,h \in G} b_h a_g h^{-1}g^{-1} = \beta^*\alpha^*$
\item[(iii)]  $\left(\left( \srg{g}{G}{a}\right)^*\right)^* = \left(\sum_{g\in G}a_gg^{-1}\right)^* = \srg{g}{G}{a}$ \qedhere
\end{itemize}
\end{proof}


%----------------------------------------------------------> Ideales de Aumento
\section{Ideales de aumento}

En lo que sigue es de mucho interés encontrar condiciones de $R$ y $G$ que permitan descomponer a $RG$ como sumas directas de ciertos subanillos. Será de especial interés conocer cuando $RG$ es un anillo semisimple para así poder escribirlo como sumas directas de ideales minimales. \\


Con este fin se hará un estudio de la relación que hay entre los subgrupos de $G$ y los ideales de $RG$. Está relación tendrá mucho utilidad cuando se trate con problemas concernientes a la estructura y propiedades de $RG$. Estas relaciones aparecieron por primera vez en un artículo publicado por A. Jennings (dar cita) y, en la forma que se presentará en este trabajo, en el trabajo hecho por W. E. Deskins (dar cita). La idea de aplicarlo por primera vez en el estudio de la reducibilidad completa (como se hará en la siguiente sección) fue de I.G. Connell (dar cita).

Ya en materia de hecho, considérese el grupo $G$ y el anillo $R$, se denotará con $\mathcal{S}(G)$  el conjunto de todos los subgrupos de $G$ y con $\mathcal{I}(RG)$ el conjunto de los ideales por izquierda de $RG$.

\begin{definicion}
Para un subgrupo $H \in \mathcal{S}(G)$ se denota por $\Delta_{R}(G,H)$ el anillo por izquierda de $RG$ generado por el conjunto $\{h-1: h \in H \}$. Esto es, 
\begin{equation}
\Delta_{R}(G,H) = \left\{ \sum_{h \in H} \alpha_h(h-1) : \alpha_h \in RG \right\}
\end{equation}
\end{definicion}

Cuando se esté trabajando con un anillo fijo $R$ se omitirá el subindíce y por lo tanto al ideal anterior se le denotará simplemente como $\Delta(G,H)$. Nótese que el ideal $\Delta(G,G)$ coincide con $\Delta(G)$, del cual se habló en la sección anterior.

\begin{lema}
Sea $H$ un subgrupo de un grupo $G$ y sea $S$ el conjunto de los generadores de $H$. Entonces, el conjunto $\{ s-1 : s \in S \}$ es un conjunto de generadores de $\Delta (G,H) $ como ideal por izquierda de $RG$ 

\end{lema}

\begin{proof}
Como $S$ es un conjunto de generadores de $H$, cada elemento $1 \neq h \in H$ puede ser escrito en la forma $h=s_1^{\epsilon_1} s_2^{\epsilon_2} \cdot s_r^{\epsilon_r} $ donde $s_i \in S$ y $\epsilon_i =\pm 1$, $1 \leq i \leq r$. Por lo tanto es suficiente probar que todo elemento de la forma $h-1$ con $h \in H$  pertenece al ideal generado por $\{s-1 : s \in S \}$. Para hacer esto se procede por inducción matemática sobre $r$. \\

\textbf{Caso Base:} Nótese que el menor caso sucede en $r =2$. Por lo tanto sea $h \in H$ entonces $h-1 = s_1^{\epsilon_1} s_2^{\epsilon_2} = s_1^{\epsilon_1}( s_2^{\epsilon_2} -1 ) + ( s_1^{\epsilon_1} -1 ) \in (S)  $ donde (S) es el ideal generado por $\{ s-1 : s \in S \}$ \\ 

\textbf{Hipótesis de Inducción} 
Supóngase que cualquier expresión de la forma $ (s_1^{\epsilon_1} s_2^{\epsilon_2} \cdots s_k^{\epsilon_k} -1)\in (S) $


\textbf{Conclusión}
Considérese la expresión de la forma $ (s_1^{\epsilon_1} s_2^{\epsilon_2} \cdots s_k^{\epsilon_k}s_{k+1}^{\epsilon_{k+1}} -1) $, hágase la sustitución $x =  s_1^{\epsilon_1} s_2^{\epsilon_2} \cdots s_k^{\epsilon_k} $ entonces $ (s_1^{\epsilon_1} s_2^{\epsilon_2} \cdots s_k^{\epsilon_k}s_{k+1}^{\epsilon_{k+1}} -1)  = x s_{k+1}^{\epsilon_{k+1}} -1 = x( s_{k+1}^{\epsilon_{k+1}} -1 ) + (x-1) \in (S) $ ya que $x-1, x( s_{k+1}^{\epsilon_{k+1}} -1 ) \in (S)$ por la hipótesis de inducción. La prueba está casi completa, sola falta decir que si apareciera algún $\epsilon_i = -1$ se aplica la factorización $y^{-1}-1 = y^{-1}(1-y)$ y el problema está resuelto.\qedhere
\end{proof}


Para dar un mejor caracterización de $\Delta_R (G,H)$, denótese con $\mathcal{T} = \{q_i\}_{i \in I}$ un conjunto completo de representantes de clases izquierdas de $H$ en $G$, un \textit{transversal} de $H$ en $G$. Se asumirá que siempre se elige como representante de la clase $H$ en $\mathcal{T}$ a la unidad de $G$. De esa manera todo elemento $g \in G$ puede ser escrito de manera única en la forma $g = q_ih_j$ con $q_i \in \mathcal{T}$ y $h_j \in H$


\begin{proposicion}
El conjunto $B_H = \{q(h-1) : q \in \mathcal{T}, h \in H, h \neq 1  \}$ es una base de $\Delta_R(G,H)$ sobre $R$.
\end{proposicion} 

\begin{proof}
Se procede en dos partes, primero se debe probar que el conjunto dado es linealmente independiente y luego que también es un generador de $\Delta_R(G,H)$. 

\textbf{Independecia Lineal} Supóngase que se tiene una combinación lineal de elementos de $B_H$ que se anula, esto es $\sum_{i,j}r_{ij}q_i(h_j-1) =0 $ con $r_{ij} \in R$. De lo anterior se sigue que $\sum_{i,j}r_{ij}q_i(h_j)-\sum_{i,j}r_{ij}q_i = 0$ por lo tanto $ \sum_{i,j}r_{ij}q_i(h_j) = \sum_{i,j}r_{ij}q_i $ lo cual se puede rescribir como $\sum_{i,j}r_{ij}q_ih_j =  \sum_i\left( \sum_jr_{ij}q_i \right)$. En la igualdad anterior se puede observar que como $h_j \neq 1$ entonces necesariamente el lado izquierdo de la ecuación tienen distinto soporto que el lado derecho, por lo tanto ambos deben ser igual a cero, pero los elementos de $G$ son linealmente independientes sobre $R$ entonces $r_{ij} = 0$  para todo $i,j$.

\textbf{Generador}
Se debe probar que $B_H$ es generador de $\Delta_R(G,H)$ para esto es suficiente demostrar que $g(h-1)$  se puede expresar  como combinación lineal de elemtos de $B_H$. Para esto basta recordar que $g = q_ih_j$ para algún $q_i \in \mathcal{T}$ y $h_j \in H$ entonces $g(h-1) = q_ih_j(h-1) = q_i(h_jh-1)+ (q_i-1) $ con lo que se demuestra lo que se pedía. \qedhere
\end{proof}


\begin{nota}
Es claro que si $G=H$ en la proposición anterior entonces $\mathcal{T} = \left\{ 1 \right\}$ y por lo tanto $B_H = \left\{ (h-1 , h \in H, h \neq 1) \right\}$ y así esto se reduce a la proposición \ref{gen} 

\end{nota}

Ahora se explorará la opción usual cuando se está hablando de subgrupos, es decir, los subgrupos normales. De hecho, si $H \lhd G$ entonces el homomorfismo canónico $\omega : G \to G/H$ puede ser extendido a un epimorfismo, a saber 

\[\omega* : RG \to R(G/H)\]

tal que 

\[\omega^*\left(\srg{g}{G}{a} \right) = \sum_{g \in G} a_g\omega(g)\]

\begin{proposicion}
Con la notación anterior
$$Ker(\omega^*) =\Delta(G,H)$$
\end{proposicion}

\begin{proof}
Considérese de nuevo $\mathcal{T}$ el transversal de $H$ en $G$. Entonces, cada elemento $\alpha \in RG$ se puede escribir como $ \alpha = \sum{i,j} r_{ij} q_ih_j$, $r_{ij} \in R$, $q_i \in \mathcal{T}$, $h_i \in H$. Si se denota $\overline{q_i} = \omega(q_i)$ entonces se tiene

\[\omega^*(\alpha) = \sum_i\left(\sum_jr_{ij}\right)\overline{q_i} \]

Entonces, $\alpha \in Ker(\omega^*)$ si y sólo si $ \sum_jr_{ij} = 0 $ para cada calor de $i$. Entonces si se tiene un $\alpha \in Ker(w^*)$ se puede escribir

\begin{eqnarray}
\alpha &=& \sum_i\left(\sum_jr_{ij}\right)\overline{q_i} \\
 &=& \sum_{ij}r_{ij}q_i(h_j-1) \in \Delta(G,H)  
\end{eqnarray}

Con lo cual se tiene que $Ker(\omega^*) \subset \Delta(G,H)$. El hecho que $\Delta(G,H) \subset Ker(\omega^*)$ es trivial, por lo tanto $Ker(\omega^*) = \Delta(G,H)$ \qedhere
\end{proof}

\begin{corolario}
Sea $H$ un subgrupo normal de $G$. Entonces $\Delta(G,H)$ es un ideal bilateral de $RG$ y
\[\frac{RG}{\Delta(G,H)} \simeq R(G/H)\]
\end{corolario}

\begin{proof}
Como $Ker(\omega^*) = \Delta(G,H)$ entonces por el primer teorema de ismorfia $ \frac{RG}{\Delta(G,H)} \simeq Im(\omega^*) $ pero como $\omega^*$ es sobreyectiva entonces $Im(\omega^*) = R(G/H)$ con lo que concluye la prueba. \qedhere
\end{proof}


Hasta este punto se ha visto que hay una relación entre subgrupos normales de $G$ e ideales bilaterales de $RG$, es decir, se pueden construir funciones de $\mathcal(S) $ a $\mathcal{I}(RG)$. La pregunta es entonces, ¿Qué pasa con las funciones en la otra vía?. Para responder esa pregunta considérese 
$$ \nabla(I) = \{ g \in G \colon  g-1 \in I\}$$
Es fácil notar que $\nabla(I) = G \cap (1+I)$

\begin{lema}
$\nabla(I)$ es subgrupo de $G$
\end{lema}
\begin{proof}
Se debe probar dos cosas:
\begin{itemize}
\item[(i)] Sean $g_1,g_2 \in \nabla(I)$ entonces 

\[g_1g_2 -1 = g_1(g_2-1) + (g_2-1) \in I \]
por lo tanto $g_1g_2 \in \nabla(I)$

\item[(ii)] Si $g \$in \nabla(I)$ entonces $g^{-1} -1 = g^{-1}(1-g) \in I$ de donde se sigue que $g^{-1} \in \nabla(I)$ \qedhere
\end{itemize}
\end{proof}

\begin{lema}
Si $I$ es un ideal bilateral entonces $\nabla(I) \lhd G$
\end{lema}

\begin{proof}
Se quiere probar que $gig^{-1} \in \nabla(I)$ entonces todo se reduce a demostrar que $gig^{-1} -1 \in I$. Nótese que $gig^{-1}-1 = gi(g^{-1}-1)+(gi-1) $ como $I$ es ideal bilateral, entonces $gi(g^{-1}-1) \in I$ y $(g_i-1) \in I$  por lo tanto $gig^{-1} \in I$. \qedhere
\end{proof}


\begin{proposicion}
Si $H \in \mathcal(S)(G)$ entonces $\nabla(\Delta(G,H)) = H$ 
\end{proposicion}

\begin{proof}
Sea $1 \neq x \in \nabla(\Delta(G,H))$ entonces $x-1 \in \Delta(G,H)$ por lo tanto se puede escribir 
$$x-1 = \sum_{i,j}r_{ij}q_i(h_j-1)$$
Como 1 aparece en el lado izquierdo de la ecuación también debe aparecer en el lado derecho, por lo tanto alguno de los $q_i$ debe ser $q_1=1$ por lo tanto hay en término de la forma $r_{1j}(h_j-1)$ . Nótese que todos los elementos de $G$ del lado derecho de la ecuación son distintos a pares pero $x$ debe aparecer allí, por lo tanto $x = h_j$. De lo anterior es inmediato que $\nabla(\Delta(G,H)) \subset H$. La otra contención es trivial. \qedhere
\end{proof}

Según lo expuesto en la proposición anterior pareciera ser cierto que $\nabla$ y $\Delta/$ son funciones inversas la una de la otra, pero esto no es cierto. 
Si se toma un ideal $I \in \mathcal(I)(RG)$ entonces ¿Qué pasa con $\Delta(G,\nabla(I))$ ? Pues bien, sea $x \in \Delta(G, \nabla (I))$ entonces $x = \sum_{i,j} r_{ij}q_i(m_j-1)$ , $m_j \in \nabla(I)$ por lo tanto $m_j -1\in I$ y de alli que $x \in I$. Con eso se ha probado que $\Delta(G,\nabla(I)) \subset I$, pero la igualdad no es necesariamente cierta. Considérese $I=RG$ entonces $\nabla(RG) = G$ de donde $\Delta (G,\nabla(RG)) = \Delta G \neq RG $


%--------------------------------------------------------------------->Semisimplicidad

\section{Semisimplicidad}

Con lo visto en la anterior sección ahora es accesible determinar condiciones necesarias y suficientes de $R$ y $G$ para que $RG$ sea semisimple.
Pero antes se probarán algunos resultados técnicos acerca de aniquiladores. 

\begin{definicion}
Sea $X$ un subconjunto de $RG$. El aniquilador de $X$ por la izquierda es el conjunto
\[ Ann_{i}(X) = \left\{ \alpha \in RG : \alpha x = 0, \forall x \in X \right\} \]
y de manera análoga el aniquilador de $X$ por la derecha es el conjunto
\[ Ann_{d}(X) = \left\{ \alpha \in RG : x\alpha  = 0, \forall x \in X \right\} \]

\end{definicion}

\begin{definicion}
Dado un grupo-anillo $RG$ y un subconjunto finito $X$ del grupo $G$, se denotará por $\hat{X}$ los siguientes elementos de $RG$
\[\hat{X} = \sum_{x \in X}x\] 
\end{definicion}

\begin{lema}
Sea $H$ un subgrupo de $G$ y sea $R$ un anillo.Entonces $Ann_{d}(\Delta(G,H)) \neq \{ 0\}$ si y sólo si $H$ es finito. En ese caso, se tiene
$$Ann_d(\Delta(G,H)) = \hat{H} \cdot RG $$
Mas aún, si $H \lhd G$ entonces $\hat{H}$ es central en $RG$ y 
\[Ann_d(\Delta(G,H)) = Ann_i(\Delta(G,H)) = RG \cdot \hat{H}\]
\end{lema}

\begin{proof}
Supóngase que $Ann_d(\Delta(G,H)) = \{ 0\}$ y considérese $\alpha = \srg{g}{G}{a} \in RG$, $\alpha \in Ann_d(\Delta(G,H))$ entonces
\begin{eqnarray}
(h-1)\alpha &=& 0  \quad \mbox{para cada } h \in H  \\
h\alpha -\alpha &=&  0  \\
\sum_{g \in G} a_gah &=& \srg{g}{G}{a} \label{ep}
\end{eqnarray}

De la última ecuación se aprecia que $hg \in sop(\alpha)$ siempre y cuando $g \in sop(\alpha)$, pero $sop(\alpha)$ es finito, por tanto $H$ es finito.
De nuevo analizando la ecuación \ref{ep} se deduce que dado $g_0 \in sop(\alpha)$ entonces $hg_o \in sop(\alpha)$ para cualquier $h$ elemento de $H$. De allí que se de la siguiente igualdad:
\[ \alpha = a_{g_0}\hat{H}g_0 + \cdots + a_{g_t}\hat{H}g_t = \hat{H}\beta, \quad \beta \in RG \]

Lo anterior muestra que si $H$ es finito entonces $Ann_d(\Delta(G,H))\subset \hat{H}RG$. Por otro lado $h\hat{H} = \hat{H}$ ya que $H$ es finito, entonces $h\hat{H} -\hat{H} = 0$ y por consiguiente $(h-1)\hat{H} = 0$ de donde $\hat{H}RG \subset Ann_d(\Delta(G,H))  $

Por último si $H \lhd G$ entonces para todo $g$ elemento de $G$ se cumple que $gHg^{-1} = H$ de donde $g\hat{H}g^{-1} = \hat{H}$ de donde se concluye inmediatamente que $\hat{H}g = g\hat{H}$ lo cual prueba que $\hat{H}$ es central en $RG$ y de alli se sigue fácilmente la conclusión. \qedhere
\end{proof}

Del lema anterior se sigue el siguiente corolario.

\begin{corolario}
Sea $G$ un grupo finito. Entonces 

\begin{itemize}
\item[(i)] $Ann_i (\Delta(G)) = Ann_d(\Delta(G)) = R\cdot \hat{H}$
\item[(ii)] $Ann_d(\Delta(G)) \cap \Delta (G) = \{ a\hat{G} : a \in R , a|G| = 0\}$ 
\end{itemize}
\end{corolario}


\begin{proof}
Se procede por incisos
\begin{itemize}
\item[(i)] Ya se ha establecido que $\Delta(G,G) = G$, por lo tanto hágase $H=G$ en el teorema anterior y el resultado es inmediato.
\item[(ii)] Sea $x \in Ann_d(\Delta G) \cap \Delta G$ entonces $x = a\sum_{g\in G}g$ y además $x \in Ker(\omega^*)$ por tanto $Ker(x)= a\omega^*\hat{G} = a|G| = 0 $ \qedhere
\end{itemize}
\end{proof}


\begin{lema}
Sea $I$ un ideal bilateral de $R$. Supóngase que existe un ideal por la izquierda $J$ tal que $R = I \oplus J$ (como $R-$ módulos). Entonces $J \subset Ann_d(I) $
\end{lema}

\begin{proof}
Sea $x \in J$ y $y \in I $ entonces $yx \in J$, $yx \in I$ entonces $yx \in J\cap I$ por lo tanto $yx=0$ de donde $x \in Ann_d(I)$ y por consiguiente $J \subset Ann_d(J)$ \qedhere 
\end{proof}

\begin{lema}\label{aumento}
Si el ideal de aumento de $RG$ es un sumando directo de $RG$ como un $RG-$módulo entonces $G$ es finito y $|G|$ es invertible en $R$
\end{lema}

\begin{proof}
Las condiciones anteriores aseguran que existe $J$ como en el lema anterior tal que $RG = \Delta G \oplus J$ de donde $J \subset \Delta G$  y por tanto $\Delta G \neq \{ 0 \}$, con lo cual $G$ es necesariamente finito.
Por otra parte $1 \in RG$ entonces $1 = e_1 + e_2$ donde $e_1 \in \Delta G$ y $e_2=a\hat{G}$, de lo cual se sigue que $\epsilon(1) = 1 = \epsilon(e_1) + \epsilon(e_2)$ pero $\epsilon(e_1) = 0$ por ser $\Delta G$ el núcleo de $\epsilon$ por ende se tiene $a|G| = 1$ con lo que se ha mostrado lo pedido. \qedhere
\end{proof}

Ahora se está en disposición de determinar condiciones necesarias y suficientes en $R$ y $G$ para que el grupo-anillo $RG$ sea semisimple. Los primeros resultados que apuntaron en esta dirección fueron dados por Maschkes, logros que están plasmados en el siguiente teorema:

\begin{teorema}[Maschke]
Sea $G$ un grupo. Entonces, el grupo-anillo $RG$ es semisimple si y sólo si las siguientes condiciones son verdaderas:
\begin{itemize}
\item[(i)] $R$ es un anillo semisimple
\item[(ii)] $G$ es finito
\item[(iii)] $|G|$ es invertible en $R$
\end{itemize}
\end{teorema}

\begin{proof}
Se procederá a probar las implicaciones en ambos sentidos:
\begin{itemize}
\item En esta parte se asume que $RG$ es semisimple, por lo tanto se puede utilizar el hecho que $\frac{RG}{\Delta (G)} = R$. De lo anterior se deduce que $R$ es un cociente y ya se ha demostrado que los cocientes son simples. Por otro lado se sabe que $\Delta (G)$ es un ideal y de la semisimplicidad de $RG$ se sabe que $\Delta (G)$ es sumando directo y del lema \ref{aumento} se asegura que las condiciones (ii) y (iii) se satisfacen. 

\item  Para mostrar la segunda implicación, asúmase que (i), (ii) y (iii) son verdaderas. 
De (i) se sigue que $RG$ es semisimple como $R-modulo$. \footnote{Recordar que esto es por una propiedad de anillos que debo poner en el capitulo 1} Considérese $M$ como $RG-modulo$, tal que $M \in RG$, entonces existe $N$ como $R$-modulo tal que 
\[RG = M \oplus N\]

Sea $\pi  RG \to M$ la proyección canónica asociada con la suma directa. Se define $\pi ^ *  \colon RG \to M$ tal que:

\[x \mapsto \frac{1}{|G|}\sum_{g \in G} g^{-1}\pi (gx) \quad   \mbox{para cada } x \in RG \] 

Es claro que dicha función existe, ya que $G$ es finito por (ii) y es $\frac{1}{|G|} < \infty$ por (iii). Se desea probar que $\pi ^ * $ es un $RG$ - homomorfismo tal que $(\pi ^*)^2 = \pi ^* $ y $M = Im(\pi ^*)$, lo cual se muestra en dos partes a continuación:

\textbf{Homomorfismo:} \newline
Basta demostrar que $\pi ^* (ax) = a \pi^*(x) \quad \mbox{para cada } a, g \in G$, ya que $\pi^* $ ya es un $R$ - homomorfismo. En efecto $\pi^* (ax) = \frac{1}{|G|} \sum_{g \in G}g^{-1}\pi (gax) = \frac{a}{|G|} \sum_{g \in G}(ga)^{-1}\pi ((ga)x) $.

Ahora se tiene que $ga \in G$, por ser $G$ un grupo, por lo tanto cuando $g$ recorre todo $G$ el producto $ga$ también lo hará, ya que $a$ es un elemento dado fijo. Por lo tanto la última expresión se puede volver a escribir como:

\[\pi^* (ax) = \frac{a}{|G|} \sum_{t \in G}t^{-1}\pi (tx) = a\pi^*(x)\]

\textbf{Sobreyectiva y Composición:} \newline
Nótese que $gm \in M$ ya que $M$ es un $RG$ - modulo, así que $\pi (gm) = gm$ y por lo tanto 

\[\pi ^* (m) = \frac{1}{|G|} \sum_{g \in G}g^{-1}\pi (gm) = \frac{1}{|G|} \sum_{g \in G}g^{-1}(gm) = \frac{1}{|G|}|G|m  = m\]

De lo anterior se sigue que $Im(\pi^*) \subset M$ y además $(\pi^*)^2 = \pi $. Por otro lado sea $m \in M$, entonces $\pi ^* (m) = m \in Im(\pi^*)$, de donde $M \in Im(\pi^*)$. 

Por lo anteriormente expuesto se sigue directamente que $Ker(\pi^*)$ es un $RG$ - submodulo tal que $RG = M \oplus ker(\pi^*)$  \qedhere

\end{itemize}
\end{proof} 

Como es usual en ciencias, se explorará un caso particular del teorema anterior con la interrogante natural ¿Qué pasa si en lugar de un anillo se considera un campo?. La pregunta anterior se reduce a contemplar el caso $R = K$, donde $K$ es un campo. Un campo siempre es semisimple, además se sabe que $|G|$ es invertible siempre y cuando $|G| \neq 0$, es decir, $car(K) \nmid |G|$, de donde se sigue el siguiente corolario:

\begin{corolario}
Sea $G$ un grupo finito y $K$ un campo. Entonces $KG$ es semisimple si y solo si $car(K) \nmid |G|$
\end{corolario}


Aunque no es el objetivo de este trabajo de graduación dar una descripción de los grupo-álgebra, resulta tentador replantear el teorema de Wedderburn-Artin en este contexto, con lo cual se brinda mas información acerca de la estructura algebraica de un grupo-álgebra.

\begin{teorema}
Sea $G$ un grupo finito y sea $K$ un campo tal que $car(K) \nmid |G|$. Entonces:
\begin{itemize}
\item $KG$ es suma directa de un numero finito de ideales bilaterales $\{B_i\}_{ 1 \leq i \leq r}$, los componentes simples de $KG$. Cada $B_i$ es una anillo simple.
\item Todo ideal bilateral de $KG$ es suma directa de algunos de los miembros de la familia $\{B_i\}_{ 1 \leq i \leq r}$
\item Cada componente simple $B_i$ es isomorfo a un anillo completo de matrices de la forma $M_{n_i}(D_i)$, donde $D_i$ es un anillo de división conteniendo una copia isomorfa de $K$ en su centro. Además el isomorfismo
\[KG \simeq \oplus_{i=1}^{r} M_{n_i}(D_i) \]
es un isomorfismo de álgebras. 
\item En cada anillo de matrices $M_{n_i}(D_i)$, el conjunto
\[ I_i =  \left\{ \begin{bmatrix}
x_1 & 0 & \dots & 0 \\
x_2 & 0 & \dots & 0 \\
\hdotsfor{4} \\
x_{n_i} & 0 & \dots & 0 \\
\end{bmatrix} : x_1, x_2, \dots, x_{n_i} \in D_i \right\} \simeq D_i^{n_i} \] 
es un ideal minimal izquierdo. 
Dado $x \in KG$, se considera $\phi (x) = (\alpha_1, \dots, \alpha_r) \in \oplus_{i=1}^r M_{n_i}(D_i)$ y se define el producto de $x$ por un elemento $m_i \in I_i$ como $xm_i = \alpha_im_i$. Con esta definición, $I_i$ se convierte en un $KG$ - módulo simple.

\item $I_i \not \simeq I_j$, si $ i \neq j$
\item Cualquier $KG$ - módulo simple es isomorfo a algún $I_i, \quad 1 \leq i \leq r$ 
\end{itemize}
\end{teorema}

Se ha hecho énfasis en este resultado, ya que en el siguiente capitulo de este trabajo, se explorará la conexión entre este resultado y la teoría de representación de grupos. 

\begin{corolario}
Sea $G$ un grupo finito y $K$ un campo algebraicamente cerrado tal que $car(K) \nmid |G| $. Entonces:
\[Kg \simeq \oplus_{i=1}^{r} M_{n_i}(K)\]ss
y $n_1^2 + n_2^2 + \dots + n_r^2 = |G| $
\end{corolario}

\begin{proof}
Como $car(K) \nmid |G|$ es inmediato que 
\[KG \simeq \oplus_{i=1}^{r} M_{n_i}(D_i)\]
donde $D_i$ es un anillo de división conteniendo una copia de $K$ en su centro. Calculando la dimensión sobre $K$ en ambos lados de la ecuación se tiene:

\[ |G| = \sum_{i=1}^{r} n_i^2[D_i:K]\]

de donde se sique que cada anillo de división $D_i$ es finito dimensional. Sea $0 \neq d_i \in D_i$ entonces $kd_i = 0$ implica que $k=0$. Similarmente, dado $a_i \in D_i$ tal que $kd_i 0 a_i$ se tiene que $k = a_id_i^{-1} \in K$ por ser $K$ algebraicamente cerrado y por lo tanto $[D_i:K = 1]$ y $D_i = K$ para $1 \leq i \leq r$, con lo cual concluye la demostración. 
\end{proof}

%------------------------------> grupo-algebras de grupos abelianos
\section{Grupo-Algebras de grupos abelianos}

 En esta sección se dará una descripción completa de grupo-anillo cuando el grupo es finito y además abeliano. 
 
 Como en la parte final de la sección anterior, se supone que $K$ es un campo tal que $car(K) \nmid |G|$. Esta caracterización fue dada por primera vez por S. Perlis y G Walker (dar la referencia). 
 
 Se comenzará con el caso donde $G$ es un grupo cíclico, así que se asume que $G = <a \colon a^n = 1>$ y que $K$ 4s un campo tal que $car(K) \nmid |G|$. Considérese la función $\phi \colon K[X] \to KG$ dada por 
 \[K[X] \ni f \mapsto f(a) \in KG \]
 Debido a que la función $\phi$ consiste en tomar un polinomio de $K[G]$ y evaluarlo en $a$, es obvio que $\phi$ es un epimorfismo de anillos y por lo tanto:
 \[KG \simeq \frac{K[X]}{Ker(\phi)}\]
donde $ker(\phi) = \{f \in K[X] : f(a) = 0 \}$. Como $K[X]$ es un dominio\footnote{poner esto en el capitulo uno y hacer referencia} de ideales principales se deduce que $Ker(\phi)$ es un ideal generado por el polinomio mónico $f_0$, de menor grado posible, tal que $f_0(a) = 0$.

Nótese que bajo el isomorfismo anterior, es claro que el elemento $a \in RG$ se mapea en $X + (f_0) \in \frac{K[X]}{(f_0)}$. Además de $a^n = 1$ se sigue que $X^n -1 \in Ker(\phi)$, ya que si existiera un polinomio $f = \sum_{i=0}^{r}k_ix^i$ con $r < n$ entonces $f(a) \neq 0$ debido a que los elementos de $\{1,a,a^2, \dots, a^r \}$ son linealmente independientes sobre $K$.De esa manera se puede asegurar que $Ker(\phi) = (X^n -1)$ por lo que se satisface 

\[KG \simeq \frac{K[X]}{(X^n -1 )} \]

Sea $ X^n -1 = f_1f_2\cdots f_t$ la descomposición de $X^n -1$ como producto de polinomios irreducibles en $K[X]$. Como se está asumiendo que $char(K) \nmid n$, este polinomio debe ser separable y por lo tanto $f_i \neq f_j$ si $i \neq j$. Utilizando el teorema chino del residuo \footnote{tambien en la parte inicial y luego referencia a el} se puede escribir:

\[KG \simeq \frac{K[X]}{f_1} \oplus \frac{K[X]}{f_2} \oplus \cdots \oplus \frac{K[X]}{f_t} \]


Utilizando este isomorfismo es fácil notar que el generador $a$ tiene imagen $( X + (f_1)  , \dots, X + (f_t) ) $. 

Considérese $\zeta_i$ una raíz de $f_i$, $1 \leq i \leq t$. Entonces, se tiene $\frac{K[X]}{(f_i)} \simeq K(\zeta_i)$. Por lo tanto

\[ KG \simeq K(\zeta_1) \oplus K(\zeta_2) \oplus \dots \oplus K(\zeta_t)  \]

Como todos los elementos $\zeta_i , \quad 1 \leq i \leq t$ son raíces de $X^n -1$, se ha probado que $KG$ es isomorfo a la suma directa de extensiones ciclotómicas de $K$. Finalmente baja este 
ultimo isomorfismo el elemento $a$ tiene imagen $(\zeta_1 , \zeta_2, \dots ,\zeta_t)$

Antes de continuar, se presentan algunos ejemplos para estudiar y comprender de mejor manera como trabajan las conclusiones anteriores.

\begin{ejemplo}
Sea $G = < a \colon a^7 = 1>$ y $K = \mathds{Q} $. 
En este caso la descomposición de $ X^7 -1$ en $\mathds{Q}$ es 
\[ X^7 -1 = (X-1)(X^6 + X^5 + X^4 + X^3 + X^2 + X + 1) \]

de esta forma si $\zeta$ es una raíz de la unidad de orden 7 distinta de 1, se puede escribir lo siguiente

\[  \mathds{Q}G = \mathds{Q}(1) \oplus \mathds{Q}(\zeta) = \mathds{Q} \oplus \mathds{Q}(\zeta)   \]
\end{ejemplo}

\begin{ejemplo}
Sea $G = < a \colon a^6 = 1 >$ y $K = \mathds{Q}$. La descomposición de $X^6 - 1 $ en $\mathds{Q}[X]$ es 

\[ X^6 - 1 = (X-1)(X+1)(X^2 + X + 1)(X^2-X+1) \]
entonces se obtiene 
\[  \mathds{Q}G \simeq \mathds{Q} \oplus \mathds{Q} \oplus \mathds{Q}\left( \frac{-1+i\sqrt{3}}{2} \right) \oplus \mathds{Q}\left( \frac{1+i\sqrt{3}}{2} \right)   \]

donde $\frac{-1 + \sqrt{3}}{2}$ es raíz de $X^2+X+1$ y $\frac{1+i\sqrt{3}}{2}$ es raíz de $X^2-X+1$, pero $\mathds{Q}\left( \frac{-1+i\sqrt{3}}{2} \right) \simeq \mathds{Q}\left( \frac{-1-i\sqrt{3}}{2} \right) \simeq \mathds{Q}\left( \frac{-(1+i\sqrt{3})}{2} \right) \simeq \mathds{Q}\left( \frac{1+i\sqrt{3}}{2} \right)$

por lo que en realidad los últimos dos sumandos son iguales, dejando la expresión de la siguiente manera:

\[  \mathds{Q}G \simeq  \mathds{Q} \oplus \mathds{Q}\left( \frac{-1+i\sqrt{3}}{2} \right)   \]





\end{ejemplo}

Los resultados anteriores dan una muy buena descripción de los grupos anillos cuando el anillo es un campo y el grupo es abeliano, por lo cual ahora se trabajará en un caso mas general. 


Para poder hacer esto, se tratará de calcular todos los sumando directos en la descomposición de $KG$.  

El lector debe recordar que para un $d$ entero positivo dado, el polinomio ciclotómico de orden $d$, denotado por $\Phi_d$ , es el producto $\Phi_d = \prod_{j}(x-\zeta_j)$, donde $\zeta_j$ hace el recorrido por todas las raíces primitivas de la unidad de orden $d$. Tambień es conocido que $X^n -1 = \prod_{d\mid n} \Phi_d $, es decir que $X^n -1 $ se puede expresar como el producto de todos los polinomios ciclotómicos $\Phi_d$ en $K[X]$, donde $d$ es un divisor de $n$. Para cada $d$ sea $\Phi_d = \prod_{i=1}^{a_d}f_{d_i}$ la descomposición de $\Phi_d$ como producto de polinomios irreducibles en $K[X]$

Entonces la descomposición de $KG$ puede ser escrita en la forma:

\[ KG \simeq \oplus_{d \mid n} \oplus_{i=1}^{a_d} \frac{K[X]}{(f_{d_i})} \simeq \oplus_{d \mid n} \oplus_{i=1}^{a_d} K(\zeta _{d_i}) \]

donde $\zeta_{d_i}$ denota una raíz de $f_{d_i}  \mbox{,} \quad 1 \leq i \leq a_d $. Para un $d$ fijo, todos los elementos $\zeta_{d_i}$ son raíces primitivas de la unidad de orden $d$, por lo tanto, todos los campos de la forma $K(\zeta_{d_i}), \quad 1\leq i \leq a_d$ son iguales y se puede escribir simplemente 

\[ KG \simeq \oplus_{d \mid n} a_dK(\zeta_d)\]

donde $\zeta_d$ es una raíz primitiva de orden $d$ y $a_dK[\zeta_d]$ denota la suma directa de $a_d$ campos diferentes, todos ellos isomorfos a $K(\zeta_d)$.

Por otro lado, como $grad(f_{d_i}) = [K(\zeta_d):K]$, se deduce que todos los polinomios tienen el mismo grado para $1 \leq i \leq a_d$. De esta forma, calculando el grado en la descomposición de $\Phi_d$, se tiene

\[ \phi(d) = a_d[K(\zeta_d):K] \]

donde $\phi$ es la función totiente de Euler. Como $G$ es un grupo cíclico de orden $n$, para cada divisor de $n$, el número de elementos de orden $d$ en $G$, que se denota con $n_d$, es precisamente $\phi(d)$, entonces:

\[ a_d = \frac{n_d}{| K(\zeta_d) : K|}  \]

\begin{ejemplo}
Sea $G = < a \colon a^n = 1 >$  un grupo cíclico de orden $n$ y $K = \mathds{Q} $. Es conocido que el polinomio $X^n-1$ se descompone en $\mathds{Q}[X]$ como un producto de polinomios ciclotómicos, a saber:

\[ X^n -1 = \prod_{d \mid n}\Phi_d(X)  \]

y los polinomios $\Phi_d$ son irreducibles en $\mathds{Q}[Q]$. Por lo tanto, en este caso en particular, la descomposición de $\mathds{Q}G$ es:

\[ \mathds{Q}G \simeq \oplus_{d\mid n} \mathds{Q}(\zeta_d)  \]

Hay que notar, que como en casos anteriores, bajo este isomorfismo el generador $a$ corresponde a la tupla cuyas entradas son raíces primitivas de la unidad de orden $d$, donde $d$ es cualquier divisor positivo de $n$. 
\end{ejemplo}

Finalmente se cerrará esta sección demostrando un hecho muy importante, a saber, que la caracterización anteriormente dada  también es válida en los grupo-anillos con grupos  abelianos finitos.

\begin{teorema}{Perlis-Walker}
Sea $G$ un grupo finito abeliano de orden $n$ y sea $K$ un campo tal que $char(K)\nmid n$. Entonces

\[  KG \simeq \oplus_{d\mid n}a_dK(\zeta) \]

donde $\zeta_d$ es una raíz primitiva de la unidad de orden $d$ y $a_d = \frac{n_d}{[K(\zeta_d):K]}$. En este fórmula $n_d$ denota el número de elementos de orden $d$ en $G$.
\end{teorema} 

\begin{proof}
Para proceder con la demostración es necesario enunciar y demostrar los siguientes lemas:

\begin{lema}\label{lema1}
Sea $R$ un anillo conmutativo y $G, H$ grupos, entonces $R(G \times H) \simeq (RG)H$ (el grupo-anillo de $H$ sobre el anillo $RG$)
\end{lema}

\begin{proof}
Considérese el conjunto $M_{n,\gamma} = \{ g: (g,h) \in sop(\gamma)\}$. y la función $f \colon R(G \times H) \to (RG)H$ tal que $\gamma \mapsto \beta $ donde $\beta = \sum_{h \in H} \alpha_hh $  con $\alpha_h = \sum_{g \in M_{h,\gamma}}a_{gh}g$. Se debe demostrar que $f$ es una función biyectiva y además es un homomorfismo de anillos.

\textbf{Homomorfismo:} 
\begin{enumerate}
\item \textbf{Conserva sumas:} Sea $\gamma_1, \gamma_2 \in R(G \times H),\quad \gamma_1 = \sum_{g \in G, h \in H}a_{gh}(g,h), \quad  \gamma_2 = \sum_{g\in G, h\in H}b_{gh}(g,h)$. De esta forma se tiene $f(\gamma_1) = \sum_{h \in H}\beta_hh, \quad \beta_h = \sum_{g\in M_{h,\gamma_1}}a_{gh}h$ y $f(\gamma_2) = \sum_{h\in H}\xi_hh,\quad \xi_h = \sum_{g \in M_{h,\gamma_2}}b_{gh}h$.

Haciendo la operatoria se tiene:
\[ f(\gamma_1) + f(\gamma_2) = \sum_{h\in H}(\beta_h+\xi_h)h = \sum_{h\in H}\alpha_hh\]

en donde $\alpha_h = \beta_h + \xi_h$

Por otro lado:

\[ f(\gamma_1) + f(\gamma_2) = f\left(\sum_{g\in G, h\in H} (a_{gh} + b_{gh})g\right) = \sum_{h \in H}\alpha_hh, \quad \alpha_h = \sum_{g \in M_h, \gamma_1 + \gamma_2}(a_{gh} + b_{gh})g  \]

De lo anterior se deduce fácilmente que $\alpha_h = \sum_{g \in M_h, \gamma_1}a_{gh}g + \sum_{g \in M_h, \gamma_2}b_{gh}g = \beta_h + \xi_h$

\item \textbf{Conserva productos:} Sean $\gamma_1,\quad \gamma_2 \in R(G \times H)$, entonces haciendo la operatoria:

\[  \gamma_1\gamma_2 = \sum_{g,m \in G, h,n \in H} a_{gh}b_{mn}(g,h)(m,n) \]

Como ya se ha probado que $f$ conserva sumas, ahora es suficiente demostrar que dados $(g,h),\quad  (m,n) \in (G \times H)$ se cumple que $f((g,h)(m,n)) = f((g,h))f((m,n))$ y que además $f$ es $R-\mbox{lineal}$. En efecto, por un lado 

\[ f((g,h))f((m,n)) = (gh)(nm) = gnhm  \]

y por el otro lado se tiene:

\[  f((g,h)(n,m)) = f((gn,hm)) = gnhm  \]

El hecho de que $f$ es $R-\mbox{lineal}$ se sigue directamente de la definición de $f$.

\item \textbf{$f$ es inyectiva:} Para demostrar que $f$ es inyectiva se debe demostrar que el único elemento que anula a $f$ es elemento neutro de $R(G\times H)$. Para el efecto, considérese $\gamma \in R(G\times H), \quad \gamma = \sum_{g \in G, h\in H}a_{gh}(g,h) $ tal que $f(\gamma) = \sum_{h \in H}\alpha_hh = 0 , \quad \alpha_h = \sum_{g \in M_{h,\gamma} = a_{gh}h}$, lo cual implica que $a_{gh} = 0 \mbox{ para cada} g \in G, h\in H$, de donde $\gamma = 0$

\item \textbf{$f$ es sobreyectiva:} Dado $\sum_{h \in h}\alpha_hh \in (RG)H$ se construye $\gamma = \sum_{g \in G, h\in H}a_{gh}(g,h)$, donde $a_{gh}$, es decir, el coeficiente de $(g,h)$ es el mismo que el de $g$ en $\alpha_h$. Con lo que concluye la prueba. 
\qedhere
\end{enumerate}
\end{proof}

\begin{lema}\label{lema2}
Sea $\{R_i\}_{i\in I}$ una familia de anillos y sea $R = \oplus_{i \in I}R_i$. Entonces para cualquier grupo $G$ se tiene $RG \simeq \oplus_{i \in I}R_iG$
\end{lema}

\begin{proof}
Considérese la función $f \colon \oplus_{i \in I}R_iG \to RG$ dado por $(\alpha_1, \cdots, \alpha_n) \mapsto \sum_{g \in G}a_gg, \quad a_g = (a_g^{(1)}, \cdots, a_g ^{(n)})$, donde $a_g^{(i)}$ es el coeficiente de $g$ en $\alpha_i = \sum_{g \in G}a_g^{(i)}g$. Se debe comprobar que $f$ es un homomorfismo de anillos.
\begin{enumerate}
\item \textbf{Conserva sumas:} Sean $\alpha = (\alpha_1, \cdots, \alpha_n), \quad \beta = (\beta_1, \cdots, \beta_n) \in \oplus_{i \in I}R_iG$, entonces su suma viene dada por $\gamma = (\alpha_1 + \beta_1, \cdots, \alpha_n + \alpha_n)$, y con ello la imagen de la suma seria $f(\gamma) = \sum_{g \in G}c_gg, \quad c_g = (a_g^{(1)}, \cdots, a_g^{(n)})$. 

Por otro lado, se tiene:
\begin{eqnarray*}
f(\alpha) + f(\beta) &=& \sum_{g \in G}a_gg + \sum_{g\in G}b_gg\\
 &=& \sum_{g \in G}(a_g + b_g)g \\
  & = &  \sum_{g \in G}d_gg, \quad d_g = (a_g^{(1)} + b_g{(1)}, \cdots, a_g^{(n)} + b_g{(n)} )
\end{eqnarray*}

por lo tanto $f(\alpha + \beta) = f(\alpha) + f(\beta)$

\item \textbf{Conserva productos:} Como en el caso anterior, se tiene $\gamma = \alpha\beta = (\alpha_1\beta_1, \cdots, \alpha_n\beta_n)$, y por lo tanto, su imagen bajo $f$, es $f(\gamma) = \sum_{u \in G}c_uu, \quad c_u = (c_u^{(1)}, \cdots, c_u^{(n)}), \quad c_u^{(i)} = \sum_{gh=u}a_g^{(i)}b_h^{(i)}$.

Por otro lado, $f(\alpha) = \sum_{g \in G}a_gg, \quad f(\beta) = \sum_{g \in G}b_gg$, multiplicando, se obtiene $f(\alpha)f(\beta) = \sum_{u \in G}d_uu, \quad d_u = \sum_{gh = u }a_gb_h = \left( \sum_{gh=u}a_g^{(1)}b_g^{(1)}, \cdots, \sum_{gh=u}a_g^{(n)}b_g^{(n) } \right) = c_u$

\item \textbf{$f$ es inyectiva:} Supóngase que $f(\alpha) = \sum_{g\in G}a_gg = 0$ entonces $a_g = (0, \cdots, 0)$, de donde $\alpha = (0, \cdots, 0)$ 

\item \textbf{$f$ es sobreyectiva: } Dado $\sum_{g \in G}a_gg, \quad a_g = (a_g^{(1)}, \cdots, a_g^{(n)})$. Entonces se construye $\alpha = \left( \sum_{g\in G}a_g^{(1)}g, \cdots, \sum_{g\in G}a_g^{(n)}g  \right)$ y es fácil verificar que $f(\alpha) = \sum_{g \in G}a_gg$ \qedhere
\end{enumerate}
\end{proof}



Para demostrar el teorema se procede por inducción sobre el orden de $G$. Supóngase que el resultado es válido para cualquier grupo abeliano de orden menor que $n$. 

Sea $G$ tal que $|G| = n$. Si $G$ es generado no hay algo que demostrar. Si $G$ no fuera un grupo generado se puede utilizar el teorema de estructura \footnote{poner en capitulo 1} de los grupos finitos abelianos para escribir $G = G_1 x H$ donde $H$ 4s generado y $|G_1| = n_1 < n$. Por hipótesis de inducción se puede escribir 

\[ RG_1 \simeq \oplus_{d_1\mid n_1} a_{d_1}K(\zeta_{d_1})   \]
donde $a_{d_1} = \frac{n_{d_1}}{[K(\zeta_{d_1}):K]}$ y $n_{d_1}$ denota el numero de elementos de orden $d_1$ en $G_1$. Aplicando el lema \ref{lema1} se cumple

\[  RG = R(G_1 x H) \simeq (RG_1)H \simeq \left(\oplus_{d_1\mid n_1}a_{d_1}K(\zeta_{d_1})\right) H   \]

utilizando el lema \ref{lema2} se obtiene

\[ \left( \oplus_{d_1\mid n_1}a_{d_1}K(\zeta_{d_1}) \right)H \simeq \oplus_{d_1 \mid n_1} a_{d_1}K(\zeta_{d_1})H   \]

Como $H$ es cíclico se puede escribir

\[ \oplus_{d_1\mid n_1}\oplus_{d_2\mid |H|}a_{d_1}a_{d_2}K(\zeta_{d_1}, \zeta_{d_2})  \]

donde $a_{d_2} = \frac{n_{d_2}}{[K(\zeta_{d_1}, \zeta_{d_2} ): K(\zeta_{d_1})]}$ y $n_{d_2}$ es el número de elementos en $H$ de orden $d_2$.

Sea $d = [d_1, d_2]$ entonces por el teorema del elemento primitivo, se tiene $K(\zeta_d) = K(d_1,d_2)$ por tanto 

\[ KG \simeq \oplus_{d\mid n}a_dK(\zeta_d)  \]

con $a_d = \sum_{d1,d2}a_{d_1}a_{d_2}$ y donde la  suma recorre todos los $d_1,d_2$ son números naturales tales que $[d_1,d_2] = d$. Por otro lado, del hecho que $[K(\zeta_d):K] = [K(\zeta_{d_1, \zeta_{d_2}}): K(\zeta_{d_1})][K(\zeta_{d_1}):K] $ se tiene que: 


\[ a_d[K(\zeta_{d} :K)] = \sum_{d_1,d_2}a_{d_1}a_{d_2}[K(\zeta_{d_1, \zeta_{d_2}}): K(\zeta_{d_1})][K(\zeta_{d_1}):K] = \sum_{d_1,d_2}n_{d_1}n_{d_2}   \]

\end{proof}

\include{representaciones}

\include{chap4}

\chapter{UNIDADES DE LOS GRUPO-ANILLOS}
\section{\quad Algunas formas de construir unidades}
Sea $R$ un anillo. Se entiende por $\mathcal{U}(R) = \{x \in R \colon (\exists y \in R)xy=yx=1\}$. 
En particular, dado un grupo $G$ y un anillo $R$, $\mathcal{U}(RG)$ denota al grupo de unidades del grupo-anillo $RG$. Como la función de aumento $\mathcal{E} \colon RG \to R$, dada por $\mathcal{E} \left( \sum a(g)g \right) = \sum a(g)$, es un homomorfismo de anillos, se tiene que $\mathcal{E}(u) \in \mathcal{U}(R)$, para todo $u \in \mathcal{U}(RG).$ Se denotará como $\mathcal{U}_1(RG)$ el subgrupo de unidades de aumento $1$ en $\mathcal{U}(RG)$, a saber
\[\mathcal{U}_1(RG) = \{u \in \mathcal{U}(RG) \colon \mathcal{E}(u) = 1 \}.\] Para una unidad $u$ del grupo-anillo integral $\mathcal{Z}G$ se tiene que $\mathcal{E}(u) = \pm1$, entonces es claro que \[ \mathcal{U}(\mathcal{Z}G) = \pm \mathcal U_1(\mathcal{Z}G) .\]
De la misma manera, para un anillo $R$ arbitario se tiene que \[ \mathcal{U}(RG) = \mathcal{U}(R) \times \mathcal{U}_1(RG). \]
No se conocen muchas formas para construir unidades. La mayoría de las construcciones conocidas son antiguas y elementales. A lo largo de este capítulo, se mostrará y describirá algunas de estas construcciones, donde se trabajará principalmente con grupo-álgebras $KG$ sobre un campo $K$ y con el grupo-anillo integral $\mathcal{Z}G$.
\begin{ejemplo}[Unidades Triviales]
Un elemento de la forma $rg$, donde $r \in \mathcal{U}(R)$ y $g \in G$, tiene inversa $r^{-1}g^{-1}$. Los elementos de esta forma son llamados \textbf{unidades triviales} de $RG$. De esta manera, por ejemplo, los elementos $\pm g, g \in G$ son las unidades triviales del grupo-anillo integral $\mathcal{Z}G$. Si $K$ es un campo, entonces las unidades triviales de $KG$ son los elementos de la forma $kg, k \in K, k \neq 0, g \in G$. Hablando de manera general, los grupo-anillos contienen unidades no triviales.
\end{ejemplo}
\begin{ejemplo}\label{ejem:unipotentes}
Sea $\eta \in R$ tal que $\eta^2 =0$, entonces se tiene $(1+\eta)(1-\eta)=1$. De este hecho, tanto $1+\eta$ como $1-\eta$ son unidades de $R$. De la misma manera, si $\eta \in R$ es tal que $\eta ^k =0$ para algún entero positivo $k$, entonces se tiene que
\begin{equation*}
(1-\eta)(1+\eta+\eta^2+\cdots+\eta^{k-1}) =  1-\eta^{k} = 1,
\end{equation*}
\begin{equation*}
(1+\eta)(1-\eta+\eta^2+\cdots\pm \eta^{k-1}) =  1 \pm \eta^{k} = 1.
\end{equation*}
Así, $1\pm \eta$ son unidades de $R$. Estas unidades son llamadas \textbf{unidades unipotentes} de $R$. En un grupo-álgebra $KG$ sobre un campo de característica $p>0$ se puede iniciar la búsqueda de unidades unipotentes investigando a los elementos nilpotentes. Si $g \in G$ es de orden $p^n$, entonces $(1-g)^{p^n} = 0$, de esta forma se demuestra que $\mu = 1-g$ es nilpotente.

En este caso $1-\eta = g$ es trivial, pero $1+\eta = 2-g$ es no trivial, a menos qu $\car(K)=2$. Nótese que $g-g^2=g(1-g)$ también es nilpotente, entonces $1+g-g^2$ es una unidad no trivial si $g^2 \neq 1$.

En el teorema \ref{teo:caracCar0} y \ref{teo:caracCarEntera} se clasificaron  todos los grupos finitos tal que el grupo-álgebra $KG$ no tiene elementos nilpotentes. Se vera entonces que las grupo-álgebras de grupos finitos casi siempre tienen unidades no triviales.
\end{ejemplo}




\chapter{\quad Aplicaciones}
En este capítulo se darán los conceptos básicos de teoría de códigos. Se empezará dando una descripción del sistema de comunicación como lo propuso Claude E. Shannon \footnote{Claude Elwood Shannon (Míchigan, 30 de abril de 1916 - 24 de febrero de 2001) fue un ingeniero electrónico y matemático estadounidense, recordado como «el padre de la teoría de la información»} en 1948. En esta parte también se introducirán todos los conceptos básicos del sistema de comunicación como canal, codificador, decodificador y código. Una vez la teoría básica está dada se hace un breve estudio de los códigos lineales, para terminar este capítulo con una clase de códigos en particular, los cíclicos. 

\section{\quad Sistema de Comunicación}
La figura \ref{fig:sistemaComunicacion} muestra un sistema de comunicación de una \textbf{fuente} a un \textbf{destino} mediante un \textbf{canal}. La comunicación puede ser en el dominio del espacio - es decir, de un punto a otro - o en el dominio del tiempo - es decir, al guardar información en algún punto en el tiempo para ser recuperada posteriormente-.
\begin{figure}
\centering
\caption{Sistema de Comunicación propuesto por Shannon}
\includegraphics[angle=-90,width=0.7\linewidth]{sistema.pdf}
\caption*{Fuente: Elaboración propia con software Dia.}
\label{fig:sistemaComunicacion}
\end{figure}
La codificación de la fuente tiene doble propósito. Primero, servir como traductor entre la salida de la fuente y la entrada al canal. Por ejemplo, si la información transmitida de la fuente al destino está en señal análoga y el canal espera recibir señal digital, se necesitará una conversión de análoga a digital en la fase de codificación y un convertidor de señal digital a análoga en la gase de decodificación. Como segunda función se podría requerir que el codificador de la fuente comprima la salida de la fuente para economizar en la longitud de la transmisión, eso significa que en el otro extremo, el decodificador de la fuente necesitará descomprimir la señal.

Algunas aplicaciones necesitan que el decodificador restaure la información para que sea idéntica a la original, en cuyo caso se dice que la compresión es \textbf{sin pérdidas}.

Otras aplicaciones, como la mayoría de transmisiones de audio e imágenes, permiten una diferencia -controlada- o distorsión entre la información original y la restaurada, así que esta posibilidad es usada para lograr mayor compresión. En este caso se dice que la compresión es \textbf{con pérdidas.}

Los canales no son perfectos debido a limitaciones físicas y de ingeniería, es decir, su salida puede diferir de su entrada debido al ruido o a defectos de fabricación.

Más aún, en algunos casos el diseño requiere que el formato de la información de salida del canal difiera del formato de entrada. Además hay aplicaciones tales como los medios de almacenamiento masivo mágnetico y óptico, donde no se permiten ciertos patrones en el flujo de bits a transmitir. Dado esto, el rol principal del codificador del canal, es superar estas limitaciones y hacer el canal tan transparente  como sea posible, tanto desde el punto de vista de la fuente como del destino. 

Es así como entran a participar los códigos. Los códigos fueron inventados para corregir errores en los canales de comunicación debido al ruido. Por ejemplo, supóngase que hay un cable telegráfico desde Ciudad de Guatemala hasta Ciudad de Panamá, mediante el cual se pueden transmitir unos y ceros. Usualmente cuando un cero es enviado se recibe un cero, pero ocasionalmente un cero puede ser recibido como un uno o un uno como un cero. Supóngase que en promedio, 1 de cada 100 símbolos se recibe de forma errónea, es decir, por cada símbolo hay una probabilidad $p=1/100$ de que ocurra un error en el canal. A esto se le llama un canal binario simétrico y se denota como BSC.
\begin{figure}
\centering
\caption{Canal Binario Simétrico}
\includegraphics[angle=-90,width=0.6\linewidth]{CBS.pdf}
\caption*{Fuente: Elaboración propia con software Dia.}
\label{fig:CBS}
\end{figure} 
Además supóngase que se enviarán muchos mensajes importantes por ese cable y se necesita enviarlos de manera rápida y segura. Los mensajes ya se encuentran escritos como cadenas de ceros y unos, producidos, quizás, por alguna computadora.

Se van a \textbf{codificar} estos mensajes para darles una protección en contra del ruido del canal. Un bloque de $k$ símbolos del mensaje $u = u_1\dots u_k, u_i = 0 \mbox{ o } 1$, será codificado como una \textbf{palabra-código} $x = x_1 \dots x_n, x_i = 0 \mbox{ o } 1$ donde $n \geq k$ (véase la figura \ref{fig:codificacion}). Estas palabra-códigos forman un código.
\begin{figure}
\caption{Proceso de Codificación}
\centering
% Graphic for TeX using PGF
% Title: /mnt/DataWin/Dropbox/Tesis/Tesis Actual/codi.dia
% Creator: Dia v0.97.2
% CreationDate: Wed Dec 11 21:23:59 2013
% For: hugo
% \usepackage{tikz}
% The following commands are not supported in PSTricks at present
% We define them conditionally, so when they are implemented,
% this pgf file will use them.
\ifx\du\undefined
  \newlength{\du}
\fi
\setlength{\du}{15\unitlength}
\begin{tikzpicture}
\pgftransformxscale{0.700000}
\pgftransformyscale{-1.000000}
\definecolor{dialinecolor}{rgb}{0.000000, 0.000000, 0.000000}
\pgfsetstrokecolor{dialinecolor}
\definecolor{dialinecolor}{rgb}{1.000000, 1.000000, 1.000000}
\pgfsetfillcolor{dialinecolor}
\pgfsetlinewidth{0.100000\du}
\pgfsetdash{}{0pt}
\pgfsetdash{}{0pt}
\pgfsetmiterjoin
\definecolor{dialinecolor}{rgb}{1.000000, 1.000000, 1.000000}
\pgfsetfillcolor{dialinecolor}
\fill (3.731981\du,6.945629\du)--(3.731981\du,13.482575\du)--(9.631981\du,13.482575\du)--(9.631981\du,6.945629\du)--cycle;
\definecolor{dialinecolor}{rgb}{0.000000, 0.000000, 0.000000}
\pgfsetstrokecolor{dialinecolor}
\draw (3.731981\du,6.945629\du)--(3.731981\du,13.482575\du)--(9.631981\du,13.482575\du)--(9.631981\du,6.945629\du)--cycle;
% setfont left to latex
\definecolor{dialinecolor}{rgb}{0.000000, 0.000000, 0.000000}
\pgfsetstrokecolor{dialinecolor}
\node[anchor=west] at (4.006981\du,9.767031\du){Fuente del };
% setfont left to latex
\definecolor{dialinecolor}{rgb}{0.000000, 0.000000, 0.000000}
\pgfsetstrokecolor{dialinecolor}
\node[anchor=west] at (4.006981\du,11.089242\du){Mensaje};
\pgfsetlinewidth{0.100000\du}
\pgfsetdash{}{0pt}
\pgfsetdash{}{0pt}
\pgfsetmiterjoin
\definecolor{dialinecolor}{rgb}{1.000000, 1.000000, 1.000000}
\pgfsetfillcolor{dialinecolor}
\fill (13.386981\du,8.142031\du)--(13.386981\du,12.282031\du)--(21.636981\du,12.282031\du)--(21.636981\du,8.142031\du)--cycle;
\definecolor{dialinecolor}{rgb}{0.000000, 0.000000, 0.000000}
\pgfsetstrokecolor{dialinecolor}
\draw (13.386981\du,8.142031\du)--(13.386981\du,12.282031\du)--(21.636981\du,12.282031\du)--(21.636981\du,8.142031\du)--cycle;
% setfont left to latex
\definecolor{dialinecolor}{rgb}{0.000000, 0.000000, 0.000000}
\pgfsetstrokecolor{dialinecolor}
\node[anchor=west] at (14.561981\du,10.357031\du){Codificador};
\pgfsetlinewidth{0.100000\du}
\pgfsetdash{}{0pt}
\pgfsetdash{}{0pt}
\pgfsetbuttcap
{
\definecolor{dialinecolor}{rgb}{0.000000, 0.000000, 0.000000}
\pgfsetfillcolor{dialinecolor}
% was here!!!
\pgfsetarrowsstart{to}
\definecolor{dialinecolor}{rgb}{0.000000, 0.000000, 0.000000}
\pgfsetstrokecolor{dialinecolor}
\draw (13.386981\du,10.212031\du)--(9.631981\du,10.214102\du);
}
\pgfsetlinewidth{0.100000\du}
\pgfsetdash{}{0pt}
\pgfsetdash{}{0pt}
\pgfsetbuttcap
{
\definecolor{dialinecolor}{rgb}{0.000000, 0.000000, 0.000000}
\pgfsetfillcolor{dialinecolor}
% was here!!!
\pgfsetarrowsstart{to}
\definecolor{dialinecolor}{rgb}{0.000000, 0.000000, 0.000000}
\pgfsetstrokecolor{dialinecolor}
\draw (26.770935\du,10.256598\du)--(21.636981\du,10.212031\du);
}
% setfont left to latex
\definecolor{dialinecolor}{rgb}{0.000000, 0.000000, 0.000000}
\pgfsetstrokecolor{dialinecolor}
\node[anchor=west] at (9.731981\du,11.042031\du){Mensaje};
% setfont left to latex
\definecolor{dialinecolor}{rgb}{0.000000, 0.000000, 0.000000}
\pgfsetstrokecolor{dialinecolor}
\node[anchor=west] at (9.731981\du,11.842031\du){$u_1\dots u_k$};
% setfont left to latex
\definecolor{dialinecolor}{rgb}{0.000000, 0.000000, 0.000000}
\pgfsetstrokecolor{dialinecolor}
\node[anchor=west] at (21.381981\du,11.292031\du){Palabra-código};
% setfont left to latex
\definecolor{dialinecolor}{rgb}{0.000000, 0.000000, 0.000000}
\pgfsetstrokecolor{dialinecolor}
\node[anchor=west] at (21.881981\du,12.092031\du){$x_1\dots x_n$};
\pgfsetlinewidth{0.100000\du}
\pgfsetdash{}{0pt}
\pgfsetdash{}{0pt}
\pgfsetmiterjoin
\definecolor{dialinecolor}{rgb}{1.000000, 1.000000, 1.000000}
\pgfsetfillcolor{dialinecolor}
\fill (26.820292\du,7.014162\du)--(26.820292\du,13.551108\du)--(32.720292\du,13.551108\du)--(32.720292\du,7.014162\du)--cycle;
\definecolor{dialinecolor}{rgb}{0.000000, 0.000000, 0.000000}
\pgfsetstrokecolor{dialinecolor}
\draw (26.820292\du,7.014162\du)--(26.820292\du,13.551108\du)--(32.720292\du,13.551108\du)--(32.720292\du,7.014162\du)--cycle;
% setfont left to latex
\definecolor{dialinecolor}{rgb}{0.000000, 0.000000, 0.000000}
\pgfsetstrokecolor{dialinecolor}
\node[anchor=west] at (28.299923\du,10.484211\du){Canal};
% setfont left to latex
\definecolor{dialinecolor}{rgb}{0.000000, 0.000000, 0.000000}
\pgfsetstrokecolor{dialinecolor}
\node[anchor=west] at (28.168743\du,15.632869\du){Ruido};
\pgfsetlinewidth{0.100000\du}
\pgfsetdash{}{0pt}
\pgfsetdash{}{0pt}
\pgfsetbuttcap
{
\definecolor{dialinecolor}{rgb}{0.000000, 0.000000, 0.000000}
\pgfsetfillcolor{dialinecolor}
% was here!!!
\pgfsetarrowsstart{to}
\definecolor{dialinecolor}{rgb}{0.000000, 0.000000, 0.000000}
\pgfsetstrokecolor{dialinecolor}
\draw (29.770292\du,13.551108\du)--(29.802718\du,15.192715\du);
}
\end{tikzpicture}

\caption*{Fuente: Elaboración propia con software Dia y exportado a \TeX}
\label{fig:codificacion}
\end{figure}
La primera parte de la palabra-código consiste en ele mensaje mismo: \[ x_1 = u_1, x_2 = u_2, \dots , x_k = u_k, \] seguido de $n-k$ símbolos de comparación $x_{k+1, \dots , x_n}$. 
Los símbolos de comparación son elegidos de tal forma que las palabra-códigos satisfagan 
\[ H \begin{pmatrix}
x_1 \\ 
x_2 \\
\vdots \\
x_n
\end{pmatrix} = Hx^{t} = 0, \] donde la matriz $H$ de $(n-k)\times k$ es la matriz de comparación de paridad del código, dada por 
\begin{equation}\label{ecu:defCodigo}
H = [A \mid I_{n-k}],
\end{equation}  
donde $A$ es una matriz fija de $(n-k)\times k$ de ceros y unos y \[ I = \begin{pmatrix}
1 & 0 & \dots & 0 \\
0 & 1 & \dots & 0 \\
\vdots & \vdots & \ddots & \vdots \\
0 & 0 & \dots & 1
\end{pmatrix} \] es la matriz identidad de $(n-k) \times (n-k)$. La aritmética en la ecuación \ref{ecu:defCodigo} se hace en módulo 2, es decir que se está trabajando con el campo $\mathds{Z}_2$. 
\begin{ejemplo}
La matriz de comparación de paridad 
\[ H = \left(\begin{array}{ccc|ccc}
0 & 1 & 1 & 1 & 0 & 0 \\
1 & 0 & 1 & 0 & 1 & 0 \\
1 & 1 & 0 & 0 & 0 & 1
\end{array}\right) \] define un código con $k=3$ y $n=6$. Para este código 
\[ A = \begin{pmatrix}
0 & 1 & 1\\
1 & 0 & 1 \\
1 & 1 & 0
\end{pmatrix}.  \]
El mensaje $u_1u_2u_3$ es codificado en la palabra-código $x = x_1x_2x_3x_4x_5x_6$ que empieza con el propio mensaje:
\[ x_1 = u_1, x_2 = u_2, x_3 = u_3, \] seguido de tres símbolos de comparación $x_4x_5x_6$ tales que $Hx^t = 0$, es decir, se cumple
\begin{eqnarray}
x_2 + x_3 + x_4 = 0, \\

\end{eqnarray}
\end{ejemplo}




\backmatter     % --------------------------------------------------  Apartados finales

\chapter{CONCLUSIONES}

\vspace{6pt}

\begin{enumerate}[1.\hspace{5ex}]
    \item Conclusiones ($c\_y\_r$.tex)
\end{enumerate}

% -------------------------------------------------------------------------------

\chapter{RECOMENDACIONES}

\vspace{6pt}

\begin{enumerate}[1.\hspace{5ex}]
    \item Recomendaciones ($c\_y\_r$.tex)
\end{enumerate}
     % Conclusiones y recomendaciones

\par}

\begin{thebibliography}{99}
    \addcontentsline{toc}{chapter}{BIBLIOGRAFÍA}
    \bibitem{bib:historia} T. Hawkins. \textit{The origins of the Theory of Group Characters}, Archive Hist. Exact Sci. $7$ (1970-71). p. 142-170.
    \bibitem{bib:burnside} WILLIAM, Burnside, \textit{The theory of Groups of Finite Order}. 	2da ed. Cambridge: Cambridge University Press, 1911.
    \bibitem{bib:grupsfact}  GOLDSHMIDT, David. \textit{A group theoretic proof of the $p^aq^b$ theorem for odd primes}. Math. Z. 13 (1970). p 373-375.
    \bibitem{bib:solubilidad} WALTER, Feit y JHON, Thompson. \textit{The solvability of groups of odd order}. Pacific J. Math. 15 (1963). p 775-1029.
    \bibitem{bib:lang}SERGE, Lang. \textit{Linear Algebra}. 3ra ed. Nueva York: Springer-Verlag, 2004. 308 p.
    \bibitem{bib:passman} DONALD, Passman. \textit{The algebraic structure of group rings}. New York: Wiley-Interscience, 1977. 550 p.
    \bibitem{bib:moser} CLAUDE, Moser. \textit{Representation de -1 comme somme de carres dans un corps cyclotomique quelconque}. J. Number Theory 5 (1973), 138-141.
    \bibitem{bib:Sehgal} SUDARSHAN, Sehgal.  \textit{Topics in Group Rings}.New York: Marcel Dekker, 1978. 233 p. 
    \bibitem{bib:herstein} NATHAN, Herstein. \textit{Topics in Algebra}. 2nda ed.  New York: Macmillah, 1986 . 381 p. 

\end{thebibliography}
   % Bibliograf�a
\end{document}




